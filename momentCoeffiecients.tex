%
% Copyright � 2016 Peeter Joot.  All Rights Reserved.
% Licenced as described in the file LICENSE under the root directory of this GIT repository.
%
%{
%\input{../blogpost.tex}
%\renewcommand{\basename}{momentCoeffiecients}
%%\renewcommand{\dirname}{notes/phy1520/}
%\renewcommand{\dirname}{notes/ece1228-electromagnetic-theory/}
%%\newcommand{\dateintitle}{}
%%\newcommand{\keywords}{}
%
%\input{../latex/peeter_prologue_print2.tex}
%
%\usepackage{peeters_layout_exercise}
%\usepackage{peeters_braket}
%\usepackage{peeters_figures}
%\usepackage{siunitx}
%%\usepackage{txfonts} % \ointclockwise
%
%\beginArtNoToc
%
%\generatetitle{Dipole and Quadrupole electrostatic potential moments and coefficients}
%\chapter{Dipole and Quadropole electrostatic potential moments and coefficents}
%\label{chap:momentCoeffiecients}
%
\section{Explicit moment and quadrupole expansion.}
%
We calculated the \( q_{1,1} \) coefficient of the electrostatic moment, as covered in \citep{jackson1975cew} chapter 4.  Let's verify the rest, as well as the tensor sum formula for the quadrupole moment, and the spherical harmonic sum that yields the dipole moment potential.
%
%%%XX
%%%\begin{equation}\label{eqn:momentCoeffiecients:20}
%%%q_{l,m} =
%%%\int (r')^l \rho(\Bx')
%%%Y^\conj_{l,m}(\theta', \phi')
%%%d^3 x',
%%%\end{equation}
%%%
%%%The class notes also give the results for \( q_{0,0}, q_{1,0}, q_{2,2}, q_{2,1}, q_{2,0} \).  Let's verify those
%%%
%%%\paragraph{\(q_{0,0}\)}
%%%
%%%\begin{equation}\label{eqn:momentCoeffiecients:40}
%%%\begin{aligned}
%%%q_{0,0}
%%%&=
%%%\int (r')^0 \rho(\Bx')
%%%Y^\conj_{0,0}(\theta', \phi')
%%%d^3 x'
%%%\\ &=
%%%\inv{4\pi}
%%%\int \rho(\Bx') d^3 x'
%%%\\ &=
%%%\frac{q}{4\pi}
%%%
%%%\end{aligned}
%%%\end{equation}
%%%
%%%\paragraph{\(q_{1,0}\)}
%%%
%%%\begin{equation}\label{eqn:momentCoeffiecients:60}
%%%\begin{aligned}
%%%q_{1,0}
%%%&=
%%%\int r' \rho(\Bx')
%%%Y^\conj_{1,0}(\theta', \phi')
%%%d^3 x'
%%%\\ &=
%%%\sqrt{\frac{3}{4\pi}}
%%%\int r' \rho(\Bx')
%%%\cos\theta'
%%%d^3 x'
%%%\\ &=
%%%\sqrt{\frac{3}{4\pi}}
%%%\int r' \rho(\Bx') \cos\theta' d^3 x'
%%%\\ &=
%%%\sqrt{\frac{3}{4\pi}}
%%%\int z' \rho(\Bx') d^3 x'
%%%\\ &=
%%%\sqrt{\frac{3}{4\pi}} p_z
%%%\end{aligned}
%%%\end{equation}
%%%
%%%\paragraph{\(q_{2,2}\)}
%%%
%%%\begin{equation}\label{eqn:momentCoeffiecients:80}
%%%\begin{aligned}
%%%q_{2,2}
%%%&=
%%%\int (r')^2 \rho(\Bx')
%%%Y^\conj_{2,2}(\theta', \phi')
%%%d^3 x'
%%%\\ &=
%%%\sqrt{\frac{15}{32 \pi}}
%%%\int (r')^2 \rho(\Bx')
%%%\sin^2 \theta e^{-2 i\phi}
%%%d^3 x'
%%%\\ &=
%%%\sqrt{\frac{15}{32 \pi}}
%%%\int (r')^2 \rho(\Bx')
%%%\sin^2 \theta \lr{ \cos \phi - i \sin\phi }^2
%%%d^3 x'
%%%\\ &=
%%%\sqrt{\frac{15}{32 \pi}}
%%%\int (r')^2 \rho(\Bx')
%%%\sin^2 \theta \lr{ \cos^2\phi - \sin^2\phi - 2 i \cos\phi \sin\phi }
%%%d^3 x'
%%%\\ &=
%%%\sqrt{\frac{15}{32 \pi}}
%%%\int \rho(\Bx') \lr{
%%%(x')^2
%%%- (y')^2
%%%- 2 i x' y' } d^3 x'
%%%\\ &=
%%%\sqrt{\frac{15}{32 \pi}}
%%%\int \rho(\Bx') \lr{
%%%x' - i y'
%%%}^2 d^3 x'
%%%%\\ &=
%%%%\sqrt{\frac{15}{32 \pi}}
%%%%\lr{ p_x - i p_y }^2
%%%\end{aligned}
%%%\end{equation}
%%%
%%%\paragraph{\(q_{2,1}\)}
%%%
%%%\begin{equation}\label{eqn:momentCoeffiecients:100}
%%%\begin{aligned}
%%%q_{2,1}
%%%&=
%%%\int (r')^2 \rho(\Bx')
%%%Y^\conj_{2,1}(\theta', \phi')
%%%d^3 x'
%%%\\ &=
%%%-\sqrt{\frac{15}{8 \pi}}
%%%\int (r')^2 \rho(\Bx')
%%%\sin\theta' \cos\theta' e^{-i \phi}
%%%d^3 x'
%%%\\ &=
%%%-\sqrt{\frac{15}{8 \pi}}
%%%\int (r')^2 \rho(\Bx')
%%%\sin\theta' \cos\theta' \lr{ \cos\phi - i \sin\phi }
%%%d^3 x'
%%%\\ &=
%%%-\sqrt{\frac{15}{8 \pi}}
%%%\int \rho(\Bx')
%%%\lr{ x' z' - i y' z' }
%%%d^3 x'
%%%%\\ &=
%%%%-\sqrt{\frac{15}{8 \pi}} p_z \lr{ p_x - i p_y }.
%%%\end{aligned}
%%%\end{equation}
%%%
%%%\paragraph{\(q_{2,0}\)}
%%%
%%%\begin{equation}\label{eqn:momentCoeffiecients:260}
%%%\begin{aligned}
%%%q_{2,0}
%%%&=
%%%\int (r')^2 \rho(\Bx')
%%%Y^\conj_{2,0}(\theta', \phi')
%%%d^3 x'
%%%\\ &=
%%%\int (r')^2 \rho(\Bx') \sqrt{\frac{5}{4\pi}} \lr{ \frac{3}{2} \cos^2\theta - \inv{2} }
%%%d^3 x'
%%%\\ &=
%%%\inv{2} \sqrt{\frac{5}{4\pi}}
%%%\int \rho(\Bx')
%%%\lr{ 3 (z')^2 - (r')^2 }
%%%d^3 x'.
%%%\end{aligned}
%%%\end{equation}
%%%
%%%\paragraph{\(Q_{ij}\)}
%%%XX
The quadrupole term of the potential was stated to be
%
\begin{equation}\label{eqn:momentCoeffiecients:120}
\begin{aligned}
\inv{4 \pi \epsilon_0} &\frac{4 \pi}{5 r^3} \sum_{m=-2}^2 \int (r')^2 \rho(\Bx') Y_{lm}^\conj(\theta', \phi') Y_{lm}(\theta, \phi) \\
&=
\inv{2} \sum_{ij} Q_{ij} \frac{x_i x_j}{r^5},
\end{aligned}
\end{equation}
%
where
%
\begin{equation}\label{eqn:momentCoeffiecients:140}
Q_{i,j} = \int \lr{ 3 x_i' x_j' - \delta_{ij} (r')^2 } \rho(\Bx') d^3 x'.
\end{equation}
%
Let's verify this.  First note that
%
\begin{equation}\label{eqn:momentCoeffiecients:160}
Y_{l,m} = \sqrt{\frac{2 l + 1}{4 \pi} \frac{(l-m)!}{(l+m)!}} P_l^m(\cos\theta) e^{i m \phi},
\end{equation}
%
and
\begin{equation}\label{eqn:momentCoeffiecients:180}
P_l^{-m}(x) =
(-1)^m \frac{(l-m)!}{(l+m)!} P_l^m(x),
\end{equation}
%
so
\begin{equation}\label{eqn:momentCoeffiecients:200}
\begin{aligned}
Y_{l,-m}
&= \sqrt{\frac{2 l + 1}{4 \pi} \frac{(l+m)!}{(l-m)!} }
P_l^{-m}(\cos\theta)
e^{-i m \phi}
\\ &=
(-1)^m
\sqrt{\frac{2 l + 1}{4 \pi} \frac{(l-m)!}{(l+m)!} }
P_l^m(x)
e^{-i m \phi}
\\ &=
(-1)^m Y_{l,m}^\conj.
\end{aligned}
\end{equation}
%
That means
%
\begin{equation}\label{eqn:momentCoeffiecients:220}
\begin{aligned}
q_{l,-m}
&=
\int (r')^l \rho(\Bx')
Y^\conj_{l,-m}(\theta', \phi')
d^3 x'
\\ &=
(-1)^m
\int (r')^l \rho(\Bx')
Y_{l,m}(\theta', \phi')
d^3 x'
\\ &=
(-1)^m q_{lm}^\conj.
\end{aligned}
\end{equation}
%
In particular, for \( m \ne 0 \)
%
\begin{equation}\label{eqn:momentCoeffiecients:320}
(r')^l Y_{l, m}^\conj (\theta', \phi') r^l Y_{l, m}(\theta, \phi)
+ (r')^l Y_{l, -m}^\conj (\theta', \phi') r^l Y_{l, -m}(\theta, \phi)
=
(r')^l Y_{l, m}^\conj (\theta', \phi') r^l Y_{l, m}(\theta, \phi)
+ (r')^l Y_{l, m} (\theta', \phi') r^l Y_{l, m}^\conj(\theta, \phi) ,
\end{equation}
%
or
\begin{equation}\label{eqn:momentCoeffiecients:340}
(r')^l Y_{l, m}^\conj (\theta', \phi') r^l Y_{l, m}(\theta, \phi)
+ (r')^l Y_{l, -m}^\conj (\theta', \phi') r^l Y_{l, -m}(\theta, \phi)
=
2 \Real \lr{ (r')^l Y_{l, m}^\conj (\theta', \phi') r^l Y_{l, m}(\theta, \phi) }.
\end{equation}
%
To verify the quadrupole expansion formula in a compact way it is helpful to compute some intermediate results.
%
\begin{equation}\label{eqn:momentCoeffiecients:360}
\begin{aligned}
r Y_{1, 1}
&= -r \sqrt{\frac{3}{8 \pi}} \sin\theta e^{i\phi}
\\ &= -\sqrt{\frac{3}{8 \pi}} (x + i y),
\end{aligned}
\end{equation}
%
\begin{equation}\label{eqn:momentCoeffiecients:380}
\begin{aligned}
r Y_{1, 0}
&= r \sqrt{\frac{3}{4 \pi}} \cos\theta
\\ &= \sqrt{\frac{3}{4 \pi}} z,
\end{aligned}
\end{equation}
%
\begin{equation}\label{eqn:momentCoeffiecients:400}
\begin{aligned}
r^2 Y_{2, 2}
&= -r^2 \sqrt{\frac{15}{32 \pi}} \sin^2\theta e^{2 i\phi}
\\ &= - \sqrt{\frac{15}{32 \pi}} (x + i y)^2,
\end{aligned}
\end{equation}
%
\begin{equation}\label{eqn:momentCoeffiecients:420}
\begin{aligned}
r^2 Y_{2, 1}
&= r^2 \sqrt{\frac{15}{8 \pi}} \sin\theta \cos\theta e^{i\phi}
\\ &= \sqrt{\frac{15}{8 \pi}} z ( x + i y ),
\end{aligned}
\end{equation}
%
\begin{equation}\label{eqn:momentCoeffiecients:440}
\begin{aligned}
r^2 Y_{2, 0}
&= r^2 \sqrt{\frac{5}{16 \pi}} \lr{ 3 \cos^2\theta - 1 }
\\ &= \sqrt{\frac{5}{16 \pi}} \lr{ 3 z^2 - r^2 }.
\end{aligned}
\end{equation}
%
Given primed coordinates and integrating the conjugate of each of these with \( \rho(\Bx') dV' \), we obtain the \( q_{lm} \) moment coefficients.  Those are
%
\begin{equation}\label{eqn:momentCoeffiecients:460}
q_{11}
= -\sqrt{\frac{3}{8 \pi}} \int d^3 x' \rho(\Bx') (x - i y),
\end{equation}
%
\begin{equation}\label{eqn:momentCoeffiecients:480}
q_{1, 0}
= \sqrt{\frac{3}{4 \pi}} \int d^3 x' \rho(\Bx') z',
\end{equation}
%
\begin{equation}\label{eqn:momentCoeffiecients:500}
q_{2, 2}
= - \sqrt{\frac{15}{32 \pi}} \int d^3 x' \rho(\Bx') (x' - i y')^2,
\end{equation}
%
\begin{equation}\label{eqn:momentCoeffiecients:520}
q_{2, 1}
= \sqrt{\frac{15}{8 \pi}} \int d^3 x' \rho(\Bx') z' ( x' - i y' ),
\end{equation}
%
\begin{equation}\label{eqn:momentCoeffiecients:540}
q_{2, 0}
= \sqrt{\frac{5}{16 \pi}} \int d^3 x' \rho(\Bx') \lr{ 3 (z')^2 - (r')^2 }.
\end{equation}
%
For the potential we are interested in
%
\begin{equation}\label{eqn:momentCoeffiecients:560}
\begin{aligned}
2 \Real q_{11} r^2 Y_{11}(\theta, \phi)
&= 2 \frac{3}{8 \pi} \int d^3 x' \rho(\Bx') \Real \lr{ (x' - i y')( x + i y) }
\\ &= \frac{3}{4 \pi} \int d^3 x' \rho(\Bx') \lr{ x x' + y y' },
\end{aligned}
\end{equation}
%
\begin{equation}\label{eqn:momentCoeffiecients:580}
q_{1, 0} r Y_{1,0}(\theta, \phi)
= \frac{3}{4 \pi} \int d^3 x' \rho(\Bx') z' z,
\end{equation}
%
\begin{equation}\label{eqn:momentCoeffiecients:600}
\begin{aligned}
2 &\Real q_{22} r^2 Y_{22}(\theta, \phi) \\
&= 2 \frac{15}{32 \pi} \int d^3 x' \rho(\Bx') \Real \lr{
(x' - i y')^2
(x + i y)^2
}
\\ &= \frac{15}{16 \pi} \int d^3 x' \rho(\Bx') \Real \lr{
((x')^2 - 2 i x' y' -(y')^2)
(x^2 + 2 i x y -y^2)
}
\\ &= \frac{15}{16 \pi} \int d^3 x' \rho(\Bx') \lr{
((x')^2 -(y')^2) (x^2 -y^2)
+ 4 x x' y y'
},
\end{aligned}
\end{equation}
%
\begin{equation}\label{eqn:momentCoeffiecients:620}
\begin{aligned}
2 \Real q_{21} r^2 Y_{21}(\theta, \phi)
&= 2 \frac{15}{8 \pi} \int d^3 x' \rho(\Bx') z \Real \lr{ ( x' - i y' ) (x + i y) }
\\ &= \frac{15}{4 \pi} \int d^3 x' \rho(\Bx') z \lr{ x x' + y y' },
\end{aligned}
\end{equation}
%
and
\begin{equation}\label{eqn:momentCoeffiecients:640}
q_{2, 0} r^2 Y_{20}(\theta, \phi)
= \frac{5}{16 \pi} \int d^3 x' \rho(\Bx') \lr{ 3 (z')^2 - (r')^2 } \lr{ 3 z^2 - r^2 }.
\end{equation}
%
The dipole term of the potential is
%
\begin{equation}\label{eqn:momentCoeffiecients:660}
\begin{aligned}
&\inv{ 4 \pi \epsilon_0 } \frac{4 \pi}{3 r^3}
\lr{
\frac{3}{4 \pi} \int d^3 x' \rho(\Bx') \lr{ x x' + y y' }
+
\frac{3}{4 \pi} \int d^3 x' \rho(\Bx') z' z
} \\
&=
\inv{ 4 \pi \epsilon_0 r^3}
\Bx \cdot \int d^3 x' \rho(\Bx') \Bx'
\\ &=
\frac{\Bx \cdot \Bp}{ 4 \pi \epsilon_0 r^3},
\end{aligned}
\end{equation}
%
as obtained directly when a strict dipole approximation was used.
%
Summing all the terms for the quadrupole gives
%
\begin{equation}\label{eqn:momentCoeffiecients:680}
\begin{aligned}
\inv{ 4 \pi \epsilon r^5 } \frac{ 4 \pi }{5}
\biglr{
&\frac{15}{16 \pi} \int d^3 x' \rho(\Bx') \lr{
((x')^2 -(y')^2) (x^2 -y^2)
+ 4 x x' y y'
} \\
&+
\frac{15}{4 \pi} \int d^3 x' \rho(\Bx') z z' \lr{ x x' + y y' } \\
&+
\frac{5}{16 \pi} \int d^3 x' \rho(\Bx') \lr{ 3 (z')^2 - (r')^2 } \lr{ 3 z^2 - r^2 }
} \\
=
\inv{ 4 \pi \epsilon r^5 }
\int d^3 x' \rho(\Bx')
\inv{4}
\biglr{
   &3
   \lr{
   ((x')^2 -(y')^2) (x^2 -y^2)
   + 4 x x' y y'
   } \\
   &+
   12
   z z' \lr{ x x' + y y' } \\
   &+
   \lr{ 3 (z')^2 - (r')^2 } \lr{ 3 z^2 - r^2 }
}.
\end{aligned}
\end{equation}
%
The portion in brackets is
%
\begin{equation}\label{eqn:momentCoeffiecients:700}
\begin{aligned}
   3
   &\lr{
      ((x')^2 -(y')^2) (x^2 -y^2)
      + 4 x x' y y'
   } \\
   +
   12
   & z z' \lr{ x x' + y y' }  \\
   +
   &\lr{ 2 (z')^2 - (x')^2 - (y')^2} \lr{ 2 z^2 - x^2 -y^2 } \\
=
x^2 &\lr{
     3 (x')^2 - 3(y')^2
-
   \lr{ 2 (z')^2 - (x')^2 - (y')^2}
} \\
+
y^2 &\lr{
      -3 (x')^2 + 3 (y')^2
-
   \lr{ 2 (z')^2 - (x')^2 - (y')^2}
} \\
+
2 z^2 &\lr{
   2 (z')^2 - (x')^2 - (y')^2
} \\
+
&12{ x x' y y' + x x' z z' + y y' z z' } \\
=
2 x^2 &\lr{
     2 (x')^2 - (y')^2 - (z')^2
} \\
+
2 y^2 &\lr{
     2 (y')^2 - (x')^2 - (z')^2
} \\
+
2 z^2 &\lr{
   2 (z')^2 - (x')^2 - (y')^2
} \\
+
&12{ x x' y y' + x x' z z' + y y' z z' }.
\end{aligned}
\end{equation}
%
The quadrupole sum can now be written as
\begin{equation}\label{eqn:momentCoeffiecients:720}
\begin{aligned}
&\inv{2}
\inv{ 4 \pi \epsilon r^5 }
\int d^3 x' \rho(\Bx') \\
&\quad\biglr{
x^2 \lr{ 3 (x')^2 - (r')^2 }
+y^2 \lr{ 3 (y')^2 - (r')^2 }
+z^2 \lr{ 3 (z')^2 - (r')^2 }
+
3 \lr{
x y x' y'
+y x y' x'
+x z x' z'
+z x z' x'
+y z y' z'
+z y z' y'
}
},
\end{aligned}
\end{equation}
%
which is precisely \cref{eqn:momentCoeffiecients:120}, the quadrupole potential stated in the text and class notes.
%
%}
%\EndArticle
