%
% Copyright � 2016 Peeter Joot.  All Rights Reserved.
% Licenced as described in the file LICENSE under the root directory of this GIT repository.
%
\makeproblem{Tangential magnetic field boundary conditions.}{emt:problemSet3:1}{
%
\index{boundary conditions!tangential magnetic field}
In the class notes we showed that when there were no sources at the interface between two
media and neither of the two media was a perfect conductor \( \sigma_1, \sigma_2 \ne \infty \) the boundary condition
on the tangential magnetic field was given by
%
\begin{equation}\label{eqn:emtProblemSet3Problem1:20}
\ncap \cross \lr{ \BH_2 - \BH_1 } = 0.
\end{equation}
%
Here, show that when \( \BJ_i + \BJ_c = \BJ_{ic} \ne 0 \), the boundary condition is given by
%
\begin{equation}\label{eqn:emtProblemSet3Problem1:40}
\ncap \cross \lr{ \BH_2 - \BH_1 } = \BJ_s,
\end{equation}
%
where
\begin{equation}\label{eqn:emtProblemSet3Problem1:60}
\BJ_s = \lim_{\Delta y \rightarrow 0} \BJ_{ic} \Delta y.
\end{equation}
%
Note: Use the geometry provided in
\cref{fig:boundaryPs3:boundaryPs3Fig1}
for your proof.
\imageFigure{../figures/ece1228-electromagnetic-theory/boundaryPs3Fig1}{Boundary geometry.}{fig:boundaryPs3:boundaryPs3Fig1}{0.3}
} % makeproblem
%
\skipIfRedacted{
\makeanswer{emt:problemSet3:1}{
%
Instead of integrating over a loop as done in class, a better way to tackle this problem is to integrate the curl over the same sort of pillbox that we use for deriving the boundary conditions from the divergence Maxwell's equations.
%
\index{Stokes' theorem}
The form of Stokes' theorem that we want, following the notation of \citep{aMacdonaldVAGC}, is
\begin{equation}\label{eqn:emtProblemSet3Problem1:80}
\int_V d^3 \Bx \cdot \lr{ \boldpartial \wedge \BA } = \oint_{\partial V} d^2 \Bx \cdot \BA.
\end{equation}
The \R{3} translation of this relation into traditional vector algebra, after applying some duality relations, is
\begin{equation}\label{eqn:emtProblemSet3Problem1:100}
\int_V dV \spacegrad \cross \BA = \oint_{\partial V} dA \ncap \cross \BA,
\end{equation}
where \( \ncap \) is the outwards normal.  Proving the general multivector Stokes relationship is beyond the scope of this problem, but we can validate
\cref{eqn:emtProblemSet3Problem1:100} by integrating the LHS over the infinitesimal rectangular prism sketched in \cref{fig:ps3Problem1ElementalVolume:ps3Problem1ElementalVolumeFig1}.
\imageFigure{../figures/ece1228-electromagnetic-theory/ps3Problem1ElementalVolumeFig1}{Elemental volume.}{fig:ps3Problem1ElementalVolume:ps3Problem1ElementalVolumeFig1}{0.2}
\begin{equation}\label{eqn:emtProblemSet3Problem1:120}
\begin{aligned}
\oint_{\partial V} dA \ncap \cross \BA
&=
\oint_{\partial V} dx dy \Be_3 \cross \lr{ \BA(z_0 + \Delta z) - \BA(z_0) } \\
&+\oint_{\partial V} dy dz \Be_1 \cross \lr{ \BA(x_0 + \Delta x) - \BA(x_0) } \\
&+\oint_{\partial V} dz dx \Be_2 \cross \lr{ \BA(y_0 + \Delta y) - \BA(y_0) } \\
&=
\int_{V} dx dy \Be_3 \cross \lr{ dz \PD{z}{\BA} }
+\int_{V} dy dz \Be_1 \cross \lr{ dx \PD{x}{\BA} } \\
&\quad +\int_{V} dz dx \Be_2 \cross \lr{ dy \PD{y}{\BA} } \\
&=
\int_{V} dx dy dz \spacegrad \cross \BA.
\end{aligned}
\end{equation}
Now, let's apply this to Ampere-Maxwell equation
\begin{equation}\label{eqn:emtProblemSet3Problem1:140}
\spacegrad \cross \BH = \BJ_{ic} + \PD{t}{\BD},
\end{equation}
where \( \BJ_{ic} = \BJ_s \delta(y) \).  We have
\begin{equation}\label{eqn:emtProblemSet3Problem1:160}
\oint dA \ncap \cross \BH = \int dV \lr{ \BJ_s \delta(y) + \PD{t}{\BD} }.
\end{equation}
This integral will be evaluated using the pillbox configuration of \cref{fig:ps3Problem1Pillbox:ps3Problem1PillboxFig1}.
\imageFigure{../figures/ece1228-electromagnetic-theory/ps3Problem1PillboxFig1}{Pillbox integration volume.}{fig:ps3Problem1Pillbox:ps3Problem1PillboxFig1}{0.2}
%
The delta function picks up only the contribution of \( \int dA \BJ_s(y=0) \), but \( \BJ_s \) only has a value on that surface anyways.  Taking the pillbox volume to zero in the \( \Delta y \rightarrow 0 \) limit, the LHS integral has only contributions from the top and bottom faces of the pillbox, and the \( \BD \) term, which is assumed finite, will get killed.  That leaves
\begin{equation}\label{eqn:emtProblemSet3Problem1:180}
\int dA \ncap \cross \lr{ \BH_2 - \BH_1 } = \int dA \BJ_s.
\end{equation}
Both sets of integrands can now be brought under one integral
\begin{equation}\label{eqn:emtProblemSet3Problem1:200}
\int dA \lr{ \ncap \cross \lr{ \BH_2 - \BH_1 } - \BJ_s } = 0.
\end{equation}
This is valid for any pillbox surface, so the integrand must be zero,
which proves the desired boundary relation
%\begin{equation}\label{eqn:emtProblemSet3Problem1:220}
\boxedEquation{eqn:emtProblemSet3Problem1:240}{
\ncap \cross \lr{ \BH_2 - \BH_1 } - \BJ_s = 0.
}
%\end{equation}
%
Except for having arbitrarily picked the y-axis as the normal direction in the delta function representation of \( \BJ_{ic} \),
this derivation has the advantage of being coordinate free.  This is in contrast to
the procedure of \citep{balanis1989advanced} followed in class where multiple loop orientations across the boundary are required to prove the general result.
}}
