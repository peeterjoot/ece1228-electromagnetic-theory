%
% Copyright � 2016 Peeter Joot.  All Rights Reserved.
% Licenced as described in the file LICENSE under the root directory of this GIT repository.
%
\makeproblem{Electric field across dielectric boundary.}{emt:problemSet3:3}{
\index{boundary conditions!electric field}
\index{dielectric!boundary conditions}
The plane \( 3x + 2y + z = 12 \) [\si{m}] describes the interface between a dielectric and free space.
The origin side of the interface has \( \epsilon_{r 1} = 3 \) and \( \BE_1 = 2 \xcap + 5 \zcap \) [\si{V/m}]. What is \(\BE_2\)
(the field on the other side of the interface)?
} % makeproblem
%
\skipIfRedacted{
\makeanswer{emt:problemSet3:3}{
The geometry of the problem is sketched roughly in \cref{fig:ps3Problem3Plane:ps3Problem3PlaneFig1}.
\imageFigure{../figures/ece1228-electromagnetic-theory/ps3Problem3PlaneFig1}{Interfaces on sides of a plane.}{fig:ps3Problem3Plane:ps3Problem3PlaneFig1}{0.3}
Assuming that there are no sources, the relationships between the fields on each side of the interface are
\begin{equation}\label{eqn:emtProblemSet3Problem3:20}
\begin{aligned}
%\ncap \cdot \lr{ \epsilon_0 \epsilon_{r2} \BE_2 - \epsilon_0 \epsilon_{r1} \BE_1 } &= 0
\ncap \cdot \lr{ \BD_2 - \BD_1 } &= 0,\\
\ncap \cross \lr{ \BE_2 - \BE_1 } &= 0.
\end{aligned}
\end{equation}
%
%Since \( \epsilon_{r2} = 1 \),
After cancelling common factors of \( \epsilon_0 \) the first relationship can be written
\begin{equation}\label{eqn:emtProblemSet3Problem3:40}
\ncap \cdot \BE_2 = \frac{\epsilon_{r1}}{\epsilon_{r2}} \ncap \cdot \BE_1.
\end{equation}
Adding the normal and the tangential components of \( \BE_2 \), we have
\begin{equation}\label{eqn:emtProblemSet3Problem3:60}
\begin{aligned}
\BE_2
&=
\ncap \lr{ \ncap \cdot \BE_2 } -
\ncap \cross \lr{ \ncap \cross \BE_2 }
\\ &=
\frac{\epsilon_{r1}}{\epsilon_{r2}} \ncap \lr{ \ncap \cdot \BE_1 }
- \ncap \cross \lr{ \ncap \cross \BE_1 }.
\end{aligned}
\end{equation}
By expanding the tangential projection (normal rejection) of a vector as
\begin{equation}\label{eqn:emtProblemSet3Problem3:160}
\begin{aligned}
\BA_t
&=
- \ncap \cross \lr{ \ncap \cross \BA }
\\ &=
\BA - \ncap (\ncap \cdot \BA),
\end{aligned}
\end{equation}
we find
\begin{equation}\label{eqn:emtProblemSet3Problem3:180}
\begin{aligned}
\BE_2
&=
\frac{\epsilon_{r1}}{\epsilon_{r2}} \ncap \lr{ \ncap \cdot \BE_1 }
+ \lr{ \BE_1 - \ncap (\ncap \cdot \BE_1) }
\\ &=
\BE_1 + \lr{\frac{\epsilon_{r1}}{\epsilon_{r2}} -1} \ncap \lr{ \ncap \cdot \BE_1 },
\end{aligned}
\end{equation}
or
\boxedEquation{eqn:emtProblemSet3Problem3:320}{
\BE_2
=
\BE_1 + \frac{\frac{\epsilon_{r1}}{\epsilon_{r2}} -1}{\Abs{\Bn}^2} \Bn \lr{ \Bn \cdot \BE_1 }.
}
The rest of the problem is routine algebra.
\begin{subequations}
\label{eqn:emtProblemSet3Problem3:220}
\begin{equation}\label{eqn:emtProblemSet3Problem3:240}
\begin{aligned}
\Bn^2 &= (3,2,1) \cdot (3,2,1) \\ &= 9 + 4 + 1 \\ &= 14,
\end{aligned}
\end{equation}
\begin{equation}\label{eqn:emtProblemSet3Problem3:260}
\begin{aligned}
\Bn \cdot \BE_1
&=
(3,2,1) \cdot (2,0,5)
\\ &=
6+5
\\ &=11,
\end{aligned}
\end{equation}
\end{subequations}
so
\begin{equation}\label{eqn:emtProblemSet3Problem3:280}
\begin{aligned}
\BE_2
&=
(2,0,5) + \frac{2 \times 11}{14} (3,2,1)
\\ &=
\inv{7}( (14,0,35) + (33,22,11) ),
\end{aligned}
\end{equation}
which is
\boxedEquation{eqn:emtProblemSet3Problem3:300}{
\BE_2
=
\frac{47}{7} \xcap + \frac{22}{7} \ycap + \frac{46}{7} \zcap
\qquad[\si{V/m}].
}
}}
