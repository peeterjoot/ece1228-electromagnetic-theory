%
% Copyright � 2016 Peeter Joot.  All Rights Reserved.
% Licenced as described in the file LICENSE under the root directory of this GIT repository.
%
%{
%\input{../blogpost.tex}
%\renewcommand{\basename}{fresnelSumAndDifferenceAngleFormulas}
%%\renewcommand{\dirname}{notes/phy1520/}
%\renewcommand{\dirname}{notes/ece1228-electromagnetic-theory/}
%%\newcommand{\dateintitle}{}
%%\newcommand{\keywords}{}
%
%\input{../latex/peeter_prologue_print2.tex}
%
%\usepackage{peeters_layout_exercise}
%\usepackage{peeters_braket}
%\usepackage{peeters_figures}
%\usepackage{siunitx}
%%\usepackage{mhchem} % \ce{}
%%\usepackage{macros_bm} % \bcM
%%\usepackage{txfonts} % \ointclockwise
%
%\beginArtNoToc
%
%\generatetitle{Fresnel angular sum and difference formulas}
%\chapter{Fresnel angular sum and difference formulas}
%\label{chap:fresnelSumAndDifferenceAngleFormulas}
%
\makeoproblem{Fresnel sum and difference formulas.}{problem:fresnelSumAndDifferenceAngleFormulas:1}{\citep{hecht1998hecht} pr. 4.39}{
%
Given a \( \mu_1 = \mu_2 \) constraint, show that the Fresnel equations have the form
%
\begin{subequations}
\label{eqn:fresnelSumAndDifferenceAngleFormulas:260}
\begin{equation}\label{eqn:fresnelSumAndDifferenceAngleFormulas:280}
r^{\textrm{TE}}
=
\frac {
\sin( \theta_t - \theta_i )
} {
\sin( \theta_t + \theta_i )
}
\end{equation}
\begin{equation}\label{eqn:fresnelSumAndDifferenceAngleFormulas:300}
r^{\textrm{TM}}
=
\frac
{\tan(\theta_i -\theta_t)}
{\tan(\theta_i +\theta_t)}
\end{equation}
\begin{equation}\label{eqn:fresnelSumAndDifferenceAngleFormulas:320}
t^{\textrm{TE}}
= \frac{ 2  \sin\theta_t \cos\theta_i }
{ \sin(\theta_i + \theta_t) }
\end{equation}
\begin{equation}\label{eqn:fresnelSumAndDifferenceAngleFormulas:340}
t^{\textrm{TM}}
=
{ \sin(\theta_i + \theta_t) \cos(\theta_i - \theta_t) }.
\end{equation}
\end{subequations}
} % problem
%
\makeanswer{problem:fresnelSumAndDifferenceAngleFormulas:1}{
%\withproblemsetsParagraph{
%
We need a couple trig identities to start with.
%
\begin{equation}\label{eqn:fresnelSumAndDifferenceAngleFormulas:20}
\begin{aligned}
\sin(a + b)
&=
\Imag\lr{ e^{j(a + b)} }
\\ &=
\Imag\lr{
e^{ja} e^{+ jb}
}
\\ &=
\Imag\lr{
(\cos a + j \sin a) (\cos b + j \sin b)
}
\\ &=
\sin a \cos b + \cos a \sin b.
\end{aligned}
\end{equation}
%
Allowing for both signs we have
%
\begin{equation}\label{eqn:fresnelSumAndDifferenceAngleFormulas:240}
\begin{aligned}
\sin(a + b) &= \sin a \cos b + \cos a \sin b \\
\sin(a - b) &= \sin a \cos b - \cos a \sin b.
\end{aligned}
\end{equation}
%
The mixed sine and cosine product can be expressed as a sum of sines
%
\begin{equation}\label{eqn:fresnelSumAndDifferenceAngleFormulas:40}
2 \sin a \cos b = \sin(a + b) + \sin(a - b).
\end{equation}
%
With \( 2 x = a + b, 2 y = a - b \), or \( a = x + y, b = x - y \), we find
%
\begin{equation}\label{eqn:fresnelSumAndDifferenceAngleFormulas:60}
\begin{aligned}
2 \sin(x + y) \cos (x - y) &= \sin( 2 x ) + \sin( 2 y ) \\
2 \sin(x - y) \cos (x + y) &= \sin( 2 x ) - \sin( 2 y ).
\end{aligned}
\end{equation}
%
Returning to the problem.  When \( \mu_1 = \mu_2 \) the Fresnel equations were found to be
%
\begin{equation}\label{eqn:fresnelSumAndDifferenceAngleFormulas:100}
\begin{aligned}
r^{\textrm{TE}} &= \frac { n_1 \cos\theta_i - n_2 \cos\theta_t } { n_1 \cos\theta_i + n_2 \cos\theta_t } \\
r^{\textrm{TM}} &= \frac{n_2 \cos\theta_i - n_1 \cos\theta_t }{ n_2 \cos\theta_i + n_1 \cos\theta_t } \\
t^{\textrm{TE}} &= \frac{ 2 n_1 \cos\theta_i } { n_1 \cos\theta_i + n_2 \cos\theta_t } \\
t^{\textrm{TM}} &= \frac{2 n_1 \cos\theta_i }{ n_2 \cos\theta_i + n_1 \cos\theta_t }.
\end{aligned}
\end{equation}
%
Using Snell's law, one of \( n_1, n_2 \) can be eliminated, for example
%
\begin{equation}\label{eqn:fresnelSumAndDifferenceAngleFormulas:120}
n_1 = n_2 \frac{\sin \theta_t}{\sin\theta_i}.
\end{equation}
%
Inserting this and proceeding with the application of the trig identities above, we have
%
\begin{subequations}
\label{eqn:fresnelSumAndDifferenceAngleFormulas:140}
\begin{equation}\label{eqn:fresnelSumAndDifferenceAngleFormulas:160}
\begin{aligned}
r^{\textrm{TE}}
&= \frac { n_2 \frac{\sin\theta_t}{\sin\theta_i} \cos\theta_i - n_2 \cos\theta_t } { n_2 \frac{\sin\theta_t}{\sin\theta_i} \cos\theta_i + n_2 \cos\theta_t }
\\ &=
\frac {
\sin\theta_t \cos\theta_i - \cos\theta_t \sin\theta_i
} {
\sin\theta_t \cos\theta_i + \cos\theta_t \sin\theta_i
}
\\ &=
\frac {
\sin( \theta_t - \theta_i )
} {
\sin( \theta_t + \theta_i )
}
\end{aligned}
\end{equation}
\begin{equation}\label{eqn:fresnelSumAndDifferenceAngleFormulas:180}
\begin{aligned}
r^{\textrm{TM}}
&= \frac{n_2 \cos\theta_i - n_2 \frac{\sin\theta_t}{\sin\theta_i} \cos\theta_t }{ n_2 \cos\theta_i + n_2 \frac{\sin\theta_t}{\sin\theta_i} \cos\theta_t }
\\ &= \frac{
\sin\theta_i \cos\theta_i - \sin\theta_t \cos\theta_t
}{
\sin\theta_i \cos\theta_i + \sin\theta_t \cos\theta_t
}
\\ &= \frac{\inv{2} \sin(2 \theta_i) -  \inv{2} \sin(2 \theta_t) }{ \inv{2} \sin(2 \theta_i) +  \inv{2} \sin(2 \theta_t) }
\\ &= \frac
{\sin(\theta_i - \theta_t)\cos(\theta_i + \theta_t) }
{\sin(\theta_i + \theta_t)\cos(\theta_i - \theta_t) }
\\ &=
\frac
{\tan(\theta_i -\theta_t)}
{\tan(\theta_i +\theta_t)}
\end{aligned}
\end{equation}
\begin{equation}\label{eqn:fresnelSumAndDifferenceAngleFormulas:200}
\begin{aligned}
t^{\textrm{TE}}
&= \frac{ 2 n_2 \frac{\sin\theta_t}{\sin\theta_i} \cos\theta_i } { n_2 \frac{\sin\theta_t}{\sin\theta_i} \cos\theta_i + n_2 \cos\theta_t }
\\ &= \frac{ 2  \sin\theta_t \cos\theta_i } { \sin\theta_t \cos\theta_i + \cos\theta_t \sin\theta_i }
\\ &= \frac{ 2  \sin\theta_t \cos\theta_i }
{ \sin(\theta_i + \theta_t) }
\end{aligned}
\end{equation}
\begin{equation}\label{eqn:fresnelSumAndDifferenceAngleFormulas:220}
\begin{aligned}
t^{\textrm{TM}}
&= \frac{2 n_2 \frac{\sin\theta_t}{\sin\theta_i} \cos\theta_i }{ n_2 \cos\theta_i + n_2 \frac{\sin\theta_t}{\sin\theta_i} \cos\theta_t }
\\ &= \frac{2  \sin\theta_t \cos\theta_i }{ \sin\theta_i \cos\theta_i +  \sin\theta_t \cos\theta_t }
\\ &= \frac{2  \sin\theta_t \cos\theta_i }
{ \inv{2} \sin(2 \theta_i) +  \inv{2} \sin(2 \theta_t) }
\\ &= \frac{2 \sin\theta_t \cos\theta_i }
{ \sin(\theta_i + \theta_t) \cos(\theta_i - \theta_t) }.
\end{aligned}
\end{equation}
\end{subequations}
%}
} % answer
%
%}
%\EndArticle
