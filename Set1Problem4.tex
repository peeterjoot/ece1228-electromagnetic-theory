%
% Copyright � 2016 Peeter Joot.  All Rights Reserved.
% Licenced as described in the file LICENSE under the root directory of this GIT repository.
%
\makeproblem{Conducting sheet with hole.}{emt:problemSet1:4}{
\index{conducting sheet}
%Figure
\Cref{fig:emtLect2:emtLect2Fig4}.
shows a flat, positive, non-conducting sheet of charge with uniform charge density \( \sigma \) [\si{C/m^2}]. A small circular hole of radius \(R \) is cut in the middle of the surface as shown.
%
\imageFigure{../figures/ece1228-electromagnetic-theory/emtLect2Fig4}{Conducting sheet with a hole.}{fig:emtLect2:emtLect2Fig4}{0.3}
%
Calculate the electric field intensity \(\BE\) at point \(P\), a distance \(z\) from the centre of the hole along its axis.
%
Hint 1: Ignore the field fringe effects around all edges.
Hint 2: Calculate the field due to a disk of radius \(R\) and use superposition.
%
} % makeproblem
%
\makeanswer{emt:problemSet1:4}{\withproblemsetsParagraph{
%
For a charged circular disk of radius \( R \) the electric field at the position z above the sheet is
%
\begin{dmath}\label{eqn:emtProblemSet1Problem4:20}
\BE(z)
= \frac{\sigma}{4 \pi \epsilon_0} \iint dA' \frac{z \Be_3 - \Bx'}{\Abs{z^2 + {\Bx'}^2}^{3/2}}
= \frac{\sigma}{4 \pi \epsilon_0} \int_{r=0}^R \int_{\theta = 0}^{2\pi} r dr d\theta \frac{z \Be_3 - r \Be_1 \cos\theta - r \Be_2 \sin\theta }{\Abs{z^2 + r^2}^{3/2}}
= \frac{2 \pi \sigma z^2}{4 \pi \epsilon_0} \int_{r=0}^R (r/z) (dr/z) \frac{z \Be_3}{\Abs{z}^3 \Abs{1 + (r/z)^2}^{3/2}}
= \frac{\sigma \sgn(z) \Be_3 }{2 \epsilon_0} \int_{u=0}^{R/z} \frac{u du}{\Abs{1 + u^2}^{3/2}}
= \frac{\sigma \sgn(z) \Be_3 }{2 \epsilon_0} \evalrange{\lr{-\inv{\sqrt{u^2 + 1}}}}{u=0}{R/z}
= \frac{\sigma \sgn(z) \Be_3 }{2 \epsilon_0} \lr{ 1 -\inv{\sqrt{(R/z)^2 + 1}} }.
\end{dmath}
%
In a gross sense, this result can be used for both the large sheet and the hole.  If the large sheet is modelled as a circular disk with radius \( R \gg z \), one way of completely neglecting any edge effects, then the field above the centre is from this portion of the sheet is approximately
%
\begin{dmath}\label{eqn:emtProblemSet1Problem4:40}
\BE_1(z) = \frac{\sigma \sgn(z) \Be_3 }{2 \epsilon_0}.
\end{dmath}
%
Subtracting the field for the disk of radius \( R \) above, we have
\begin{dmath}\label{eqn:emtProblemSet1Problem4:60}
\BE(z)
= \frac{\sigma \sgn(z) \Be_3 }{2 \epsilon_0} \lr{ 1 - \lr{ 1 -\inv{\sqrt{(R/z)^2 + 1}} } }
= \frac{\sigma \sgn(z) \Be_3 }{2 \epsilon_0} \inv{\sqrt{(R/z)^2 + 1}}
= \frac{\sigma \sgn(z) \Be_3 }{2 \epsilon_0} \frac{\Abs{z}}{R} \inv{\sqrt{1 + (z/R)^2}},
\end{dmath}
%
or
\begin{dmath}\label{eqn:emtProblemSet1Problem4:80}
\BE(z) = \frac{\sigma z \Be_3 }{2 \epsilon_0 R \sqrt{1 + (z/R)^2}}.
\end{dmath}
}}
