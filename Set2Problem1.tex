%
% Copyright � 2016 Peeter Joot.  All Rights Reserved.
% Licenced as described in the file LICENSE under the root directory of this GIT repository.
%
\makeproblem{Electric Dipole.}{emt:problemSet2:1}{
\index{dipole!electric}
%
An electric dipole is shown in \cref{fig:problemset2:problemset2Fig1}.
%
\imageFigure{../figures/ece1228-electromagnetic-theory/problemset2Fig1}{Electric dipole configuration.}{fig:problemset2:problemset2Fig1}{0.3}
%
\makesubproblem{}{emt:problemSet2:1b}
\index{potential!electric dipole}
Find the Potential \( V \) at an arbitrary point \( \BA \).
%
\makesubproblem{}{emt:problemSet2:1a}
Calculate the field \( \BE \) from the above potential.
%
(show that it is the same result we obtained in the class).
} % makeproblem
%
\skipIfRedacted{
\makeanswer{emt:problemSet2:1}{
%
\makeSubAnswer{}{emt:problemSet2:1a}
%
Following \cref{fig:dipoleSignConventionL3:dipoleSignConventionL3Fig3}, the vector from the origin to the observation point is
%
%\imageFigure{../figures/ece1228-electromagnetic-theory/dipoleSignConventionL3Fig3}{Dipole sign convention.}{fig:dipoleSignConventionL3:dipoleSignConventionL3Fig3}{0.3}
%
\begin{equation}\label{eqn:emtProblemSet2Problem1:20}
\Br = \BR_1 + \Bd/2
= \BR_2 - \Bd/2,
\end{equation}
%
or
%
\begin{equation}\label{eqn:emtProblemSet2Problem1:40}
\begin{aligned}
\BR_1 &= \Br - \Bd/2 \equiv \BR_{+} \\
\BR_2 &= \Br + \Bd/2 \equiv \BR_{-}.
\end{aligned}
\end{equation}
%
The potential for this superposition is
\begin{dmath}\label{eqn:emtProblemSet2Problem1:60}
V
=
\inv{4 \pi \epsilon_0} \lr{
\frac{q}{\Abs{\BR_{+}}} -
\frac{q}{\Abs{\BR_{-}}}
}
=
\frac{q}{4 \pi \epsilon_0} \lr{
\frac{1}{\Abs{\BR_{+}}} -
\frac{1}{\Abs{\BR_{-}}}
}.
\end{dmath}
%
The magnitudes can be expanded in Taylor series
%
\begin{dmath}\label{eqn:emtProblemSet2Problem1:80}
\Abs{\BR_{\pm}}^{-1}
=
\lr{
\lr{ \Br \mp \Bd/2 } \cdot \lr{ \Br \mp \Bd/2 }
}^{-1/2}
=
\lr{
\lr{ \Br^2 + (\Bd/2)^2 \mp 2 \Br \cdot \Bd/2 }
}^{-1/2}
=
\lr{
\lr{ \Br^2 + (\Bd/2)^2 \mp \Br \cdot \Bd }
}^{-1/2}
=
(\Br^2)^{-1/2}
\lr{
\lr{ 1 + \lr{\frac{\Bd}{2 r}}^2 \mp \rcap \cdot \frac{\Bd}{r} }
}^{-1/2}
=
r^{-1}
\lr{
1
-\frac{1}{2}
\lr{ \lr{\frac{\Bd}{2 r}}^2 \mp \rcap \cdot \frac{\Bd}{r} }
+\lr{\frac{-1}{2}}
\lr{\frac{-3}{2}} \inv{2!}
\lr{ \lr{\frac{\Bd}{2 r}}^2 \mp \rcap \cdot \frac{\Bd}{r} }^2
+ \cdots
}.
\end{dmath}
%
Here \( r = \Abs{\Br} \), and the Taylor series was taken in the \( \Bd \ll r \) limit.  The sums and differences of these magnitudes, are to first order
%
\begin{dmath}\label{eqn:emtProblemSet2Problem1:100}
\inv{\Abs{\BR_{+}}}
-
\inv{\Abs{\BR_{-}}}
\approx
2 \frac{1}{r} \lr{\frac{-1}{2}} \lr{-\rcap \cdot \frac{\Bd}{r}}
=
\frac{1}{r^2} \rcap \cdot \Bd,
\end{dmath}
%
for
%\begin{dmath}\label{eqn:emtProblemSet2Problem1:120}
\boxedEquation{eqn:emtProblemSet2Problem1:120}{
   V = \frac{\rcap \cdot \Bd}{4 \pi \epsilon_0 r^2 }.
}
%\end{dmath}
%
\makeSubAnswer{}{emt:problemSet2:1b}
%
The electric field follows from \( \BE = -\spacegrad V \).  First note that
%
\begin{dmath}\label{eqn:emtProblemSet2Problem1:140}
\spacegrad \inv{r^n}
=
\Be_k \partial_k (x_m x_m)^{-n/2}
=
-\frac{n}{2} \Be_k \frac{2 x_m \delta_{k m}}{r^{n+2}}
=
-n \frac{\rcap}{r^{n+1}}.
\end{dmath}
%
Computing the gradient of the dot product, we find
\begin{dmath}\label{eqn:emtProblemSet2Problem1:160}
\spacegrad \frac{\rcap}{r^2} \cdot \Bd
=
\spacegrad \frac{\Br}{r^3} \cdot \Bd
=
\Be_k \partial_k \frac{x_m d_m}{r^3}
=
\Be_k \frac{\delta_{k m} d_m}{r^3}
+ \Br \cdot \Bd \spacegrad \inv{r^3}
=
\frac{\Bd}{r^3}
-3 \Br \cdot \Bd \frac{\rcap}{r^4}
=
\frac{\Bd - 3 (\rcap \cdot \Bd) \rcap}{r^3},
\end{dmath}
%
so
%
\begin{dmath}\label{eqn:emtProblemSet2Problem1:180}
V(\Br)
= \frac{q}{4 \pi \epsilon_0}
\frac{3 (\rcap \cdot \Bd) \rcap -\Bd}{r^3}.
\end{dmath}
%
With \( \Bp = q \Bd \), this is the result found in class
%
%\begin{dmath}\label{eqn:emtProblemSet2Problem1:200}
\boxedEquation{eqn:emtProblemSet2Problem1:200}{
V(\Br)
= \frac{1}{4 \pi \epsilon_0}
\frac{3 (\rcap \cdot \Bp) \rcap -\Bp}{r^3}.
}
%\end{dmath}
}}
