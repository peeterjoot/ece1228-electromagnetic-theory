%
% Copyright � 2016 Peeter Joot.  All Rights Reserved.
% Licenced as described in the file LICENSE under the root directory of this GIT repository.
%
%\input{../blogpost.tex}
%\renewcommand{\basename}{emt7}
%\renewcommand{\dirname}{notes/ece1228/}
%\newcommand{\keywords}{ECE1228H}
%\input{../latex/peeter_prologue_print2.tex}
%
%%\usepackage{ece1228}
%\usepackage{peeters_braket}
%%\usepackage{peeters_layout_exercise}
%\usepackage{peeters_figures}
%\usepackage{mathtools}
%\usepackage{siunitx}
%\usepackage{macros_bm}
%
%\beginArtNoToc
%\generatetitle{ECE1228H Electromagnetic Theory.  Lecture 8: Wave equation.  Taught by Prof.\ M. Mojahedi}
\mychapter{Wave equation.}
%\label{chap:emt7}
%
%\paragraph{Disclaimer}
%
%Peeter's lecture notes from class.  These may be incoherent and rough.
%
%These are notes for the UofT course ECE1228H, Electromagnetic Theory, taught by Prof. M. Mojahedi, covering \textchapref{{1}} \citep{balanis1989advanced} content.
%
\paragraph{Wave equation.}
Using an expansion of the triple cross product in terms of the Laplacian
\begin{dmath}\label{eqn:emtLecture7:40}
\spacegrad \cross \lr{ \spacegrad \cross \Bf }
=
-\spacegrad \cdot \lr{ \spacegrad \wedge \Bf }
=
-\spacegrad^2 \Bf
+ \spacegrad \lr{ \spacegrad \cdot \Bf },
\end{dmath}
we can evaluate the cross products
\begin{equation}\label{eqn:emtLecture7:60}
\begin{aligned}
\spacegrad \cross \lr{ \spacegrad \cross \bcE } &= \spacegrad \cross \lr{ -\PD{t}{\bcB} - \bcM },\\
\spacegrad \cross \lr{ \spacegrad \cross \bcH } &= \spacegrad \cross \lr{ \PD{t}{\bcD} + \bcJ },
\end{aligned}
\end{equation}
or
\begin{equation}\label{eqn:emtLecture7:80}
\begin{aligned}
-\spacegrad^2 \bcE + \spacegrad \lr{ \spacegrad \cdot \bcE } &= -\mu \PD{t}{} \spacegrad \cross \bcH - \spacegrad \cross \bcM, \\
-\spacegrad^2 \bcH + \spacegrad \lr{ \spacegrad \cdot \bcH } &= \epsilon \PD{t}{} \lr{ \spacegrad \cross \bcE } + \spacegrad \cross \bcJ,
\end{aligned}
\end{equation}
%
or
%
\begin{equation}\label{eqn:emtLecture7:100}
\begin{aligned}
-\spacegrad^2 \bcE + \inv{\epsilon} \spacegrad \rho_{ev} &= -\mu \PD{t}{} \lr{ \PD{t}{\bcD} + \bcJ } - \spacegrad \cross \bcM,\\
-\spacegrad^2 \bcH + \inv{\mu} \spacegrad \rho_{mv} &= \epsilon \PD{t}{} \lr{ -\PD{t}{\bcB} - \bcM } + \spacegrad \cross \bcJ.
\end{aligned}
\end{equation}
%
This decouples the equations for the electric and the magnetic fields
%
\begin{equation}\label{eqn:emtLecture7:120}
\begin{aligned}
\spacegrad^2 \bcE &=
   \mu \epsilon \PDSq{t}{\bcE} +
   \inv{\epsilon} \spacegrad \rho_{ev} +
   \mu \PD{t}{\bcJ } +
   \spacegrad \cross \bcM, \\
\spacegrad^2 \bcH &=
   \epsilon \mu \PDSq{t}{\bcH} +
   \inv{\mu} \spacegrad \rho_{mv} +
   \epsilon \PD{t}{\bcM } -
   \spacegrad \cross \bcJ.
\end{aligned}
\end{equation}
%
Splitting the current between induced and bound (?) currents
%
\begin{equation}\label{eqn:emtLecture7:260}
\bcJ = \bcJ_i + \bcJ_c = \bcJ_i + \sigma \bcE,
\end{equation}
%
these become
%
\begin{equation}\label{eqn:emtLecture7:160}
\begin{aligned}
\spacegrad^2 \bcE &=
   \mu \epsilon \PDSq{t}{\bcE} +
   \inv{\epsilon} \spacegrad \rho_{ev} +
   \mu \sigma \PD{t}{\bcE} +
   \spacegrad \cross \bcM +
   \mu \PD{t}{\bcJ_i}, \\
\spacegrad^2 \bcH &=
   \epsilon \mu \PDSq{t}{\bcH} +
   \inv{\mu} \spacegrad \rho_{mv} +
   \epsilon \PD{t}{\bcM } +
   \sigma \mu \PD{t}{\bcH} +
   \sigma \bcM
-
   \spacegrad \cross \bcJ_i
.
\end{aligned}
\end{equation}
%
\paragraph{Time harmonic form}
%
Assuming time harmonic dependence \( \bcX = \BX e^{j\omega t} \), we find
%
\begin{equation}\label{eqn:emtLecture7:140}
\begin{aligned}
\spacegrad^2 \BE &=
   \lr{ - \omega^2 \mu \epsilon +
   j \omega \mu \sigma } \BE +
   \inv{\epsilon} \spacegrad \rho_{ev} +
   \spacegrad \cross \BM +
   j \omega \mu \BJ_i, \\
\spacegrad^2 \BH &=
   \lr{ -\omega^2 \epsilon \mu +
   j \omega \sigma \mu } \BH +
   \inv{\mu} \spacegrad \rho_{mv} +
   (j \omega \epsilon + \sigma) \BM
-
   \spacegrad \cross \BJ_i.
\end{aligned}
\end{equation}
%
For a lossy medium where \( \epsilon = \epsilon' -j \omega \epsilon'' \), the leading term factor is
%
\begin{dmath}\label{eqn:emtLecture7:180}
- \omega^2 \mu \epsilon + j \omega \mu \sigma
=
- \omega^2 \mu \epsilon' + j \omega \mu \lr{ \sigma + \omega \epsilon'' }.
\end{dmath}
%
With the definition
\begin{equation}\label{eqn:emtLecture7:200}
\gamma^2 = \lr{ \alpha + j \beta }^2 = - \omega^2 \mu \epsilon' + j \omega \mu \lr{ \sigma + \omega \epsilon'' },
\end{equation}
%
the wave equations have the form
%
\begin{equation}\label{eqn:emtLecture7:220}
\begin{aligned}
\spacegrad^2 \BE &=
\gamma^2 \BE +
   \inv{\epsilon} \spacegrad \rho_{ev} +
   \spacegrad \cross \BM +
   j \omega \mu \BJ_i, \\
\spacegrad^2 \BH &=
\gamma^2 \BH +
   \inv{\mu} \spacegrad \rho_{mv} +
   (j \omega \epsilon + \sigma) \BM
-
   \spacegrad \cross \BJ_i.
\end{aligned}
\end{equation}
%
Here
%
\begin{itemize}
\item \( \alpha \) is the attenuation constant [\si{Np/m}],
\item \( \beta \) is the phase velocity [\si{rad/m}],
\item \( \gamma \) is the propagation constant [\si{1/m}].
\end{itemize}

We are usually interested in solutions in regions free of magnetic currents, induced electric currents, and free of any charge densities, in which case the wave equations are just
%
\begin{equation}\label{eqn:emtLecture7:240}
\begin{aligned}
\spacegrad^2 \BE &= \gamma^2 \BE, \\
\spacegrad^2 \BH &= \gamma^2 \BH.
\end{aligned}
\end{equation}
%
%\EndArticle
%\EndNoBibArticle
