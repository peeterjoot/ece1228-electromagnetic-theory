   \section{Tunnelling.}
In class, we walked through splitting up the wave equation into components, and separation of variables.  I didn't take notes on that.

Winding down that discussion, however, was a mention of phase and group velocity, and a phenomena called superluminal velocity.  This latter is analogous to quantum electron tunnelling where a wave can make it through an aperture with a damped solution \( e^{-\alpha x} \) in the aperture interval, and sinusoidal solutions in the incident and transmitted regions as sketched in \cref{fig:L7:L7Fig1}.  The time \( \tau \) to get through the aperture is called the tunnelling time.
\imageFigure{../figures/ece1228-electromagnetic-theory/L7Fig1}{Superluminal tunnelling.}{fig:L7:L7Fig1}{0.3}
