%
% Copyright � 2016 Peeter Joot.  All Rights Reserved.
% Licenced as described in the file LICENSE under the root directory of this GIT repository.
%
\makeproblem{Magnetic field for a current loop.}{emt:problemSet3:2}{
%
\index{current loop}
A loop of wire located in x-y plane carrying current \(I\) is shown in \cref{fig:currentLoopPs3:currentLoopPs3Fig2}.
The loop's radius is \(R_l\).
\imageFigure{../figures/ece1228-electromagnetic-theory/currentLoopPs3Fig2}{Current loop.}{fig:currentLoopPs3:currentLoopPs3Fig2}{0.3}
\makesubproblem{}{emt:problemSet3:2a}
Calculate the magnetic field flux density, \( \BB \), along the loop axis at a distance \( z \) from its centre.
%
\makesubproblem{}{emt:problemSet3:2b}
Simplify the results in
\partref{emt:problemSet3:2a}
for large distances along the z-axis (\( z \gg R_l \)).
%
\makesubproblem{}{emt:problemSet3:2c}
Express the results in
\partref{emt:problemSet3:2b}
in terms of magnetic dipole
moment. Make sure you write the expression in vector
form.
\makesubproblem{}{emt:problemSet3:2d}
In keeping with your understanding of magnetic bar's
north and south poles, designate the north and south poles
for the current carrying loop shown in the figure.
%
\index{Biot-Savart}
\paragraph{Hint:} Use Biot-Savart law which states the following: A
differential current element, \( I d\Bl' \), produces a differential
magnetic field, \( d\BB \),
at a distance \( R \) from the current
element given by
%
\begin{dmath}\label{eqn:emtProblemSet3Problem2:20}
d\BB = \frac{\mu_0}{4 \pi} \frac{I d\Bl' \cross \BR }{R^3},
\end{dmath}
%
or
\begin{dmath}\label{eqn:emtProblemSet3Problem2:40}
\BB = \frac{\mu_0}{4 \pi} \int \frac{I d\Bl' \cross \BR }{R^3},
\end{dmath}
%
Note that integration is carried over the source (current) and \( R \) points from the current elements
to the point of observation.
} % makeproblem
%
\makeanswer{emt:problemSet3:2}{\withproblemsetsParagraph{
\makeSubAnswer{}{emt:problemSet3:2a}
%
The integral for the general observation point is straightforward to write down.
The observation point for this problem is
%
\begin{dmath}\label{eqn:emtProblemSet3Problem2:220}
\Br = z \Be_3.
\end{dmath}
%
The charge point is
\begin{dmath}\label{eqn:emtProblemSet3Problem2:240}
\Br' = R_l \lr{ \Be_1 \cos\theta' + \Be_2 \sin\theta' },
\end{dmath}
%
and the element of the loop path is
\begin{dmath}\label{eqn:emtProblemSet3Problem2:260}
d\Bl' = R_l d\theta' \lr{ \Be_2 \cos\theta' - \Be_1 \sin\theta'}.
\end{dmath}
%
The difference vector from the charge position to the observation point is
%
\begin{dmath}\label{eqn:emtProblemSet3Problem2:300}
\BR
= \Br - \Br'
=
z \Be_3 - R_l \lr{ \Be_1 \cos\theta' + \Be_2 \sin\theta' },
\end{dmath}
%
with squared length
%
\begin{dmath}\label{eqn:emtProblemSet3Problem2:320}
\BR^2
=
z^2 + R_l^2.
\end{dmath}
%
\index{elemental arc}
Finally, the cross product of the elemental arc with the difference vector is
%
\begin{dmath}\label{eqn:emtProblemSet3Problem2:280}
d\Bl' \cross \BR
=
R_l d\theta'
\lr{
\Be_2 \cos\theta' - \Be_1 \sin\theta'
}
\cross
\lr{
z \Be_3 - R_l \lr{ \Be_1 \cos\theta' + \Be_2 \sin\theta' }
}
=
R_l d\theta'
\lr{
z \lr{ \Be_1 \cos\theta' + \Be_2 \sin\theta' }
+
R_l \Be_3 \lr{
\cos^2\theta' + \sin^2 \theta'
}
}
=
R_l d\theta'
\lr{
z %\Be_1 e^{i\theta'}
\lr{ \Be_1 \cos\theta' + \Be_2 \sin\theta' }
+ R_l \Be_3
}.
\end{dmath}
%
The magnetic field integral is
\boxedEquation{eqn:emtProblemSet3Problem2:201}{
%\begin{dmath}\label{eqn:emtProblemSet3Problem2:201}
\BB = \frac{I \mu_0 R_l}{4\pi} \int_0^{2\pi} d\theta'
\frac
{ z
%\Be_1 e^{i\theta'}
\lr{ \Be_1 \cos\theta' + \Be_2 \sin\theta' }
+ \Be_3 R_l }
{ \lr{ z^2 + R_l^2 }^{3/2} }.
}
%\end{dmath}
%
%This is consistent with \cref{eqn:emtProblemSet3Problem2:200} as expected.
This integral is particularly easy to evaluate.  All the trigonometric contributions are killed integrating over the \( [0, 2 \pi] \) interval, leaving just
\begin{equation}\label{eqn:emtProblemSet3Problem2:340}
\BB = \frac{I \mu_0 R_l}{4\pi}
\frac
{ 2 \pi \Be_3 R_l }
{ \lr{ z^2 + R_l^2 }^{3/2} },
\end{equation}
%
or
\boxedEquation{eqn:emtProblemSet3Problem2:360}{
\BB = \frac{\Be_3 I \mu_0 }{2\pi}
\frac
{ \lr{\pi R_l^2} }
{ \lr{ z^2 + R_l^2 }^{3/2} },
}
%
\makeSubAnswer{}{emt:problemSet3:2b}
%
Taylor expanding the denominator gives
%
\begin{dmath}\label{eqn:emtProblemSet3Problem2:380}
\BB
= \frac{\Be_3 I \mu_0 }{2\pi}
\frac
{ \lr{\pi R_l^2} }
{ z^3 }
\lr{
1 - \frac{3}{2} \lr{ \frac{R_l}{z} }^2
+ \frac{(3)(5)}{2^3} \lr{ \frac{R_l}{z} }^4
+ \cdots
}.
\end{dmath}
%
If \( z \gg R_l \) this is dominated by the zero order term of the expansion, leaving
\boxedEquation{eqn:emtProblemSet3Problem2:460}{
\BB
\approx
\Be_3 I \frac{\mu_0 }{2\pi}
\frac
{ \lr{\pi R_l^2} }
{ z^3 }.
}
%
\makeSubAnswer{}{emt:problemSet3:2c}
%
If an element of magnetic moment for the loop is \( d\Bm = \ncap I ds \), the magnetic moment for the whole loop is
%
\begin{dmath}\label{eqn:emtProblemSet3Problem2:420}
\BM
= \Be_3 I (2 \pi R_l)
\end{dmath}
%
so the magnetic field is
%
%\begin{dmath}\label{eqn:emtProblemSet3Problem2:440}
\boxedEquation{eqn:emtProblemSet3Problem2:440}{
\BB
=
\BM \frac{\mu_0}{4\pi}
\frac{ R_l}{z^3}.
}
%\end{dmath}
%
\makeSubAnswer{}{emt:problemSet3:2d}
%
\index{magnetic pole}
I couldn't recall the N vs. S convention for field lines in a magnet, and couldn't find anything in my electromagnetism books on this.
However, the ``Field of a cylindrical bar magnet calculated with Amp\`ere's model'' figure in \citep{wiki:magnet}, indicates that the magnetic field lines emanate from the North pole of a magnet.  This means that the current loop in this problem can be thought of as representing a magnet with north and south poles in the \( z > 0 \) and \( z < 0 \) regions respectively, as sketched in \cref{fig:magnetPoleForCurrentLoop:magnetPoleForCurrentLoopFig1}.
\imageFigure{../figures/ece1228-electromagnetic-theory/magnetPoleForCurrentLoopFig1}{Magnet pole orientations for current loop.}{fig:magnetPoleForCurrentLoop:magnetPoleForCurrentLoopFig1}{0.3}
%
\paragraph{Off axis field.}
%
%
% Copyright © 2016 Peeter Joot.  All Rights Reserved.
% Licenced as described in the file LICENSE under the root directory of this GIT repository.
%
%\section{Appendix I.  Problem 2. Current loop integral off axis.}
%
Initially I was curious what the current loop magnetic field integral would look like in general, allowing for an off axis observation point.
%
I found it natural to do that compuation using Geometric Algebra to express vector rotation in a plane and the other geometrical constructs of this problem.  The basic rules in that Algebra are that unit vectors square to unity (\(\Be_k^2 = 1 \)), and that orthogonal vectors anticommute (\( \Be_1 \Be_2 = -\Be_2 \Be_1 \)).  For example, letting \( i = \Be_1 \Be_2 \) the radial unit vector can be expessed as
%
\begin{dmath}\label{eqn:emtProblemSet3Problem2:160}
\rhocap(\theta)
=
\Be_1 e^{i \theta}
= \Be_1 \lr{ \cos\theta + \Be_1 \Be_2 \sin\theta }
= \Be_1 \cos\theta + (\Be_1^2) \Be_2 \sin\theta
= \Be_1 \cos\theta + \Be_2 \sin\theta,
\end{dmath}
%
and the \( \thetacap \) direction vector is
\begin{dmath}\label{eqn:emtProblemSet3Appendix:240}
\thetacap(\theta)
=
\Be_2 e^{i \theta}
= \Be_2 \lr{ \cos\theta + \Be_1 \Be_2 \sin\theta }
= \Be_2 \cos\theta + \Be_2 \Be_1 \Be_2 \sin\theta
= \Be_2 \cos\theta + \Be_2 (-\Be_2 \Be_1) \sin\theta
= \Be_2 \cos\theta - \Be_1 \sin\theta.
\end{dmath}
%
This allows for a compact expression of an off-axis observation point
%
\begin{dmath}\label{eqn:emtProblemSet3Problem2:60}
\Br = z \Be_3 + \rho \Be_1 e^{i\theta}.
\end{dmath}
%
Similarly, the charge point is
\begin{dmath}\label{eqn:emtProblemSet3Problem2:80}
\Br' = R_l \Be_1 e^{i \theta'},
\end{dmath}
%
and the element of the loop path is
\begin{dmath}\label{eqn:emtProblemSet3Problem2:100}
d\Bl' = R_l \Be_2 e^{i\theta'} d\theta'.
\end{dmath}
%
The difference vector from the charge position to the observation point is
%
\begin{dmath}\label{eqn:emtProblemSet3Problem2:120}
\BR
= \Br - \Br'
=
z \Be_3 + \rho \Be_1 e^{i\theta}
-
R_l \Be_1 e^{i \theta'},
\end{dmath}
%
with squared length
%
\begin{dmath}\label{eqn:emtProblemSet3Problem2:140}
\BR^2
=
z^2 +
\lr{ \rho \Be_1 e^{i\theta}
-
R_l \Be_1 e^{i \theta'}
}
\cdot
\lr{ \rho \Be_1 e^{i\theta}
-
R_l \Be_1 e^{i \theta'}
}
=
z^2 + \rho^2 + R_l^2 - 2 \rho R_l \cos\lr{ \theta - \theta' }.
\end{dmath}
%
For the cross product, using a bivector duality transformation \( \Ba \cross \Bb = -\Be_1 \Be_2 \Be_3 (\Ba \wedge \Bb) \), and expressing the wedge product as a grade two selection, we have
%
\begin{dmath}\label{eqn:emtProblemSet3Problem2:180}
d\Bl' \cross \BR
=
-\Be_1 \Be_2 \Be_3 R_l d\theta' \gpgradetwo{
\Be_2 e^{i \theta'}
\lr{
z \Be_3 + \rho \Be_1 e^{i\theta}
-
R_l \Be_1 e^{i \theta'}
}
}
=
R_l d\theta' \lr{
z \Be_1 e^{i\theta'}
-
\Be_3 \rho \cos\lr{ \theta - \theta' }
+ \Be_3 R_l
}.
\end{dmath}
%
The final integral can now be assembled
%
\boxedEquation{eqn:emtProblemSet3Appendix:220}{
%\begin{dmath}\label{eqn:emtProblemSet3Problem2:200}
\BB = \frac{I \mu_0 R_l}{4\pi} \int_0^{2\pi} d\theta'
\frac
{ z \Be_1 e^{i\theta'} - \Be_3 \rho \cos\lr{ \theta - \theta' } + \Be_3 R_l }
{ \lr{z^2 + \rho^2 + R_l^2 - 2 \rho R_l \cos\lr{ \theta - \theta' }}^{3/2} }.
%\end{dmath}
}
%
This is consistent with the traditional vector algebra derivation that led to \cref{eqn:emtProblemSet3Problem2:201} where \( \rho = 0 \) was assumed.
It is clear now, why the problem statement asked only to consider the z-axis observation points where \( \rho = 0 \).  With \( \theta' \) dependencies in the denominator, performing the integral above for \( \rho \ne 0 \) looks spectacularly unpleasant.
%

}}
