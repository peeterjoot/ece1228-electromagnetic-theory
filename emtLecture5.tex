%
% Copyright � 2016 Peeter Joot.  All Rights Reserved.
% Licenced as described in the file LICENSE under the root directory of this GIT repository.
%
%\input{../blogpost.tex}
%\renewcommand{\basename}{emt5}
%\renewcommand{\dirname}{notes/ece1228/}
%\newcommand{\keywords}{ECE1228H}
%\input{../latex/peeter_prologue_print2.tex}
%
%%\usepackage{ece1228}
%\usepackage{peeters_braket}
%%\usepackage{peeters_layout_exercise}
%\usepackage{peeters_figures}
%\usepackage{macros_cal}
%\usepackage{macros_bm}
%\usepackage{mathtools}
%\usepackage{siunitx}
%
%\beginArtNoToc
%\generatetitle{ECE1228H Electromagnetic Theory.  Lecture 5: Poynting vector.  Taught by Prof.\ M. Mojahedi}
%\mychapter{Poynting vector, and time harmonic (phasor) fields.}
\label{chap:emt5}
%
%\section{Poynting.}
\index{Poynting vector}
\index{Poynting theorem}
%
The cross product terms of Maxwell's equation are
\begin{equation}\label{eqn:emtLecture5:120}
\spacegrad \cross \BE
= -\BM_i - \PD{t}{\BB}
= -\BM_i - \BM_d,
\end{equation}
%
where \(\BM_d\) is called the magnetic displacement current here.  For the magnetic curl we have
%
\begin{equation}\label{eqn:emtLecture5:140}
\spacegrad \cross \BH
= \BJ_i + \BJ_c + \PD{t}{\BD}
= \BJ_i + \BJ_c + \BJ_d.
\end{equation}
%
It is left as an exercise to show that
%
\begin{equation}\label{eqn:emtLecture5:160}
\spacegrad \cdot \lr{ \BE \cross \BH } + \BH \cdot \lr{ \BM_i + \BM_d }  + \BE \cdot \lr{ \BJ_i + \BJ_c + \BJ_d } = 0,
\end{equation}
%
or
\begin{equation}\label{eqn:emtLecture5:180}
\begin{aligned}
\oint &d\Ba \cdot \lr{ \BE \cross \BH } + \int dV \lr{ \BH \cdot \lr{ \BM_i + \BM_d }  + \BE \cdot \lr{ \BJ_i + \BJ_c + \BJ_d }} \\
&= 0,
\end{aligned}
\end{equation}
%
or
\begin{equation}\label{eqn:emtLecture5:200}
\begin{aligned}
   0 &=
   \oint d\Ba \cdot \lr{ \BE \cross \BH } \\
   &+ \int dV \BH \cdot \BM_i
+ \int dV \BE \cdot \BJ_i
+ \int dV \BE \cdot \BJ_c \\
&+ \int dV \lr{ \BH \cdot \PD{t}{\BB} + \BE \cdot \PD{t}{\BD} }.
\end{aligned}
\end{equation}
%
\index{supplied power density}
Define a supplied power density \( \rho_{\textrm{supp}} \)
%
\begin{equation}\label{eqn:emtLecture5:220}
-\rho_{\textrm{supp}}
=
 \int dV \BH \cdot \BM_i
+ \int dV \BE \cdot \BJ_i.
\end{equation}
%
When the medium is not dispersive or lossy, we have
%
\begin{equation}\label{eqn:emtLecture5:240}
\begin{aligned}
\int dV \BH \cdot \PD{t}{\BB}
&=
\mu \int dV \BH \cdot \PD{t}{\BH}
\\ &=
\PD{t}{} \int dV \mu \Abs{\BH}^2.
\end{aligned}
\end{equation}
%
\index{magnetic energy density}
% FIXME: Is this supposed to be Wb (webers)?  chatgpt says \mu_0 H^2 has dimensions of Pa (scals).
The units of \( [\mu \Abs{\BH}^2] \) are \si{W}, so one can defined a magnetic energy density \( \mu \Abs{\BH}^2 \), and
%
\begin{equation}\label{eqn:emtLecture5:260}
W_m =
\int dV \mu \Abs{\BH}^2,
\end{equation}
%
for
%
\begin{equation}\label{eqn:emtLecture5:280}
\int dV \BH \cdot \PD{t}{\BB}
=
\PD{t}{W_m}.
\end{equation}
%
\index{stored magnetic energy}
This is the rate of change of stored magnetic energy [\si{J/s} = \si{W}].
%
Similarly
\begin{equation}\label{eqn:emtLecture5:300}
\begin{aligned}
\int dV \BE \cdot \PD{t}{\BD}
&=
\epsilon
\int dV \BE \cdot \PD{t}{\BE}
\\ &=
\PD{t}{} \int dV \epsilon \Abs{\BE}^2.
\end{aligned}
\end{equation}
%
\index{electric energy density}
The electric energy density is \( \epsilon \Abs{\BE}^2 \).  Let
%
\begin{equation}\label{eqn:emtLecture5:320}
W_e =
\int dV \epsilon \Abs{\BE}^2,
\end{equation}
%
and
\begin{equation}\label{eqn:emtLecture5:340}
\int dV \BE \cdot \PD{t}{\BD}
=
\PD{t}{W_e}.
\end{equation}
%
We also have a term
%
\begin{equation}\label{eqn:emtLecture5:360}
\begin{aligned}
\int dV \BE \cdot \BJ_c
&=
\int dV \BE \cdot (\sigma \BE)
\\ &=
\int dV \sigma \Abs{\BE}^2.
%\equiv ...
\end{aligned}
\end{equation}
%
\index{stored electric energy}
This is the rate of change of stored electric energy.
%
The remaining term is
\begin{equation}\label{eqn:emtLecture5:380}
\oint d\Ba \cdot \lr{ \BE \cross \BH }.
\end{equation}
%
This is a density of the power that is leaving the volume.  The vector \( \BE \cross \BH \) is special, called the Poynting vector, and coincidentally points in the direction that the energy leaves the bounding surface per unit time.  We write
%
\begin{equation}\label{eqn:emtLecture5:400}
\BS = \BE \cross \BH.
\end{equation}
%
In vacuum the phase velocity \( \Bv_p \), group velocity \( \Bv_g \) and packet(?) velocity \( \Bv_p \) all line up.  This isn't the case in the media.
%
It turns out that without dissipation
%
\begin{equation}\label{eqn:emtLecture5:420}
\int \BH \cdot \PD{t}{\BB} = \int \BE \cdot \PD{t}{\BD}.
\end{equation}
%
\index{LC circuit}
For example in an LC circuit \cref{fig:lecture4LCCircuit:lecture4LCCircuitFig1}
half the cycle the energy is stored in the inductor, and in the other half of the cycle the energy is stored in the capacitor.
%
\imageFigure{../figures/ece1228-electromagnetic-theory/lecture4LCCircuitFig1}{LC circuit.}{fig:lecture4LCCircuit:lecture4LCCircuitFig1}{0.2}
%
Summarizing
%
\begin{equation}\label{eqn:emtLecture5:440}
\oint \lr{ \BE \cross \BH } \cdot d\Ba = P_{\textrm{exit}}.
\end{equation}
%
