%
% Copyright � 2016 Peeter Joot.  All Rights Reserved.
% Licenced as described in the file LICENSE under the root directory of this GIT repository.
%
%----------------------------------------------------------------------------------------
%
% Prof. decoder ring:
%
% antigreat      == integrate
% ambeeguis      == ambiguous
% eikenvector    == eigenvector
% esmal          == small
% sine "of a" x  == sin(x)
% beta suba x    == beta_x
% constrain      == constraint
% havea expr     == have expr (example: havea e^x), plug this expression for beta -> plug this expression for abeta.
% isa            == is : f_{c1} isa bigger than f_{c2}
% den            == then
% ispectra       == spectrum
% smood          == smooth
% togeder        == together
%
%----------------------------------------------------------------------------------------
%\part{Lecture notes}
   %\mychapter{Electromagnetic fields}
   %
% Copyright � 2016 Peeter Joot.  All Rights Reserved.
% Licenced as described in the file LICENSE under the root directory of this GIT repository.
%
%\input{../blogpost.tex}
%\renewcommand{\basename}{emt1}
%\renewcommand{\dirname}{notes/ece1228/}
%\newcommand{\keywords}{ECE1228H}
%\input{../latex/peeter_prologue_print2.tex}
%
%%\usepackage{ece1228}
%\usepackage{peeters_braket}
%%\usepackage{peeters_layout_exercise}
%\usepackage{peeters_figures}
%\usepackage{mathtools}
%\usepackage{siunitx}
%
%\beginArtNoToc
%\generatetitle{ECE1228H Electromagnetic Theory.  Lecture 1: Introduction.  Taught by Prof.\ M. Mojahedi}
%\mychapter{Introduction.}
%\label{chap:emt1}
%
%\paragraph{Disclaimer}
%
%Peeter's lecture notes from class.  These may be incoherent and rough.
%
%These are notes for the UofT course ECE1228H, Electromagnetic Theory, taught by Prof. M. Mojahedi, covering \textchapref{{1}} \%citep{balanis1989advanced} content.
%
\section{Conventions for Maxwell's equations.}
\index{Maxwell's equations!time domain}
%
In these course notes, Maxwell's equations will be written in one of two forms.  The first is the standard bold face vectors, where the fields are assumed to be real.
%
\begin{itemize}
\item Faraday's Law
\begin{equation}\label{eqn:emtLecture1:20}
\spacegrad \cross \BE( \Br, t ) = - \PD{t}{\BB}(\Br, t) - \BM_i,
\end{equation}
\item Ampere-Maxwell equation
\begin{equation}\label{eqn:emtLecture1:40}
\spacegrad \cross \BH( \Br, t ) = \BJ_\txtc(\Br, t) + \PD{t}{\BD}(\Br, t),
\end{equation}
\item Gauss's law
\begin{equation}\label{eqn:emtLecture1:80}
\spacegrad \cdot \BD(\Br, t) = \rho_{\txte\txtv}(\Br, t),
\end{equation}
\item Gauss's law for magnetism
\begin{equation}\label{eqn:emtLecture1:100}
\spacegrad \cdot \BB(\Br, t) = \rho_{\txtm\txtv}(\Br, t).
\end{equation}
\end{itemize}
%
In chapters where frequency domain analysis is used, Maxwell's equations will be written in script
%
\begin{equation}\label{eqn:emtLecture1:160}
\begin{aligned}
\spacegrad \cross \bcE &= -\PD{t}{\bcB} - \bcM \\
\spacegrad \cross \bcH &= \PD{t}{\bcD} + \bcJ \\
\spacegrad \cross \bcB &= q_{mv} \\
\spacegrad \cross \bcD &= q_{ev} \\
\end{aligned}
\end{equation}
%
with bold face reserved for complex valued field variables.  In the frequency domain (called time harmonic form in this class), the frequency dependence is of the form
%
\begin{equation}\label{eqn:emtLecture1:140}
\bcX = \Real\lr{ \BX e^{j\omega t} }.
\end{equation}
%
In this form, Maxwell's equations are
%
\begin{equation}\label{eqn:emtLecture1:180}
\begin{aligned}
\spacegrad \cross \BE &= -j \omega \BB - \BM \\
\spacegrad \cross \BH &= j \omega \BD + \BJ \\
\spacegrad \cross \BB &= \rho_{mv} \\
\spacegrad \cross \BD &= \rho_{ev}.
\end{aligned}
\end{equation}
%
Where there is no ambiguity, bold face vectors will be used, even in the time domain.
%
\section{Units.}
%
Regardless of the conventions, after unpacking, we have a total of eight equations, with four vectoral field variables, and 8 sources, all interrelated by partial derivatives in space and time coordinates.
%
It will be left to homework to show that without the displacement current \( \PDi{t}{\BD} \), these equations will not satisfy conservation relations.
%
The fields are and sources are
\index{units}
\begin{itemize}
\item \( \BE \) Electric field intensity \si{V/m},
\item \( \BB \) Magnetic flux density \si{V s/m^2} (or Tesla),
\item \( \BH \) Magnetic field intensity \si{A/m},
\item \( \BD \) Electric flux density \si{C/m^2},
\item \( \rho_{\txte\txtv} \) Electric charge volume density,
\item \( \rho_{\txtm\txtv} \) Magnetic charge volume density,
\item \( \BJ_{\txtc} \) Impressed (source) electric current ,ensity \si{A/m^2}.  This is the charge passing through a plane in a unit time.  Here \( \txtc \) is for ``conduction''.
\item \( \BM_{\txti} \) Impressed (source) magnetic current density \si{V/m^2}.
\end{itemize}
%
In an undergrad context we'll have seen the electric and magnetic fields in the Lorentz force law
%
\begin{equation}\label{eqn:emtLecture1:120}
\BF = q \Bv \cross \BB + q\BE.
\end{equation}
%
In SI there are 7 basic units.  These include
%
\begin{itemize}
\item length \si{m},
\item mass \si{kg},
\item time \si{s},
\item ampere \si{A},
\index{unit!ampere}
\item kelvin \si{K} (temperature),
\index{unit!kelvin}
\item candela (luminous intensity),
\index{unit!candela}
\item mole (amount of substance),
\index{unit!mole}
\end{itemize}
%
\index{unit!coulomb}
Note that the coulomb is not a fundamental unit, but the ampere is.  This is because it is easier to measure.

For homework: show that magnetic field lines must close on themselves when there are no magnetic sources (zero divergence).  This is opposed to electric fields that spread out from the charge.
%
%\EndNoBibArticle

   %
% Copyright � 2016 Peeter Joot.  All Rights Reserved.
% Licenced as described in the file LICENSE under the root directory of this GIT repository.
%
%\input{../blogpost.tex}
%\renewcommand{\basename}{emt2}
%\renewcommand{\dirname}{notes/ece1228/}
%\newcommand{\keywords}{ECE1228H}
%\input{../latex/peeter_prologue_print2.tex}
%
%%\usepackage{ece1228}
%\usepackage{peeters_braket}
%%\usepackage{peeters_layout_exercise}
%\usepackage{peeters_figures}
%\usepackage{mathtools}
%\usepackage{siunitx}
%
%\beginArtNoToc
%\generatetitle{ECE1228H Electromagnetic Theory.  Lecture 2: Boundaries.  Taught by Prof.\ M. Mojahedi}
\chapter{Boundaries}
\label{chap:emt2}

%\paragraph{Disclaimer}
%
%Peeter's lecture notes from class.  These may be incoherent and rough.
%
%These are notes for the UofT course ECE1228H, Electromagnetic Theory, taught by Prof. M. Mojahedi, covering \textchapref{{1}} \citep{balanis1989advanced} content.
%
\paragraph{Integral forms}

Given Maxwell's equations at a point

\begin{dmath}\label{eqn:emtLecture2:20}
\begin{aligned}
\spacegrad \cross \BE &= -\PD{t}{\BB} \\
\spacegrad \cross \BH &= \BJ + \PD{t}{\BD} \\
\spacegrad \cdot \BD &= \rho_\txtv \\
\spacegrad \cdot \BB &= 0
\end{aligned}
\end{dmath}

what happens when we have different fields and currents on two sides of a boundary?  To answer these questions, we want to use the integral forms of Maxwell's equations, over the geometries illustrated in \cref{fig:loopAndPillbox:loopAndPillboxFig1}.
\index{boundary}

\imageFigure{../figures/ece1228-electromagnetic-theory/loopAndPillboxFig1}{Loop and pillbox configurations.}{fig:loopAndPillbox:loopAndPillboxFig1}{0.2}

To do so, we use Stokes' and the divergence theorems relating the area and volume integrals to the surfaces of those geometries.

These are

\index{Stokes' Theorem}
\index{Divergence Theorem}
\begin{dmath}\label{eqn:emtLecture2:40}
\begin{aligned}
\iint_S \lr{ \spacegrad \cross \BA } \cdot d\Bs &= \oint_C \BA \cdot d\Bl \\
\iint_V \lr{ \spacegrad \cdot \BA } d\Bs &= \oint_A \BA \cdot d\Bs \\
\end{aligned}
\end{dmath}

\index{Faraday's law}
Application of the Stokes' to Faraday's law we get

\begin{dmath}\label{eqn:emtLecture2:60}
\oint_C \BE \cdot d\Bl = -\PD{t}{} \iint \BB \cdot d\Bs
\end{dmath}

UNITS: \( V/m \times m \)

The quantity
\begin{dmath}\label{eqn:emtLecture2:80}
\iint \BB \cdot d\Bs,
\end{dmath}

is called the magnetic flux of \( \BB \), and changing of this flux is responsible for the generation of electromotive force.
\index{magnetic flux}

%F2:
Similarly

\begin{dmath}\label{eqn:emtLecture2:100}
\begin{aligned}
\oint \BH \cdot d\Bl &= \iint \BJ \cdot d\Bs + \PD{t}{} \iint \BD \cdot d\Bs \\
\oint \BD \cdot d\Bs &= \iiint \rho_\txtv dV = Q_\txte \\
\oint \BB \cdot d\Bs &= 0.
\end{aligned}
\end{dmath}

\index{constitutive relations}
\paragraph{Constitutive relations}

With 12 unknowns in \( \BE, \BB, \BD, \BH \) and 8 equations in Maxwell's equations (or 6 if the divergence equations are considered redundant), things don't look too good for solutions.  In simple media, in the frequency domain, relations of the form

\begin{dmath}\label{eqn:emtLecture2:120}
\begin{aligned}
\BD( \Br, \omega ) &= \epsilon \BE( \Br, \omega ) \\
\BB( \Br, \omega ) &= \mu \BH( \Br, \omega ).
\end{aligned}
\end{dmath}

\index{permeability}
\index{macroscopic}
The permeabilities \( \epsilon \) and \( \mu \) are macroscopic beasts, determined either experimentally, or theoretically using an averaging process involving many (millions, or billions, or more) particles.  However, the theoretical determinations that have been attempted do not work well in practise and usually end up considerably different than the measured values.  We are referred to \citep{jackson1975cew} for one attempt to model the statistical microscopic effects non-quantum mechanically to justify the traditional macroscopic form of Maxwell's equations.

These can be position dependent, as in the grating sketched in \cref{fig:gratingL2:gratingL2Fig3}.

\imageFigure{../figures/ece1228-electromagnetic-theory/gratingL2Fig3}{Grating.}{fig:gratingL2:gratingL2Fig3}{0.2}

\index{capacitor}
\index{breakdown voltage}
The permeabilities can also depend on the strength of the fields.  An example, application of an electric field to gallium arsenide or glass can change the behaviour in the material.  We can also have non-linear effects, such as the effect on a capacitor when the voltage is increased.  The response near the breakdown point where the capacitor blows up demonstrates this spectacularly.  We can also have materials for which the permeabilities depend on the direction of the field, or the temperature, or the pressure in the environment, the tensile or compression forces on the material, or many other factors.  There are many other possible complicating factors, for example, the electric response \( \epsilon \) can depend on the magnetic field strength \( \Abs{\BB} \).  We could then write

\begin{dmath}\label{eqn:emtLecture2:140}
\epsilon = \epsilon( \Br, \Abs{\BE}, \BE/\Abs{\BE}, T, P, \Abs{\Beta}, \omega, k ).
\end{dmath}

Further complicating things is that \( \epsilon \) is a complex number (for fields specified in the frequency domain).

\index{anisotropic}
We can also have anisotropic situations where the electric and displacement fields are no longer colinear as sketched in \cref{fig:constituativeRelationsL2:constituativeRelationsL2Fig4}.

\imageFigure{../figures/ece1228-electromagnetic-theory/constituativeRelationsL2Fig4}{Anisotropic field relations.}{fig:constituativeRelationsL2:constituativeRelationsL2Fig4}{0.2}

which indicates that the permittivity \( \epsilon \) in the relation

\begin{dmath}\label{eqn:emtLecture2:160}
\BD = \epsilon \BE,
\end{dmath}

can be modelled as a matrix or as a second rank tensor.  When the off diagonal entries are zero, and the diagonal values are all equal, we have the special case where \( \epsilon \) is reduced to a function.  That function may still be complex-valued, and dependent on many factors, but it least it is scalar valued in this situation.

\index{polarization}
\index{magnetization}
\paragraph{Polarization and magnetization}

If we have a material (such as glass), we can generally assume that the induced field can be related to the vacuum field according to

\begin{dmath}\label{eqn:emtLecture2:180}
\BE = \BP + \epsilon_0 \BE,
\end{dmath}
and
\begin{dmath}\label{eqn:emtLecture2:200}
\BB = \mu_0 \BM + \mu_0 \BH = \mu_0 \lr{ \BM + \BH }.
\end{dmath}

\index{permittivity!vacuum}
Here the vacuum permittivity \( \epsilon_0 \) has the value \( 8.85 \times 10^{-12} \si{F/m} \).  When we are ignoring (fictional) magnetic sources, we have a constant relation between the magnetic fields \( \BB = \mu_0 \BH \).

Assuming \( \BP = \epsilon_0 \chi_\txte \BE \), then

\begin{dmath}\label{eqn:emtLecture2:220}
\BD
= \epsilon_0 \BE + \epsilon_0 \chi_\txte \BE
= \epsilon_0 ( 1 + \chi_\txte ) \BE ,
\end{dmath}

so with \( \epsilon_r = 1 + \chi_\txte \), and \( \epsilon = \epsilon_0 \epsilon_r \) we have

\begin{dmath}\label{eqn:emtLecture2:240}
\BD = \epsilon \BE.
\end{dmath}

\index{permittivity!relative}
Note that the relative permittivity \( \epsilon_r \) is dimensionless, whereas the vacuum permittivity has units of \si{F/m}.  We call \(\epsilon\) the (unqualified) permittivity.  The relative permittivity \( \epsilon_r\) is sometimes called the relative permittivity.

\index{index of refraction}
Another useful quantity is the index of refraction

\begin{dmath}\label{eqn:emtLecture2:260}
\eta = \sqrt{ \epsilon_r \mu_r } \approx \sqrt{\epsilon_r}.
\end{dmath}

Similar to the above we can write \( \BM = \chi_\txtm \BH \) then

\begin{dmath}\label{eqn:emtLecture2:280}
\BM = \mu_0 \BH + \mu_0 \BM = \mu_0 \lr{ 1 + \chi_\txtm } \BH
= \mu_0 \mu_r \BH
\end{dmath}

so with \( \mu_r = 1 + \chi_\txtm \), and \( \mu = \mu_0 \mu_r \) we have

\begin{dmath}\label{eqn:emtLecture2:300}
\BB = \mu \BH.
\end{dmath}

\paragraph{Linear and angular momentum in light}

\index{photon!momentum}
\index{photon!angular momentum}
It was pointed out that we have two relations in mechanics that relate momentum and forces

\begin{dmath}\label{eqn:emtLecture2:320}
\begin{aligned}
\BF &= \ddt{\BP} \\
\Btau &= \ddt{\BL},
\end{aligned}
\end{dmath}

where \( \BP = m \Bv \) is the linear momentum, and \( \BL = \Br \cross \Bp \) is the angular momentum.  In quantum electrodynamics, the photon can be described using a relationship between wave-vector and momentum

\begin{dmath}\label{eqn:emtLecture2:340}
\Bp
= \Hbar \Bk
= \Hbar \frac{ 2\pi}{\lambda}
= \frac{h}{2\pi} \frac{ 2\pi}{\lambda}
= \frac{h}{\lambda},
\end{dmath}

where \( \hbar = 6.522 \times 10^{-16} \si{ev.s} \).

Photons are also governed by

\begin{dmath}\label{eqn:emtLecture2:360}
E = \Hbar \omega = h \nu.
\end{dmath}

\index{De-Broglie relation}
(De-Broglie's relations).

ASIDE: optical fibre at 1550 has the lowest amount of optical attenuation.

Since photons have linear momentum, we can move things around using light.  With photons having both linear momentum and energy relationships, and there is a relation between between torque and linear momentum, it seems that there must be the possibility of light having angular momentum.

Is it possible to utilize the angular momentum to impose patterns on beams (such as laser beams).  For example, what if a beam could have a geometrical pattern along its line of propagation, being off in some regions, on in others.  This is in fact possible, generating beams that are ``self healing''.

The question was posed ``Is it possible to solve electromagnetic problems utilizing the force concepts?'', using the Lorentz
force equation

\begin{dmath}\label{eqn:emtLecture2:380}
\BF = q \Bv \cross \BB + q\BE.
\end{dmath}

This was not thought to be a productive approach due to the complexity.

FIXME: It appeared that this animated talk (probably not captured well) about momentum in light was linked to the idea of the Helmholtz theorem.  Exactly how was not clear to me.

\index{Helmholtz's theorem}
\paragraph{Helmholtz's theorem}

Suppose that we have a linear material where

\begin{dmath}\label{eqn:emtLecture2:400}
\begin{aligned}
\spacegrad \cross \BE &= -\PD{t}{\BB} \\
\spacegrad \cross \BH &= \BJ + \PD{t}{\BD} \\
\spacegrad \cdot \BE &= \frac{\rho_\txtv}{\epsilon_0} \\
\spacegrad \cdot \BH &= 0
\end{aligned}
\end{dmath}

We have relations between the divergence and curl of \( \BE \) given the sources.  Is that sufficient to determine \( \BE \) itself?  The answer is yes, which is due to the Helmholtz theorem.

Extra homework question (bonus) : can knowledge of the tangential components of the fields also be used to uniquely determine \( \BE \)?

%Also homework: read notes about irrotational fields and solenoidal fields.

%\EndArticle

   \section{Problems.}
      %
% Copyright � 2016 Peeter Joot.  All Rights Reserved.
% Licenced as described in the file LICENSE under the root directory of this GIT repository.
%
%{
%\input{../blogpost.tex}
%\renewcommand{\basename}{continuityDisplacement}
%%\renewcommand{\dirname}{notes/phy1520/}
%\renewcommand{\dirname}{notes/ece1228-electromagnetic-theory/}
%%\newcommand{\dateintitle}{}
%%\newcommand{\keywords}{}
%
%\input{../latex/peeter_prologue_print2.tex}
%
%\usepackage{peeters_layout_exercise}
%\usepackage{peeters_braket}
%\usepackage{peeters_figures}
%\usepackage{siunitx}
%%\usepackage{mhchem} % \ce{}
%%\usepackage{macros_bm} % \bcM
%%\usepackage{txfonts} % \ointclockwise
%
%\beginArtNoToc
%
%\generatetitle{Continuity equation and Ampere's law}
%\chapter{Continuity equation and Ampere's law}
%\label{chap:continuityDisplacement}
%
\makeproblem{Displacement current and Ampere's law.}{problem:continuityDisplacement:1}{
Show that without the displacement current \( \PDi{t}{\BD} \), Maxwell's equations will not satisfy conservation relations.
} % problem
%
\makeanswer{problem:continuityDisplacement:1}{
%\withproblemsetsParagraph{
%
Without the displacement current, Maxwell's equations are
\begin{equation}\label{eqn:continuityDisplacement:20}
\begin{aligned}
\spacegrad \cross \BE( \Br, t ) &= - \PD{t}{\BB}(\Br, t) \\
\spacegrad \cross \BH( \Br, t ) &= \BJ \\
\spacegrad \cdot \BD(\Br, t) &= \rho_{\txtv}(\Br, t) \\
\spacegrad \cdot \BB(\Br, t) &= 0.
\end{aligned}
\end{equation}
%
Assuming that the continuity equation must hold, we have
\begin{dmath}\label{eqn:continuityDisplacement:40}
0
= \spacegrad \cdot \BJ + \PD{t}{\rho_\txtv}
= \spacegrad \cdot \lr{ \spacegrad \cross \BH } + \PD{t}{} (\spacegrad \cdot \BD)
= \PD{t}{} (\spacegrad \cdot \BD)
\ne 0.
\end{dmath}
%
This shows that the current in Ampere's law must be transformed to
%
\begin{dmath}\label{eqn:continuityDisplacement:60}
\BJ \rightarrow \BJ + \PD{t}{\BD},
\end{dmath}
%
should we wish the continuity equation to be satisfied.  With such an addition we have
%
\begin{dmath}\label{eqn:continuityDisplacement:80}
0
= \spacegrad \cdot \BJ + \PD{t}{\rho_\txtv}
= \spacegrad \cdot \lr{ \spacegrad \cross \BH - \PD{t}{\BD} } + \PD{t}{} (\spacegrad \cdot \BD)
= \spacegrad \cdot \lr{ \spacegrad \cross \BH } - \spacegrad \cdot \PD{t}{\BD} + \PD{t}{} (\spacegrad \cdot \BD).
\end{dmath}
%
The first term is zero (assuming sufficient continuity of \(\BH\)) and the second two terms cancel when the space and time derivatives of one are commuted.
%}
} % answer
%
%}
%\EndNoBibArticle

      %
% Copyright � 2016 Peeter Joot.  All Rights Reserved.
% Licenced as described in the file LICENSE under the root directory of this GIT repository.
%
%{
%\input{../blogpost.tex}
%\renewcommand{\basename}{griffithsEM2_7}
%\renewcommand{\dirname}{notes/ece1228/}
%%\newcommand{\dateintitle}{}
%%\newcommand{\keywords}{}
%
%\input{../latex/peeter_prologue_print2.tex}
%
%\usepackage{peeters_layout_exercise}
%\usepackage{peeters_braket}
%\usepackage{peeters_figures}
%\usepackage{siunitx}
%
%\beginArtNoToc
%
%\generatetitle{Electric field due to spherical shell}
%\chapter{electric field due to spherical shell}
%\label{chap:griffithsEM2_7}
%
\makeoproblem{Electric field due to spherical shell.}{problem:griffithsEM2_7:1}{\citep{griffiths1999introduction} pr. 2.7}{
\index{electric field!spherical shell}
Calculate the field due to a spherical shell.  The field is
%
\begin{dmath}\label{eqn:griffithsEM2_7:20}
\BE = \frac{\sigma}{4 \pi \epsilon_0} \int \frac{(\Br - \Br')}{\Abs{\Br - \Br'}^3} da',
\end{dmath}
%
where \( \Br' \) is the position to the area element on the shell.  For the test position, let \( \Br = z \Be_3 \).
} % problem
%
\makeanswer{problem:griffithsEM2_7:1}{
%\withproblemsetsParagraph{
\index{geometric algebra}
We need to parameterize the area integral.  A complex-number like geometric algebra representation works nicely.
%
\begin{dmath}\label{eqn:griffithsEM2_7:40}
\Br'
= R \lr{ \sin\theta \cos\phi, \sin\theta \sin\phi, \cos\theta }
= R \lr{ \Be_1 \sin\theta \lr{ \cos\phi + \Be_1 \Be_2 \sin\phi } + \Be_3 \cos\theta }
= R \lr{ \Be_1 \sin\theta e^{i\phi} + \Be_3 \cos\theta }.
\end{dmath}
%
\index{pseudoscalar}
Here \( i = \Be_1 \Be_2 \) has been used to represent to horizontal rotation plane.
%
The difference in position between the test vector and area-element is
%
\begin{dmath}\label{eqn:griffithsEM2_7:60}
\Br - \Br'
= \Be_3 \lr{ z - R \cos\theta } - R \Be_1 \sin\theta e^{i \phi},
\end{dmath}
%
with an absolute squared length of
%
\begin{dmath}\label{eqn:griffithsEM2_7:80}
\Abs{\Br - \Br' }^2
= \lr{ z - R \cos\theta }^2 + R^2 \sin^2\theta
= z^2 + R^2 - 2 z R \cos\theta.
\end{dmath}
%
As a side note, this is a kind of fun way to prove the old ``cosine-law'' identity.  With that done, the field integral can now be expressed explicitly
%
\begin{dmath}\label{eqn:griffithsEM2_7:100}
\BE
= \frac{\sigma}{4 \pi \epsilon_0} \int_{\phi = 0}^{2\pi} \int_{\theta = 0}^\pi R^2 \sin\theta d\theta d\phi
\frac{\Be_3 \lr{ z - R \cos\theta } - R \Be_1 \sin\theta e^{i \phi}}
{
\lr{z^2 + R^2 - 2 z R \cos\theta}^{3/2}
}
= \frac{2 \pi R^2 \sigma \Be_3}{4 \pi \epsilon_0} \int_{\theta = 0}^\pi \sin\theta d\theta
\frac{z - R \cos\theta}
{
\lr{z^2 + R^2 - 2 z R \cos\theta}^{3/2}
}
= \frac{2 \pi R^2 \sigma \Be_3}{4 \pi \epsilon_0} \int_{\theta = 0}^\pi \sin\theta d\theta
\frac{ R( z/R - \cos\theta) }
{
(R^2)^{3/2} \lr{ (z/R)^2 + 1 - 2 (z/R) \cos\theta}^{3/2}
}
= \frac{\sigma \Be_3}{2 \epsilon_0} \int_{u = -1}^{1} du
\frac{ z/R - u}
{
\lr{1 + (z/R)^2 - 2 (z/R) u}^{3/2}
}.
\end{dmath}
%
Observe that all the azimuthal contributions get killed.  We expect that due to the symmetry of the problem.  We are left with an integral that submits to Mathematica, but doesn't look fun to attempt manually.  Specifically
%
\begin{dmath}\label{eqn:griffithsEM2_7:120}
\int_{-1}^1 \frac{a-u}{\lr{1 + a^2 - 2 a u}^{3/2}} du
=
\left\{
\begin{array}{l l}
\frac{2}{a^2} & \quad \mbox{if \( a > 1 \) } \\
0 & \quad \mbox{if \( a < 1 \),}
\end{array}
\right.
\end{dmath}
%
so
%
%\begin{dmath}\label{eqn:griffithsEM2_7:140}
\boxedEquation{eqn:griffithsEM2_7:160}{
\BE
=
\left\{
\begin{array}{l l}
\frac{\sigma (R/z)^2 \Be_3}{\epsilon_0}
& \quad \mbox{if \( z > R \) } \\
0 & \quad \mbox{if \( z < R \).}
\end{array}
\right.
}
%\end{dmath}
%
In the problem, it is pointed out to be careful of the sign when evaluating \( \sqrt{ R^2 + z^2 - 2 R z } \), however, I don't see where that is even useful?
%}
} % answer
%
%}
%\EndArticle

      %
% Copyright � 2016 Peeter Joot.  All Rights Reserved.
% Licenced as described in the file LICENSE under the root directory of this GIT repository.
%
\index{solenoidal}
\index{non-solenoidal}
\index{divergence free}
\makeproblem{Solenoidal fields.}{emt:problemSet1:1}{
For the electric fields graphically shown below indicate whether the fields are solenoidal (divergence free) or not. In the case of non-solenoidal fields indicate the charge generating the field is positive or negative. Justify your answer.
%
%\cref{fig:emtLect2:emtLect2Fig1}.
\imageFigure{../figures/ece1228-electromagnetic-theory/emtLect2Fig1}{Field lines.}{fig:emtLect2:emtLect2Fig1}{0.2}
} % makeproblem
%
\makeanswer{emt:problemSet1:1}{
%\withproblemsetsParagraph{
%
\begin{enumerate}[(a)]
\item
The first set of field lines has the appearance of non-solenoidal.  To demonstrate this a graphical-numeric approximation of \( \int \spacegrad \cdot \BE \propto \sum_i \ncap \cdot \BE_i \) is sketched in \cref{fig:nonSolinoidal:nonSolinoidalFig2}.
%
\imageFigure{../figures/ece1228-electromagnetic-theory/nonSolinoidalFig2}{Graphical divergence integration.}{fig:nonSolinoidal:nonSolinoidalFig2}{0.15}
%
For each field line \( \BE_i \), passing through this square integration volume, the length of the projection onto the \( x \) axis is shorter on the right side of the box than the left.  Suppose the left hand projections of \( \BE \) onto \( \xcap \) are \( 0.9 \), and \( 0.8 \) vs. \( 0.7\), and \(0.6\) on the right for the bottom and top red field lines respectively.  The flux of those field lines is proportional to
%
\begin{dmath}\label{eqn:emtProblemSet1Problem1:20}
\sum_i \ncap \cdot \BE \approx (0.7 - 0.9) + (0.6 - 0.8) = -0.4,
\end{dmath}
%
so this field appears to be non-solenoidal.  As for the charges generating the field, this field has the look of a small portion of a dipole field as sketched in \cref{fig:dipole:dipoleFig1}, with the lines in the supplied figure flowing out of a positive charge to a negative.
%
\imageFigure{../figures/ece1228-electromagnetic-theory/dipoleFig1}{Crude sketch of dipole field.}{fig:dipole:dipoleFig1}{0.15}
%
\item
This next figure has the appearance of the electric field lines coming out of a single positive charge
%
\begin{dmath}\label{eqn:emtProblemSet1Problem1:40}
\BE = \frac{q}{4 \pi \epsilon_0} \frac{\rcap}{r^2}.
\end{dmath}
%
Such a field is divergence free everywhere but the origin.  For \( \Br \ne 0 \)
%
\begin{dmath}\label{eqn:emtProblemSet1Problem1:60}
\spacegrad \cdot \BE
=
\frac{q}{4 \pi \epsilon_0} \spacegrad \cdot \frac{\Br}{r^3}
=
\frac{q}{4 \pi \epsilon_0} \lr{ \frac{\spacegrad \cdot \Br }{r^3} + \lr{ \spacegrad \inv{r^3} } \cdot \Br }
=
\frac{q}{4 \pi \epsilon_0} \lr{ \frac{ 3 }{r^3} + \lr{ -\frac{3}{2} 2 \frac{\Br}{r^5} } \cdot \Br }
=
0.
\end{dmath}
%
Because of the singularity at the origin, this is still a solenoidal field, as shown by the divergence integral
%
\begin{dmath}\label{eqn:emtProblemSet1Problem1:80}
\int_V \spacegrad \cdot \BE dV
=
\oint_{\partial V} \ncap \cdot \BE dA
=
\frac{q}{4 \pi \epsilon_0} \iint \rcap \cdot \ncap r^2 \sin\theta d\theta d\phi
=
\frac{q}{4 \pi \epsilon_0} \iint \ncap \cdot \frac{\rcap}{r^2} r^2 \sin\theta d\theta d\phi
=
\frac{q}{4 \pi \epsilon_0} 4 \pi
=
\frac{q}{\epsilon_0}.
\end{dmath}
%
\item
This last field is solenoidal, since the field lines are all of equal magnitude and direction.  Suppose that field was
%
\begin{dmath}\label{eqn:emtProblemSet1Problem1:100}
\BE = \xcap E,
\end{dmath}
%
where \( E \) is constant.  The divergence is then
%
\begin{dmath}\label{eqn:emtProblemSet1Problem1:120}
\spacegrad \cdot \BE = \PD{x}{E} = 0.
\end{dmath}
%
\end{enumerate}
%}
} % answer

      %
% Copyright � 2016 Peeter Joot.  All Rights Reserved.
% Licenced as described in the file LICENSE under the root directory of this GIT repository.
%
\index{field lines!electric}
\makeproblem{Electric field lines.}{emt:problemSet1:2}{
Can either or both of the vector fields shown below represent an electrostatic field \( \BE \). Justify your answer.
\imageTwoFigures
{../figures/ece1228-electromagnetic-theory/emtLect2Fig2}
{../figures/ece1228-electromagnetic-theory/emtLect2Fig3}
{Field lines.}{fig:emtLect2:emtLect2Fig2}{scale=0.3}
} % makeproblem
\makeanswer{emt:problemSet1:2}{\withproblemsetsParagraph{
\begin{enumerate}[(a)]
\item
The first field line configuration looks like it could be the superposition of a set of infinite planes (because of the straight line fields).  Two such planes do not have the degrees of freedom to supply the variability of this field, so let's try three.  Let's assume that we are looking at the configuration sketched in \cref{fig:threePlanes:threePlanesFig3}, where the magnitude (and signs) of the field strengths \( a, b, c \) are not known.
\imageFigure{../figures/ece1228-electromagnetic-theory/threePlanesFig3}{Three infinite planes.}{fig:threePlanes:threePlanesFig3}{0.3}
In regions 1, 2, 3, the field strengths have the following proportionate relationships
\begin{equation}\label{eqn:emtProblemSet1Problem2:20}
\begin{aligned}
a - b - c &= 1 \\
a + b - c &= 2 \\
a - b - c &= 1
\end{aligned}
\end{equation}
The solution to these equations is \( (a,b,c) = (1, 1/2, -1/2) \), so this field configuration can be obtained by a sequence of fields with charge densities in this proportion.  The supplied image represents the blue boxed region of the sketch, a dipole configuration with a positive surface charge density twice that of the positive centre surface charge density.
\item
If this second field configuration is the result of an electrostatic configuration, it is not obvious how.  We can superimpose any number of infinite planes to vary the field strength in regions between the planes, still maintaining the straight line configuration.  However, there's no such superposition that could also vary the field strength side to side from the centre as in the picture.
\end{enumerate}
}}

      %
% Copyright � 2016 Peeter Joot.  All Rights Reserved.
% Licenced as described in the file LICENSE under the root directory of this GIT repository.
%
\index{solenoidal}
\index{irrotational}
\makeproblem{Solenoidal and irrotational fields.}{emt:problemSet1:3}{
In terms of \(\BE\) or \(\BH\) give an example for each of the following conditions:
%
\makesubproblem{}{emt:problemSet1:3a}
Field is solenoidal and irrotational.
\makesubproblem{}{emt:problemSet1:3b}
Field is solenoidal and rotational.
\makesubproblem{}{emt:problemSet1:3c}
Field is non-solenoidal and irrotational.
\makesubproblem{}{emt:problemSet1:3d}
Field is non-solenoidal and rotational.
} % makeproblem
%
\skipIfRedacted{
\makeanswer{emt:problemSet1:3}{
\makeSubAnswer{}{emt:problemSet1:3a}
%
We can find a \( \BH \) for which \( \spacegrad \cdot \BH = 0 \) and \( \spacegrad \cross \BH = 0 \).
For simple media where \( \BB = \mu \BH \), let \( \BH = \mu \spacegrad \cross \BA \).  This automatically satisfies \( \spacegrad \cdot \BH = 0 \), and has curl
%
\begin{dmath}\label{eqn:emtProblemSet1Problem3:20}
\spacegrad \cross \BH
=
\mu \spacegrad \cross ( \spacegrad \cross \BA )
=
\mu \lr{ \spacegrad (\spacegrad \cdot \BA) -\spacegrad^2 \BA} .
\end{dmath}
%
Seeking a \( \BA \) for which this is zero, a trial function such as \( \BA = \Be_1 \psi \) seems like a reasonable first try, for which we have
%
\begin{dmath}\label{eqn:emtProblemSet1Problem3:40}
\spacegrad \cross \BH
=
\mu \lr{ \spacegrad (\partial_x \psi) - \Be_1 \spacegrad^2 \psi}.
\end{dmath}
%
If \( \psi \) is any polynomial in \( y \) or \( z\) that has degrees less than two, this will be zero.  One such function is
\( \BA = A \Be_1 y \), which gives
%
\begin{dmath}\label{eqn:emtProblemSet1Problem3:60}
\BH =
\mu \begin{vmatrix}
\Be_1 & \Be_2 & \Be_3 \\
\partial_x & \partial_y & \partial_z \\
A f(y) & 0 & 0 \\
\end{vmatrix}
=
-\mu A \Be_3.
\end{dmath}
As there is no positional dependence, this clearly has both \( \spacegrad \cdot \BH = 0 \), and \( \spacegrad \cross \BH = 0 \), the desired respective solenoidal and irrotational properties.
\makeSubAnswer{}{emt:problemSet1:3b}
A field \( \BF \) for which \( \spacegrad \cdot \BF = 0 \) and \( \spacegrad \cross \BF \ne 0 \) is sought.
Any solution to the magnetostatics equations in simple media \( \BB = \mu \BH \) suffices
\begin{equation}\label{eqn:emtProblemSet1Problem3:80}
\begin{aligned}
   \spacegrad \cross \BH &= \mu \BJ \\
   \spacegrad \cdot \BH &= 0,
\end{aligned}
\end{equation}
provided \( \BJ \) is non-zero in some region. One such solution is provided by the volume form of the Biot-Savart law
\begin{dmath}\label{eqn:emtProblemSet1Problem3:100}
   \BH = \inv{4\pi} \iiint_V dV \frac{\BJ \cross \rcap'}{\Abs{\Br'}^2}.
\end{dmath}
This has the desired solenoidal
\( \spacegrad \cdot \BH = 0 \)
and rotational
\( \spacegrad \cross \BH \ne 0 \)
properties.
\makeSubAnswer{}{emt:problemSet1:3c}
A field \( \BF \) that is non-solenoidal \( \spacegrad \cdot \BF \ne 0 \) and irrotational \( \spacegrad \cross \BF = 0 \) is sought.
Any solution to the electrostatics equations in simple media \( \BD = \epsilon \BE \) suffices
\begin{equation}\label{eqn:emtProblemSet1Problem3:120}
\begin{aligned}
   \spacegrad \cdot \BE &= \inv{\epsilon} \rho \\
   \spacegrad \cross \BE &= 0,
\end{aligned}
\end{equation}
provided \( \rho \) is non-zero in some region.  The standard Coulomb solution provides a field with the desired
non-solenoidal and irrotational properties
\begin{dmath}\label{eqn:emtProblemSet1Problem3:140}
   \BE = \inv{4\pi \epsilon} \iiint_V dV \frac{\rho}{\Abs{\Br}^2}.
\end{dmath}
\makeSubAnswer{}{emt:problemSet1:3d}
A field \( \BF \) that is non-solenoidal \( \spacegrad \cdot \BF \ne 0 \) and rotational \( \spacegrad \cross \BF \ne 0 \) is sought.
%
Any electric field solution to the general Maxwell's equations for which there is a non-zero charge density, and a magnetic field component has these characteristics.  In simple media where \( \BD = \epsilon \BE \) such a solution will satisfy both Faraday's law and Gauss's law
\begin{equation}\label{eqn:emtProblemSet1Problem3:160}
\begin{aligned}
   \spacegrad \cross \BE &= -\partial_t \BB \\
   \spacegrad \cdot \BE &= \inv{\epsilon} \rho.
\end{aligned}
\end{equation}
One such solution is the superposition of an electrostatics solution with any solution describing the propagation of light, for example
\begin{dmath}\label{eqn:emtProblemSet1Problem3:180}
   \BE = \inv{4\pi \epsilon} \iiint_V dV \frac{\rho}{\Abs{\Br}^2} +
\Be_2 E_0 \sin\lr{ \frac{\pi}{a} x } \cos\lr{ \omega t - \beta_\txtz z }.
\end{dmath}
This is the superposition of the field from problem 6 with an electrostatics solution.
}}

      %
% Copyright � 2016 Peeter Joot.  All Rights Reserved.
% Licenced as described in the file LICENSE under the root directory of this GIT repository.
%
\makeproblem{Conducting sheet with hole.}{emt:problemSet1:4}{
\index{conducting sheet}
%Figure
\Cref{fig:emtLect2:emtLect2Fig4}.
shows a flat, positive, non-conducting sheet of charge with uniform charge density \( \sigma \) [\si{C/m^2}]. A small circular hole of radius \(R \) is cut in the middle of the surface as shown.
%
\imageFigure{../figures/ece1228-electromagnetic-theory/emtLect2Fig4}{Conducting sheet with a hole.}{fig:emtLect2:emtLect2Fig4}{0.3}
%
Calculate the electric field intensity \(\BE\) at point \(P\), a distance \(z\) from the center of the hole along its axis.
%
Hint 1: Ignore the field fringe effects around all edges.
Hint 2: Calculate the field due to a disk of radius \(R\) and use superposition.
%
} % makeproblem
%
\skipIfRedacted{
\makeanswer{emt:problemSet1:4}{
%
For a charged circular disk of radius \( R \) the electric field at the position z above the sheet is
%
\begin{dmath}\label{eqn:emtProblemSet1Problem4:20}
\BE(z)
= \frac{\sigma}{4 \pi \epsilon_0} \iint dA' \frac{z \Be_3 - \Bx'}{\Abs{z^2 + {\Bx'}^2}^{3/2}}
= \frac{\sigma}{4 \pi \epsilon_0} \int_{r=0}^R \int_{\theta = 0}^{2\pi} r dr d\theta \frac{z \Be_3 - r \Be_1 \cos\theta - r \Be_2 \sin\theta }{\Abs{z^2 + r^2}^{3/2}}
= \frac{2 \pi \sigma z^2}{4 \pi \epsilon_0} \int_{r=0}^R (r/z) (dr/z) \frac{z \Be_3}{\Abs{z}^3 \Abs{1 + (r/z)^2}^{3/2}}
= \frac{\sigma \sgn(z) \Be_3 }{2 \epsilon_0} \int_{u=0}^{R/z} \frac{u du}{\Abs{1 + u^2}^{3/2}}
= \frac{\sigma \sgn(z) \Be_3 }{2 \epsilon_0} \evalrange{\lr{-\inv{\sqrt{u^2 + 1}}}}{u=0}{R/z}
= \frac{\sigma \sgn(z) \Be_3 }{2 \epsilon_0} \lr{ 1 -\inv{\sqrt{(R/z)^2 + 1}} }.
\end{dmath}
%
In a gross sense, this result can be used for both the large sheet and the hole.  If the large sheet is modeled as a circular disk with radius \( R \gg z \), one way of completely neglecting any edge effects, then the field above the center is from this portion of the sheet is approximately
%
\begin{dmath}\label{eqn:emtProblemSet1Problem4:40}
\BE_1(z) = \frac{\sigma \sgn(z) \Be_3 }{2 \epsilon_0}.
\end{dmath}
%
Subtracting the field for the disk of radius \( R \) above, we have
\begin{dmath}\label{eqn:emtProblemSet1Problem4:60}
\BE(z)
= \frac{\sigma \sgn(z) \Be_3 }{2 \epsilon_0} \lr{ 1 - \lr{ 1 -\inv{\sqrt{(R/z)^2 + 1}} } }
= \frac{\sigma \sgn(z) \Be_3 }{2 \epsilon_0} \inv{\sqrt{(R/z)^2 + 1}}
= \frac{\sigma \sgn(z) \Be_3 }{2 \epsilon_0} \frac{\Abs{z}}{R} \inv{\sqrt{1 + (z/R)^2}},
\end{dmath}
%
or
\begin{dmath}\label{eqn:emtProblemSet1Problem4:80}
\BE(z) = \frac{\sigma z \Be_3 }{2 \epsilon_0 R \sqrt{1 + (z/R)^2}}.
\end{dmath}
}}

      %
% Copyright � 2016 Peeter Joot.  All Rights Reserved.
% Licenced as described in the file LICENSE under the root directory of this GIT repository.
%
\makeproblem{Helmholtz theorem.}{emt:problemSet1:5}{
\index{Helmholtz's theorem}
Prove the first Helmholtz's theorem, i.e. if vector \(\BM\) is defined by its divergence
%
\begin{equation}\label{eqn:emtProblemSet1Problem5:20}
\spacegrad \cdot \BM = s
\end{equation}
%
and its curl
\begin{equation}\label{eqn:emtProblemSet1Problem5:40}
\spacegrad \cross \BM = \BC.
\end{equation}
%
within a region and its normal component \( \BM_{\txtn} \) over the boundary, then \( \BM \) is
uniquely specified.
%
%Note: Assume there is a vector \( \BN \) with its divergence and curl equal to \( s \) and \( \BC \) respectively, then show that \( \BM = \BN \) .
} % makeproblem
%
\skipIfRedacted{
\makeanswer{emt:problemSet1:5}{
%
%I attempted this problem in two different ways.  My first approach assembled the divergence and curl relations above into a single (geometric algebra) multivector gradient equation and applied the vector valued Green's function for the gradient to invert that equation.  That approach logically led from the differential equation for \( \BM \) to the solution for \( \BM \) in terms of \( s \) and \( \BC \).  However, this strategy introduced some complexities
%that make me doubt the correctness of the associated boundary analysis.
%
%Even if the details of the boundary handling in my multivector approach is not correct, I thought that approach was interesting enough to share, and have placed it in an appendix to this problem set.  It is accompanied with a primer on geometric algebra that is hopefully enough to allow the reader to grasp the basic ideas of the approach, but is probably not sufficient to understand all the details without further study.
%
%The answer obtained in that first attempt at this problem, when \( \Norm{\BM}/r^2 \), and \( \Norm{\BC}/r \) both vanish on an infinite sphere, is that the field has a unique value determined by \( s \) and \( \BC \), namely
%
%\begin{equation}\label{eqn:emtProblemSet1Problem5:60}
%\BM =
%-\spacegrad \int_V dV' \frac{ s(\Bx')}{ 4 \pi \Norm{\Bx - \Bx'} }
%+\spacegrad \cross \int_V dV' \frac{ \BC(\Bx') }{ 4 \pi \Norm{\Bx - \Bx'} }.
%\end{equation}
%
%It's possible to work backwards from this result to obtain second order gradient terms applied to \( \BM(\Bx')/\Norm{\Bx - \Bx'} \) .  This suggests that a Laplacian (i.e. scalar) representation of the delta function may be a superior way to tackle this problem, perhaps also yielding a simpler result for the boundary term.  This is in fact the case, and the logical starting point
%
We can compute the relationship between the vector \( \BM \) using a convolution representation of the vector function \( \BM \) and a Laplacian representations of the delta function
%
\begin{equation}\label{eqn:emtProblemSet1Problem5:80}
\BM(\Bx) = \int_V dV' \delta(\Bx - \Bx') \BM(\Bx').
\end{equation}
%
\index{Laplacian}
\index{delta function!Laplacian}
The Laplacian representation of the delta function in \R{3} is
%
\begin{equation}\label{eqn:emtProblemSet1Problem5:100}
\delta(\Bx - \Bx') = -\inv{4\pi} \spacegrad^2 \inv{\Norm{\Bx - \Bx'}},
\end{equation}
%
\index{convolution}
so \( \BM \) can be represented as the following convolution
%
\begin{equation}\label{eqn:emtProblemSet1Problem5:120}
\BM(\Bx) = -\inv{4\pi} \int_V dV' \spacegrad^2 \inv{\Norm{\Bx - \Bx'}} \BM(\Bx').
\end{equation}
%
As noted in \cref{eqn:usefulFormulas:700}
%\cref{eqn:emtProblemSet1Appendix:460}
the Laplacian of a vector can be factored as
%
\begin{equation}\label{eqn:emtProblemSet1Problem5:140}
\spacegrad^2 \Ba
=
\spacegrad (\spacegrad \cdot \Ba)
-
\spacegrad \cross (\spacegrad \cross \Ba).
\end{equation}
%
%Note that the last term can be written in cross product notation using \( \Bc \cdot (\Ba \wedge \Bb) = -\Bc \cross (\Ba \cross \Bb) \) if desired.
%
Using this relation and proceeding with a few applications of the chain rule, plus the fact that \(
\spacegrad 1/\Norm{\Bx - \Bx'} = -\spacegrad' 1/\Norm{\Bx - \Bx'} \), we find
%
\begin{equation}\label{eqn:emtProblemSet1Problem5:160}
\begin{aligned}
&-4 \pi \BM(\Bx) \\
&= \int_V dV' \spacegrad^2 \inv{\Norm{\Bx - \Bx'}} \BM(\Bx')
\\ &=
\spacegrad \int_V dV' \spacegrad \cdot \inv{\Norm{\Bx - \Bx'}} \BM(\Bx')
- \spacegrad \cross \int_V dV' \spacegrad \cross \inv{\Norm{\Bx - \Bx'}} \BM(\Bx')
\\ &=
-\spacegrad \int_V dV' \lr{ \spacegrad' \cdot \inv{\Norm{\Bx - \Bx'}}} \BM(\Bx')
+\spacegrad \cross \int_V dV' \lr{ \spacegrad' \inv{\Norm{\Bx - \Bx'}} } \cross \BM(\Bx')
\\ &=
-\spacegrad \int_V dV' \spacegrad' \cdot \frac{ \BM(\Bx') }{\Norm{\Bx - \Bx'}}
+\spacegrad \int_V dV' \frac{ \spacegrad' \cdot \BM(\Bx')}{\Norm{\Bx - \Bx'}}
+\spacegrad \cross \int_V dV' \spacegrad' \cross \frac{\BM(\Bx')}{\Norm{\Bx - \Bx'}}
-\spacegrad \cross \int_V dV' \frac{ \spacegrad' \cross \BM(\Bx')}{\Norm{\Bx - \Bx'}}
%\\ &=
%-\spacegrad \int_{\partial V} dA' \ncap \cdot \frac{ \BM(\Bx') }{\Norm{\Bx - \Bx'}}
%+\spacegrad \int_V dV' \frac{ s(\Bx')}{\Norm{\Bx - \Bx'}}
%-\spacegrad \cdot \int_{\partial V} dV' \ncap \wedge \frac{\BM(\Bx')}{\Norm{\Bx - \Bx'}}
%+\spacegrad \cdot \int_V dV' \frac{ I \BC(\Bx')}{\Norm{\Bx - \Bx'}}
,
\end{aligned}
\end{equation}
%
or
%
%\begin{equation}\label{eqn:emtProblemSet1Problem5:180}
\boxedEquation{eqn:emtProblemSet1Problem5:200}{
\begin{aligned}
\BM(\Bx)
&=
-\spacegrad \int_V dV' \frac{ s(\Bx')}{\Norm{\Bx - \Bx'}}
+\spacegrad \cross \int_V dV' \frac{ \BC(\Bx')}{\Norm{\Bx - \Bx'}} \\
&+\spacegrad \int_{\partial V} dA' \ncap \cdot \frac{ \BM(\Bx') }{\Norm{\Bx - \Bx'}}
-\spacegrad \cross \int_{\partial V} dV' \ncap \cross \frac{\BM(\Bx')}{\Norm{\Bx - \Bx'}}
.
\end{aligned}
%\end{equation}
}
%
If the surface in question is taken to be an infinite sphere over all space, then we require that \( \Norm{\BM(\Bx')}/\Norm{\Bx - \Bx'} \) vanishes, killing the surface integral.  Such a solution is uniquely specified by the divergence and the curl
%
\begin{equation}\label{eqn:emtProblemSet1Problem5:220}
\BM_\infty(\Bx)
=
-\spacegrad \int_V dV' \frac{ s(\Bx')}{\Norm{\Bx - \Bx'}}
+\spacegrad \cross \int_V dV' \frac{ \BC(\Bx')}{\Norm{\Bx - \Bx'}}
.
\end{equation}
%
For a finite integration surface, the divergence and curl no longer specify a unique solution.  To that solution a particular solution involving the normal and tangential components of the to-be-determined vector \( \BM \) must be added.  That is
%
\begin{equation}\label{eqn:emtProblemSet1Problem5:240}
\begin{aligned}
\BM(\Bx)
&=
\BM_\infty(\Bx)
+\spacegrad \int_{\partial V} dA' \ncap \cdot \frac{ \BM(\Bx') }{\Norm{\Bx - \Bx'}} \\
&\qquad -\spacegrad \cross \int_{\partial V} dV' \ncap \cross \frac{\BM(\Bx')}{\Norm{\Bx - \Bx'}}
.
\end{aligned}
\end{equation}
%
%When the surface is finite, then we must add to thatthe the solutio
%The vector \( \BM \) is unique if
% on the infinite sphere.  For solutions valid in and on a specific surface, the general solution is found by a superposition of the unique infinite-sphere solution, with a specific integral equation solution dependent on the normal and tangential components of \( \BM \) on the surface of interest.
}}

      %
% Copyright � 2016 Peeter Joot.  All Rights Reserved.
% Licenced as described in the file LICENSE under the root directory of this GIT repository.
%
\makeproblem{Waveguide field.}{emt:problemSet1:6}{
\index{waveguide}
The instantaneous electric field inside a conducting parallel plate waveguide is given by
%
\begin{dmath}\label{eqn:emtProblemSet1Problem6:20}
\bcE(\Br, t) = \Be_2 E_0 \sin\lr{ \frac{\pi}{a} x } \cos\lr{ \omega t - \beta_\txtz z },
\end{dmath}
%
where \( \beta_\txtz \) is the waveguide's phase constant and \( a \) is the waveguide width (a constant).
Assuming there are no sources within the free-space-filled pipe, determine
\makesubproblem{}{emt:problemSet1:6a}
The corresponding instantaneous magnetic field components inside the conducting pipe.
\makesubproblem{}{emt:problemSet1:6b}
The phase constant \( \beta_z \).
} % makeproblem
%
\makeanswer{emt:problemSet1:6}{\withproblemsetsParagraph{
\makeSubAnswer{}{emt:problemSet1:6a}
%
Following the notation of \citep{balanis1989advanced} the phasor for the electric field is
%
\begin{dmath}\label{eqn:emtProblemSet1Problem6:40}
\BE(\Br, t) = \Be_2 E_0 \sin\lr{ \frac{\pi}{a} x } e^{ -j\beta_\txtz z },
\end{dmath}
%
and Maxwell's equations in source free linear media are
\index{Maxwell's equations!frequency domain}
\begin{subequations}
\label{eqn:emtProblemSet1Problem6:60}
\begin{dmath}\label{eqn:emtProblemSet1Problem6:80}
\spacegrad \cross \BE = -j \omega \BB,
\end{dmath}
\begin{dmath}\label{eqn:emtProblemSet1Problem6:100}
\spacegrad \cross \BB = -j \omega \mu \epsilon \BE,
\end{dmath}
\begin{dmath}\label{eqn:emtProblemSet1Problem6:120}
\spacegrad \cdot \BE = 0,
\end{dmath}
\begin{dmath}\label{eqn:emtProblemSet1Problem6:140}
\spacegrad \cdot \BB = 0.
\end{dmath}
\end{subequations}
%
Rearranging for \( \BB \) in \cref{eqn:emtProblemSet1Problem6:60} gives
%
\begin{dmath}\label{eqn:emtProblemSet1Problem6:160}
\BB
= \frac{j}{\omega} \spacegrad \cross \BE
= \frac{j E_0}{\omega}
\begin{vmatrix}
\Be_1 & \Be_2 & \Be_3 \\
\partial_x & \partial_y & \partial_z \\
0 & \sin\lr{ \frac{\pi}{a} x } e^{ -j\beta_\txtz z } & 0
\end{vmatrix}
=
\frac{j E_0}{\omega}
\lr{
\Be_1 \beta \sin\lr{ \frac{\pi}{a} x } + \Be_3 \frac{\pi}{a} \cos\lr{ \frac{\pi}{a} x }
} e^{ -j\beta_\txtz z }.
\end{dmath}
%
In the time domain, the magnetic field is
\begin{dmath}\label{eqn:emtProblemSet1Problem6:180}
\bcB =
\Real \BB e^{j \omega t}
=
-\frac{E_0}{\omega}
\lr{
\Be_1 \beta \sin\lr{ \frac{\pi}{a} x } + \Be_3 \frac{\pi}{a} \cos\lr{ \frac{\pi}{a} x }
} \sin\lr{ \omega t - \beta_\txtz z },
\end{dmath}
%
or
%\begin{dmath}\label{eqn:emtProblemSet1Problem6:200}
\boxedEquation{eqn:emtProblemSet1Problem6:200}{
\bcH =
-\frac{E_0}{\omega \mu}
\lr{
\Be_1 \beta \sin\lr{ \frac{\pi}{a} x } + \Be_3 \frac{\pi}{a} \cos\lr{ \frac{\pi}{a} x }
} \sin\lr{ \omega t - \beta_\txtz z }.
}
%\end{dmath}
%
\makeSubAnswer{}{emt:problemSet1:6b}
%
\index{wave equation}
To determine the constraints on \( \beta_\txtz \) it can be observed that the fields obey the wave equation (or Helmholtz equation in the frequency domain), which follows from the expansion of the curl of the curl
%
\begin{dmath}\label{eqn:emtProblemSet1Problem6:220}
\spacegrad \cross \lr{ \spacegrad \cross \BA }
=
\spacegrad \lr{ \spacegrad \cdot \BA } - \spacegrad^2 \BA.
\end{dmath}
Applying this to the electric field equations, we find
%
\begin{dmath}\label{eqn:emtProblemSet1Problem6:300}
\spacegrad \cancel{\spacegrad \cdot \BE } - \spacegrad^2 \BE
=
\spacegrad \cross \lr{ \spacegrad \cross \BE }
= -j \omega \spacegrad \cross \BB
= -j \omega (j \omega \mu \epsilon) \BE,
\end{dmath}
%
or
\begin{dmath}\label{eqn:emtProblemSet1Problem6:240}
\spacegrad^2 \BE
= -\mu \epsilon \omega^2 \BE.
\end{dmath}
%
The Laplacian of the electric field of this problem is
%
\begin{dmath}\label{eqn:emtProblemSet1Problem6:260}
\spacegrad^2 \BE
=
\lr{ -\lr{\frac{\pi}{a}}^2 + (-j\beta_\txtz)^2 }
\Be_2 E_0 \sin\lr{ \frac{\pi}{a} x } e^{ -j\beta_\txtz z },
\end{dmath}
%
\index{dispersion rate}
so the constraint on the dispersion rate is
%
%\begin{dmath}\label{eqn:emtProblemSet1Problem6:280}
\boxedEquation{eqn:emtProblemSet1Problem6:280}{
\beta_\txtz^2 = \mu \epsilon \omega^2 - \lr{\frac{\pi}{a}}^2.
}
%\end{dmath}
%
}}

      %\input{Set1Appendix.tex}
      %
% Copyright � 2016 Peeter Joot.  All Rights Reserved.
% Licenced as described in the file LICENSE under the root directory of this GIT repository.
%
\makeproblem{Infinite line charge.}{emt:problemSet2:2}{
\index{line charge}

An infinitely long straight line charge has a constant charge density \( \rho_l \) [\si{C/m}].

\makesubproblem{}{emt:problemSet2:2a}
Using the integral formulation for \( \BE \)
discussed in the class calculate the electric
field at an arbitrary point \( \BA(\rho, \phi, z) \).
\makesubproblem{}{emt:problemSet2:2b}

Using the Gauss law calculate the same as \partref{emt:problemSet2:2a}.

\makesubproblem{}{emt:problemSet2:2c}

Now suppose that our uniformly charged (\( \rho_l \) constant)
has a finite extension from \( z = a \) to \( z = b \), as sketched in \cref{fig:problemset2:problemset2Fig2}.
\imageFigure{../figures/ece1228-electromagnetic-theory/problemset2Fig2}{Line charge.}{fig:problemset2:problemset2Fig2}{0.3}
Find the electric field at the
arbitrary point \( \BA \).

Note: Express your results in cylindrical coordinate
system.
} % makeproblem

\makeanswer{emt:problemSet2:2}{\withproblemsetsParagraph{
\makeSubAnswer{}{emt:problemSet2:2a}

Since the line charge is of infinite length and rotationally symmetric with respect to \( \phi \), the observation point can be positioned anywhere convenient, such as

\begin{dmath}\label{eqn:emtProblemSet2Problem2:20}
\BA = \rho \rhocap.
\end{dmath}

Let the point on the wire be
\begin{dmath}\label{eqn:emtProblemSet2Problem2:40}
\Br' = z \zcap.
\end{dmath}

The absolute distance between the observation point and the element of charge is
\begin{dmath}\label{eqn:emtProblemSet2Problem2:60}
\Abs{\BA - \Br'} = \sqrt{ (\rho \rhocap)^2 + (z \zcap)^2} = \sqrt{ \rho^2 + z^2 }.
\end{dmath}

The electric field can now be expressed in integral form
\begin{dmath}\label{eqn:emtProblemSet2Problem2:80}
\BE(\BA)
=
\inv{ 4 \pi \epsilon_0 } \int_{-\infty}^\infty dz \rho_l \frac{ \rho \rhocap - z \zcap }{ \lr{\rho^2 + z^2}^{3/2} }
=
\frac{\sigma_l \BA }{ 4 \pi \epsilon_0 } \int_{-\infty}^\infty dz \inv{ \lr{\rho^2 + z^2}^{3/2} }
=
\frac{\sigma_l \BA }{ 4 \pi \epsilon_0 \rho^3 } \int_{-\infty}^\infty dz \inv{ \lr{1 + (z/\rho)^2}^{3/2} }
=
\frac{\sigma_l \BA }{ 4 \pi \epsilon_0 \rho^2 } \int_{-\infty}^\infty du \inv{ \lr{1 + u^2}^{3/2} }.
\end{dmath}

The \( \zcap \) term was killed since the integrand was an odd function in \( z \).  Note that

\begin{dmath}\label{eqn:emtProblemSet2Problem2:100}
   \int \frac{du}{ \lr{1 + u^2}^{3/2} }
= \frac{u}{\sqrt{1 + u^2}},
\end{dmath}

which has a PV limit over \([-\infty,\infty]\) of 2.  This gives

%\begin{dmath}\label{eqn:emtProblemSet2Problem2:120}
\boxedEquation{eqn:emtProblemSet2Problem2:120}{
\BE(\rho)
=
\inv{ 4 \pi \epsilon_0 }
\frac{2 \sigma_l }{\rho} \rhocap.
}
%\end{dmath}

\makeSubAnswer{}{emt:problemSet2:2b}

Using Gauss's law integrating over a cylindrical segment surrounding the wire, we have

\begin{dmath}\label{eqn:emtProblemSet2Problem2:140}
\Delta z \frac{\rho_l}{\epsilon_0}
= \int \spacegrad \cdot \BE dV
= \oint \ncap \cdot \BE dA
= E_\rho(\rho) (2 \pi \rho \Delta z),
\end{dmath}

This assumes the end surfaces at infinity contribute nothing to the surface integral, and gives after rearrangement

\begin{dmath}\label{eqn:emtProblemSet2Problem2:160}
E_\rho(\rho)
= \inv{2 \pi \rho } \frac{\rho_l}{\epsilon_0}.
\end{dmath}

This matches \cref{eqn:emtProblemSet2Problem2:120} as expected.

\makeSubAnswer{}{emt:problemSet2:2c}
\index{cylindrical coordinates}
For the finite problem, there is still rotational symmetry, but the \( \zcap \) term of the field is no longer cancelled out, and we have a \( z \) dependence in the field.

Let
\begin{equation}\label{eqn:emtProblemSet2Problem2:240}
\begin{aligned}
\BA &= z \zcap + \rho\rhocap \\
\Br' &= z' \zcap,
\end{aligned}
\end{equation}
for which
\begin{equation}\label{eqn:emtProblemSet2Problem2:260}
\begin{aligned}
\BA - \Br' &= (z - z') \zcap + \rho \rhocap \\
\Abs{\BA - \Br'}^2 &= (z - z')^2 + \rho^2.
\end{aligned}
\end{equation}
\begin{dmath}\label{eqn:emtProblemSet2Problem2:180}
\BE(\BA)
=
\frac{\rho_l}{ 4 \pi \epsilon_0 } \int_{a}^b dz \frac{ \rho \rhocap + (z - z') \zcap }{ \lr{\rho^2 + (z-z')^2}^{3/2} }.
\end{dmath}
Let \( z' - z = \rho u \), for
\begin{dmath}\label{eqn:emtProblemSet2Problem2:280}
\BE(\rho, z)
=
\frac{\rho_l}{ 4 \pi \epsilon_0 \rho } \int_{(a-z)/\rho}^{(b-z)/\rho} du \frac{ \rhocap - u \zcap }{ \lr{1 + u^2}^{3/2} }.
\end{dmath}
Note that
\begin{dmath}\label{eqn:emtProblemSet2Problem2:200}
\int du \frac{ u }{ \lr{1 + u^2}^{3/2} } = -\inv{\sqrt{1 + u^2}},
\end{dmath}
so
\begin{dmath}\label{eqn:emtProblemSet2Problem2:220}
\BE(\rho, z) =
\frac{\rho_l }{ 4 \pi \epsilon_0 \rho } \lr{
\evalrange{ \frac{\zcap}{\sqrt{1 + u^2}} }{(a-z)/\rho}{(b-z)/\rho}
+
\evalrange{ \frac{u \rhocap}{\sqrt{1 + u^2}} }{(a-z)/\rho}{(b-z)/\rho}
}.
\end{dmath}
This expands to
%\begin{dmath}\label{eqn:emtProblemSet2Problem2:300}
\boxedEquation{eqn:emtProblemSet2Problem2:320}{
\BE(\rho, z)
=
\frac{\rho_l }{ 4 \pi \epsilon_0 \rho } \lr{
\frac{\rho \zcap + (b-z)\rhocap}{\sqrt{\rho^2 + (b-z)^2}}
-\frac{\rho \zcap + (a-z)\rhocap}{\sqrt{\rho^2 + (a-z)^2}}
}.
}
%\end{dmath}
Observe that we recover \cref{eqn:emtProblemSet2Problem2:120} when \( z = 0, a = -b, b \rightarrow \infty \), as expected.
}}

      %
% Copyright � 2016 Peeter Joot.  All Rights Reserved.
% Licenced as described in the file LICENSE under the root directory of this GIT repository.
%
\makeproblem{Gradient in cylindrical coordinates.}{emt:problemSet2:3}{
\index{gradient!cylindrical coordinates}
%
If gradient of a scalar function \( \psi \) rectangular coordinate system is given by
%
\begin{dmath}\label{eqn:emtproblemSet2Problem3:20}
\spacegrad \psi =
\xcap_1 \PD{x}{\psi}
+\ycap_2 \PD{y}{\psi}
+\zcap \PD{z}{\psi},
\end{dmath}
%
using coordinate transformation and chain rule show
that the gradient of \( \psi \) in cylindrical coordinates is given by
%
\begin{dmath}\label{eqn:emtproblemSet2Problem3:40}
\spacegrad \psi =
\rhocap \PD{\rho}{\psi}
+\phicap \inv{\rho} \PD{\phi}{\psi}
+\zcap \PD{z}{\psi}.
\end{dmath}
} % makeproblem
%
\skipIfRedacted{
\makeanswer{emt:problemSet2:3}{
%
The components of the gradient in the cylindrical coordinate system can be factored into a product of a rotation matrix and the Cartesian components of the gradient
\index{rotation matrix}
\index{basis vectors}
\begin{dmath}\label{eqn:emtProblemSet2Problem3:60}
\begin{bmatrix}
\rhocap \cdot \spacegrad \psi \\
\phicap \cdot \spacegrad \psi \\
\zcap \cdot \spacegrad \psi \\
\end{bmatrix}
=
\begin{bmatrix}
\rhocap \cdot \lr{ \xcap \PDi{x}{\psi} +\ycap \PDi{y}{\psi} +\zcap \PDi{z}{\psi} } \\
\phicap \cdot \lr{ \xcap \PDi{x}{\psi} +\ycap \PDi{y}{\psi} +\zcap \PDi{z}{\psi} } \\
\zcap \cdot \lr{ \xcap \PDi{x}{\psi} +\ycap \PDi{y}{\psi} +\zcap \PDi{z}{\psi} } \\
\end{bmatrix}
=
\begin{bmatrix}
\rhocap \cdot \xcap & \rhocap \cdot \ycap & \rhocap \cdot \zcap \\
\phicap \cdot \xcap & \phicap \cdot \ycap & \phicap \cdot \zcap \\
\zcap \cdot \xcap & \zcap \cdot \ycap & \zcap \cdot \zcap \\
\end{bmatrix}
\begin{bmatrix}
\PDi{x}{\psi} \\
\PDi{y}{\psi} \\
\PDi{z}{\psi}
\end{bmatrix}
\end{dmath}
%
\index{coordinate transformation!cylindrical}
The coordinate transformation is
\begin{equation}\label{eqn:emtProblemSet2Problem3:80}
\begin{aligned}
x &= \rho \cos\phi \\
y &= \rho \sin\phi,
\end{aligned}
\end{equation}
where the unit vectors are
%\footnote{The exponential \( \exp\lr{\xcap \ycap \phi} = \cos\phi + \xcap \ycap \sin\phi \) is geometric algebra notation for a (non-commutative) multivector rotation operator int the x-y plane.  The bivector \( \xcap \ycap \) squares to -1, allowing for complex number like rotations of vectors.}
\begin{equation}\label{eqn:emtProblemSet2Problem3:100}
\begin{aligned}
\rhocap &= 
%\xcap e^{\xcap \ycap \phi} = 
\xcap \cos \phi + \ycap \sin\phi \\
\phicap &= 
%\ycap e^{\xcap \ycap \phi} = 
\ycap \cos \phi - \xcap \sin\phi,
\end{aligned}
\end{equation}
so
\begin{dmath}\label{eqn:emtProblemSet2Problem3:120}
\begin{bmatrix}
\rhocap \cdot \spacegrad \psi \\
\phicap \cdot \spacegrad \psi \\
\zcap \cdot \spacegrad \psi \\
\end{bmatrix}
=
\begin{bmatrix}
\cos\phi & \sin\phi & 0 \\
- \sin\phi & \cos\phi & 0 \\
0 & 0 & 1
\end{bmatrix}
\begin{bmatrix}
\PDi{x}{\psi} \\
\PDi{y}{\psi} \\
\PDi{z}{\psi}
\end{bmatrix}
\end{dmath}
The \( x, y \) partials on the RHS can be expanded by chain rule
\begin{equation}\label{eqn:emtProblemSet2Problem3:140}
\begin{aligned}
\PD{x}{\psi} &= \PD{\rho}{\psi}\PD{x}{\rho} + \PD{\phi}{\psi} \PD{x}{\phi} \\
\PD{y}{\psi} &= \PD{\rho}{\psi}\PD{y}{\rho} + \PD{\phi}{\psi} \PD{y}{\phi},
\end{aligned}
\end{equation}
or, in matrix form
\begin{dmath}\label{eqn:emtProblemSet2Problem3:160}
\begin{bmatrix}
\PDi{x}{\psi} \\
\PDi{y}{\psi} \\
\PDi{z}{\psi}
\end{bmatrix}
=
\begin{bmatrix}
\PDi{x}{\rho} & \PDi{x}{\phi} & 0 \\
\PDi{y}{\rho} & \PDi{y}{\phi} & 0 \\
0 & 0 & 1
\end{bmatrix}
\begin{bmatrix}
\PDi{\rho}{\psi} \\
\PDi{\phi}{\psi} \\
\PDi{z}{\psi}
\end{bmatrix}.
\end{dmath}
The partials of \( \rho \) can be computed from \( \rho^2 = x^2 + y^2 \), and are
\begin{equation}\label{eqn:emtProblemSet2Problem3:180}
\begin{aligned}
2 \rho \PD{x}{\rho} &= 2 x \\
2 \rho \PD{y}{\rho} &= 2 y,
\end{aligned}
\end{equation}
or
\begin{equation}\label{eqn:emtProblemSet2Problem3:200}
\begin{aligned}
\PD{x}{\rho} &= \cos\phi \\
\PD{y}{\rho} &= \sin\phi.
\end{aligned}
\end{equation}
We can compute the partials with respect to \( \phi \) from
\begin{equation}\label{eqn:emtProblemSet2Problem3:300}
\begin{aligned}
\cos\phi &= \frac{x}{\sqrt{x^2 + y^2}} \\
\sin\phi &= \frac{y}{\sqrt{x^2 + y^2}}.
\end{aligned}
\end{equation}
For example
\begin{dmath}\label{eqn:emtProblemSet2Problem3:220}
-\sin \phi \PD{x}{\phi}
= \inv{\sqrt{x^2 + y^2}} + \frac{x (-1/2) 2 x}{\lr{x^2 + y^2}^{3/2}}
= \frac{x^2 + y^2 - x^2}{\lr{x^2 + y^2}^{3/2}}
= \frac{y^2}{\lr{x^2 + y^2}^{3/2}}.
= \frac{y^2}{\rho^3}
= \frac{\sin^2\phi}{\rho}.
\end{dmath}
Similarly
\begin{dmath}\label{eqn:emtProblemSet2Problem3:240}
\cos \phi \PD{y}{\phi}
= \frac{x^2}{\rho^3}
= \frac{\cos^2\phi}{\rho}.
\end{dmath}
In terms of \( \rho, \phi \), these are
\begin{equation}\label{eqn:emtProblemSet2Problem3:260}
\begin{aligned}
\PD{x}{\phi} &= -\frac{\sin\phi}{\rho} \\
\PD{y}{\phi} &= \frac{\cos\phi}{\rho},
\end{aligned}
\end{equation}
so
\begin{dmath}\label{eqn:emtProblemSet2Problem3:280}
\begin{bmatrix}
\PDi{x}{\psi} \\
\PDi{y}{\psi} \\
\PDi{z}{\psi}
\end{bmatrix}
=
\begin{bmatrix}
\cos\phi & -\sin\phi/\rho & 0 \\
\sin\phi & \cos\phi/\rho & 0 \\
0 & 0 & 1
\end{bmatrix}
\begin{bmatrix}
\PDi{\rho}{\psi} \\
\PDi{\phi}{\psi} \\
\PDi{z}{\psi}
\end{bmatrix}.
\end{dmath}
Substitution back into \cref{eqn:emtProblemSet2Problem3:120} gives
\begin{dmath}\label{eqn:emtProblemSet2Problem3:320}
\begin{bmatrix}
\rhocap \cdot \spacegrad \psi \\
\phicap \cdot \spacegrad \psi \\
\zcap \cdot \spacegrad \psi \\
\end{bmatrix}
=
\begin{bmatrix}
\cos\phi & \sin\phi & 0 \\
- \sin\phi & \cos\phi & 0 \\
0 & 0 & 1
\end{bmatrix}
\begin{bmatrix}
\cos\phi & -\sin\phi/\rho & 0 \\
\sin\phi & \cos\phi/\rho & 0 \\
0 & 0 & 1
\end{bmatrix}
\begin{bmatrix}
\PDi{\rho}{\psi} \\
\PDi{\phi}{\psi} \\
\PDi{z}{\psi}
\end{bmatrix}
=
\begin{bmatrix}
1 & 0 & 0 \\
0 & \inv{\rho} & 0 \\
0 & 0 & 1
\end{bmatrix}
\begin{bmatrix}
\PDi{\rho}{\psi} \\
\PDi{\phi}{\psi} \\
\PDi{z}{\psi}
\end{bmatrix}
=
\begin{bmatrix}
\PDi{\rho}{\psi} \\
\inv{\rho} \PDi{\phi}{\psi} \\
\PDi{z}{\psi}
\end{bmatrix}.
\end{dmath}
%
Finally, the gradient can be reassembled in terms of its projections onto each of the basis vectors in the cylindrical coordinate system
\begin{dmath}\label{eqn:emtProblemSet2Problem3:340}
\spacegrad \psi
=
\rhocap (\rhocap \cdot \spacegrad \psi)
+\phicap (\phicap \cdot \spacegrad \psi)
+\zcap (\zcap \cdot \spacegrad \psi),
\end{dmath}
or
%\begin{dmath}\label{eqn:emtProblemSet2Problem3:360}
\boxedEquation{eqn:emtProblemSet2Problem3:380}{
\spacegrad \psi
=
\rhocap \PD{\rho}{\psi}
+\frac{\phicap }{\rho}\PD{\phi}{\psi}
+\zcap \PD{z}{\psi}.
}
%\end{dmath}
}}

      %
% Copyright � 2016 Peeter Joot.  All Rights Reserved.
% Licenced as described in the file LICENSE under the root directory of this GIT repository.
%
\makeproblem{Point charge.}{emt:problemSet2:5}{
\index{point charge}
\index{Maxwell's equations!point charge}
\makesubproblem{}{emt:problemSet2:5a}
Consider a point charge \( q \). Using Maxwell equations, derive an expression for the
electric field \( \BE \)
generated by \( q \) at the distance \( \Br \) from it.  Clearly express your
assumptions and justify them.
\makesubproblem{}{emt:problemSet2:5b}
Derive an expression for the force experience by the charge \( q' \) located at distance \( \Br \)
from the charge \( q \). (This is called Coulomb force)
\makesubproblem{}{emt:problemSet2:5c}
Derive an expression for the electrostatic potential \( V \) at the distance \( \Br \) from the
charge \( q \) with respect to the electrostatic potential at infinity. For convenience, set the
value of electrostatic potential at infinity to zero.
} % makeproblem
\skipIfRedacted{
\makeanswer{emt:problemSet2:5}{
\makeSubAnswer{}{emt:problemSet2:5a}
\index{statics}
Maxwell's equations in linear media, for a static configuration (no time derivatives), are
\begin{equation}\label{eqn:emtProblemSet2Problem5:20}
\begin{aligned}
\spacegrad \cdot \BE &= \frac{\rho}{\epsilon}, \\
\spacegrad \cross \BE &= 0, \\
\spacegrad \cdot \BB &= 0, \\
\spacegrad \cross \BB &= \mu \BJ.
\end{aligned}
\end{equation}

Let's assume that the ``point'' charge in question is uniformly and symmetrically distributed in a spherical configuration clustered around its position.  For convenience assume that this charge is situated at the origin.  Integrating Gauss's law over a spherical volume, of radius \( R \), that completely encloses the charge we have
\begin{dmath}\label{eqn:emtProblemSet2Problem5:40}
\int \spacegrad' \cdot \BE(\Br') dV' = \inv{\epsilon} \int \rho(\Br') dV' = \frac{q}{\epsilon}.
\end{dmath}
The divergence integral can be evaluated with the divergence theorem, giving
\begin{dmath}\label{eqn:emtProblemSet2Problem5:60}
\frac{q}{\epsilon}
= \oint_{\Abs{\Br'} = R} \ncap \cdot \BE(\Br') dV'
= E_n(R) 4 \pi R^2,
\end{dmath}
or
\begin{dmath}\label{eqn:emtProblemSet2Problem5:80}
E_n(R) = \inv{4 \pi \epsilon} \frac{q}{R^2}.
\end{dmath}
Here \( E_n \) is the normal component of the electric field, so the total electric field at any point \( \Br \) away from
the region where the charge is located is
\begin{dmath}\label{eqn:emtProblemSet2Problem5:100}
\BE(\Br)
= \inv{4 \pi \epsilon} \frac{\rcap q}{\Abs{\Br}^2}
+ \phicap E_\phi(\Br)
+ \thetacap E_\theta(\Br).
\end{dmath}

Consider integration of the electric field on curve that is restricted to the surface fixed at \( \Abs{\Br} = R \).  Integrating on a closed curve we have
\begin{equation}\label{eqn:emtProblemSet2Problem5:120}
\oint \BE \cdot d\Bl
= \int \lr{ \spacegrad \cross \BE} \cdot d\Ba
= 0.
\end{equation}
This means that any field lines on the surface must close on themselves.  A couple such curves are sketched in \cref{fig:closedCurvesOnSphere:closedCurvesOnSphereFig1}.
\imageFigure{../figures/ece1228-electromagnetic-theory/closedCurvesOnSphereFig1}{Closed curves on sphere.}{fig:closedCurvesOnSphere:closedCurvesOnSphereFig1}{0.2}

There are no such closed curves that are rotationally invariant, whereas it was assumed that the ``charge distribution'' of the point charge was uniform and spherically symmetric (i.e. rotationally invariant).  In order to resolve this contradiction it must be assumed that all the tangential components of the electric field on the surface of the sphere containing the charge are zero.  This means that the field is strictly radial
%\begin{dmath}\label{eqn:emtProblemSet2Problem5:260}
\boxedEquation{eqn:emtProblemSet2Problem5:280}{
\BE(\Br)
= \frac{q}{4 \pi \epsilon} \frac{\rcap}{\Abs{\Br}^2}.
}
%\end{dmath}
That said, if one assumes that the point charges of interest are protons, nuclei, or electrons, the assumptions made in this problem are not entirely physically reasonable.
From a quantum mechanical perspective, modeling an electron, or any similarly localized charged particle as a point particle at a fixed point in space is incompatible with the Heisenberg uncertainty principle, since a fixed point charge location imbues infinite momentum.  Also, we
know that electrons have an orientation (i.e. spin), observable (indirectly, using ions) using a Stern-Gerlach apparatus.  That spin is not modeled by Maxwell's equations, and invalidates the idea of modeling the electron as a uniform symmetric entity.
Interestingly, experimental apparatus for measuring free electron spin, a more difficult experiment than measuring ionic spin, is still being discussed \citep{garraway1999observing}.
\makeSubAnswer{}{emt:problemSet2:5b}
The force on a charge is given by that charge times the electric field at that point, so the force on a test charge \( q' \) situated at \( \Br \), with the point charge \( q \) at the origin is just
\begin{dmath}\label{eqn:emtProblemSet2Problem5:240}
\BF(\Br)
= q' \BE(\Br)
= \inv{4 \pi \epsilon} \frac{q q' \rcap}{\Abs{\Br}^2}.
\end{dmath}
\makeSubAnswer{}{emt:problemSet2:5c}
The potential implicitly defined by \( \BE = - \spacegrad V \) can be recovered by integrating over a curve connected by endpoints \( \Ba, \Bb \)
\begin{dmath}\label{eqn:emtProblemSet2Problem5:140}
\int_\Ba^\Bb \BE \cdot d\Bl
=
-\int_\Ba^\Bb \spacegrad V \cdot d\Bl
=
-\lr{ V(\Bb) - V(\Ba) },
\end{dmath}
or
\begin{dmath}\label{eqn:emtProblemSet2Problem5:160}
V(\Br) = V(\Br_0) - \int_{\Br_0}^\Br \BE \cdot d\Bl,
\end{dmath}
where \( V(\Br_0) \) is the value of the potential at a reference point \( \Br_0 \).  For the electrostatic field, this integral can be performed over a radial path \( \Bl \propto \rcap \) from infinity to \( r\rcap \), which is
\begin{dmath}\label{eqn:emtProblemSet2Problem5:180}
V(\Br)
= V(\infty) - \int_{\infty}^r E_n dr
= -\int_{\infty}^r \frac{q}{4 \pi \epsilon r^2} dr
= \frac{q}{4 \pi\epsilon} \evalrange{\lr{\inv{r}}}{\infty}{r},
\end{dmath}
or
%\begin{dmath}\label{eqn:emtProblemSet2Problem5:200}
\boxedEquation{eqn:emtProblemSet2Problem5:220}{
V(\Br)
= \frac{q}{4 \pi\epsilon r}.
}
%\end{dmath}
}}

   %
% Copyright � 2016 Peeter Joot.  All Rights Reserved.
% Licenced as described in the file LICENSE under the root directory of this GIT repository.
%
%\input{../blogpost.tex}
%\renewcommand{\basename}{emt3}
%\renewcommand{\dirname}{notes/ece1228/}
%\newcommand{\keywords}{ECE1228H}
%\input{../latex/peeter_prologue_print2.tex}
%
%%\usepackage{ece1228}
%\usepackage{peeters_braket}
%%\usepackage{peeters_layout_exercise}
%\usepackage{peeters_figures}
%\usepackage{mathtools}
%\usepackage{siunitx}
%
%\beginArtNoToc
%\generatetitle{ECE1228H Electromagnetic Theory.  Lecture 3: Electrostatics and dipoles.  Taught by Prof.\ M. Mojahedi}
%\mychapter{Electrostatics and dipoles.}
\index{electrostatics}
\label{chap:emt3}
%
%\section{Disclaimer}
%
%Peeter's lecture notes from class.  These may be incoherent and rough.
%
%These are notes for the UofT course ECE1228H, Electromagnetic Theory, taught by Prof. M. Mojahedi.
%%, covering \textchapref{{1}} \citep{balanis1989advanced} content.
%
\index{polarization}
\index{magnetization}
\section{Polarization and Magnetization.}
%
The importance of the polarization and magnetization given by
%
\begin{equation}\label{eqn:emtLecture3:20}
\begin{aligned}
\BD &= \epsilon_0 \BE + \BP \\
\BP &= \epsilon_0 \chi_\txte \BE,
\end{aligned}
\end{equation}
%
where
\begin{equation}\label{eqn:emtLecture3:40}
\begin{aligned}
\BD &= \epsilon \BE \\
\epsilon &= \epsilon_0 \epsilon_r \\
\epsilon_r &= 1 + \chi_\txte.
\end{aligned}
\end{equation}
%
\section{Point charge.}
\index{point charge}
%
\begin{dmath}\label{eqn:emtLecture3:60}
\BE
= \frac{q}{4 \pi \epsilon_0} \frac{\rcap}{\Br^2}
= \frac{q}{4 \pi \epsilon_0} \frac{\Br}{\Abs{\Br}^3}
= \frac{q}{4 \pi \epsilon_0} \frac{\Br}{r^3}.
\end{dmath}
%
In more complex media the \( \epsilon_0 \) here can be replaced by \( \epsilon \).
Here the vector \( \Br \) points from the charge to the observation point.
%
Note that the class notes use \( \hat{a}_R \) instead of \( \rcap \).
%
When the charge isn't located at the origin, we must modify this accordingly
%
\begin{dmath}\label{eqn:emtLecture3:80}
\BE
= \frac{q}{4 \pi \epsilon_0} \frac{\BR}{\Abs{\BR}^3}
= \frac{q}{4 \pi \epsilon_0} \frac{\BR}{R^3},
\end{dmath}
%
\index{electric field!direction vector}
where \( \BR = \Br - \Br' \) still points from the location of the charge to the point of observation, as sketched in \cref{fig:VectorFromChargeToObservationL3:VectorFromChargeToObservationL3Fig1}.
%
\imageFigure{../figures/ece1228-electromagnetic-theory/VectorFromChargeToObservationL3Fig1}{Vector distance from charge to observation point.}{fig:VectorFromChargeToObservationL3:VectorFromChargeToObservationL3Fig1}{0.2}
%
\index{superposition}
This can be further generalized to collections of point charges by superposition
%
\begin{dmath}\label{eqn:emtLecture3:100}
\BE
= \frac{1}{4 \pi \epsilon_0} \sum_i q_i \frac{\Br - \Br_i'}{\Abs{\Br - \Br_i'}^3}.
\end{dmath}
%
\index{gradient}
Observe that a potential that satisfies \( \BE = - \spacegrad V \) can be defined as
%
\begin{dmath}\label{eqn:emtLecture3:120}
V
= \frac{1}{4 \pi \epsilon_0} \sum_i \frac{q_i}{\Abs{\Br - \Br_i'}}.
\end{dmath}
%
When we are considering real world scenarios (like touching your hair, and then the table), how do we deal with the billions of charges involved.  This can be done by considering the charges so small that they can be approximated as a continuous distribution of charges.
%
This can be done by introducing the concept of a continuous charge distribution \( \rho_\txtv(\Br') \).
The charge that is in a small differential volume element \( dV' \) is \( \rho(\Br') dV' \), and
the superposition has the form
%
\begin{dmath}\label{eqn:emtLecture3:140}
\BE
= \frac{1}{4 \pi \epsilon_0} \iiint dV' \rho_\txtv(\Br') \frac{\Br - \Br'}{\Abs{\Br - \Br'}^3},
\end{dmath}
%
with potential
%
\begin{dmath}\label{eqn:emtLecture3:160}
V
= \frac{1}{4 \pi \epsilon_0} \iiint dV' \frac{\rho_\txtv(\Br')}{\Abs{\Br - \Br'}}.
\end{dmath}
%
\index{surface charge density}
The surface charge density analogue of this is
%
\begin{dmath}\label{eqn:emtLecture3:180}
\BE
= \frac{1}{4 \pi \epsilon_0} \iint dA' \rho_\txts(\Br') \frac{\Br - \Br'}{\Abs{\Br - \Br'}^3},
\end{dmath}
%
with potential
%
\index{potential!electric}
\begin{dmath}\label{eqn:emtLecture3:200}
V
= \frac{1}{4 \pi \epsilon_0} \iint dA' \frac{\rho_\txts(\Br')}{\Abs{\Br - \Br'}}.
\end{dmath}
%
The line charge density analogue of this is
%
\begin{dmath}\label{eqn:emtLecture3:220}
\BE
= \frac{1}{4 \pi \epsilon_0} \int dl' \rho_\txtl(\Br') \frac{\Br - \Br'}{\Abs{\Br - \Br'}^3},
\end{dmath}
%
with potential
%
\begin{dmath}\label{eqn:emtLecture3:240}
V
= \frac{1}{4 \pi \epsilon_0} \int dl' \frac{\rho_\txtl(\Br')}{\Abs{\Br - \Br'}}.
\end{dmath}
%
The difficulty with any of these approaches is the charge density is hardly ever known.  When the charge density is known, this sorts of integrals may not be analytically calculable, but they do yield to numeric calculation.
%
We may often prefer the potential calculations of the field calculations because they are much easier, having just one component to deal with.
%
\section{Electric field of a dipole.}
\index{dipole!electric}
%
An equal charge dipole configuration is sketched in \cref{fig:dipoleSignConventionL3:dipoleSignConventionL3Fig3}.
%
\imageFigure{../figures/ece1228-electromagnetic-theory/dipoleSignConventionL3Fig3}{Dipole sign convention.}{fig:dipoleSignConventionL3:dipoleSignConventionL3Fig3}{0.2}
%
\begin{equation}\label{eqn:emtLecture3:280}
\begin{aligned}
\Br_1 &= \Br - \frac{\Bd}{2}, \\
\Br_2 &= \Br + \frac{\Bd}{2}.
\end{aligned}
\end{equation}
%
The electric field is
\begin{dmath}\label{eqn:emtLecture3:300}
\BE
= \frac{q}{4 \pi \epsilon_0} \lr{
\frac{\Br_1}{r_1^3} - \frac{\Br_2}{r_2^3} }
=
\frac{q}{4 \pi \epsilon_0} \lr{
\frac{\Br - \Bd/2}{\Abs{\Br - \Bd/2}^3} - \frac{\Br + \Bd/2}{\Abs{\Br + \Bd/2}^3} }.
\end{dmath}
%
For \( r \gg \Abs{\Bd} \), this can be reduced using the normal first order reduction techniques, left to an exercise.
% (homework question).
%In that homework, we are to perform this calculation in terms of vector expressions, and not coordinates.
%\frac{\Br - \Bd/2}{\Abs{\Br - \Bd/2}^3} - \frac{\Br + \Bd/2}{\Abs{\Br + \Bd/2}^3} }
%
This is essentially requires an expansion of
%
\begin{dmath}\label{eqn:emtLecture3:320}
\Abs{\Br \pm \Bd/2}^{-3/2} = \lr{
(\Br \pm \Bd/2) \cdot (\Br \pm \Bd/2) }^{-3/2}.
\end{dmath}
%
The final result with \( \Bp = q \Bd \) (the dipole moment) can be found to be
%
\begin{dmath}\label{eqn:emtLecture3:340}
\BE
= \frac{1}{4 \pi \epsilon_0 r^3} \lr{ 3 \frac{\Br \cdot \Bp }{r^2} \Br - \Bp }
\end{dmath}
%
With \( \Bp = q \zcap \), we have spherical coordinates for the observation point, and Cartesian for the dipole moment.  To convert the moment to spherical we can use
%
\index{spherical coordinates!rotation matrix}
\begin{dmath}\label{eqn:emtLecture3:360}
\begin{bmatrix}
A_r \\
A_\theta \\
A_\phi
\end{bmatrix}
=
\begin{bmatrix}
\sin\theta \cos\phi & \sin\theta\sin\phi & \cos\theta \\
\cos\theta \cos\phi & \cos\theta\sin\phi & -\sin\theta \\
-\sin\phi & \cos\phi & 0
\end{bmatrix}
\begin{bmatrix}
A_x \\
A_y \\
A_z
\end{bmatrix}.
\end{dmath}
%
All such rotation matrices can be found in the appendix of \citep{balanis1989advanced} for example.  For the dipole vector this gives
%
\begin{dmath}\label{eqn:emtLecture3:380}
\begin{bmatrix}
p_r \\
p_\theta \\
p_\phi
\end{bmatrix}
=
\begin{bmatrix}
\cos\theta p \\
-\sin\theta p \\
0
\end{bmatrix}.
\end{dmath}
%
or
\begin{equation}\label{eqn:emtLecture3:400}
\Bp = p \zcap = p \lr{ \cos\theta \rcap - \sin\theta \thetacap }.
\end{equation}
%
Plugging in this eventually gives
\begin{dmath}\label{eqn:emtLecture3:420}
\BE = \frac{p}{4 \pi \epsilon_0 r^3} \lr{ 2 \cos\theta \rcap + \sin\theta \thetacap },
\end{dmath}
%
where \( \Abs{\Br} = r \).
%
It will be left to a problem to show that
the potential for an electric dipole is given by
%
\begin{dmath}\label{eqn:emtLecture3:440}
V = \frac{\Bp \cdot \rcap}{4 \pi \epsilon_0 r^2 }.
\end{dmath}
%
Observe that the dipole field drops off faster than the field for a single electric charge.  This is true generally, with quadrupole and higher order moments dropping off faster as the degree is increased.
%
\section{Bound (polarized) surface and volume charge densities.}
\index{bound charge}
\index{capacitor}
%
When an electric field is applied to a volume, bound charges are induced on the surface of the material, and bound charges induced in the volume.  Both of these are related to the polarization \( \BP \), and the displacement current in the material, in a configuration such as the capacitor sketched in \cref{fig:displacementCurrentL3:displacementCurrentL3Fig5}.
%
\imageFigure{../figures/ece1228-electromagnetic-theory/displacementCurrentL3Fig5}{Circuit with displacement current.}{fig:displacementCurrentL3:displacementCurrentL3Fig5}{0.2}
%
Consider, for example, a capacitor using glass as a dielectric.  The charges are not able to move within the insulating material, but dipole configurations can be induced on the surface and in the bulk of the material, as sketched in \cref{fig:glassCapacitorInsulatorBoundChargesL3:glassCapacitorInsulatorBoundChargesL3Fig6}.
%
\index{dielectric}
\imageFigure{../figures/ece1228-electromagnetic-theory/glassCapacitorInsulatorBoundChargesL3Fig6}{Glass dielectric capacitor bound charge dipole configurations.}{fig:glassCapacitorInsulatorBoundChargesL3:glassCapacitorInsulatorBoundChargesL3Fig6}{0.2}
%
\index{polarization}
How many materials behave is largely determined by electric dipole effects.  In particular, the polarization \( \BP \) can be considered the density of electric dipoles.
%
\begin{dmath}\label{eqn:emtLecture3:260}
\BP = \lim_{\Delta v' \rightarrow 0} \sum_k^{N \Delta v'} \frac{\Bp_k}{\Delta v'},
\end{dmath}
%
\index{number density}
where \( N \) is the number density in the volume at that point, and \( \Delta v' \) is the differential volume element.
Dimensions:
%
\begin{itemize}
\item \([\Bp] = \si{C . m} \),
\item \([\BP] = \si{C / m^2} \).
\end{itemize}
%
In particular, when the electron cloud density of a material is not symmetric, as is the case in the p-orbital roughly sketched in \cref{fig:pOrbitalSketchL3:pOrbitalSketchL3Fig4}, then we have a dipole configuration in each atom.  When the atom is symmetric, by applying an electric field, a dipole configuration can be created.
%
\imageFigure{../figures/ece1228-electromagnetic-theory/pOrbitalSketchL3Fig4}{A p-orbital dipole like electronic configuration.}{fig:pOrbitalSketchL3:pOrbitalSketchL3Fig4}{0.2}
%
As the volume shrinks to zero, the dipole moment can be expressed as
%
\begin{dmath}\label{eqn:emtLecture3:460}
\BP = \frac{d\Bp}{dv}.
\end{dmath}
%
\index{dipole!elemental}
For an elemental dipole \( d\Bp = \BP dv' \), the contribution to the potential is
%
\begin{dmath}\label{eqn:emtLecture3:480}
dV
= \frac{d \Bp \cdot \rcap}{4 \pi \epsilon_0 R^2}
= \frac{\BP \cdot \rcap}{4 \pi \epsilon_0 R^2} dv'
\end{dmath}
%
Since
%
\begin{dmath}\label{eqn:emtLecture3:500}
\spacegrad' \inv{R} = \frac{\rcap}{R^2},
\end{dmath}
%
this can be written as
%
\begin{dmath}\label{eqn:emtLecture3:520}
V
=
\inv{4 \pi \epsilon_0 }
\int_{v'} dv' \BP \cdot \spacegrad' \inv{R}
=
\inv{4 \pi \epsilon_0 }
\int_{v'} dv' \spacegrad' \cdot \frac{\BP}{R}
-
\inv{4 \pi \epsilon_0 }
\int_{v'} dv' \frac{\spacegrad' \cdot \BP}{R}
=
\inv{4 \pi \epsilon_0 }
\lr{
\oint_{S'} ds' \ncap \cdot \frac{\BP}{R}
-
\int_{v'} dv' \frac{\spacegrad' \cdot \BP}{R}
}.
\end{dmath}
%
Looking back to the potentials in their volume density \cref{eqn:emtLecture3:160}
and surface charge density \cref{eqn:emtLecture3:180}
forms, we see that
identifications can be made with the volume and surface charge densities
%
\begin{equation}\label{eqn:emtLecture3:620}
\begin{aligned}
\rho_\txts' &= \BP \cdot \ncap, \\
\rho_\txtv' &= \spacegrad' \cdot \BP.
\end{aligned}
\end{equation}
%
Dropping primes, these are respectively
%
\begin{itemize}
\item Bound or polarized surface charge density: \( \rho_{sP} = \BP \cdot \ncap \), in [\si{C/m^2}]
\item Bound or polarized volume charge density: \( \rho_{vP} = \spacegrad \cdot \BP \), in [\si{C/m^3}]
\end{itemize}
%
Recall that in Maxwell's equations for the vacuum we have
%
\begin{dmath}\label{eqn:emtLecture3:640}
\spacegrad \cdot \BE = \frac{\rho_\txtv}{\epsilon_0}.
\end{dmath}
%
Here \( \rho_\txtv \) represents ``free'' charge density.  Adding in potential bound charges we have
\begin{dmath}\label{eqn:emtLecture3:660}
\spacegrad \cdot \BE =
\frac{\rho_\txtv}{\epsilon_0}
+
\frac{\rho_{\txtv\txtP}}{\epsilon_0}
=
\frac{\rho_\txtv}{\epsilon_0}
-
\frac{\spacegrad \cdot \BP}{\epsilon_0}.
\end{dmath}
%
Rearranging we can write
\begin{dmath}\label{eqn:emtLecture3:680}
\spacegrad \cdot \lr{ \epsilon_0 \BE + \BP } = \rho_\txtv.
\end{dmath}
%
\index{Gauss's law!matter}
This finally justifies the Maxwell equation
\begin{dmath}\label{eqn:emtLecture3:700}
\spacegrad \cdot \BD = \rho_\txtv,
\end{dmath}
%
where \( \BD = \epsilon_0 \BE + \BP \).
%
Assuming a relationship between the polarization vector and the electric field of the form
%
\begin{dmath}\label{eqn:emtLecture3:720}
\BP = \epsilon_0 \chi_e \BE,
\end{dmath}
%
possibly a tensor relationship.  The bound charges in the material are seen to related the displacement current and the electric field
%
\begin{dmath}\label{eqn:emtLecture3:740}
\BD
= \epsilon_0 \BE + \BP
= \epsilon_0 \BE + \epsilon_0 \chi_e \BE,
= \epsilon_0 \lr{ 1 + \chi_e } \BE,
= \epsilon_0 \epsilon_r \BE,
= \epsilon \BE.
\end{dmath}

Question: Think about why do we ignore the surface charges here?   Answer: we are not considering boundaries... they are at infinity.
%
%\EndArticle

   \section{Problems.}
      %
% Copyright � 2016 Peeter Joot.  All Rights Reserved.
% Licenced as described in the file LICENSE under the root directory of this GIT repository.
%
\makeproblem{Electric Dipole.}{emt:problemSet2:1}{
\index{dipole!electric}
%
An electric dipole is shown in \cref{fig:problemset2:problemset2Fig1}.
%
\imageFigure{../figures/ece1228-electromagnetic-theory/problemset2Fig1}{Electric dipole configuration.}{fig:problemset2:problemset2Fig1}{0.3}
%
\makesubproblem{}{emt:problemSet2:1b}
\index{potential!electric dipole}
Find the Potential \( V \) at an arbitrary point \( \BA \).
%
\makesubproblem{}{emt:problemSet2:1a}
Calculate the field \( \BE \) from the above potential.
%
(show that it is the same result we obtained in the class).
} % makeproblem
%
\skipIfRedacted{
\makeanswer{emt:problemSet2:1}{
%
\makeSubAnswer{}{emt:problemSet2:1a}
%
Following \cref{fig:dipoleSignConventionL3:dipoleSignConventionL3Fig3}, the vector from the origin to the observation point is
%
%\imageFigure{../figures/ece1228-electromagnetic-theory/dipoleSignConventionL3Fig3}{Dipole sign convention.}{fig:dipoleSignConventionL3:dipoleSignConventionL3Fig3}{0.3}
%
\begin{equation}\label{eqn:emtProblemSet2Problem1:20}
\Br = \BR_1 + \Bd/2
= \BR_2 - \Bd/2,
\end{equation}
%
or
%
\begin{equation}\label{eqn:emtProblemSet2Problem1:40}
\begin{aligned}
\BR_1 &= \Br - \Bd/2 \equiv \BR_{+} \\
\BR_2 &= \Br + \Bd/2 \equiv \BR_{-}.
\end{aligned}
\end{equation}
%
The potential for this superposition is
\begin{equation}\label{eqn:emtProblemSet2Problem1:60}
\begin{aligned}
V
&=
\inv{4 \pi \epsilon_0} \lr{
\frac{q}{\Abs{\BR_{+}}} -
\frac{q}{\Abs{\BR_{-}}}
}
\\ &=
\frac{q}{4 \pi \epsilon_0} \lr{
\frac{1}{\Abs{\BR_{+}}} -
\frac{1}{\Abs{\BR_{-}}}
}.
\end{aligned}
\end{equation}
%
The magnitudes can be expanded in Taylor series
%
\begin{equation}\label{eqn:emtProblemSet2Problem1:80}
\begin{aligned}
\Abs{\BR_{\pm}}^{-1}
&=
\lr{
\lr{ \Br \mp \Bd/2 } \cdot \lr{ \Br \mp \Bd/2 }
}^{-1/2}
\\ &=
\lr{
\lr{ \Br^2 + (\Bd/2)^2 \mp 2 \Br \cdot \Bd/2 }
}^{-1/2}
\\ &=
\lr{
\lr{ \Br^2 + (\Bd/2)^2 \mp \Br \cdot \Bd }
}^{-1/2}
\\ &=
(\Br^2)^{-1/2}
\lr{
\lr{ 1 + \lr{\frac{\Bd}{2 r}}^2 \mp \rcap \cdot \frac{\Bd}{r} }
}^{-1/2}
\\ &=
r^{-1}
\lr{
1
-\frac{1}{2}
\lr{ \lr{\frac{\Bd}{2 r}}^2 \mp \rcap \cdot \frac{\Bd}{r} }
+\lr{\frac{-1}{2}}
\lr{\frac{-3}{2}} \inv{2!}
\lr{ \lr{\frac{\Bd}{2 r}}^2 \mp \rcap \cdot \frac{\Bd}{r} }^2
+ \cdots
}.
\end{aligned}
\end{equation}
%
Here \( r = \Abs{\Br} \), and the Taylor series was taken in the \( \Bd \ll r \) limit.  The sums and differences of these magnitudes, are to first order
%
\begin{equation}\label{eqn:emtProblemSet2Problem1:100}
\inv{\Abs{\BR_{+}}}
-
\inv{\Abs{\BR_{-}}}
\approx
2 \frac{1}{r} \lr{\frac{-1}{2}} \lr{-\rcap \cdot \frac{\Bd}{r}}
=
\frac{1}{r^2} \rcap \cdot \Bd,
\end{equation}
%
for
%\begin{equation}\label{eqn:emtProblemSet2Problem1:120}
\boxedEquation{eqn:emtProblemSet2Problem1:120}{
   V = \frac{\rcap \cdot \Bd}{4 \pi \epsilon_0 r^2 }.
}
%\end{equation}
%
\makeSubAnswer{}{emt:problemSet2:1b}
%
The electric field follows from \( \BE = -\spacegrad V \).  First note that
%
\begin{equation}\label{eqn:emtProblemSet2Problem1:140}
\begin{aligned}
\spacegrad \inv{r^n}
&=
\Be_k \partial_k (x_m x_m)^{-n/2}
\\ &=
-\frac{n}{2} \Be_k \frac{2 x_m \delta_{k m}}{r^{n+2}}
\\ &=
-n \frac{\rcap}{r^{n+1}}.
\end{aligned}
\end{equation}
%
Computing the gradient of the dot product, we find
\begin{equation}\label{eqn:emtProblemSet2Problem1:160}
\begin{aligned}
\lr{\spacegrad \frac{\rcap}{r^2} } \cdot \Bd
&=
\lr{ \spacegrad \frac{\Br}{r^3} } \cdot \Bd
\\ &=
\Be_k \partial_k \frac{x_m d_m}{r^3}
\\ &=
\Be_k \frac{\delta_{k m} d_m}{r^3}
+ \Br \cdot \Bd \spacegrad \inv{r^3}
\\ &=
\frac{\Bd}{r^3}
-3 \Br \cdot \Bd \frac{\rcap}{r^4}
\\ &=
\frac{\Bd - 3 (\rcap \cdot \Bd) \rcap}{r^3},
\end{aligned}
\end{equation}
%
so
%
\begin{equation}\label{eqn:emtProblemSet2Problem1:180}
V(\Br)
= \frac{q}{4 \pi \epsilon_0}
\frac{3 (\rcap \cdot \Bd) \rcap -\Bd}{r^3}.
\end{equation}
%
With \( \Bp = q \Bd \), this is the result found in class
%
%\begin{equation}\label{eqn:emtProblemSet2Problem1:200}
\boxedEquation{eqn:emtProblemSet2Problem1:200}{
V(\Br)
= \frac{1}{4 \pi \epsilon_0}
\frac{3 (\rcap \cdot \Bp) \rcap -\Bp}{r^3}.
}
%\end{equation}
}}

      %
% Copyright � 2016 Peeter Joot.  All Rights Reserved.
% Licenced as described in the file LICENSE under the root directory of this GIT repository.
%
%{
\makeproblem{Dipole moment density for disk.}{emt:problemSet2:4}{
\index{dipole moment density}
A dielectric circular disk of radius \( a \) and thickness \( d \) is permanently polarized with a
dipole moment per unit volume \( \BP \) [\si{C/m^2}], where \( \Abs{\BP} \)
is
constant and parallel to the disk axis (z-axis here) as
shown in \cref{fig:problemset2:problemset2Fig3}.
%
\imageFigure{../figures/ece1228-electromagnetic-theory/problemset2Fig3}{Circular disk geometry.}{fig:problemset2:problemset2Fig3}{0.3}
%
\makesubproblem{}{emt:problemSet2:4a}
Calculate the potential along the disk axis for \( z > 0 \).
\makesubproblem{}{emt:problemSet2:4b}
Approximate the result obtained in \partref{emt:problemSet2:4a} for the case
of \( Z \gg d \).
} % makeproblem
%
\skipIfRedacted{
\makeanswer{emt:problemSet2:4}{
\makeSubAnswer{}{emt:problemSet2:4a}
%
In class the potential for a discrete dipole pair was found to be
%
\begin{dmath}\label{eqn:emtProblemSet2Problem4:20}
V(\Br) = \inv{4 \pi \epsilon_0} \frac{\rcap \cdot \Bp}{r^2},
\end{dmath}
%
so in the continuum, the element of dipole moment is \( d\Bp = \BP(\Br') dV' \), for a total potential of
%
\begin{dmath}\label{eqn:emtProblemSet2Problem4:40}
V(\Br)
= \inv{4 \pi \epsilon_0} \int dV' \frac{\lr{ \Br - \Br'} \cdot \BP(\Br')}{\lr{\Br - \Br'}^3}.
\end{dmath}
%
Restricting \( \Br \) to the z-axis, with \( \Br(z) = z \zcap \), and using cylindrical coordinates for the disk
%
\begin{dmath}\label{eqn:emtProblemSet2Problem4:60}
\Br' = z'\zcap + \rho'\rhocap,
\end{dmath}
%
where \( z' \in [-d,0] \) and \( \rho' \in [0, a] \).  The difference vector between the charge and the observation point is
%
\begin{dmath}\label{eqn:emtProblemSet2Problem4:80}
\Br - \Br' = (z - z')\zcap - \rho' \rhocap,
\end{dmath}
%
with magnitude
\begin{dmath}\label{eqn:emtProblemSet2Problem4:100}
\Abs{\Br - \Br'}^2 = (z - z')^2 + (\rho')^2.
\end{dmath}
%
With \( \BP = \Abs{\BP} \zcap \), the potential along the axis is
%
\begin{dmath}\label{eqn:emtProblemSet2Problem4:120}
V(z)
=
\frac{2 \pi \Abs{\BP} }{4 \pi \epsilon_0}
\int_0^a \rho' d\rho \int_{-d}^0 dz' \frac{\lr{ (z - z')\zcap - \rho' \rhocap} \cdot \zcap}{\lr{(z - z')^2 + (\rho')^2}^{3/2}}
=
\frac{\Abs{\BP} }{2 \epsilon_0}
\int_0^a \rho' d\rho \int_{-d}^0 dz' \frac{ (z - z') }{\lr{(z - z')^2 + (\rho')^2}^{3/2}}
=
% u = z' - z
% v = \rho'
-\frac{\Abs{\BP} }{2 \epsilon_0}
\int_0^a dv \int_{-d-z}^{-z} du \frac{ u v }{\lr{u^2 + v^2}^{3/2}}
=
\frac{\Abs{\BP} }{2 \epsilon_0}
\int_{-d-z}^{-z} du \evalrange{\lr{\frac{ u }{\lr{u^2 + v^2}^{1/2}}}}{v =0}{a}
=
\frac{\Abs{\BP} }{2 \epsilon_0}
\int_{-d-z}^{-z} du \lr{ \frac{ u }{\sqrt{u^2 + a^2}} - \frac{u}{\Abs{u}} }
=
\frac{\Abs{\BP} }{2 \epsilon_0}
\int_{-d-z}^{-z} du \lr{ \frac{ u }{\sqrt{u^2 + a^2}} - \sgn{u}}.
\end{dmath}
%
Looking at the values of the integration variable \( u \), note that
for \( z > 0 \), \( u < 0 \) and for \( z < -d \), \( u > 0 \), so the potential outside of the disk is
%
\begin{dmath}\label{eqn:emtProblemSet2Problem4:140}
V(z)
=
\frac{\Abs{\BP} }{2 \epsilon_0} \lr{ d \sgn(z) + \int_{-d-z}^{-z} du \frac{ u }{\sqrt{u^2 + a^2}} }
=
\frac{\Abs{\BP} }{2 \epsilon_0} \lr{ d \sgn(z) +
\evalrange{\lr{\sqrt{u^2 + a^2}}}{-d-z}{-z}
},
\end{dmath}
%
or
%
%\begin{dmath}\label{eqn:emtProblemSet2Problem4:160}
\boxedEquation{eqn:emtProblemSet2Problem4:180}{
V(z)
=
\frac{\Abs{\BP} }{2 \epsilon_0} \lr{ d \sgn(z) +
\sqrt{ z^2 + a^2 } - \sqrt{ (d+z)^2 + a^2 }
}.
}
%\end{dmath}
%
\makeSubAnswer{}{emt:problemSet2:4b}
%
Let
%
\begin{dmath}\label{eqn:emtProblemSet2Problem4:200}
f(z) = \sqrt{ z^2 + a^2 },
\end{dmath}
%
which has a derivative of
%This has a derivative that is zero at the origin
%
\begin{dmath}\label{eqn:emtProblemSet2Problem4:220}
f'(z) = \frac{z}{\sqrt{z^2 + a^2}}.
\end{dmath}
%
%The second derivative is
%\begin{dmath}\label{eqn:emtProblemSet2Problem4:240}
%f''(z)
%= \frac{1}{\sqrt{z^2 + a^2}} + (-1/2) \frac{ z(2z) }{\lr{z^2 + a^2}^{3/2}}
%= \frac{ a^2 }{\lr{z^2 + a^2}^{3/2}}.
%\end{dmath}
%
The first order Taylor approximation is
%
\begin{dmath}\label{eqn:emtProblemSet2Problem4:260}
f(z + d) - f(z)
\approx f'(z) d
=
\frac{z d}{\sqrt{z^2 + a^2}},
\end{dmath}
%
so for \( \Abs{z} \gg d \) the potential is approximated by
%
\begin{dmath}\label{eqn:emtProblemSet2Problem4:280}
%\boxedEquation{eqn:emtProblemSet2Problem4:300}{
V(z)
\approx
\frac{\Abs{\BP} }{2 \epsilon_0} \lr{ d \sgn(z) - \frac{z d}{\sqrt{z^2 + a^2}} },
%}
\end{dmath}
%
or
%\begin{dmath}\label{eqn:emtProblemSet2Problem4:360}
\boxedEquation{eqn:emtProblemSet2Problem4:380}{
V(z)
=
\frac{\Abs{\BP} d \sgn(z)}{2 \epsilon_0} \lr{ 1 -
\frac{1}{\sqrt{1+ (a/z)^2}}
}.
}
%\end{dmath}
%
Note that if \( \Abs{z} \gg a \) too, this can be further approximated as
%
%\boxedEquation{eqn:emtProblemSet2Problem4:340}{
\begin{dmath}\label{eqn:emtProblemSet2Problem4:320}
V(z)
\approx
\frac{\Abs{\BP} d \sgn(z)}{2 \epsilon_0} \lr{ 1 - (1 + (-1/2) (a/z)^2) }
=
\inv{4 \pi \epsilon_0} \frac{\Abs{\BP} \sgn(z) (d \pi a^2)}{z^2}.
\end{dmath}
%}
%
This is a slightly tidier result that shows the asymptotic inverse-square \( z \) dependence of the potential more clearly than \cref{eqn:emtProblemSet2Problem4:380}, a result that assumed \( z \gg d \), but did not assume \( z \gg a \).  We also have the volume of the disk as an explicit factor in this approximation.
%, but requires \( z \\gg a,d \), a more strict limiting value than the \( z \gg d \) requested in the problem.
}}
%}

      %
% Copyright � 2016 Peeter Joot.  All Rights Reserved.
% Licenced as described in the file LICENSE under the root directory of this GIT repository.
%
%{
%\input{../blogpost.tex}
%\renewcommand{\basename}{dipoleMoment}
%%\renewcommand{\dirname}{notes/phy1520/}
%\renewcommand{\dirname}{notes/ece1228-electromagnetic-theory/}
%%\newcommand{\dateintitle}{}
%%\newcommand{\keywords}{}
%
%\input{../latex/peeter_prologue_print2.tex}
%
%\usepackage{peeters_layout_exercise}
%\usepackage{peeters_braket}
%\usepackage{peeters_figures}
%\usepackage{siunitx}
%
%\beginArtNoToc
%
%\generatetitle{Dipole moment}
%\chapter{Dipole moment}
%\label{chap:dipoleMoment}
%
\makeproblem{Field for an electric dipole.}{problem:dipoleMoment:dipole}{
\index{dipole!electric}
%
An equal charge dipole configuration is sketched in \cref{fig:dipoleSignConventionL3:dipoleSignConventionL3Fig3}.  Compute the electric field.
%
% L3:
%\imageFigure{../figures/ece1228-electromagnetic-theory/dipoleSignConventionL3Fig3}{Dipole sign convention.}{fig:dipoleSignConventionL3:dipoleSignConventionL3Fig3}{0.3}
} % problem
%
\makeanswer{problem:dipoleMoment:dipole}{\withproblemsetsParagraph{
The vector from the origin to the observation point is
%
\begin{equation}\label{eqn:dipoleMoment:20}
\Br = \BR_1 + \Bd/2
= \BR_2 - \Bd/2,
\end{equation}
%
or
%
\begin{equation}\label{eqn:dipoleMoment:40}
\begin{aligned}
\BR_1 &= \Br - \Bd/2 \equiv \BR_{+} \\
\BR_2 &= \Br + \Bd/2 \equiv \BR_{-}.
\end{aligned}
\end{equation}
%
The electric field for this superposition is
\begin{dmath}\label{eqn:dipoleMoment:60}
\BE
=
\inv{4 \pi \epsilon_0} \lr{
\frac{q \BR_{+}}{\Abs{\BR_{+}}^3} -
\frac{q \BR_{-}}{\Abs{\BR_{-}}^3}
}
=
\frac{q}{4 \pi \epsilon_0} \lr{
\frac{\Br - \Bd/2}{\Abs{\BR_{+}}^3} -
\frac{\Br + \Bd/2}{\Abs{\BR_{-}}^3}
}
=
\frac{q}{4 \pi \epsilon_0} \lr{
\Br \lr{
\inv{\Abs{\BR_{+}}^3}
 -
\inv{\Abs{\BR_{-}}^3}
}
-
\frac{\Bd}{2} \lr{
\inv{\Abs{\BR_{+}}^3}
+
\inv{\Abs{\BR_{-}}^3}
}
}.
\end{dmath}
%
The magnitudes can be expanded in Taylor series
%
\begin{dmath}\label{eqn:dipoleMoment:80}
\Abs{\BR_{\pm}}^{3}
=
\lr{
\lr{ \Br \mp \Bd/2 } \cdot \lr{ \Br \mp \Bd/2 }
}^{-3/2}
=
\lr{
\lr{ \Br^2 + (\Bd/2)^2 \mp 2 \Br \cdot \Bd/2 }
}^{-3/2}
=
\lr{
\lr{ \Br^2 + (\Bd/2)^2 \mp \Br \cdot \Bd }
}^{-3/2}
=
(\Br^2)^{-3/2}
\lr{
\lr{ 1 + \lr{\frac{\Bd}{2 r}}^2 \mp \rcap \cdot \frac{\Bd}{r} }
}^{-1/2}
=
r^{-3}
\lr{
1
-\frac{3}{2}
\lr{ \lr{\frac{\Bd}{2 r}}^2 \mp \rcap \cdot \frac{\Bd}{r} }
+\lr{\frac{-3}{2}}
\lr{\frac{-5}{2}} \inv{2!}
\lr{ \lr{\frac{\Bd}{2 r}}^2 \mp \rcap \cdot \frac{\Bd}{r} }^2
+ \cdots
}.
\end{dmath}
%
Here \( r = \Abs{\Br} \), and the Taylor series was taken in the \( \Bd/r \ll 1 \) limit.  The sums and differences of these magnitudes, are to first order
%
\begin{dmath}\label{eqn:dipoleMoment:100}
\inv{\Abs{\BR_{+}}^3}
-
\inv{\Abs{\BR_{-}}^3}
=
2 \frac{1}{r^3} \lr{\frac{-3}{2}} \lr{-\rcap \cdot \frac{\Bd}{r}}
\approx
\frac{3}{r^4} \rcap \cdot \Bd,
\end{dmath}
%
and
%
\begin{dmath}\label{eqn:dipoleMoment:120}
\inv{\Abs{\BR_{+}}^3}
+
\inv{\Abs{\BR_{-}}^3}
\approx
\frac{2}{r^3}.
\end{dmath}
%
The \( \Br \gg \Bd \) limiting expression for the electric field is
%
\begin{dmath}\label{eqn:dipoleMoment:140}
\BE
\approx
\frac{q}{4 \pi \epsilon_0 r^3} \lr{
3 \rcap \lr{ \rcap \cdot \Bd }
-
2 \frac{\Bd}{2}
},
\end{dmath}
%
or, with \( \Bp = q \Bd \)
%
%\begin{dmath}\label{eqn:dipoleMoment:180}
\boxedEquation{eqn:dipoleMoment:180}{
\BE =
\frac{1}{4 \pi \epsilon_0 r^3} \lr{
3 \rcap \lr{ \rcap \cdot \Bp }
-\Bp
}.
}
%\end{dmath}
}} % answer
%
%}
%\EndNoBibArticle

      %
% Copyright � 2016 Peeter Joot.  All Rights Reserved.
% Licenced as described in the file LICENSE under the root directory of this GIT repository.
%
%{
%\input{../blogpost.tex}
%\renewcommand{\basename}{dipolePotential}
%%\renewcommand{\dirname}{notes/phy1520/}
%\renewcommand{\dirname}{notes/ece1228-electromagnetic-theory/}
%%\newcommand{\dateintitle}{}
%%\newcommand{\keywords}{}
%
%\input{../latex/peeter_prologue_print2.tex}
%
%\usepackage{peeters_layout_exercise}
%\usepackage{peeters_braket}
%\usepackage{peeters_figures}
%\usepackage{siunitx}
%
%\beginArtNoToc
%
%\generatetitle{Electric dipole potential}
%\chapter{Electric dipole potential}
%\label{chap:dipolePotential}
%
\makeproblem{Electric dipole potential}{problem:dipolePotential:1}{
\index{potential!electric dipole}
\index{dipole!potential}
%
Having shown that
%
\begin{dmath}\label{eqn:dipolePotential:20}
\BE =
\frac{1}{4 \pi \epsilon_0 r^3} \lr{
3 \rcap \lr{ \rcap \cdot \Bp }
-\Bp
},
\end{dmath}
%
find the expression for the electric potential for this field.
} % problem
%
\makeanswer{problem:dipolePotential:1}{\withproblemsetsParagraph{
%
With the electric potential defined indirectly by
\begin{dmath}\label{eqn:dipolePotential:40}
\BE = -\spacegrad V,
\end{dmath}
%
we can integrate to find the difference in potential between two points
\begin{dmath}\label{eqn:dipolePotential:60}
\int_\Ba^\Bb \BE \cdot d\Bl =
- \int
\int_\Ba^\Bb \spacegrad V \cdot d\Bl
=
- \lr{ V(\Bb) - V(\Ba) },
\end{dmath}
%
or
\begin{dmath}\label{eqn:dipolePotential:80}
V(\Bb) - V(\Ba) = -
\int_\Ba^\Bb \BE \cdot d\Bl.
\end{dmath}
%
Since the dipole potential is zero at \( \Br = \infty \), we have
%
\begin{dmath}\label{eqn:dipolePotential:100}
V(\Br)
= -\int_\infty^\Br \BE \cdot d\Bl.
\end{dmath}
%
Let's integrate this on the radial path \( \Br(r') = r'\rcap \), for \( r' \in [\infty, r] \)
%
\begin{dmath}\label{eqn:dipolePotential:120}
V(\Br)
= -\int_\infty^\Br \BE \cdot d\Bl
= -\int_\infty^\Br \BE \cdot \rcap dr'
=
-
\frac{1}{4 \pi \epsilon_0 }
\int_\infty^r \frac{dr'}{{r'}^3}
\rcap
\cdot
\lr{
3 \rcap \lr{ \rcap \cdot \Bp }
-\Bp
}
=
-\frac{2}{4 \pi \epsilon_0 }
\int_\infty^r dr' \frac{\rcap\cdot \Bp}{{r'}^3}
=
\frac{\rcap \cdot \Bp}{4 \pi \epsilon_0 } \evalrange{ \inv{{r'}^2} }{\infty}{r},
\end{dmath}
%
so
%\begin{dmath}\label{eqn:dipolePotential:160}
\boxedEquation{eqn:dipolePotential:140}{
V(\Br) =
\frac{ \rcap \cdot \Bp}{4 \pi \epsilon_0 }.
}
%\end{dmath}
}} % answer
%
%}
%\EndNoBibArticle

   %
% Copyright � 2016 Peeter Joot.  All Rights Reserved.
% Licenced as described in the file LICENSE under the root directory of this GIT repository.
%
%\input{../blogpost.tex}
%\renewcommand{\basename}{emt4}
%\renewcommand{\dirname}{notes/ece1228/}
%\newcommand{\keywords}{ECE1228H}
%\input{../latex/peeter_prologue_print2.tex}
%
%%\usepackage{ece1228}
%\usepackage{peeters_braket}
%%\usepackage{peeters_layout_exercise}
%\usepackage{peeters_figures}
%\usepackage{mathtools}
%\usepackage{siunitx}
%
%\beginArtNoToc
%\generatetitle{ECE1228H Electromagnetic Theory.  Lecture 4: Magnetic moment, and Boundary value conditions.  Taught by Prof.\ M. Mojahedi}
\mychapter{Magnetic moment, and Boundary value conditions.}
%\label{chap:emt4}
%
%\paragraph{Disclaimer}
%
%Peeter's lecture notes from class.  These may be incoherent and rough.
%
%These are notes for the UofT course ECE1228H, Electromagnetic Theory, taught by Prof. M. Mojahedi, covering \textchapref{{1}} \citep{balanis1989advanced} content.
%
\paragraph{Magnetic moment.}
\index{magnetic moment}
\index{moment!magnetic}
%
Using a semi-classical model of an electron, assuming that the electron circles the nuclei.  This is a completely wrong model, but useful.  In reality, electrons are random and probabilistic and do not follow defined paths.  We do however have a magnetic moment associated with the electron, and one associated with the spin of the electron, and a moment associated with the spin of the nuclei.  All of these concepts can be used to describe a more accurate model and such a model is discussed in \citep{jackson1975cew} chapters 11,12,13.
%
Ignoring the details of how the moments really occur physically, we can take it as a given that they exist, and model them as elemenetal magnetic dipole moments of the form
%
\begin{dmath}\label{eqn:emtLecture4:20}
d\Bm_i = \ncap_i I_i ds_i \qquad [\si{A m^2}].
\end{dmath}
%
Note that \( ds_i \) is an element of surface area, not arc length!
%
Here the normal is defined in terms of the right hand rule with respect to the direction of the current as sketched in \cref{fig:emtLecture4:emtLecture4Fig1}.
\imageFigure{../figures/ece1228-electromagnetic-theory/emtLecture4Fig1}{Orientation of current loop.}{fig:emtLecture4:emtLecture4Fig1}{0.3}
%
Such dipole moments are actually what an MRI measures.  The noises that people describe from MRI machines are actually when the very powerful magnets are being rotated, allowing for the magnetic moments in the atoms of the body to be measured in different directions.
%
\index{magnetic polarization}
\index{magnetization}
The magnetic polarization, or magnetization \( \BM \), in [\si{A/m}]] is given by
%
\begin{dmath}\label{eqn:emtLecture4:40}
\BM
= \lim_{\Delta v \rightarrow 0} \lr{ \inv{\Delta v} \Bm_i }
= \lim_{\Delta v \rightarrow 0} \lr{ \inv{\Delta v} \sum_{i = 1}^{N \delta v} d\Bm_i }
= \lim_{\Delta v \rightarrow 0} \lr{ \inv{\Delta v} \sum_{i = 1}^{N \delta v} \ncap_i I_i ds_i } .
\end{dmath}
%
In materials the magnetization within the atoms are usually random, however, application of a magnetic field can force these to line up, as sketched in \cref{fig:emtLecture4:emtLecture4Fig2}.
%
\imageFigure{../figures/ece1228-electromagnetic-theory/emtLecture4Fig2}{External magnetic field alignment of magnetic moments.}{fig:emtLecture4:emtLecture4Fig2}{0.3}
%
\index{torque}
This is accomplished because an applied magnetic field acting on the magnetic moment introduces a torque, as also occured with dipole moments under applied electric fields
%
\begin{equation}\label{eqn:emtLecture4:60}
\begin{aligned}
\Btau_B &= d\Bm \cross \BB_a \\
\Btau_E &= d\Bp \cross \BE_a.
\end{aligned}
\end{equation}
%
\index{energy!torque}
There is an energy associated with this torque
%
\begin{equation}\label{eqn:emtLecture4:80}
\begin{aligned}
\Delta U_B &= -d\Bm \cdot \BB_a \\
\Delta U_E &= -d\Bp \cdot \BE_a.
\end{aligned}
\end{equation}
%
In analogy with the electric dipole moment analysis, it can be assumed that there is a linear relationship between the magnetic polarization and the applied magnetic field
%
\begin{dmath}\label{eqn:emtLecture4:100}
\BB = \mu_0 \BH_a + \mu_0 \BM = \mu_0\lr{ \BH_a + \BM },
\end{dmath}
%
where
\begin{dmath}\label{eqn:emtLecture4:120}
\BM = \chi_m \BH_a,
\end{dmath}
%
so
\begin{equation}\label{eqn:emtLecture4:140}
\BB
= \mu_0\lr{ 1 + \chi_m } \BH_a
\equiv \mu \BH_a.
\end{equation}
%
Like electric dipoles, in a volume, we can have bound currents on the surface [\si{A/m}], as well as bound volume currents [\si{A/m^2}].
% as sketched in
%
%F3
%
It can be shown, as with the electric dipoles related bound charge densities of \cref{eqn:emtLecture3:620}, that magnetic currents can be defined
%
\begin{equation}\label{eqn:emtLecture4:160}
\begin{aligned}
\BJ_{sm} &= \BM \cross \ncap \\
\BJ_{vm} &= \spacegrad \cross \BM,
\end{aligned}
\end{equation}
%
\paragraph{Conductivity}
\index{conductivity}
%
\index{constitutive relationships}
We have two constitutive relationships so far
\begin{equation}\label{eqn:emtLecture4:180}
\begin{aligned}
\BD &= \epsilon \BE \\
\BB &= \mu \BH
\end{aligned}
\end{equation}
%
but this needs to be augmented by
%
\index{Ohm's law}
\begin{dmath}\label{eqn:emtLecture4:200}
\BJ_c = \epsilon \BE.
\end{dmath}
%
There are a couple ways to discuss this.  One is to model \( \epsilon \) as a complex number.  Such a model is not entirely unconstrained.  Like with the Cauchy-Riemann conditions that relate derivatives of the real and imaginary parts of a complex number, there is a relationship (Kramers-Kronig \citep{wiki:kramersKronig}), an integral relationship that relates the real and imaginary parts of the permittivity \( \epsilon \).
%
\paragraph{Boundary conditions.}
\index{boundary conditions}
%
The boundary conditions are
%
\begin{itemize}
\item \( \ncap \cross \lr{ \BE_2 - \BE_1 } = - \BM_s \)
This means that the tangential components of \( \BE \) is continuous accross the boundary (those components of \(\BE_1,\BE_2\) are equal on the boundary), when \( \BM_s \) is zero.
%
Here \( \BM_s \) is the (fictitious) magnetic current density in [\si{V/m}].
%
\item \( \ncap \cross \lr{ \BH_2 - \BH_1 } = \BJ_s \)
%
\index{tangential field component}
This means that the tangential components of the magnetic fields \( \BH \) are discontinous when the electric surface current density \( \BJ_s \) [\si{A/m}] is non-zero, but continuous otherwise.  The latter is sketched in \cref{fig:emtLecture4:emtLecture4Fig5}.
%
\imageFigure{../figures/ece1228-electromagnetic-theory/emtLecture4Fig5}{Equal tangential fields.}{fig:emtLecture4:emtLecture4Fig5}{0.2}
%
Here \( \BJ_s \) is the movement of the free current on the surface.  The bound charges are incorporated into \( \BD \).
%
\item \( \ncap \cdot \lr{ \BD_2 - \BD_1 } = \rho_{es} \)
%
\index{normal field component}
Here \( \rho_{es} \) is the electric surface charge density [\si{C/m^2}].
%
This means that the normal component of the electric displacement field \( \BD \) is discontinuous accross the boundary in the presence of electric surface charge densities, but continuous when that is zero.
%
\item \( \ncap \cdot \lr{ \BB_2 - \BB_1 } = \rho_{ms} \)
%
\index{magnetic surface charge density}
Here \( \rho_{ms} \) is the (fictional) magnetic surface charge density [\si{Weber/m^2}].
%
This means that the magnetic fields \( \BB \) are continous in the abscense of (fictional) magnetic surface charge densities.
%
\end{itemize}
%
In the abscence of any free charges or currents, these relationships are considerably simplified
%
\begin{subequations}
\label{eqn:emtLecture4:220}
\begin{dmath}\label{eqn:emtLecture4:240}
\ncap \cross \lr{ \BE_2 - \BE_1 } = 0
\end{dmath}
\begin{dmath}\label{eqn:emtLecture4:260}
\ncap \cross \lr{ \BH_2 - \BH_1 } = 0
\end{dmath}
\begin{dmath}\label{eqn:emtLecture4:280}
\ncap \cdot \lr{ \BD_2 - \BD_1 } = 0
\end{dmath}
\begin{dmath}\label{eqn:emtLecture4:420}
\ncap \cdot \lr{ \BB_2 - \BB_1 } = 0
\end{dmath}
\end{subequations}
%
To get an idea where these come from, consider the derivation of \cref{eqn:emtLecture4:260}, relating the tangential components of \( \BH \), as sketched in \cref{fig:emtLecture4:emtLecture4Fig4}.
%
\imageFigure{../figures/ece1228-electromagnetic-theory/emtLecture4Fig4}{Boundary geometry.}{fig:emtLecture4:emtLecture4Fig4}{0.3}
%
Integrating over such a loop, the integral version of the Ampere-Maxwell equation \cref{eqn:emtLecture1:40}, with \( \BJ = \sigma \BE \) is
%
\begin{dmath}\label{eqn:emtLecture4:300}
\oint_C \BH \cdot d\Bl = \int_S \sigma \BE \cdot d\Bs + \PD{t}{} \int_S \BD \cdot d\Bs.
\end{dmath}
%
In the limit, with the height \( \Delta y \rightarrow 0 \), this is
%
\begin{dmath}\label{eqn:emtLecture4:320}
\oint_C \BH \cdot d\Bl
\approx
H_1 \cdot (\Delta x \xcap)
-H_2 \cdot (\Delta x \xcap)
\end{dmath}
%
Similarly
\begin{dmath}\label{eqn:emtLecture4:340}
\int_S \BD \cdot d\Bs
\approx
\BD \cdot \zcap \Delta x \Delta y,
\end{dmath}
%
and
\begin{equation}\label{eqn:emtLecture4:360}
\int_S \BJ \cdot d\Bs
=
\int_S \sigma \BE \cdot d\Bs
\approx
\sigma \BE \cdot \zcap \Delta x \Delta y,
\end{equation}
%
However, if \( \Delta y \) approaches zero, both of these terms are killed.
%
This gives
%
\begin{dmath}\label{eqn:emtLecture4:380}
\xcap \cdot \lr{ \BH_2 - \BH_1 } = 0.
\end{dmath}
%
If you were to perform the same calculation using a loop in the y-z plane you'd find
%
\begin{dmath}\label{eqn:emtLecture4:460}
\zcap \cdot \lr{ \BH_2 - \BH_1 } = 0.
\end{dmath}
%
Either way, the tangential component of \( \BH \) is continous on the boundary.
%
This derivation, using explicit components, follows \citep{balanis1989advanced}.  Non coordinate derivations are also possible (reference?).
%
The idea is that
%
\begin{dmath}\label{eqn:emtLecture4:440}
\ncap \cross \lr{ (\BH_2 - \BH_{2n}) -(\BH_1 - \BH_{1n}) }
=
\ncap \cross \lr{ \BH_2 - \BH_1 }
= 0.
\end{dmath}
%
What if there is a surface current?
%
\begin{dmath}\label{eqn:emtLecture4:400}
\lim_{\Delta y \rightarrow 0} \BJ_{ic} \Delta y = \BJ_s.
\end{dmath}
%
When this is the case the \( \BJ = \sigma \BE \) needs to be fixed up a bit, and showing how is left to a problem.
%.  In a problem \( \BJ_s \) is the current that is going through the elemental surface considered.
%
In the notes the other boundary relations are derived.  The normal ones follow by integrating over a pillbox volume.
%
Variations include the cases when one of the surfaces is made a perfect conductor.  Such a case can be treated by noting that the \( \BE \) field must be zero.
%
\paragraph{Conducting media.}
%
It will be left to homework to show, using the continuity equation and Gauss's law that
inside a conductor, that free charges distribute themselves exclusively on the surface on the medium.  Because of this there is no electric field inside the medium (Gauss's law).  What does this imply about the magnetic field in the same medium.  We must have
%
\begin{dmath}\label{eqn:emtLecture5:20}
\spacegrad \cross \BE = - \PD{t}{\BB}
\end{dmath}
%
so if \( \BE \) is zero in the medium the magnetic field must be either constant with respect to time, or zero.  In a general electrodynamic configuration, both the magnetic and electric fields vary with time, which seems to imply that \( \BB \) must be zero if \( \BE \) is zero in that space.
%
However, this is not consistent with what we see with an iron core inductor.  In such an inductor, the iron is used to
concentrate the magnetic field.  Clearly we have magnetic fields in the iron bar, since that is the purpose of it being there.  It turns out that if the frequencies are low enough (and even some smaller GHz frequencies are), then we can consider the system to be quasi-electrostatic, with zero electric fields inside a conductor, yet with finite approximately time independent magnetic fields.  As the frequencies are increased, the magnetic fields are forced out of the conductor into the surrounding space.
%
The transition point that defines the boundary between electrostatic and quasi-electrostatic will depend on the precision desired.
%
\paragraph{Boundary conditions with zero magnetic fields in a conductor}
%
For many calculations, we can proceed with the assumption that there are no appreciable electric nor magnetic fields inside of a conductor.  When that is the case, outside of a conducting medium, we have
%
\begin{dmath}\label{eqn:emtLecture5:40}
\ncap \cross \BE_2 = 0,
\end{dmath}
%
so there is no tangential component to an electric field of a conductor.  We also have
%
\begin{dmath}\label{eqn:emtLecture5:60}
\ncap \cdot \BD_2 = \rho_{es}
\end{dmath}
%
Assuming there is also no magnetic field either in the conductor, we also have
%
\begin{dmath}\label{eqn:emtLecture5:80}
\ncap \cross \BH_2 = \BJ_s,
\end{dmath}
%
and
\begin{dmath}\label{eqn:emtLecture5:100}
\ncap \cdot \BB_2 = 0.
\end{dmath}
%
There is no normal component to the magnetic field at the surface of a conductor, and the tangential component is determined by the surface current density.
%
%\EndArticle

      \section{Problems.}
      %
% Copyright � 2016 Peeter Joot.  All Rights Reserved.
% Licenced as described in the file LICENSE under the root directory of this GIT repository.
%
%{
%\input{../blogpost.tex}
%\renewcommand{\basename}{magneticMomentJackson}
%%\renewcommand{\dirname}{notes/phy1520/}
%\renewcommand{\dirname}{notes/ece1228-electromagnetic-theory/}
%%\newcommand{\dateintitle}{}
%%\newcommand{\keywords}{}
%
%\input{../latex/peeter_prologue_print2.tex}
%
%\usepackage{peeters_layout_exercise}
%\usepackage{peeters_braket}
%\usepackage{peeters_figures}
%\usepackage{siunitx}
%%\usepackage{txfonts} % \ointclockwise
%
%\beginArtNoToc
%
%\generatetitle{Magnetic moment for a localized magnetostatic current}
%\chapter{Magnetic moment for a localized magnetostatic current}
%\label{chap:magneticMomentJackson}
% \citep{sakurai2014modern} pr X.Y
% \citep{pozar2009microwave}
% \citep{qftLectureNotes}
% \citep{doran2003gap}
%\paragraph{Motivation.}
%
\makeproblem{Magnetic moment for localized current.}{problem:magneticMomentJackson:1}{
Jackson \citep{jackson1975cew} \S 5.6 derives an expression for the magnetic moment of a localized current distribution, far from the source.  Repeat this derivation, filling in the details.
%.  This time I found that his presentation of magnetic moment didn't really make sense to me.  Here's my own pass through it, filling in a number of details.  As I did last time, I'll also translate into SI units as I go.
} % problem
%
\makeanswer{problem:magneticMomentJackson:1}{
%\withproblemsetsParagraph{
%\paragraph{Vector potential.}
%
\index{Biot-Savart}
The Biot-Savart expression for the magnetic field can be factored into a curl expression using the usual tricks
%
\begin{dmath}\label{eqn:magneticMomentJackson:20}
\BB
= \frac{\mu_0}{4\pi} \int \frac{\BJ(\Bx') \cross (\Bx - \Bx')}{\Abs{\Bx - \Bx'}^3} d^3 x'
= -\frac{\mu_0}{4\pi} \int \BJ(\Bx') \cross \spacegrad \inv{\Abs{\Bx - \Bx'}} d^3 x'
= \frac{\mu_0}{4\pi} \spacegrad \cross \int \frac{\BJ(\Bx')}{\Abs{\Bx - \Bx'}} d^3 x',
\end{dmath}
%
so the vector potential, through its curl, defines the magnetic field \( \BB = \spacegrad \cross \BA \) is given by
%
\begin{dmath}\label{eqn:magneticMomentJackson:40}
\BA(\Bx) = \frac{\mu_0}{4 \pi} \int \frac{J(\Bx')}{\Abs{\Bx - \Bx'}} d^3 x'.
\end{dmath}
%
If the current source is localized (zero outside of some finite region), then there will always be a region for which \( \Abs{\Bx} \gg \Abs{\Bx'} \), so the denominator yields to Taylor expansion
%
\begin{dmath}\label{eqn:magneticMomentJackson:60}
\inv{\Abs{\Bx - \Bx'}}
=
\inv{\Abs{\Bx}} \lr{1 + \frac{\Abs{\Bx'}^2}{\Abs{\Bx}^2} - 2 \frac{\Bx \cdot \Bx'}{\Abs{\Bx}^2} }^{-1/2}
\approx
\inv{\Abs{\Bx}} \lr{ 1 + \frac{\Bx \cdot \Bx'}{\Abs{\Bx}^2} }
=
\inv{\Abs{\Bx}} + \frac{\Bx \cdot \Bx'}{\Abs{\Bx}^3}.
\end{dmath}
%
so the vector potential, far enough away from the current source is
\begin{dmath}\label{eqn:magneticMomentJackson:80}
\BA(\Bx)
=
\frac{\mu_0}{4 \pi} \int \frac{J(\Bx')}{\Abs{\Bx}} d^3 x'
+\frac{\mu_0}{4 \pi} \int \frac{(\Bx \cdot \Bx')J(\Bx')}{\Abs{\Bx}^3} d^3 x'.
\end{dmath}
%
Jackson uses a sneaky trick to show that the first integral is killed for a localized source.  That trick appears to be based on evaluating the following divergence
%
\begin{dmath}\label{eqn:magneticMomentJackson:100}
\spacegrad \cdot (\BJ(\Bx) x_i)
=
(\spacegrad \cdot \BJ) x_i
+
(\spacegrad x_i) \cdot \BJ
=
(\Be_k \partial_k x_i) \cdot\BJ
=
\delta_{ki} J_k
=
J_i.
\end{dmath}
%
Note that this made use of the fact that \( \spacegrad \cdot \BJ = 0 \) for magnetostatics.  This provides a way to rewrite the current density as a divergence
%
\begin{dmath}\label{eqn:magneticMomentJackson:120}
\int \frac{J(\Bx')}{\Abs{\Bx}} d^3 x'
=
\Be_i \int \frac{\spacegrad' \cdot (x_i' \BJ(\Bx'))}{\Abs{\Bx}} d^3 x'
=
\frac{\Be_i}{\Abs{\Bx}} \int \spacegrad' \cdot (x_i' \BJ(\Bx')) d^3 x'
=
\frac{1}{\Abs{\Bx}} \oint \Bx' (d\Ba' \cdot \BJ(\Bx')).
\end{dmath}
%
When \( \BJ \) is localized, this is zero provided we pick the integration surface for the volume outside of that localization region.
%
It is now desired to rewrite \( \int \Bx \cdot \Bx' \BJ \) as a triple cross product since the dot product of such a triple cross product has exactly this term in it
%
\begin{dmath}\label{eqn:magneticMomentJackson:140}
- \Bx \cross \int \Bx' \cross \BJ
=
\int (\Bx \cdot \Bx') \BJ
-
\int (\Bx \cdot \BJ) \Bx'
=
\int (\Bx \cdot \Bx') \BJ
-
\Be_k x_i \int J_i x_k',
\end{dmath}
%
so
\begin{dmath}\label{eqn:magneticMomentJackson:160}
\int (\Bx \cdot \Bx') \BJ
=
- \Bx \cross \int \Bx' \cross \BJ
+
\Be_k x_i \int J_i x_k'.
\end{dmath}
%
To get of this second term, the next sneaky trick is to consider the following divergence
%
\begin{dmath}\label{eqn:magneticMomentJackson:180}
\oint d\Ba' \cdot (\BJ(\Bx') x_i' x_j')
=
\int dV' \spacegrad' \cdot (\BJ(\Bx') x_i' x_j')
=
\int dV' (\spacegrad' \cdot \BJ)
+
\int dV' \BJ \cdot \spacegrad' (x_i' x_j')
=
\int dV' J_k \cdot \lr{ x_i' \partial_k x_j' + x_j' \partial_k x_i' }
=
\int dV' \lr{ J_k x_i' \delta_{kj} + J_k x_j' \delta_{ki} }
=
\int dV' \lr{ J_j x_i' + J_i x_j'}.
\end{dmath}
%
The surface integral is once again zero, which means that we have an antisymmetric relationship in integrals of the form
%
\begin{dmath}\label{eqn:magneticMomentJackson:200}
\int J_j x_i' = -\int J_i x_j'.
\end{dmath}
%
Now we can use the tensor algebra trick of writing \( y = (y + y)/2 \),
%
\begin{dmath}\label{eqn:magneticMomentJackson:220}
\int (\Bx \cdot \Bx') \BJ
=
- \Bx \cross \int \Bx' \cross \BJ
+
\Be_k x_i \int J_i x_k'
=
- \Bx \cross \int \Bx' \cross \BJ
+
\inv{2} \Be_k x_i \int \lr{ J_i x_k' + J_i x_k' }
=
- \Bx \cross \int \Bx' \cross \BJ
+
\inv{2} \Be_k x_i \int \lr{ J_i x_k' - J_k x_i' }
=
- \Bx \cross \int \Bx' \cross \BJ
+
\inv{2} \Be_k x_i \int (\BJ \cross \Bx')_j \epsilon_{ikj}
=
- \Bx \cross \int \Bx' \cross \BJ
-
\inv{2} \epsilon_{kij} \Be_k x_i \int (\BJ \cross \Bx')_j
=
- \Bx \cross \int \Bx' \cross \BJ
-
\inv{2} \Bx \cross \int \BJ \cross \Bx'
=
- \Bx \cross \int \Bx' \cross \BJ
+
\inv{2} \Bx \cross \int \Bx' \cross \BJ
=
-\inv{2} \Bx \cross \int \Bx' \cross \BJ,
\end{dmath}
%
so
%
\begin{dmath}\label{eqn:magneticMomentJackson:240}
\BA(\Bx) \approx \frac{\mu_0}{4 \pi \Abs{\Bx}^3} \lr{ -\frac{\Bx}{2} } \int \Bx' \cross \BJ(\Bx') d^3 x'.
\end{dmath}
%
Letting
%
\index{magnetic moment}
%\begin{dmath}\label{eqn:magneticMomentJackson:260}
\boxedEquation{eqn:magneticMomentJackson:260}{
\Bm = \inv{2} \int \Bx' \cross \BJ(\Bx') d^3 x',
}
%\end{dmath}
%
\index{vector potential}
the far field approximation of the vector potential is
%\begin{dmath}\label{eqn:magneticMomentJackson:280}
\boxedEquation{eqn:magneticMomentJackson:280}{
\BA(\Bx) = \frac{\mu_0}{4 \pi} \frac{\Bm \cross \Bx}{\Abs{\Bx}^3}.
}
%\end{dmath}
%
Note that when the current is restricted to an infinitesimally thin loop, the magnetic moment reduces to
%
\begin{dmath}\label{eqn:magneticMomentJackson:300}
\Bm(\Bx) = \frac{I}{2} \int \Bx \cross d\Bl'.
\end{dmath}
%
Referring to \citep{griffiths1999introduction} (pr. 1.60), this can be seen to be \( I \) times the ``vector-area'' integral.
%
A side effect of having evaluated this approximation is that we have shown that
%
\begin{dmath}\label{eqn:magneticMomentJackson:320}
\int \lr{ \Bx \cdot \Bx' } \BJ(\Bx') d^3 x'
=
\Bm \cross \Bx.
\end{dmath}
%
This will be required again later when evaluating the force due to an applied magnetic field in terms of the magnetic moment.
%}
} % answer
%
%}
%\EndArticle

      %
% Copyright � 2016 Peeter Joot.  All Rights Reserved.
% Licenced as described in the file LICENSE under the root directory of this GIT repository.
%
%{
%\input{../blogpost.tex}
%\renewcommand{\basename}{vectorAreaGriffiths}
%%\renewcommand{\dirname}{notes/phy1520/}
%\renewcommand{\dirname}{notes/ece1228-electromagnetic-theory/}
%%\newcommand{\dateintitle}{}
%%\newcommand{\keywords}{}
%
%\input{../latex/peeter_prologue_print2.tex}
%
%\usepackage{peeters_layout_exercise}
%\usepackage{peeters_braket}
%\usepackage{peeters_figures}
%\usepackage{siunitx}
%\usepackage{macros_qed}
%\usepackage{txfonts} % \ointclockwise
%
%\beginArtNoToc
%
%\generatetitle{Vector Area}
%%\chapter{Vector Area}
%
%One of the results of this problem is required for a later one on magnetic moments that I'd like to do.

\makeoproblem{Vector Area.}{problem:vectorAreaGriffiths:1}{\citep{griffiths1999introduction} pr. 1.61}{

The integral

\begin{dmath}\label{eqn:vectorAreaGriffiths:20}
\Ba = \int_S d\Ba,
\end{dmath}

is sometimes called the vector area of the surface \( S \).

\makesubproblem{}{problem:vectorAreaGriffiths:1:a}

Find the vector area of a hemispherical bowl of radius \( R \).
\makesubproblem{}{problem:vectorAreaGriffiths:1:b}

Show that \( \Ba = 0 \) for any closed surface.
\makesubproblem{}{problem:vectorAreaGriffiths:1:c}
Show that \( \Ba \) is the same for all surfaces sharing the same boundary.

\makesubproblem{}{problem:vectorAreaGriffiths:1:d}

Show that
\begin{dmath}\label{eqn:vectorAreaGriffiths:40}
\Ba = \inv{2} \ointctrclockwise \Br \cross d\Bl,
\end{dmath}

where the integral is around the boundary line.

\makesubproblem{}{problem:vectorAreaGriffiths:1:e}

Show that
\begin{dmath}\label{eqn:vectorAreaGriffiths:60}
\ointctrclockwise \lr{ \Bc \cdot \Br } d\Bl = \Ba \cross \Bc.
\end{dmath}
} % problem

\makeanswer{problem:vectorAreaGriffiths:1}{\withproblemsetsParagraph{
\makeSubAnswer{}{problem:vectorAreaGriffiths:1:a}

\begin{dmath}\label{eqn:vectorAreaGriffiths:80}
\Ba
=
\int_{0}^{\pi/2} R^2 \sin\theta d\theta \int_0^{2\pi} d\phi
\lr{ \sin\theta \cos\phi, \sin\theta \sin\phi, \cos\theta }
=
R^2 \int_{0}^{\pi/2} d\theta \int_0^{2\pi} d\phi
\lr{ \sin^2\theta \cos\phi, \sin^2\theta \sin\phi, \sin\theta\cos\theta }
=
2 \pi R^2 \int_{0}^{\pi/2} d\theta \Be_3
\sin\theta\cos\theta
=
\pi R^2
\Be_3
\int_{0}^{\pi/2} d\theta
\sin(2 \theta)
=
\pi R^2
\Be_3
\evalrange{\lr{\frac{-\cos(2 \theta)}{2}}}{0}{\pi/2}
=
\pi R^2
\Be_3
\lr{ 1 - (-1) }/2
=
\pi R^2
\Be_3.
\end{dmath}

\makeSubAnswer{}{problem:vectorAreaGriffiths:1:b}

As hinted in the original problem description, this follows from

\begin{dmath}\label{eqn:vectorAreaGriffiths:100}
\int dV \spacegrad T = \oint T d\Ba,
\end{dmath}

simply by setting \( T = 1 \).

\makeSubAnswer{}{problem:vectorAreaGriffiths:1:c}

Suppose that two surfaces sharing a boundary are parameterized by vectors \( \Bx(u, v), \Bx(a,b) \) respectively.  The area integral with the first parameterization is

\begin{dmath}\label{eqn:vectorAreaGriffiths:120}
\Ba
= \int \PD{u}{\Bx} \cross \PD{v}{\Bx} du dv
= \epsilon_{ijk} \Be_i \int \PD{u}{x_j} \PD{v}{x_k} du dv
=
\epsilon_{ijk} \Be_i \int
\lr{
\PD{a}{x_j}
\PD{u}{a}
+
\PD{b}{x_j}
\PD{u}{b}
}
\lr{
\PD{a}{x_k}
\PD{v}{a}
+
\PD{b}{x_k}
\PD{v}{b}
}
du dv
=
\epsilon_{ijk} \Be_i \int
du dv
\lr{
\PD{a}{x_j}
\PD{u}{a}
\PD{a}{x_k}
\PD{v}{a}
+
\PD{b}{x_j}
\PD{u}{b}
\PD{b}{x_k}
\PD{v}{b}
+
\PD{b}{x_j}
\PD{u}{b}
\PD{a}{x_k}
\PD{v}{a}
+
\PD{a}{x_j}
\PD{u}{a}
\PD{b}{x_k}
\PD{v}{b}
}
=
\epsilon_{ijk} \Be_i \int
du dv
\lr{
\PD{a}{x_j}
\PD{a}{x_k}
\PD{u}{a}
\PD{v}{a}
+
\PD{b}{x_j}
\PD{b}{x_k}
\PD{u}{b}
\PD{v}{b}
}
+
\epsilon_{ijk} \Be_i \int
du dv
\lr{
\PD{b}{x_j}
\PD{a}{x_k}
\PD{u}{b}
\PD{v}{a}
-
\PD{a}{x_k}
\PD{b}{x_j}
\PD{u}{a}
\PD{v}{b}
}.
\end{dmath}

In the last step a \( j,k \) index swap was performed for the last term of the second integral.  The first integral is zero, since the integrand is symmetric in \( j,k \).  This leaves
\begin{dmath}\label{eqn:vectorAreaGriffiths:140}
\Ba
=
\epsilon_{ijk} \Be_i \int
du dv
\lr{
\PD{b}{x_j}
\PD{a}{x_k}
\PD{u}{b}
\PD{v}{a}
-
\PD{a}{x_k}
\PD{b}{x_j}
\PD{u}{a}
\PD{v}{b}
}
=
\epsilon_{ijk} \Be_i \int
\PD{b}{x_j}
\PD{a}{x_k}
\lr{
\PD{u}{b}
\PD{v}{a}
-
\PD{u}{a}
\PD{v}{b}
}
du dv
=
\epsilon_{ijk} \Be_i \int
\PD{b}{x_j}
\PD{a}{x_k}
\frac{\partial(b,a)}{\partial(u,v)} du dv
=
-\int
\PD{b}{\Bx} \cross \PD{a}{\Bx} da db
=
\int
\PD{a}{\Bx} \cross \PD{b}{\Bx} da db.
\end{dmath}

However, this is the area integral with the second parameterization, proving that the area-integral for any given boundary is independent of the surface.

\makeSubAnswer{}{problem:vectorAreaGriffiths:1:d}

Having proven that the area-integral for a given boundary is independent of the surface that it is evaluated on, the result follows by illustration as hinted in the full problem description.  Draw a ``cone'', tracing a vector \( \Bx' \) from the origin to the position line element, and divide that cone up into infinitesimal slices as sketched in \cref{fig:coneVectorArea:coneVectorAreaFig1}.

\imageFigure{../figures/ece1228-electromagnetic-theory/coneVectorAreaFig1}{Cone configuration.}{fig:coneVectorArea:coneVectorAreaFig1}{0.2}

The area of each of these triangular slices is

\begin{dmath}\label{eqn:vectorAreaGriffiths:160}
\inv{2} \Bx' \cross d\Bl'.
\end{dmath}

Summing those triangles proves the result.

\makeSubAnswer{}{problem:vectorAreaGriffiths:1:e}

As hinted in the problem, this follows from

\begin{dmath}\label{eqn:vectorAreaGriffiths:180}
\int \spacegrad T \cross d\Ba = -\ointctrclockwise T d\Bl.
\end{dmath}

Set \( T = \Bc \cdot \Br \), for which

\begin{dmath}\label{eqn:vectorAreaGriffiths:240}
\spacegrad T
= \Be_k \partial_k c_m x_m
= \Be_k c_m \delta_{km}
= \Be_k c_k
= \Bc,
\end{dmath}

so
\begin{dmath}\label{eqn:vectorAreaGriffiths:200}
(\spacegrad T) \cross d\Ba
=
\int \Bc \cross d\Ba
=
\Bc \cross \int d\Ba
=
\Bc \cross \Ba.
\end{dmath}

so
\begin{dmath}\label{eqn:vectorAreaGriffiths:220}
\Bc \cross \Ba = -\ointctrclockwise (\Bc \cdot \Br) d\Bl,
\end{dmath}

or
\begin{dmath}\label{eqn:vectorAreaGriffiths:260}
\ointctrclockwise (\Bc \cdot \Br) d\Bl
=
\Ba \cross \Bc. \qedmarker
\end{dmath}
}} % answer

%}
%\EndArticle

      %
% Copyright � 2016 Peeter Joot.  All Rights Reserved.
% Licenced as described in the file LICENSE under the root directory of this GIT repository.
%
\makeproblem{Tangential magnetic field boundary conditions.}{emt:problemSet3:1}{

\index{boundary conditions!tangential magnetic field}
In the class notes we showed that when there were no sources at the interface between two
media and neither of the two media was a perfect conductor \( \sigma_1, \sigma_2 \ne \infty \) the boundary condition
on the tangential magnetic field was given by

\begin{dmath}\label{eqn:emtProblemSet3Problem1:20}
\ncap \cross \lr{ \BH_2 - \BH_1 } = 0.
\end{dmath}

Here, show that when \( \BJ_i + \BJ_c = \BJ_{ic} \ne 0 \), the boundary condition is given by

\begin{dmath}\label{eqn:emtProblemSet3Problem1:40}
\ncap \cross \lr{ \BH_2 - \BH_1 } = \BJ_s,
\end{dmath}

where
\begin{dmath}\label{eqn:emtProblemSet3Problem1:60}
\BJ_s = \lim_{\Delta y \rightarrow 0} \BJ_{ic} \Delta y.
\end{dmath}

Note: Use the geometry provided in
\cref{fig:boundaryPs3:boundaryPs3Fig1}
for your proof.
\imageFigure{../figures/ece1228-electromagnetic-theory/boundaryPs3Fig1}{Boundary geometry.}{fig:boundaryPs3:boundaryPs3Fig1}{0.3}
} % makeproblem

\makeanswer{emt:problemSet3:1}{\withproblemsetsParagraph{

Instead of integrating over a loop as done in class, a better way to tackle this problem is to integrate the curl over the same sort of pillbox that we use for deriving the boundary conditions from the divergence Maxwell's equations.

\index{Stokes' theorem}
The form of Stokes' theorem that we want, following the notation of \citep{aMacdonaldVAGC}, is
\begin{dmath}\label{eqn:emtProblemSet3Problem1:80}
\int_V d^3 \Bx \cdot \lr{ \boldpartial \wedge \BA } = \oint_{\partial V} d^2 \Bx \cdot \BA.
\end{dmath}
The \R{3} translation of this relation into traditional vector algebra, after applying some duality relations, is
\begin{dmath}\label{eqn:emtProblemSet3Problem1:100}
\int_V dV \spacegrad \cross \BA = \oint_{\partial V} dA \ncap \cross \BA,
\end{dmath}
where \( \ncap \) is the outwards normal.  Proving the general multivector Stokes relationship is beyond the scope of this problem, but we can validate
\cref{eqn:emtProblemSet3Problem1:100} by integrating the LHS over the infinitesimal rectangular prism sketched in \cref{fig:ps3Problem1ElementalVolume:ps3Problem1ElementalVolumeFig1}.
\imageFigure{../figures/ece1228-electromagnetic-theory/ps3Problem1ElementalVolumeFig1}{Elemental volume.}{fig:ps3Problem1ElementalVolume:ps3Problem1ElementalVolumeFig1}{0.2}
\begin{equation}\label{eqn:emtProblemSet3Problem1:120}
\begin{aligned}
\oint_{\partial V} dA \ncap \cross \BA
&=
\oint_{\partial V} dx dy \Be_3 \cross \lr{ \BA(z_0 + \Delta z) - \BA(z_0) } \\
&+\oint_{\partial V} dy dz \Be_1 \cross \lr{ \BA(x_0 + \Delta x) - \BA(x_0) } \\
&+\oint_{\partial V} dz dx \Be_2 \cross \lr{ \BA(y_0 + \Delta y) - \BA(y_0) } \\
&=
\int_{V} dx dy \Be_3 \cross \lr{ dz \PD{z}{\BA} }
+\int_{V} dy dz \Be_1 \cross \lr{ dx \PD{x}{\BA} } \\
&\quad +\int_{V} dz dx \Be_2 \cross \lr{ dy \PD{y}{\BA} } \\
&=
\int_{V} dx dy dz \spacegrad \cross \BA.
\end{aligned}
\end{equation}
Now, let's apply this to Ampere-Maxwell equation
\begin{dmath}\label{eqn:emtProblemSet3Problem1:140}
\spacegrad \cross \BH = \BJ_{ic} + \PD{t}{\BD},
\end{dmath}
where \( \BJ_{ic} = \BJ_s \delta(y) \).  We have
\begin{dmath}\label{eqn:emtProblemSet3Problem1:160}
\oint dA \ncap \cross \BH = \int dV \lr{ \BJ_s \delta(y) + \PD{t}{\BD} }.
\end{dmath}
This integral will be evaluated using the pillbox configuration of \cref{fig:ps3Problem1Pillbox:ps3Problem1PillboxFig1}.
\imageFigure{../figures/ece1228-electromagnetic-theory/ps3Problem1PillboxFig1}{Pillbox integration volume.}{fig:ps3Problem1Pillbox:ps3Problem1PillboxFig1}{0.2}

The delta function picks up only the contribution of \( \int dA \BJ_s(y=0) \), but \( \BJ_s \) only has a value on that surface anyways.  Taking the pillbox volume to zero in the \( \Delta y \rightarrow 0 \) limit, the LHS integral has only contributions from the top and bottom faces of the pillbox, and the \( \BD \) term, which is assumed finite, will get killed.  That leaves
\begin{dmath}\label{eqn:emtProblemSet3Problem1:180}
\int dA \ncap \cross \lr{ \BH_2 - \BH_1 } = \int dA \BJ_s.
\end{dmath}
Both sets of integrands can now be brought under one integral
\begin{dmath}\label{eqn:emtProblemSet3Problem1:200}
\int dA \lr{ \ncap \cross \lr{ \BH_2 - \BH_1 } - \BJ_s } = 0.
\end{dmath}
This is valid for any pillbox surface, so the integrand must be zero,
which proves the desired boundary relation
%\begin{dmath}\label{eqn:emtProblemSet3Problem1:220}
\boxedEquation{eqn:emtProblemSet3Problem1:240}{
\ncap \cross \lr{ \BH_2 - \BH_1 } - \BJ_s = 0.
}
%\end{dmath}

Except for having arbitrarily picked the y-axis as the normal direction in the delta function representation of \( \BJ_{ic} \),
this derivation has the advantage of being coordinate free.  This is in contrast to
the procedure of \citep{balanis1989advanced} followed in class where multiple loop orientations across the boundary are required to prove the general result.
}}

      %
% Copyright � 2016 Peeter Joot.  All Rights Reserved.
% Licenced as described in the file LICENSE under the root directory of this GIT repository.
%
\makeproblem{Magnetic field for a current loop.}{emt:problemSet3:2}{
%
\index{current loop}
A loop of wire located in x-y plane carrying current \(I\) is shown in \cref{fig:currentLoopPs3:currentLoopPs3Fig2}.
The loop's radius is \(R_l\).
\imageFigure{../figures/ece1228-electromagnetic-theory/currentLoopPs3Fig2}{Current loop.}{fig:currentLoopPs3:currentLoopPs3Fig2}{0.3}
\makesubproblem{}{emt:problemSet3:2a}
Calculate the magnetic field flux density, \( \BB \), along the loop axis at a distance \( z \) from its center.
%
\makesubproblem{}{emt:problemSet3:2b}
Simplify the results in
\partref{emt:problemSet3:2a}
for large distances along the z-axis (\( z \gg R_l \)).
%
\makesubproblem{}{emt:problemSet3:2c}
Express the results in
\partref{emt:problemSet3:2b}
in terms of magnetic dipole
moment. Make sure you write the expression in vector
form.
\makesubproblem{}{emt:problemSet3:2d}
In keeping with your understanding of magnetic bar's
north and south poles, designate the north and south poles
for the current carrying loop shown in the figure.
%
\index{Biot-Savart}
\paragraph{Hint:} Use Biot-Savart law which states the following: A
differential current element, \( I d\Bl' \), produces a differential
magnetic field, \( d\BB \),
at a distance \( R \) from the current
element given by
%
\begin{equation}\label{eqn:emtProblemSet3Problem2:20}
d\BB = \frac{\mu_0}{4 \pi} \frac{I d\Bl' \cross \BR }{R^3},
\end{equation}
%
or
\begin{equation}\label{eqn:emtProblemSet3Problem2:40}
\BB = \frac{\mu_0}{4 \pi} \int \frac{I d\Bl' \cross \BR }{R^3},
\end{equation}
%
Note that integration is carried over the source (current) and \( R \) points from the current elements
to the point of observation.
} % makeproblem
%
\skipIfRedacted{
\makeanswer{emt:problemSet3:2}{
\makeSubAnswer{}{emt:problemSet3:2a}
%
The integral for the general observation point is straightforward to write down.
The observation point for this problem is
%
\begin{dmath}\label{eqn:emtProblemSet3Problem2:220}
\Br = z \Be_3.
\end{dmath}
%
The charge point is
\begin{dmath}\label{eqn:emtProblemSet3Problem2:240}
\Br' = R_l \lr{ \Be_1 \cos\theta' + \Be_2 \sin\theta' },
\end{dmath}
%
and the element of the loop path is
\begin{dmath}\label{eqn:emtProblemSet3Problem2:260}
d\Bl' = R_l d\theta' \lr{ \Be_2 \cos\theta' - \Be_1 \sin\theta'}.
\end{dmath}
%
The difference vector from the charge position to the observation point is
%
\begin{dmath}\label{eqn:emtProblemSet3Problem2:300}
\BR
= \Br - \Br'
=
z \Be_3 - R_l \lr{ \Be_1 \cos\theta' + \Be_2 \sin\theta' },
\end{dmath}
%
with squared length
%
\begin{dmath}\label{eqn:emtProblemSet3Problem2:320}
\BR^2
=
z^2 + R_l^2.
\end{dmath}
%
\index{elemental arc}
Finally, the cross product of the elemental arc with the difference vector is
%
\begin{dmath}\label{eqn:emtProblemSet3Problem2:280}
d\Bl' \cross \BR
=
R_l d\theta'
\lr{
\Be_2 \cos\theta' - \Be_1 \sin\theta'
}
\cross
\lr{
z \Be_3 - R_l \lr{ \Be_1 \cos\theta' + \Be_2 \sin\theta' }
}
=
R_l d\theta'
\lr{
z \lr{ \Be_1 \cos\theta' + \Be_2 \sin\theta' }
+
R_l \Be_3 \lr{
\cos^2\theta' + \sin^2 \theta'
}
}
=
R_l d\theta'
\lr{
z %\Be_1 e^{i\theta'}
\lr{ \Be_1 \cos\theta' + \Be_2 \sin\theta' }
+ R_l \Be_3
}.
\end{dmath}
%
The magnetic field integral is
\boxedEquation{eqn:emtProblemSet3Problem2:201}{
%\begin{dmath}\label{eqn:emtProblemSet3Problem2:201}
\BB = \frac{I \mu_0 R_l}{4\pi} \int_0^{2\pi} d\theta'
\frac
{ z
%\Be_1 e^{i\theta'}
\lr{ \Be_1 \cos\theta' + \Be_2 \sin\theta' }
+ \Be_3 R_l }
{ \lr{ z^2 + R_l^2 }^{3/2} }.
}
%\end{dmath}
%
%This is consistent with \cref{eqn:emtProblemSet3Problem2:200} as expected.
This integral is particularly easy to evaluate.  All the trigonometric contributions are killed integrating over the \( [0, 2 \pi] \) interval, leaving just
\begin{equation}\label{eqn:emtProblemSet3Problem2:340}
\BB = \frac{I \mu_0 R_l}{4\pi}
\frac
{ 2 \pi \Be_3 R_l }
{ \lr{ z^2 + R_l^2 }^{3/2} },
\end{equation}
%
or
\boxedEquation{eqn:emtProblemSet3Problem2:360}{
\BB = \frac{\Be_3 I \mu_0 }{2\pi}
\frac
{ \lr{\pi R_l^2} }
{ \lr{ z^2 + R_l^2 }^{3/2} },
}
%
\makeSubAnswer{}{emt:problemSet3:2b}
%
Taylor expanding the denominator gives
%
\begin{dmath}\label{eqn:emtProblemSet3Problem2:380}
\BB
= \frac{\Be_3 I \mu_0 }{2\pi}
\frac
{ \lr{\pi R_l^2} }
{ z^3 }
\lr{
1 - \frac{3}{2} \lr{ \frac{R_l}{z} }^2
+ \frac{(3)(5)}{2^3} \lr{ \frac{R_l}{z} }^4
+ \cdots
}.
\end{dmath}
%
If \( z \gg R_l \) this is dominated by the zero order term of the expansion, leaving
\boxedEquation{eqn:emtProblemSet3Problem2:460}{
\BB
\approx
\Be_3 I \frac{\mu_0 }{2\pi}
\frac
{ \lr{\pi R_l^2} }
{ z^3 }.
}
%
\makeSubAnswer{}{emt:problemSet3:2c}
%
If an element of magnetic moment for the loop is \( d\Bm = \ncap I ds \), the magnetic moment for the whole loop is
%
\begin{dmath}\label{eqn:emtProblemSet3Problem2:420}
\BM
= \Be_3 I (2 \pi R_l),
\end{dmath}
%
so the magnetic field is
%
%\begin{dmath}\label{eqn:emtProblemSet3Problem2:440}
\boxedEquation{eqn:emtProblemSet3Problem2:440}{
\BB
=
\BM \frac{\mu_0}{4\pi}
\frac{ R_l}{z^3}.
}
%\end{dmath}
%
\makeSubAnswer{}{emt:problemSet3:2d}
%
\index{magnetic pole}
I couldn't recall the N vs. S convention for field lines in a magnet, and couldn't find anything in my electromagnetism books on this.
However, the ``Field of a cylindrical bar magnet calculated with Amp\`ere's model'' figure in \citep{wiki:magnet}, indicates that the magnetic field lines emanate from the North pole of a magnet.  This means that the current loop in this problem can be thought of as representing a magnet with north and south poles in the \( z > 0 \) and \( z < 0 \) regions respectively, as sketched in \cref{fig:magnetPoleForCurrentLoop:magnetPoleForCurrentLoopFig1}.
\imageFigure{../figures/ece1228-electromagnetic-theory/magnetPoleForCurrentLoopFig1}{Magnet pole orientations for current loop.}{fig:magnetPoleForCurrentLoop:magnetPoleForCurrentLoopFig1}{0.3}
%
\paragraph{Off axis field.}
%
%
% Copyright © 2016 Peeter Joot.  All Rights Reserved.
% Licenced as described in the file LICENSE under the root directory of this GIT repository.
%
%\section{Appendix I.  Problem 2. Current loop integral off axis.}
%
Initially I was curious what the current loop magnetic field integral would look like in general, allowing for an off axis observation point.
%
I found it natural to do that compuation using Geometric Algebra to express vector rotation in a plane and the other geometrical constructs of this problem.  The basic rules in that Algebra are that unit vectors square to unity (\(\Be_k^2 = 1 \)), and that orthogonal vectors anticommute (\( \Be_1 \Be_2 = -\Be_2 \Be_1 \)).  For example, letting \( i = \Be_1 \Be_2 \) the radial unit vector can be expessed as
%
\begin{dmath}\label{eqn:emtProblemSet3Problem2:160}
\rhocap(\theta)
=
\Be_1 e^{i \theta}
= \Be_1 \lr{ \cos\theta + \Be_1 \Be_2 \sin\theta }
= \Be_1 \cos\theta + (\Be_1^2) \Be_2 \sin\theta
= \Be_1 \cos\theta + \Be_2 \sin\theta,
\end{dmath}
%
and the \( \thetacap \) direction vector is
\begin{dmath}\label{eqn:emtProblemSet3Appendix:240}
\thetacap(\theta)
=
\Be_2 e^{i \theta}
= \Be_2 \lr{ \cos\theta + \Be_1 \Be_2 \sin\theta }
= \Be_2 \cos\theta + \Be_2 \Be_1 \Be_2 \sin\theta
= \Be_2 \cos\theta + \Be_2 (-\Be_2 \Be_1) \sin\theta
= \Be_2 \cos\theta - \Be_1 \sin\theta.
\end{dmath}
%
This allows for a compact expression of an off-axis observation point
%
\begin{dmath}\label{eqn:emtProblemSet3Problem2:60}
\Br = z \Be_3 + \rho \Be_1 e^{i\theta}.
\end{dmath}
%
Similarly, the charge point is
\begin{dmath}\label{eqn:emtProblemSet3Problem2:80}
\Br' = R_l \Be_1 e^{i \theta'},
\end{dmath}
%
and the element of the loop path is
\begin{dmath}\label{eqn:emtProblemSet3Problem2:100}
d\Bl' = R_l \Be_2 e^{i\theta'} d\theta'.
\end{dmath}
%
The difference vector from the charge position to the observation point is
%
\begin{dmath}\label{eqn:emtProblemSet3Problem2:120}
\BR
= \Br - \Br'
=
z \Be_3 + \rho \Be_1 e^{i\theta}
-
R_l \Be_1 e^{i \theta'},
\end{dmath}
%
with squared length
%
\begin{dmath}\label{eqn:emtProblemSet3Problem2:140}
\BR^2
=
z^2 +
\lr{ \rho \Be_1 e^{i\theta}
-
R_l \Be_1 e^{i \theta'}
}
\cdot
\lr{ \rho \Be_1 e^{i\theta}
-
R_l \Be_1 e^{i \theta'}
}
=
z^2 + \rho^2 + R_l^2 - 2 \rho R_l \cos\lr{ \theta - \theta' }.
\end{dmath}
%
For the cross product, using a bivector duality transformation \( \Ba \cross \Bb = -\Be_1 \Be_2 \Be_3 (\Ba \wedge \Bb) \), and expressing the wedge product as a grade two selection, we have
%
\begin{dmath}\label{eqn:emtProblemSet3Problem2:180}
d\Bl' \cross \BR
=
-\Be_1 \Be_2 \Be_3 R_l d\theta' \gpgradetwo{
\Be_2 e^{i \theta'}
\lr{
z \Be_3 + \rho \Be_1 e^{i\theta}
-
R_l \Be_1 e^{i \theta'}
}
}
=
R_l d\theta' \lr{
z \Be_1 e^{i\theta'}
-
\Be_3 \rho \cos\lr{ \theta - \theta' }
+ \Be_3 R_l
}.
\end{dmath}
%
The final integral can now be assembled
%
\boxedEquation{eqn:emtProblemSet3Appendix:220}{
%\begin{dmath}\label{eqn:emtProblemSet3Problem2:200}
\BB = \frac{I \mu_0 R_l}{4\pi} \int_0^{2\pi} d\theta'
\frac
{ z \Be_1 e^{i\theta'} - \Be_3 \rho \cos\lr{ \theta - \theta' } + \Be_3 R_l }
{ \lr{z^2 + \rho^2 + R_l^2 - 2 \rho R_l \cos\lr{ \theta - \theta' }}^{3/2} }.
%\end{dmath}
}
%
This is consistent with the traditional vector algebra derivation that led to \cref{eqn:emtProblemSet3Problem2:201} where \( \rho = 0 \) was assumed.
It is clear now, why the problem statement asked only to consider the z-axis observation points where \( \rho = 0 \).  With \( \theta' \) dependencies in the denominator, performing the integral above for \( \rho \ne 0 \) looks spectacularly unpleasant.
%

}}

      %
% Copyright � 2016 Peeter Joot.  All Rights Reserved.
% Licenced as described in the file LICENSE under the root directory of this GIT repository.
%
\makeproblem{Electric field across dielectric boundary.}{emt:problemSet3:3}{
\index{boundary conditions!electric field}
\index{dielectric!boundary conditions}
The plane \( 3x + 2y + z = 12 \) [\si{m}] describes the interface between a dielectric and free space.
The origin side of the interface has \( \epsilon_{r 1} = 3 \) and \( \BE_1 = 2 \xcap + 5 \zcap \) [\si{V/m}]. What is \(\BE_2\)
(the field on the other side of the interface)?
} % makeproblem
%
\skipIfRedacted{
\makeanswer{emt:problemSet3:3}{
The geometry of the problem is sketched roughly in \cref{fig:ps3Problem3Plane:ps3Problem3PlaneFig1}.
\imageFigure{../figures/ece1228-electromagnetic-theory/ps3Problem3PlaneFig1}{Interfaces on sides of a plane.}{fig:ps3Problem3Plane:ps3Problem3PlaneFig1}{0.3}
Assuming that there are no sources, the relationships between the fields on each side of the interface are
\begin{equation}\label{eqn:emtProblemSet3Problem3:20}
\begin{aligned}
%\ncap \cdot \lr{ \epsilon_0 \epsilon_{r2} \BE_2 - \epsilon_0 \epsilon_{r1} \BE_1 } &= 0
\ncap \cdot \lr{ \BD_2 - \BD_1 } &= 0,\\
\ncap \cross \lr{ \BE_2 - \BE_1 } &= 0.
\end{aligned}
\end{equation}
%
%Since \( \epsilon_{r2} = 1 \),
After cancelling common factors of \( \epsilon_0 \) the first relationship can be written
\begin{equation}\label{eqn:emtProblemSet3Problem3:40}
\ncap \cdot \BE_2 = \frac{\epsilon_{r1}}{\epsilon_{r2}} \ncap \cdot \BE_1.
\end{equation}
Adding the normal and the tangential components of \( \BE_2 \), we have
\begin{equation}\label{eqn:emtProblemSet3Problem3:60}
\begin{aligned}
\BE_2
&=
\ncap \lr{ \ncap \cdot \BE_2 } -
\ncap \cross \lr{ \ncap \cross \BE_2 }
\\ &=
\frac{\epsilon_{r1}}{\epsilon_{r2}} \ncap \lr{ \ncap \cdot \BE_1 }
- \ncap \cross \lr{ \ncap \cross \BE_1 }.
\end{aligned}
\end{equation}
By expanding the tangential projection (normal rejection) of a vector as
\begin{equation}\label{eqn:emtProblemSet3Problem3:160}
\begin{aligned}
\BA_t
&=
- \ncap \cross \lr{ \ncap \cross \BA }
\\ &=
\BA - \ncap (\ncap \cdot \BA),
\end{aligned}
\end{equation}
we find
\begin{equation}\label{eqn:emtProblemSet3Problem3:180}
\begin{aligned}
\BE_2
&=
\frac{\epsilon_{r1}}{\epsilon_{r2}} \ncap \lr{ \ncap \cdot \BE_1 }
+ \lr{ \BE_1 - \ncap (\ncap \cdot \BE_1) }
\\ &=
\BE_1 + \lr{\frac{\epsilon_{r1}}{\epsilon_{r2}} -1} \ncap \lr{ \ncap \cdot \BE_1 },
\end{aligned}
\end{equation}
or
\boxedEquation{eqn:emtProblemSet3Problem3:320}{
\BE_2
=
\BE_1 + \frac{\frac{\epsilon_{r1}}{\epsilon_{r2}} -1}{\Abs{\Bn}^2} \Bn \lr{ \Bn \cdot \BE_1 }.
}
The rest of the problem is routine algebra.
\begin{subequations}
\label{eqn:emtProblemSet3Problem3:220}
\begin{equation}\label{eqn:emtProblemSet3Problem3:240}
\begin{aligned}
\Bn^2 &= (3,2,1) \cdot (3,2,1) \\ &= 9 + 4 + 1 \\ &= 14,
\end{aligned}
\end{equation}
\begin{equation}\label{eqn:emtProblemSet3Problem3:260}
\begin{aligned}
\Bn \cdot \BE_1
&=
(3,2,1) \cdot (2,0,5)
\\ &=
6+5
\\ &=11,
\end{aligned}
\end{equation}
\end{subequations}
so
\begin{equation}\label{eqn:emtProblemSet3Problem3:280}
\begin{aligned}
\BE_2
&=
(2,0,5) + \frac{2 \times 11}{14} (3,2,1)
\\ &=
\inv{7}( (14,0,35) + (33,22,11) ),
\end{aligned}
\end{equation}
which is
\boxedEquation{eqn:emtProblemSet3Problem3:300}{
\BE_2
=
\frac{47}{7} \xcap + \frac{22}{7} \ycap + \frac{46}{7} \zcap
\qquad[\si{V/m}].
}
}}

      %
% Copyright � 2016 Peeter Joot.  All Rights Reserved.
% Licenced as described in the file LICENSE under the root directory of this GIT repository.
%
\makeproblem{Laplacian form of delta function.}{emt:problemSet3:4}{
\index{Laplacian!Green's function}
\index{Green's function!Laplacian representation}
Prove that
%
\begin{dmath}\label{eqn:emtProblemSet3Problem4:20}
-\spacegrad^2 \inv{r} = 4 \pi \delta^3(\Br),
\end{dmath}
%
where \( r = \Abs{\Br} \) is the position vector.
} % makeproblem
%
\skipIfRedacted{
\makeanswer{emt:problemSet3:4}{
%
The first thing to show is that \( \spacegrad^2 1/r = 0 \) everywhere that \( r \ne 0 \).  Note that
%
\begin{dmath}\label{eqn:emtProblemSet3Problem4:40}
\spacegrad r^{-n}
=
\spacegrad \lr{\Br^2}^{-n/2}
=
-\frac{n}{2} \frac{\Be_k \partial_k x_m x_m}{r^{n + 2}}
=
-\frac{n}{2} \frac{2 \Be_k \delta_{km} x_m}{r^{n + 2}}
=
-n \frac{\Br}{r^{n+2}}.
\end{dmath}
%
The Laplacian is
\begin{dmath}\label{eqn:emtProblemSet3Problem4:60}
\spacegrad^2 \inv{r}
=
\spacegrad \cdot \spacegrad \inv{r}
=
-\spacegrad \cdot
\frac{\Br}{r^{3}}
=
-\frac{\spacegrad \cdot \Br}{r^3}
-
\lr{ \spacegrad \inv{r^3} } \cdot \Br
=
-\frac{3}{r^3}
-
\lr{ -3 \frac{\Br}{r^5}} \cdot \Br
=
-\frac{3}{r^3}
+
3 \frac{ r^2}{r^5}
=
0,
\end{dmath}
%
provided \( \Br \ne 0 \), an expected property of the delta function.
%
To complete the proof, we have to show that this Laplacian has the desired filtering effect under convolution
%
\begin{dmath}\label{eqn:emtProblemSet3Problem4:80}
\int dV' f(\Br') \delta^3(\Br' -\Br) = f(\Br).
\end{dmath}
%
\index{delta function}
Inserting the assumed delta function representation we have
%
\begin{dmath}\label{eqn:emtProblemSet3Problem4:100}
\int dV' f(\Br') \delta^3(\Br' - \Br)
=
-\inv{4\pi} \int dV' f(\Br') {\spacegrad'}^2 \inv{\Abs{\Br' - \Br}}
=
-\lim_{\epsilon \rightarrow 0}
\inv{4\pi} \int_{\Abs{\Br - \Br'} < \epsilon} dV' f(\Br') \spacegrad' \cdot \spacegrad' \inv{\Abs{\Br' - \Br}}
=
\lim_{\epsilon \rightarrow 0}
\inv{4\pi} \int_{\Abs{\Br - \Br'} < \epsilon} dV' f(\Br') \spacegrad' \cdot \frac{\Br' - \Br}{\Abs{\Br' - \Br}^3}
=
\lim_{\epsilon \rightarrow 0}
\inv{4\pi} \int_{\Abs{\Br - \Br'} = \epsilon} dV' f(\Br') \ncap \cdot \frac{\Br' - \Br}{\Abs{\Br' - \Br}^3}.
\end{dmath}
%
Because the Laplacian has been shown to be zero everywhere where \( \Br \ne \Br' \) the volume integral over all space has been restricted to a small spherical volume surrounding the point \( \Br \).  The divergence theorem is then used to transform this integral into a surface integral over that spherical volume.  However, the exterior normal to this surface is \( \ncap = \lr{\Br' - \Br}/\Abs{\Br' - \Br} \), leaving
%
\begin{dmath}\label{eqn:emtProblemSet3Problem4:120}
\int dV' f(\Br') \delta^3(\Br' - \Br)
=
\lim_{\epsilon \rightarrow 0}
\inv{4\pi} \int_{\Abs{\Br - \Br'} = \epsilon} dV' f(\Br') \inv{\Abs{\Br' - \Br}^2}
=
\lim_{\epsilon \rightarrow 0}
\inv{4\pi} \int_{\Abs{\Br - \Br'} = \epsilon} dV' f(\Br') \inv{\epsilon^2}
=
\lim_{\epsilon \rightarrow 0}
\inv{4\pi} f(\Br) \frac{4 \pi \epsilon^2}{\epsilon^2}
=
f(\Br).
\end{dmath}
%
In the second last step, the it is assumed that the function \( f(\Br') \) is well behaved enough in the near the point \( \Br \) that it can be pulled out of the integral, and replaced with it's mean value in the neighbourhood of \( \Br \), which then tends to \( f(\Br)\) as \( \epsilon \rightarrow 0 \).
}}

      %%
% Copyright © 2016 Peeter Joot.  All Rights Reserved.
% Licenced as described in the file LICENSE under the root directory of this GIT repository.
%
%\section{Appendix II.  Problem 3.  Normal and tangential decomposition.}
%
The decomposition of \cref{eqn:emtProblemSet3Problem3:60} can be derived easily using Geometric Algebra
\index{Geometric Algebra}
%
\begin{dmath}\label{eqn:emtProblemSet3Problem3:80}
\BA
=
\ncap^2 \BA
=
\ncap (\ncap \cdot \BA)
+\ncap (\ncap \wedge \BA)
%=
%\ncap (\ncap \cdot \BA)
%+
%\ncap \cdot (\ncap \wedge \BA)
\end{dmath}
%
\index{grade one selection}
The last dot product can be expanded as a grade one (vector) selection
%
\begin{dmath}\label{eqn:emtProblemSet3Problem3:100}
\ncap (\ncap \wedge \BA)
=
\gpgradeone{
\ncap (\ncap \wedge \BA)
}
=
\gpgradeone{
\ncap I (\ncap \cross \BA)
}
=
I^2 \ncap \cross (\ncap \cross \BA)
=
- \ncap \cross (\ncap \cross \BA),
\end{dmath}
%
\index{normal component}
\index{tangential component}
so the decomposition of a vector \( \BA \) in terms of its normal and tangential projections is
\begin{dmath}\label{eqn:emtProblemSet3Problem3:120}
\BA
=
\ncap (\ncap \cdot \BA)
-
\ncap \cross (\ncap \cross \BA).
\end{dmath}
%
I'm not sure how to naturally determine this relationship using traditional vector algebra.  However, it can be verified by expanding the triple cross product in coordinates using tensor contraction formalism
%
\index{triple cross product}
\begin{dmath}\label{eqn:emtProblemSet3Problem3:140}
-\ncap \cross (\ncap \cross \BA)
=
-\epsilon_{xyz} \Be_x n_y \lr{\ncap \cross \BA}_z
=
-\epsilon_{xyz} \Be_x n_y \epsilon_{zrs} n_r A_s
=
-\delta_{xy}^{[rs]}
\Be_x n_y n_r A_s
=
-\Be_x n_y \lr{ n_x A_y -n_y A_x }
= -\ncap (\ncap \cdot \BA) + (\ncap \cdot \ncap) \BA
= \BA - \ncap (\ncap \cdot \BA).
\end{dmath}
%
\index{projection}
\index{rejection}
This last statement illustrates the geometry of this decomposition, showing that the tangential projection (or normal rejection) of a vector is really just the vector minus its normal projection.
%
%This can be rearranged to show that the
%\begin{dmath}\label{eqn:emtProblemSet3Problem3:100}
%
%The tangential projection, can also be expanded in dot products
%
%\begin{dmath}\label{eqn:emtProblemSet3Problem3:200}
%\ncap (\ncap \wedge \BA)
%=
%\ncap \cdot (\ncap \wedge \BA)
%=
%\BA - \ncap (\ncap \cdot \BA)
%\end{dmath}

      %
% Copyright � 2016 Peeter Joot.  All Rights Reserved.
% Licenced as described in the file LICENSE under the root directory of this GIT repository.
%
\makeproblem{Conductor charge distribution on surface.}{emt:problemSet4:4}{
\index{conductor!charge dissipation}
\index{continuity equation}
We have stated that the boundary condition for a perfect conductor is such that there is
no electric field or charge distribution inside of the conductor. Here we will study the
dynamics of this process. Start with continuity equation
\( \spacegrad \cdot \BJ = -\PDi{t}{\rho} \), where \( \BJ \)
is the
current density [\si{A/m^2}] and \( \rho \) is the charge density [\si{C/m^3}]. Show that a charge (charge
density) placed inside a conductor will decay in an exponential manner.
} % makeproblem
\makeanswer{emt:problemSet4:4}{\withproblemsetsParagraph{
If we assume that the induced charge density is related by the conductance to the electric field
\begin{dmath}\label{eqn:emtProblemSet4Problem4:20}
\BJ = \sigma \BE,
\end{dmath}
then the continuity equation can be written as
%
\begin{dmath}\label{eqn:emtProblemSet4Problem4:40}
\PD{t}{\rho}
=
-\spacegrad \cdot \BJ
=
- \sigma \spacegrad \cdot \BE
=
- \frac{\sigma}{\epsilon} \rho.
\end{dmath}
%
Now we have an equation for \( \rho \), with solution
%\begin{dmath}\label{eqn:emtProblemSet4Problem4:60}
\boxedEquation{eqn:emtProblemSet4Problem4:80}{
\rho = \rho_0 e^{-\sigma t/\epsilon}.
}
%\end{dmath}
%For copper, we have
%
% http://maxwells-equations.com/materials/conductivity.php
%\sigma = 6.3 * 10^7
Given this dispersion relation for the charge density, we can also find the normal component of the current density using the divergence theorem.  Evaluating the continuity equation in an infinitesimal spherical volume of radius \( R \) surrounding the initial charge \( \rho_0 dV \) that was placed at the centre of that sphere, far enough from the boundary of the conductor that we can ignore it, we find
\begin{dmath}\label{eqn:emtProblemSet4Problem4:100}
\oint \ncap \cdot \BJ dA = - \PD{t}{} \int \rho_0 e^{-\sigma t/\epsilon} dV,
\end{dmath}
or
\begin{dmath}\label{eqn:emtProblemSet4Problem4:120}
J_n \Delta A = \frac{\sigma}{\epsilon} \rho_0 e^{-\sigma t/\epsilon} \Delta V.
\end{dmath}
%
Assuming that such a current is radial by symmetry, this provides the dispersion relation for the current out of this same volume
\begin{equation}\label{eqn:emtProblemSet4Problem4:140}
\begin{aligned}
\BJ &= \BJ_0 e^{-\sigma t/\epsilon} \\
\BJ_0 &= \ncap \frac{\sigma}{\epsilon} \frac{R}{3} \rho_0.
\end{aligned}
\end{equation}
We have a positive current out of the volume that the initial charge \( \rho_0 \) was placed in (i.e. the charge leaving that space).  That charge density also decays with time as the charge dissipates, since there is no charge left to flow out of that space.
}}

      %
% Copyright � 2016 Peeter Joot.  All Rights Reserved.
% Licenced as described in the file LICENSE under the root directory of this GIT repository.
%
%{
%\input{../blogpost.tex}
%\renewcommand{\basename}{magneticFieldFromMoment}
%%\renewcommand{\dirname}{notes/phy1520/}
%\renewcommand{\dirname}{notes/ece1228-electromagnetic-theory/}
%%\newcommand{\dateintitle}{}
%%\newcommand{\keywords}{}
%
%\input{../latex/peeter_prologue_print2.tex}
%
%\usepackage{peeters_layout_exercise}
%\usepackage{peeters_braket}
%\usepackage{peeters_figures}
%\usepackage{siunitx}
%%\usepackage{mhchem} % \ce{}
%%\usepackage{macros_bm} % \bcM
%%\usepackage{txfonts} % \ointclockwise
%
%\beginArtNoToc
%
%\generatetitle{Calculating the magnetostatic field from the moment}
%\chapter{Calculating the magnetostatic field from the moment}
%\label{chap:magneticFieldFromMoment}
% \citep{jackson1975cew}
%
\makeproblem{Magnetic field from moment.}{problem:magneticFieldFromMoment:1}{
The vector potential, to first order, for a magnetostatic localized current distribution was found to be
%
\begin{dmath}\label{eqn:magneticFieldFromMoment:20}
\BA(\Bx) = \frac{\mu_0}{4 \pi} \frac{\Bm \cross \Bx}{\Abs{\Bx}^3}.
\end{dmath}
%
Use this to calculate the magnetic field.
} % problem
%
\makeanswer{problem:magneticFieldFromMoment:1}{\withproblemsetsParagraph{
%
\index{magnetic field}
\index{magnetic moment}
\index{triple cross product}
\begin{dmath}\label{eqn:magneticFieldFromMoment:40}
\BB
=
\frac{\mu_0}{4 \pi}
\spacegrad \cross \lr{ \Bm \cross \frac{\Bx}{r^3} }
=
-\frac{\mu_0}{4 \pi}
\spacegrad \cdot \lr{ \Bm \wedge \frac{\Bx}{r^3} }
=
-\frac{\mu_0}{4 \pi}
\lr{
(\Bm \cdot \spacegrad) \frac{\Bx}{r^3}
-\Bm \spacegrad \cdot \frac{\Bx}{r^3}
}
=
\frac{\mu_0}{4 \pi}
\lr{
-\frac{(\Bm \cdot \spacegrad) \Bx}{r^3}
- \lr{ \Bm \cdot \lr{\spacegrad \inv{r^3} }} \Bx
+\Bm (\spacegrad \cdot \Bx) \inv{r^3}
+\Bm \lr{\spacegrad \inv{r^3} } \cdot \Bx
}.
\end{dmath}
%
Here I've used \( \Ba \cross \lr{ \Bb \cross \Bc } = -\Ba \cdot \lr{ \Bb \wedge \Bc } \), and then expanded that with \( \Ba \cdot \lr{ \Bb \wedge \Bc } = (\Ba \cdot \Bb) \Bc - (\Ba \cdot \Bc) \Bb \).  Since one of these vectors is the gradient, care must be taken to have it operate on the appropriate terms in such an expansion.
%
Since we have \( \spacegrad \cdot \Bx = 3 \), \( (\Bm \cdot \spacegrad) \Bx = \Bm \), and \( \spacegrad 1/r^n = -n \Bx/r^{n+2} \), this reduces to
%
\begin{dmath}\label{eqn:magneticFieldFromMoment:60}
\BB
=
\frac{\mu_0}{4 \pi}
\lr{
- \frac{\Bm}{r^3}
+ 3 \frac{(\Bm \cdot \Bx) \Bx}{r^5} %
+ 3 \Bm \inv{r^3}
-3 \Bm \frac{\Bx}{r^5} \cdot \Bx
}
=
\frac{\mu_0}{4 \pi}
\frac{3 (\Bm \cdot \ncap) \ncap -\Bm}{r^3},
\end{dmath}
%
which is the desired result.
}} % answer
%
%}
%\EndNoBibArticle

   %
% Copyright � 2016 Peeter Joot.  All Rights Reserved.
% Licenced as described in the file LICENSE under the root directory of this GIT repository.
%
%\input{../blogpost.tex}
%\renewcommand{\basename}{emt5}
%\renewcommand{\dirname}{notes/ece1228/}
%\newcommand{\keywords}{ECE1228H}
%\input{../latex/peeter_prologue_print2.tex}
%
%%\usepackage{ece1228}
%\usepackage{peeters_braket}
%%\usepackage{peeters_layout_exercise}
%\usepackage{peeters_figures}
%\usepackage{macros_cal}
%\usepackage{macros_bm}
%\usepackage{mathtools}
%\usepackage{siunitx}
%
%\beginArtNoToc
%\generatetitle{ECE1228H Electromagnetic Theory.  Lecture 5: Poynting vector.  Taught by Prof.\ M. Mojahedi}
\mychapter{Poynting vector, and time harmonic (phasor) fields.}
\label{chap:emt5}
%
\paragraph{Poynting}
\index{Poynting vector}
\index{Poynting theorem}
%
The cross product terms of Maxwell's equation are
\begin{equation}\label{eqn:emtLecture5:120}
\spacegrad \cross \BE
= -\BM_i - \PD{t}{\BB}
= -\BM_i - \BM_d,
\end{equation}
%
where \(\BM_d\) is called the magnetic displacement current here.  For the magnetic curl we have
%
\begin{equation}\label{eqn:emtLecture5:140}
\spacegrad \cross \BH
= \BJ_i + \BJ_c + \PD{t}{\BD}
= \BJ_i + \BJ_c + \BJ_d.
\end{equation}
%
It is left as an exercise to show that
%
\begin{dmath}\label{eqn:emtLecture5:160}
\spacegrad \cdot \lr{ \BE \cross \BH } + \BH \cdot \lr{ \BM_i + \BM_d }  + \BE \cdot \lr{ \BJ_i + \BJ_c + \BJ_d } = 0,
\end{dmath}
%
or
\begin{equation}\label{eqn:emtLecture5:180}
\begin{aligned}
\oint &d\Ba \cdot \lr{ \BE \cross \BH } + \int dV \lr{ \BH \cdot \lr{ \BM_i + \BM_d }  + \BE \cdot \lr{ \BJ_i + \BJ_c + \BJ_d }} \\
&= 0,
\end{aligned}
\end{equation}
%
or
\begin{equation}\label{eqn:emtLecture5:200}
\begin{aligned}
   0 &=
   \oint d\Ba \cdot \lr{ \BE \cross \BH } \\
   &+ \int dV \BH \cdot \BM_i
+ \int dV \BE \cdot \BJ_i
+ \int dV \BE \cdot \BJ_c \\
&+ \int dV \lr{ \BH \cdot \PD{t}{\BB} + \BE \cdot \PD{t}{\BD} }.
\end{aligned}
\end{equation}
%
\index{supplied power density}
Define a supplied power density \( \rho_{\textrm{supp}} \)
%
\begin{dmath}\label{eqn:emtLecture5:220}
-\rho_{\textrm{supp}}
=
 \int dV \BH \cdot \BM_i
+ \int dV \BE \cdot \BJ_i.
\end{dmath}
%
When the medium is not dispersive or lossy, we have
%
\begin{dmath}\label{eqn:emtLecture5:240}
\int dV \BH \cdot \PD{t}{\BB}
=
\mu \int dV \BH \cdot \PD{t}{\BH}
=
\PD{t}{} \int dV \mu \Abs{\BH}^2.
\end{dmath}
%
\index{magnetic energy density}
The units of \( [\mu \Abs{\BH}^2] \) are \si{W}, so one can defined a magnetic energy density \( \mu \Abs{\BH}^2 \), and
%
\begin{dmath}\label{eqn:emtLecture5:260}
W_m =
\int dV \mu \Abs{\BH}^2,
\end{dmath}
%
for
%
\begin{dmath}\label{eqn:emtLecture5:280}
\int dV \BH \cdot \PD{t}{\BB}
=
\PD{t}{W_m}.
\end{dmath}
%
\index{stored magnetic energy}
This is the rate of change of stored magnetic energy [\si{J/s} = \si{W}].
%
Similarly
\begin{dmath}\label{eqn:emtLecture5:300}
\int dV \BE \cdot \PD{t}{\BD}
=
\epsilon
\int dV \BE \cdot \PD{t}{\BE}
=
\PD{t}{} \int dV \epsilon \Abs{\BE}^2.
\end{dmath}
%
\index{electric energy density}
The electric energy density is \( \epsilon \Abs{\BE}^2 \).  Let
%
\begin{dmath}\label{eqn:emtLecture5:320}
W_e =
\int dV \epsilon \Abs{\BE}^2,
\end{dmath}
%
and
\begin{dmath}\label{eqn:emtLecture5:340}
\int dV \BE \cdot \PD{t}{\BD}
=
\PD{t}{W_e}.
\end{dmath}
%
We also have a term
%
\begin{dmath}\label{eqn:emtLecture5:360}
\int dV \BE \cdot \BJ_c
=
\int dV \BE \cdot (\sigma \BE)
=
\int dV \sigma \Abs{\BE}^2.
%\equiv ...
\end{dmath}
%
\index{stored electric energy}
This is the rate of change of stored electric energy.
%
The remaining term is
\begin{dmath}\label{eqn:emtLecture5:380}
\oint d\Ba \cdot \lr{ \BE \cross \BH }.
\end{dmath}
%
This is a density of the power that is leaving the volume.  The vector \( \BE \cross \BH \) is special, called the Poynting vector, and coincidentally points in the direction that the energy leaves the bounding surface per unit time.  We write
%
\begin{dmath}\label{eqn:emtLecture5:400}
\BS = \BE \cross \BH.
\end{dmath}
%
In vacuum the phase velocity \( \Bv_p \), group velocity \( \Bv_g \) and packet(?) velocity \( \Bv_p \) all line up.  This isn't the case in the media.
%
It turns out that without dissipation
%
\begin{dmath}\label{eqn:emtLecture5:420}
\int \BH \cdot \PD{t}{\BB} = \int \BE \cdot \PD{t}{\BD}.
\end{dmath}
%
\index{LC circuit}
For example in an LC circuit \cref{fig:lecture4LCCircuit:lecture4LCCircuitFig1}
half the cycle the energy is stored in the inductor, and in the other half of the cycle the energy is stored in the capacitor.
%
\imageFigure{../figures/ece1228-electromagnetic-theory/lecture4LCCircuitFig1}{LC circuit.}{fig:lecture4LCCircuit:lecture4LCCircuitFig1}{0.2}
%
Summarizing
%
\begin{dmath}\label{eqn:emtLecture5:440}
\oint \lr{ \BE \cross \BH } \cdot d\Ba = P_{\textrm{exit}}.
\end{dmath}
%
\paragraph{Time harmonics}
\index{time harmonics}
%
Recall that we have differential equations to solve for each type of circuit element in the time domain.  For example in \cref{fig:lecture4inductor:lecture4inductorFig2a}, we have
%
\begin{dmath}\label{eqn:emtLecture5:980}
V_i(t) = L \ddt{i},
\end{dmath}
%
\imageFigure{../figures/ece1228-electromagnetic-theory/lecture4inductorFig2a}{Inductor.}{fig:lecture4inductor:lecture4inductorFig2a}{0.2}
%
and for the capacitor sketched in \cref{fig:lecture4cap:lecture4capFig2b}, we have
\begin{dmath}\label{eqn:emtLecture5:1000}
i_c(t) = C \ddt{V_c}.
\end{dmath}
%
\imageFigure{../figures/ece1228-electromagnetic-theory/lecture4capFig2b}{Capacitor.}{fig:lecture4cap:lecture4capFig2b}{0.2}
%
When we use Laplace or Fourier techniques to solve circuits with such differential equation elements.  The price that we paid for that was that we have to start dealing with complex-valued (phasor) quantities.  We can do this for field equations as well.  The goal is to remove the time domain coupling in Maxwell equations like
%
\begin{dmath}\label{eqn:emtLecture5:460}
\spacegrad \cross \BE(\Br, t) = -\PD{t}{\BB}(\Br, t),
\end{dmath}
\begin{dmath}\label{eqn:emtLecture5:480}
\spacegrad \cross \BH(\Br, t) = \sigma \BE + \PD{t}{\BD}(\Br, t).
\end{dmath}
%
For a single frequency, assume that the time dependency can be written as
%
\begin{dmath}\label{eqn:emtLecture5:500}
\BE(\Br, t) = \Real \lr{ \BE^\conj(\Br) e^{j \omega t} }.
\end{dmath}
%
We may now have to require \( \BE(\Br) \) to be complex valued.
We also have to be really careful about which convention of the time domain solution we are going to use, since we could just as easily use
%
\begin{dmath}\label{eqn:emtLecture5:720}
\BE(\Br, t) = \Real \lr{ \BE(\Br) e^{-j \omega t} }.
\end{dmath}
%
For example
\begin{dmath}\label{eqn:emtLecture5:840}
\Real( e^{i k z} e^{-i\omega t} ) = \cos( k z - \omega t ),
\end{dmath}
%
is identical with
\begin{dmath}\label{eqn:emtLecture5:860}
\Real( e^{-j k z} e^{j\omega t} ) = \cos( \omega t -k z),
\end{dmath}
%
showing that a solution or its complex conjugate is equally valid.
%
Engineering books use \( e^{j \omega t} \) whereas most physicists use \( e^{-i \omega t } \).
%
What if we have more complex time dependencies, such as that sketched in \cref{fig:lecture4NonSine:lecture4NonSineFig3}?
%
\imageFigure{../figures/ece1228-electromagnetic-theory/lecture4NonSineFig3}{Non-sinusoidal time dependence.}{fig:lecture4NonSine:lecture4NonSineFig3}{0.2}
%
We can do this using Fourier superposition, adding a finite or infinite set of single frequency solutions.  The first order of business is to solve the system for a single frequency.
%
Let's write our Fourier transform pairs as
\begin{subequations}
\label{eqn:emtLecture5:520}
\begin{equation}\label{eqn:emtLecture5:540}
\calF(\BA(\Br, t)) =
\BA(\Br, \omega)
=
\int_{-\infty}^\infty \BA(\Br, t) e^{-j \omega t} dt,
\end{equation}
\begin{equation}\label{eqn:emtLecture5:560}
\BA(\Br, t) = \calF^{-1}(\BA(\Br, \omega))
=
\inv{2\pi}
\int_{-\infty}^\infty \BA(\Br, \omega) e^{j \omega t} d\omega.
\end{equation}
\end{subequations}
%
In particular
%
\begin{equation}\label{eqn:emtLecture5:580}
\calF\lr{ \ddt{f(t)} } = j \omega F(\omega),
\end{equation}
%
so the Fourier transform of the Maxwell equation
\begin{dmath}\label{eqn:emtLecture5:600}
\calF\lr{ \spacegrad \cross \BE(\Br, t) }
=
\calF\lr{ -\PD{t}{\BB}(\Br, t) },
\end{dmath}
%
is
%
\begin{dmath}\label{eqn:emtLecture5:620}
\spacegrad \cross \BE(\Br, \omega) = - j\omega \BB(\Br, \omega).
\end{dmath}
%
The four Maxwell's equations can be written as
%
\begin{itemize}
\item Faraday's Law:
\begin{dmath}\label{eqn:emtLecture5:640}
\spacegrad \cross \BE( \Br, \omega ) = - j \omega \BB(\Br, \omega) - \BM_i.
\end{dmath}
\item Ampere-Maxwell equation:
\begin{dmath}\label{eqn:emtLecture5:660}
\spacegrad \cross \BH( \Br, \omega ) = \BJ_\txtc(\Br, \omega) + \BD(\Br, \omega).
\end{dmath}
\item Gauss's law:
\begin{dmath}\label{eqn:emtLecture5:680}
\spacegrad \cdot \BD(\Br, \omega) = \rho_{\txte\txtv}(\Br, \omega).
\end{dmath}
\item Gauss's law for magnetism:
\begin{dmath}\label{eqn:emtLecture5:700}
\spacegrad \cdot \BB(\Br, \omega) = \rho_{\txtm\txtv}(\Br, \omega).
\end{dmath}
\end{itemize}
%
Now we can more easily model non-simple media with
%
\begin{equation}\label{eqn:emtLecture5:740}
\begin{aligned}
\BB(\Br, \omega) &= \mu(\omega) \BH(\Br, \omega), \\
\BD(\Br, \omega) &= \epsilon(\omega) \BE(\Br, \omega).
\end{aligned}
\end{equation}
%
so Maxwell's equations are
%
\begin{dmath}\label{eqn:emtLecture5:760}
\spacegrad \cross \BE( \Br, \omega ) = - j \omega \mu(\omega) \BH(\Br, \omega) - \BM_i,
\end{dmath}
\begin{dmath}\label{eqn:emtLecture5:780}
\spacegrad \cross \BH( \Br, \omega ) = \BJ_\txtc(\Br, \omega) + \epsilon(\omega) \BE(\Br, \omega),
\end{dmath}
\begin{dmath}\label{eqn:emtLecture5:800}
\epsilon(\omega) \spacegrad \cdot \BE(\Br, \omega) = \rho_{\txte\txtv}(\Br, \omega),
\end{dmath}
\begin{dmath}\label{eqn:emtLecture5:820}
\mu(\omega) \spacegrad \cdot \BH(\Br, \omega) = \rho_{\txtm\txtv}(\Br, \omega).
\end{dmath}
%
\paragraph{Frequency domain Poynting}
\index{Poynting vector!frequency domain}
%
The frequency domain (time harmonic) equivalent of the instantaneous Poynting theorem is
%
\begin{dmath}\label{eqn:emtLecture5:880}
\inv{2} \oint d\Ba \cdot \lr{ \BE \cross \BH^\conj }
- \inv{2} \int dV \lr{ \BH^\conj \cdot \BM_i + \BE \cdot \BJ_i^\conj }
+ \inv{2} \int dV \sigma \Abs{\BE}^2
+ j \omega \inv{2} \int dV \lr{ \mu \Abs{\BH}^2 - \epsilon \Abs{\BE}^2 } = 0.
\end{dmath}
%
Showing this is left as an exercise.
%Showing this will probably be given as homework.
Since
%
\begin{dmath}\label{eqn:emtLecture5:900}
\Real(\BA) \cross \Real(\BB) \ne \Real( \BA \cross \BB ).
\end{dmath}
%
We want to find the instantaneous Poynting vector in terms of the phasor fields.  Following
\citep{balanis1989advanced}, where script is used for the instantaneous quantities and non-script for the phasors, we find
%
\begin{dmath}\label{eqn:emtLecture5:920}
\bcS(\Br, t)
= \bcE(\Br, t) \cross \bcH(\Br, t)
= \Real(\bcE(\Br, t)) \cross \Real(\bcH(\Br, t))
=
\frac{ \BE e^{j\omega t} + \BE^\conj e^{-j \omega t}}{2}
\cross
\frac{ \BH e^{j\omega t} + \BH^\conj e^{-j \omega t}}{2}
=
\inv{4}
\lr{
\BE \cross \BH^\conj + \BE^\conj \cross \BH
+
\BE \cross \BH e^{2 j\omega t}
+
\BH \cross \BE e^{-2 j\omega t}
}
=
\inv{2} \Real(\BE \cross \BH^\conj) + \inv{2} \Real( \BE \cross \BH  e^{2 j\omega t} ).
\end{dmath}
%
Should we time average over a period \( \expectation{.} = (1/T) \int_0^T (.) \) the second term is killed, so that
%
\begin{dmath}\label{eqn:emtLecture5:940}
\expectation{ \bcS }
=
\inv{2} \Real(\BE \cross \BH^\conj) + \inv{2} \Real( \BE \cross \BH  e^{2 j\omega t} ).
\end{dmath}
%
The instantaneous Poynting vector is thus
\begin{dmath}\label{eqn:emtLecture5:960}
\bcS(\Br, t) = \expectation{\BS} + \inv{2} \Real\lr{ \BE \cross \BH e^{j \omega t} }.
\end{dmath}
%
%\EndArticle

      \section{Problems.}
      %
% Copyright � 2016 Peeter Joot.  All Rights Reserved.
% Licenced as described in the file LICENSE under the root directory of this GIT repository.
%
\makeproblem{Index of refraction.}{emt:problemSet4:1}{
\index{index of refraction}
\index{doppler shift}
Transmitter \( T \) of a time-harmonic wave of frequency \( \nu \) moves with velocity \( \BU \)
at
an angle \( \theta \) relative to the direct line to a stationary receiver \( R \), as sketched in
\cref{fig:ps4:ps4Fig1}.
\imageFigure{../figures/ece1228-electromagnetic-theory/ps4Fig1}{Field refraction.}{fig:ps4:ps4Fig1}{0.15}
\makesubproblem{}{emt:problemSet4:1a}
Derive the expression for the frequency detected by the receiver \(R\), assuming that the
medium between \(T\) and \(R\) has a positive index of refraction \(n\). (Apply the appropriate
approximations.)
\makesubproblem{}{emt:problemSet4:1b}
How is the expression obtained in
\partref{emt:problemSet4:1a}
is modified if the medium is a metamaterial
with negative index of refraction.
\makesubproblem{}{emt:problemSet4:1c}
From the physical point of view, how is the situation in
\partref{emt:problemSet4:1b}
different from
\partref{emt:problemSet4:1a}
?
} % makeproblem
\skipIfRedacted{
\makeanswer{emt:problemSet4:1}{
\makeSubAnswer{}{emt:problemSet4:1a}
Instead of considering a moving source, we can flip the problem and consider a stationary source and moving target as sketched in \cref{fig:transmitRecieveMovingTarget:transmitRecieveMovingTargetFig1}.
\imageFigure{../figures/ece1228-electromagnetic-theory/transmitRecieveMovingTargetFig1}{Moving target.}{fig:transmitRecieveMovingTarget:transmitRecieveMovingTargetFig1}{0.2}
If the source is emitting a spherical wave
\begin{equation}\label{eqn:emtProblemSet4Problem1:20}
\psi = \frac{e^{j(\omega t - k r)}}{r},
\end{equation}
this wave will reach the target when
\begin{equation}\label{eqn:emtProblemSet4Problem1:40}
\begin{aligned}
r
&= \Abs{\BR'}
\\ &= \Abs{\Br_0 - \BU \Delta t}
\\ &= \sqrt{ \Br_0^2 + \BU^2 (\Delta t)^2 - 2 \Br_0 \cdot \BU \Delta t}
\\ &= r_0 \sqrt{ 1 + \frac{\BU^2 (\Delta t)^2}{r_0^2} - \frac{2 r_0 U \cos\theta \Delta t}{r_0^2} }
\approx r_0 \sqrt{ 1 - \frac{U \cos\theta \Delta t}{r_0} }
\approx r_0 - U \cos\theta \Delta t.
\end{aligned}
\end{equation}

Here a small time approximation has been used to discard the term quadratic in \( \Delta t\).  After that a first order Taylor expansion assuming \( \Br_0^2 \gg \Abs{2 \Br_0 \cdot \BU \Delta t} \).  That is, an assumption that the target isn't moving that far relative to the initial separation in the time it takes for the wave to reach the target.  Setting the initial time to zero, so that \( \Delta t = t \), the wave at the target point is approximately
%
\begin{equation}\label{eqn:emtProblemSet4Problem1:60}
\begin{aligned}
\psi
&= \frac{\exp\lr{j \lr{\omega t - k \lr{ r_0 - U \cos\theta \Delta t } } } }{\Abs{\BR'}}
\\ &= \frac{\exp\lr{j \lr{ \omega + k U \cos\theta} t - k r_0 } }{\Abs{\BR'}}.
\end{aligned}
\end{equation}
%
In order to interpret this as a frequency shift, we need the relations between \( k \) and \( \omega \).  When I wrote \( \psi \) above, I really meant one of the components of the electric or magnetic fields, subject to the wave equations
\begin{equation}\label{eqn:emtProblemSet4Problem1:80}
\begin{aligned}
\spacegrad^2 \bcE &= \mu \epsilon \PDSq{t}{\bcE}, \\
\spacegrad^2 \bcB &= \mu \epsilon \PDSq{t}{\bcB},
\end{aligned}
\end{equation}
where \( v = 1/\sqrt{\mu \epsilon} \) is the propagation speed of the wave.
%
With a time harmonic representation, say \( \bcE = \Real\lr{ \BE e^{j \omega t} } \), the electric (or magnetic) field equation takes the form
\begin{equation}\label{eqn:emtProblemSet4Problem1:100}
\spacegrad^2 \BE = -\mu \epsilon \omega^2 \BE = -k^2 \BE,
\end{equation}
so
\begin{equation}\label{eqn:emtProblemSet4Problem1:120}
\begin{aligned}
k &= \frac{\omega}{v} \\ &= \frac{\omega}{c} \frac{c}{v} \\ &= \frac{n}{c} \omega.
\end{aligned}
\end{equation}

The frequency shift observed at the target is therefore
\boxedEquation{eqn:emtProblemSet4Problem1:140}{
\nu' = \nu \lr{ 1 + \frac{n}{c} U \cos\theta }.
}
%
As a sign check, consider \( \theta = 0, n = 1 \), where the target and the source are moving directly towards each other, with the light travelling in vacuum.  In this case, we recover the desired blue shift approximation, with a higher frequency observed at the target
\begin{equation}\label{eqn:emtProblemSet4Problem1:160}
\nu' = \nu \lr{ 1 + \frac{U}{c} }.
\end{equation}
\makeSubAnswer{}{emt:problemSet4:1b}
Nothing in the analysis above depended on the sign of the index of refraction.  Should that sign be negative, \cref{eqn:emtProblemSet4Problem1:140} is still valid.  However, in such a case, the frequency is decreased, instead of increased.
\makeSubAnswer{}{emt:problemSet4:1c}
A negative index of refraction introduces a red shift instead of a blue shift when the targets are actually moving towards each other.  The observed frequency shift is as if the source and target points were actually receding from each other, instead of advancing (or advancing when actually receding).
}}

      %
% Copyright � 2016 Peeter Joot.  All Rights Reserved.
% Licenced as described in the file LICENSE under the root directory of this GIT repository.
%
\makeproblem{Phasor equality.}{emt:problemSet4:2}{
\index{phasor}
Prove that if
%
\begin{equation}\label{eqn:emtProblemSet4Problem2:20}
\Real\lr{ \BA(\Br) e^{j \omega t}} = \Real\lr{ \BB(\Br) e^{j \omega t}},
\end{equation}
%
then
\( \BA(\Br) = \BB(\Br) \).
This means that the \( \Real() \)
operator can be removed on phasors of the same frequency.
} % makeproblem
%
\skipIfRedacted{
\makeanswer{emt:problemSet4:2}{
Let
\begin{equation}\label{eqn:emtProblemSet4Problem2:40}
\begin{aligned}
\BA &= \BA_r + j \BA_i \\
\BB &= \BB_r + j \BB_i.
\end{aligned}
\end{equation}
%
Expanding \cref{eqn:emtProblemSet4Problem2:20}, for two equal time domain representation of the phasors we have
\begin{equation}\label{eqn:emtProblemSet4Problem2:60}
\BA_r \cos(\omega t) - \BA_i \sin(\omega t)
=
\BB_r \cos(\omega t) - \BB_i \sin(\omega t).
\end{equation}
%
This is required to be true for all times.  If that is the case, then at \( t = 0 \) we have
\begin{equation}\label{eqn:emtProblemSet4Problem2:80}
\BA_r
=
\BB_r,
\end{equation}
and at time \( t = -\pi/(2\omega) \),
\begin{equation}\label{eqn:emtProblemSet4Problem2:100}
\BA_i
=
\BB_i.
\end{equation}
%
Equality of \( \BA_r = \BB_r \), and \( \BA_i = \BB_r \) also implies that the imaginary parts are equal, since
\begin{equation}\label{eqn:emtProblemSet4Problem2:120}
\begin{aligned}
\Imag\lr{ \BA e^{j\omega t} } - \Imag\lr{ \BB e^{j\omega t} }
&=
\BA_r \sin(\omega t) + \BA_i \cos(\omega t)
-\lr{ \BB_r \sin(\omega t) + \BB_i \cos(\omega t) }
\\ &=
\lr{ \BA_r - \BB_r } \sin(\omega t)
+
\lr{ \BA_i - \BB_i } \cos(\omega t)
\\ &=
0.
\end{aligned}
\end{equation}
}}

      %
% Copyright � 2016 Peeter Joot.  All Rights Reserved.
% Licenced as described in the file LICENSE under the root directory of this GIT repository.
%
\makeproblem{Duality theorem.}{emt:problemSet4:3}{
\index{duality theorem}
Prove that if the time-harmonic fields \( \BE(\Br) \) and \( \BH(\Br) \)
are solutions to Maxwell's
equations in a simple, source free medium ( \( \BM_i = \BJ_i = \BJ_c = 0, \rho_{mv} = \rho_{ev} = 0 \) ),
characterized by \( \epsilon, \mu \) ; then
\( \BE'(\Br) = \eta \BH(\Br) \) and
\( \BH'(\Br) = -\frac{\BE(\Br)}{\eta} \)
are also solutions of
the Maxwell equations.
\( \eta \)
is the intrinsic impedance of the medium.
\paragraph{Remark}: By showing the above you have proved the validity of the so called duality
theorem.
} % makeproblem
\makeanswer{emt:problemSet4:3}{\withproblemsetsParagraph{
In source free simple media, Maxwell's time-harmonic equations are
\begin{equation}\label{eqn:emtProblemSet4Problem3:20}
\begin{aligned}
\spacegrad \cross \BH &= j \omega \epsilon \BE, \\
\spacegrad \cross \BE &= -j \omega \mu \BH, \\
\spacegrad \cdot \BH &= 0, \\
\spacegrad \cdot \BE &= 0.
\end{aligned}
\end{equation}
Inserting \( \BH = \BE'/\eta, \BE = -\eta \BH' \), these are
\begin{equation}\label{eqn:emtProblemSet4Problem3:40}
\begin{aligned}
\spacegrad \cross \BE' &= -j \omega \epsilon \eta^2 \BH', \\
\spacegrad \cross \BH' &= j \omega \frac{\mu}{\eta^2} \BE', \\
\spacegrad \cdot \BE' &= 0, \\
\spacegrad \cdot \BH' &= 0,
\end{aligned}
\end{equation}
we see that \( \BE', \BH' \) are solutions provided
\begin{equation}\label{eqn:emtProblemSet4Problem3:60}
\begin{aligned}
\epsilon &= \frac{\mu}{\eta^2}, \\
\mu &= \epsilon \eta^2,
\end{aligned}
\end{equation}
or
\begin{dmath}\label{eqn:emtProblemSet4Problem3:80}
\eta^2 = \frac{\mu}{\epsilon}.
\end{dmath}
}}

      %%
% Copyright � 2016 Peeter Joot.  All Rights Reserved.
% Licenced as described in the file LICENSE under the root directory of this GIT repository.
%
%{
%\input{../blogpost.tex}
%\renewcommand{\basename}{poynting}
%%\renewcommand{\dirname}{notes/phy1520/}
%\renewcommand{\dirname}{notes/ece1228-electromagnetic-theory/}
%%\newcommand{\dateintitle}{}
%%\newcommand{\keywords}{}
%
%\input{../latex/peeter_prologue_print2.tex}
%
%\usepackage{peeters_layout_exercise}
%\usepackage{peeters_braket}
%\usepackage{peeters_figures}
%\usepackage{siunitx}
%%\usepackage{txfonts} % \ointclockwise
%
%\beginArtNoToc
%
%\generatetitle{Poynting relationship}
%\chapter{Poynting relationship}
%\label{chap:poynting}
%
\makeproblem{Poynting theorem.}{problem:poynting:1}{
%
} % problem
%
\makeanswer{problem:poynting:1}{
%\withproblemsetsParagraph{
%
%}
} % answer
%
%}
%\EndArticle

      %
% Copyright � 2016 Peeter Joot.  All Rights Reserved.
% Licenced as described in the file LICENSE under the root directory of this GIT repository.
%
\makeproblem{Poynting theorem.}{emt:problemSet7:1}{
Using Maxwell's equations given in the class notes, derive the Poynting theorem in both
differential and integral form for instantaneous fields. Assume a linear, homogeneous medium
with no temporal dispersion.
} % makeproblem
%
\skipIfRedacted{
\makeanswer{emt:problemSet7:1}{
%
Given
\begin{equation}\label{eqn:emtproblemSet7Problem1:20}
\spacegrad \cross \BE
= -\BM_i - \PD{t}{\BB},
\end{equation}
%
and
\begin{equation}\label{eqn:emtproblemSet7Problem1:40}
\spacegrad \cross \BH
= \BJ_i + \BJ_c + \PD{t}{\BD},
\end{equation}
%
we want to expand the divergence of \( \BE \cross \BH \) to find the form of the Poynting theorem.
%
%
First we need the chain rule for of this sort of divergence.  Using primes to indicate the scope of the gradient operation
%
\begin{equation}\label{eqn:emtproblemSet7Problem1:60}
\begin{aligned}
\spacegrad \cdot \lr{ \BE \cross \BH }
&=
\spacegrad' \cdot \lr{ \BE' \cross \BH }
-
\spacegrad' \cdot \lr{ \BH' \cross \BE }
\\ &=
\BH \cdot \lr{ \spacegrad' \cross \BE' }
-
\BH \cdot \lr{ \spacegrad' \cross \BH' }
%\\ &= %\gpgradezero{ %\spacegrad \lr{ \BE \cross \BH }
%}
%\\ &=
%\gpgradezero{
%-I \spacegrad \lr{ \BE \wedge \BH }
%}
%\\ &=
%-I \spacegrad \wedge \lr{ \BE \wedge \BH }
%\\ &=
%-I \BH \wedge \lr{ \spacegrad \wedge \BE }
%+I \BE \wedge \lr{ \spacegrad \wedge \BH }
\\ &=
\BH \cdot \lr{ \spacegrad \cross \BE }
-
\BE \cdot \lr{ \spacegrad \cross \BH }.
\end{aligned}
\end{equation}
%
In the second step, cyclic permutation of the triple product was used.
This checks against the inside front cover of Jackson \citep{jackson1975cew}.  Now we can plug in the Maxwell equation cross products.
%
\begin{equation}\label{eqn:emtproblemSet7Problem1:80}
\begin{aligned}
\spacegrad \cdot \lr{ \BE \cross \BH }
&=
\BH \cdot \lr{ -\BM_i - \PD{t}{\BB} }
-
\BE \cdot \lr{ \BJ_i + \BJ_c + \PD{t}{\BD} }
\\ &=
-\BH \cdot \BM_i
-\mu \BH \cdot \PD{t}{\BH}
-
\BE \cdot \BJ_i
-
\BE \cdot \BJ_c
-
\epsilon \BE \cdot \PD{t}{\BE},
\end{aligned}
\end{equation}
%
or
%
%\begin{equation}\label{eqn:emtproblemSet7Problem1:100}
\boxedEquation{eqn:emtproblemSet7Problem1:120}{
0
=
\spacegrad \cdot \lr{ \BE \cross \BH }
+ \frac{\epsilon}{2} \PD{t}{} \Abs{ \BE }^2
+ \frac{\mu}{2} \PD{t}{} \Abs{ \BH }^2
+ \BH \cdot \BM_i
+ \BE \cdot \BJ_i
+ \sigma \Abs{\BE}^2.
}
%\end{equation}
%
In integral form this is
%
\begin{equation}\label{eqn:emtproblemSet7Problem1:140}
\begin{aligned}
0
&=
\int d\BA \cdot \lr{ \BE \cross \BH }
+ \inv{2} \PD{t}{} \int dV \lr{
\epsilon \Abs{ \BE }^2
+ \mu \Abs{ \BH }^2
} 
\\
&\quad
+ \int dV \BH \cdot \BM_i
+ \int dV \BE \cdot \BJ_i
+ \sigma \int dV \Abs{\BE}^2.
\end{aligned}
\end{equation}
}}

      %
% Copyright � 2016 Peeter Joot.  All Rights Reserved.
% Licenced as described in the file LICENSE under the root directory of this GIT repository.
%
%{
%\input{../blogpost.tex}
%\renewcommand{\basename}{poyntingTimeHarmonic}
%%\renewcommand{\dirname}{notes/phy1520/}
%\renewcommand{\dirname}{notes/ece1228-electromagnetic-theory/}
%%\newcommand{\dateintitle}{}
%%\newcommand{\keywords}{}
%
%\input{../latex/peeter_prologue_print2.tex}
%
%\usepackage{peeters_layout_exercise}
%\usepackage{peeters_braket}
%\usepackage{peeters_figures}
%\usepackage{siunitx}
%\usepackage{macros_bm}
%%\usepackage{txfonts} % \ointclockwise
%
%\beginArtNoToc
%
%\generatetitle{Frequency domain time averaged Poynting theorem}
%\chapter{Frequency domain time averaged Poynting theorem}
%\label{chap:poyntingTimeHarmonic}
\makeproblem{Frequency domain time averaged Poynting theorem}{problem:poyntingTimeHarmonic:1}{
The time domain Poynting relationship was found to be
\begin{equation}\label{eqn:poyntingTimeHarmonic:20}
\begin{aligned}
0
&=
\spacegrad \cdot \lr{ \BE \cross \BH }
+ \frac{\epsilon}{2} \BE \cdot \PD{t}{\BE}
+ \frac{\mu}{2} \BH \cdot \PD{t}{\BH} \\
&\qquad + \BH \cdot \BM_i
+ \BE \cdot \BJ_i
+ \sigma \BE \cdot \BE.
\end{aligned}
\end{equation}
Derive the equivalent relationship for the time averaged portion of the time-harmonic Poynting vector.
} % problem
\makeanswer{problem:poyntingTimeHarmonic:1}{\withproblemsetsParagraph{
The time domain representation of the Poynting vector in terms of the time-harmonic (phasor) vectors is
\begin{dmath}\label{eqn:poyntingTimeHarmonic:40}
\bcE \cross \bcH
= \inv{4}
\lr{
\BE e^{j\omega t}
+ \BE^\conj e^{-j\omega t}
}
\cross
\lr{
\BH e^{j\omega t}
+ \BH^\conj e^{-j\omega t}
}
=
\inv{2} \Real \lr{ \BE \cross \BH^\conj + \BE \cross \BH e^{2 j \omega t} },
\end{dmath}
so if we are looking for the relationships that effect only the time averaged Poynting vector, over integral multiples of the period, we are interested in evaluating the divergence of
\begin{dmath}\label{eqn:poyntingTimeHarmonic:60}
\inv{2} \BE \cross \BH^\conj.
\end{dmath}
The time-harmonic Maxwell's equations are
\begin{equation}\label{eqn:poyntingTimeHarmonic:80}
\begin{aligned}
\spacegrad \cross \BE &= - j \omega \mu \BH - \BM_i, \\
\spacegrad \cross \BH &= j \omega \epsilon \BE + \BJ_i + \sigma \BE.
\end{aligned}
\end{equation}
The latter after conjugation is
\begin{dmath}\label{eqn:poyntingTimeHarmonic:100}
\spacegrad \cross \BH^\conj = -j \omega \epsilon^\conj \BE^\conj + \BJ_i^\conj + \sigma^\conj \BE^\conj.
\end{dmath}
For the divergence we have
\begin{equation}\label{eqn:poyntingTimeHarmonic:120}
\begin{aligned}
\spacegrad &\cdot \lr{ \BE \cross \BH^\conj } \\
&=
\BH^\conj \cdot \lr{ \spacegrad \cdot \BE }
-\BE \cdot \lr{ \spacegrad \cdot \BH^\conj } \\
&=
\BH^\conj \cdot \lr{ - j \omega \mu \BH - \BM_i }
- \BE \cdot \lr{ -j \omega \epsilon^\conj \BE^\conj + \BJ_i^\conj + \sigma^\conj \BE^\conj },
\end{aligned}
\end{equation}
or
\begin{equation}\label{eqn:poyntingTimeHarmonic:140}
\begin{aligned}
0
&=
\spacegrad \cdot \lr{ \BE \cross \BH^\conj } \\
&\qquad +
\BH^\conj \cdot \lr{ j \omega \mu \BH + \BM_i }
+ \BE \cdot \lr{ -j \omega \epsilon^\conj \BE^\conj + \BJ_i^\conj + \sigma^\conj \BE^\conj },
\end{aligned}
\end{equation}
so
%\begin{dmath}\label{eqn:poyntingTimeHarmonic:160}
\boxedEquation{eqn:poyntingTimeHarmonic:160}{
\begin{aligned}
0
&=
\spacegrad \cdot \inv{2} \lr{ \BE \cross \BH^\conj }
+ \inv{2} \lr{ \BH^\conj \cdot \BM_i
+ \BE \cdot \BJ_i^\conj } \\
&\quad + \inv{2} j \omega \lr{ \mu \Abs{\BH}^2 - \epsilon^\conj \Abs{\BE}^2 }
+ \inv{2} \sigma^\conj \Abs{\BE}^2.
\end{aligned}
}
%\end{dmath}
}} % answer
%}
%\EndNoBibArticle

   %
% Copyright � 2016 Peeter Joot.  All Rights Reserved.
% Licenced as described in the file LICENSE under the root directory of this GIT repository.
%
%\input{../blogpost.tex}
%\renewcommand{\basename}{emt6}
%\renewcommand{\dirname}{notes/ece1228/}
%\newcommand{\keywords}{ECE1228H}
%\input{../latex/peeter_prologue_print2.tex}
%
%%\usepackage{ece1228}
%\usepackage{peeters_braket}
%%\usepackage{peeters_layout_exercise}
%\usepackage{peeters_figures}
%\usepackage{mathtools}
%\usepackage{siunitx}
%\usepackage{enumerate}
%
%\beginArtNoToc
%\generatetitle{ECE1228H Electromagnetic Theory.  Lecture 6: Lorentz-Lorenz dispersion.  Taught by Prof.\ M. Mojahedi}
%%\chapter{Lorentz-Lorenz dispersion}
%\label{chap:emt6}

%\paragraph{Disclaimer}
%
%Peeter's lecture notes from class.  These may be incoherent and rough.
%
%These are notes for the UofT course ECE1228H, Electromagnetic Theory, taught by Prof. M. Mojahedi, covering \textchapref{{1}} \citep{balanis1989advanced} content.
%
\paragraph{Lorentz-Lorenz Dispersion}
%
We will model the medium using a frequency representation of the permittivity
%
\begin{equation}\label{eqn:emtLecture6:20}
\begin{aligned}
\epsilon(\omega) &= \epsilon'(\omega) - j \epsilon''(\omega) \\
\mu(\omega) &= \mu'(\omega) - j \mu''(\omega)
\end{aligned}
\end{equation}
%
The real part is the phase, whereas the imaginary part is the loss.
%
\begin{dmath}\label{eqn:emtLecture6:40}
n = \frac{c}{v}
= \frac{\sqrt{\epsilon \mu}}{\sqrt{\epsilon_0 \mu_0}}
= \sqrt{\epsilon_r \mu_r}
\end{dmath}
%
We can also write
%
\begin{dmath}\label{eqn:emtLecture6:60}
n(\omega) = n'(\omega) - j n''(\omega)
\end{dmath}
%
If we are considering an electric dipole
%
\begin{dmath}\label{eqn:emtLecture6:80}
\BP_i = Q_i \Bx_i
\end{dmath}
%
With
%
\begin{dmath}\label{eqn:emtLecture6:100}
\BP = \epsilon_0 \chi_e \BE,
\end{dmath}
%
and a time harmonic representation for the electric field
%
\begin{dmath}\label{eqn:emtLecture6:120}
\BE = \BE_0 e^{j \omega t}.
\end{dmath}
%
The dipole moment is assumed to be
%
\begin{dmath}\label{eqn:emtLecture6:140}
\BP = \lim_{\Delta v \rightarrow 0} \frac{ \sum_{i = 1}^{N \Delta v} \BP_i }{\Delta v}
= \frac{ N \Delta v \Bp}{\Delta v}
= N \Bp
= N Q \Bx.
\end{dmath}
%
%F1:
%
We model the oscillating electron and nucleus as a mass and spring.
This electron oscillator model is often called the Lorentz model.  It is not really a model for atoms as such, but the way that an atom responds to pertubation.  At the time when Lorentz formulated the model it was not known that the nuclei havr massive mass as compared to the electrons.
The Lorentz assumption was that in the absence of applied eletric fields the centroids of positive and neagivve charges coincide, but when a field is applied, the electrons will experience a Lorentz force and will be displaced from their equilibrium position.
The wrote ``the displacement immediately gives rise to a new force by which the particle is pulled back towards its original position, and which we may therefore appropriately distinguish by the name of elastic force.''

The forces of interest are
%
\begin{equation}\label{eqn:emtLecture6:160}
\begin{aligned}
F_{\textrm{friction}} &= -D \frac{dx}{dt} = -D v \\
F_{\textrm{elastic}} &= -S x \\
F_{\textrm{external}} &= Q E = Q E_0 e^{j \omega t}
\end{aligned}
\end{equation}
%
Adding all the forces, the electrical system, in one dimension, can be assumed to have the form
%
\begin{equation}\label{eqn:emtLecture6:180}
F = m \frac{d^2 x}{dt^2}
=
-D \frac{dx}{dt}
-D v \\
-S x \\
+ Q E_0 e^{j \omega t},
\end{equation}
%
or
\begin{dmath}\label{eqn:emtLecture6:200}
\frac{d^2 x}{dt^2} + \frac{D}{m} \ddt{x} + \frac{S}{m} x = \frac{Q E_0}{m} e^{j \omega t}
\end{dmath}
%
Let's define
%
\begin{equation}\label{eqn:emtLecture6:220}
\begin{aligned}
\gamma &= \frac{D}{m} \\
\omega_0^2 &= \frac{S}{m},
\end{aligned}
\end{equation}
%
so that
%
\begin{dmath}\label{eqn:emtLecture6:240}
\frac{d^2 x}{dt^2} + \gamma \ddt{x} + \omega_0^2 x = \frac{Q E_0}{m} e^{j \omega t}.
\end{dmath}
%
\paragraph{Calculating the permittivity and susceptibility}
%
With \( x = x_0 e^{j \omega t} \) we have
%
\begin{dmath}\label{eqn:emtLecture6:260}
x_0 \lr{ -\omega^2 + j\gamma \omega + \omega_0^2 } = \frac{Q E_0}{m},
\end{dmath}
%
or (with \( E = E_0 e^{j \omega t} \)), just
%
\begin{equation}\label{eqn:emtLecture6:280}
x = x_0 e^{j\omega t}
= \frac{Q E}{m \lr{ -\omega^2 + j\gamma \omega + \omega_0^2 } }.
\end{equation}
%
\begin{enumerate}[I]
\item Assume that dipoles are identical
\item Assume no coupling between dipoles
\item There are N dipoles per unit volume.  In other words, N is the number of dipoles per unit volume.
\end{enumerate}
%
The polarization \( P(t) \) is given by
%
\begin{dmath}\label{eqn:emtLecture6:300}
P(t) = N Q x,
\end{dmath}
%
where \( Q \) is the charge associate with the unit dipole.  This has dimensions of [\si{\frac{1}{m^3} \times C \times m}], or [\si{C/m^2}].  This polarization is
%
\begin{dmath}\label{eqn:emtLecture6:440}
P(t)
= \frac{Q^2 N E/m}{\omega_0^2 -\omega^2 + j\gamma \omega }.
\end{dmath}
%
In particular, the ratio of the polarization to the electric field magnitude is
%
\begin{dmath}\label{eqn:emtLecture6:320}
\frac{P}{E}
= \frac{Q^2 N/ m}{\omega_0^2 -\omega^2 + j\gamma \omega }.
\end{dmath}
%
With \( P = \epsilon_0 \chi_e E \), we have
%
\begin{dmath}\label{eqn:emtLecture6:340}
\chi_e = \frac{Q^2 N/ m \epsilon_0}{\omega_0^2 -\omega^2 + j\gamma \omega }.
\end{dmath}
%
Define
%
\begin{dmath}\label{eqn:emtLecture6:360}
\omega_p^2 = \frac{ Q^2 N}{m \epsilon_0},
\end{dmath}
%
which has dimensions [\si{1/s^2}].  Then
%
\begin{dmath}\label{eqn:emtLecture6:380}
\chi_e = \frac{\omega_p^2}{\omega_0^2 -\omega^2 + j\gamma \omega }.
\end{dmath}
%
With \( \epsilon_r = 1 + \chi_e \) we have
%
\begin{equation}\label{eqn:emtLecture6:400}
\epsilon_r
= \frac{\epsilon}{\epsilon_0}
= 1 + \frac{\omega_p^2}{\omega_0^2 -\omega^2 + j\gamma \omega }.
\end{equation}
%
%or
%\begin{dmath}\label{eqn:emtLecture6:420}
%\epsilon_r
%= \frac{ \omega_0^2 -\omega^2 + j\gamma \omega + \omega_p^2}{\omega_0^2 -\omega^2 + j\gamma \omega }
%\end{dmath}
%
One can show that \( \epsilon_r = \epsilon_r' -j \epsilon_r'' \) are given bby
%
\begin{dmath}\label{eqn:emtLecture6:460}
\epsilon_r' = \frac{\omega_p^2 \lr{ \omega_0^2 - \omega^2 } }{ (\omega_0^2 - \omega^2)^2 + (\omega \gamma)^2 } + 1,
\end{dmath}
\begin{dmath}\label{eqn:emtLecture6:480}
\epsilon_r'' = \frac{\omega_p^2 \omega \gamma}{ (\omega_0^2 - \omega^2)^2 + (\omega \gamma)^2 }.
\end{dmath}
%
FIXME: calculate this.
%
\paragraph{No damping}
%
With \( D = 0 \), or \( \gamma = 0 \) then \( \epsilon_r'' = 0 \),
%
\begin{dmath}\label{eqn:emtLecture6:500}
x = \frac{Q E_0/m}{\omega^2 - \omega^2} e^{j \omega t},
\end{dmath}
%
and
\begin{equation}\label{eqn:emtLecture6:520}
\epsilon_r
=
\epsilon_r'
= \frac{\epsilon}{\epsilon_0}
=
1 + \frac{\omega_p^2}{\omega_0^2 - \omega^2}.
\end{equation}
%
This has a curve like \cref{fig:emtL6:emtL6Fig5a}.
%
\imageFigure{../figures/ece1228-electromagnetic-theory/emtL6Fig5a}{Undamped resonance.}{fig:emtL6:emtL6Fig5a}{0.3}
%
instead of the normal damped resonance curve like  \cref{fig:emtL6:emtL6Fig5b}.
%
\imageFigure{../figures/ece1228-electromagnetic-theory/emtL6Fig5b}{Damped resonance.}{fig:emtL6:emtL6Fig5b}{0.3}
%
As \( \omega \rightarrow \omega_0 \), then the displacement \( x \rightarrow \infty \).  The frequency \( \omega_0 \) is called the resonance frequency of the system.
%
If the resonance frequency is zero (free charges), then
%
\begin{dmath}\label{eqn:emtLecture6:540}
\epsilon_r = \epsilon_r' = 1 - \frac{\omega_p^2}{\omega^2},
\end{dmath}
%
which is negative for \( \omega_p > \omega \).
%
When damping is present, the resonance frequency is the root of the characteristic equation of the homogeneous part of \cref{eqn:emtLecture6:200}.
%
\paragraph{Multiple resonances}
%
When there are \( N \) molecules per unit volume, and each molecule has
Z electrons per molecule that have a binding frequency \( \omega_i \) and damping constant \( \gamma_i \), then it can be shown that
%
\begin{dmath}\label{eqn:emtLecture6:560}
\epsilon_r = 1 + \frac{Q N^2}{m \epsilon_0} \sum \frac{ f_i }{\omega_0^2 - \omega^2 + j \gamma \omega }
\end{dmath}
%
A quantum mechanical derivation of the transition frequencies is used to derive this multiple resonance result.
%
%\EndArticle
%\EndNoBibArticle

      \section{Problems.}
      %
% Copyright � 2016 Peeter Joot.  All Rights Reserved.
% Licenced as described in the file LICENSE under the root directory of this GIT repository.
%
\makeproblem{Passive medium.}{emt:problemSet5:1}{
Parameters for \ce{AlGaN} (a passive medium) are given as
\begin{equation}\label{eqn:emtProblemSet5Problem1:20}
\begin{aligned}
\omega_0 &= 1.921 \times 10^{14} \si{rad/s}, \\
\omega_p &= 3.328 \times 10^{14} \si{rad/s}, \\
\gamma   &= 9.756 \times 10^{12} \si{rad/s}.
\end{aligned}
\end{equation}
Assuming Lorentz model:
\makesubproblem{}{emt:problemSet5:1a}
Plot the real and imaginary parts of the index of refraction for the range of \( \omega = 0 \) to \( \omega = 6 \times 10^{14} \).
On the figure identify the region of anomalous dispersion.
\makesubproblem{}{emt:problemSet5:1b}
Plot the real and imaginary parts of the relative permittivity for the same range as in \partref{emt:problemSet5:1a}.

On the figure identify the region of anomalous dispersion.
} % makeproblem
\skipIfRedacted{
\makeanswer{emt:problemSet5:1}{
\makeSubAnswer{}{emt:problemSet5:1a}
Given the relative permittivity
\begin{equation}\label{eqn:emtProblemSet5Problem1:60}
\epsilon_r = 1 + \chi_e =
1 + \frac{ \omega_{p} }{\omega_{0}^2 - \omega^2 + j \gamma \omega },
\end{equation}
the index of refraction, for \( \mu \approx \mu_0 \), is
\begin{equation}\label{eqn:emtProblemSet5Problem1:40}
\begin{aligned}
n
&= \frac{c}{v}
\\ &= \frac{\sqrt{\epsilon_0 \epsilon_r \mu}}{\sqrt{\epsilon_0 \mu_0}}
\approx \sqrt{\epsilon_r}.
\end{aligned}
\end{equation}
This is plotted in \cref{fig:p1IndexOfRefraction:p1IndexOfRefractionFig1}.
To be able to flag the portions of the plots that are regions of anomalous dispersion, we have to know what that is.  According to \citep{wiki:opticalDispersion} these are regions where the real part of the index of refraction increases as \( \lambda = 2 \pi/\omega \) increases, or as \( \omega \) decreases.  That is, the portions of the curve where \( d\Real{n}/d\omega < 0 \).  Those portions of the curves are indicated in the plots with a boxed region.
%The plotting code is attached
\mathImageFigure{../figures/ece1228-electromagnetic-theory/p1IndexOfRefractionFig1}{Index of refraction.}{fig:p1IndexOfRefraction:p1IndexOfRefractionFig1}{0.3}{ps5:ps5.nb}
\makeSubAnswer{}{emt:problemSet5:1b}
%\cref{fig:p1EpsilonR:p1EpsilonRFig2}.
\mathImageFigure{../figures/ece1228-electromagnetic-theory/p1EpsilonRFig2}{Relative permittivity.}{fig:p1EpsilonR:p1EpsilonRFig2}{0.3}{ps5:ps5.nb}
}}

      %
% Copyright � 2016 Peeter Joot.  All Rights Reserved.
% Licenced as described in the file LICENSE under the root directory of this GIT repository.
%
\makeproblem{Medium with multiple resonances.}{emt:problemSet5:2}{
Relative permittivity for a medium with multiple resonances is given by:
\begin{equation}\label{eqn:emtProblemSet5Problem2:20}
\epsilon_r = 1 + \chi_e =
1 + \sum_{k=1} \frac{ \omega_{p,k} }{\omega_{0,k}^2 - \omega^2 + j \gamma_k \omega }.
\end{equation}
Moreover, the case of an \textit{active medium} (i.e. medium with gain) can be modeled by allowing \( \omega_{p,k} \)
in above to become purely imaginary.
Under these conditions, plot
\begin{equation}\label{eqn:emtProblemSet5Problem2:40}
\Real\lr{ n(\omega) } -1,
\end{equation}
and
\begin{equation}\label{eqn:emtProblemSet5Problem2:60}
\Imag\lr{ n(\omega) },
\end{equation}
as a function of detuning frequency,
\begin{equation}\label{eqn:emtProblemSet5Problem2:80}
\nu = \frac{\omega - \omega_c}{2 \pi},
\end{equation}
for ammonia vapour (an active medium) where
\begin{equation}\label{eqn:emtProblemSet5Problem2:100}
\begin{aligned}
\omega_{0,1} &= 2.4165825 \times 10^{15} \si{rad/s}, \\
\omega_{0,2} &= 2.4166175 \times 10^{15} \si{rad/s}, \\
\omega_{p,k} = \omega_p &= 10^{10} \si{rad/s}, \\
\gamma_{k} = \gamma &= 5 \times 10^9 \si{rad/s}, \\
\ifrac{(\omega - \omega_c)}{2 \pi} & \in [-7,7] \si{GHz}, \\
\omega_c &= 2.4166 \times 10^{15} \si{rad/s}.
\end{aligned}
\end{equation}
} % makeproblem
\skipIfRedacted{
\makeanswer{emt:problemSet5:2}{
Making the \( \omega_{p,k} \rightarrow j \omega_{p,k} \) transformation toggles the sign of the \(  \omega_{p,k}^2 \) terms in \cref{eqn:emtProblemSet5Problem2:20}, or
\begin{equation}\label{eqn:emtProblemSet5Problem2:41}
\epsilon_r
=
1 - \sum_{k=1} \frac{ \omega_{p,k} }{\omega_{0,k}^2 - \omega^2 + j \gamma_k \omega }.
\end{equation}
The real and imaginary parts of the index of refraction \( n(\omega) = \sqrt{\epsilon_r} \) are plotted as instructed in \cref{fig:p2n:p2nFig3}.
\mathImageFigure{../figures/ece1228-electromagnetic-theory/p2nFig3}{Active medium index of refraction.}{fig:p2n:p2nFig3}{0.3}{ps5:ps5.nb}
}}

      %
% Copyright � 2016 Peeter Joot.  All Rights Reserved.
% Licenced as described in the file LICENSE under the root directory of this GIT repository.
%
\makeproblem{Susceptibility kernel.}{emt:problemSet5:3}{
\makesubproblem{}{emt:problemSet5:3a}
Assuming that a medium is described by the time harmonic relationship \( \BD(\Bx, \omega) = \epsilon(\omega) \BE(\Bx, \Bomega) \), show that the
time domain relation between the electric flux density \( \BD \) and the electric field \( \BE \) is given by,
\begin{equation}\label{eqn:emtProblemSet5Problem3:20}
\BD(\Bx, t) = \epsilon_0
\lr{
\BE(\Bx, t)
+ \int_{-\infty}^\infty G(\tau) \BE(\Bx, t - \tau) d\tau,
}
\end{equation}
where \( G(\tau) \) is the susceptibility kernel given by
\begin{equation}\label{eqn:emtProblemSet5Problem3:40}
G(\tau) =
%\inv{\sqrt{2 \pi}}
\inv{2 \pi}
\int_{-\infty}^\infty
\lr{\frac{\epsilon(\omega)}{\epsilon_0} - 1}
e^{-j \omega t} d\tau.
\end{equation}
\makesubproblem{}{emt:problemSet5:3b}
Show that
\begin{equation}\label{eqn:emtProblemSet5Problem3:60}
\epsilon(-\omega) = \epsilon^\conj(\omega).
\end{equation}
\makesubproblem{}{emt:problemSet5:3c}
Show that for \( \epsilon(\omega) = \epsilon'(\omega) + j \epsilon''(\omega) \),
\( \epsilon'(\omega) \) is even
and \( \epsilon''(\omega) \) is odd.
} % makeproblem
\skipIfRedacted{
\makeanswer{emt:problemSet5:3}{
\makeSubAnswer{}{emt:problemSet5:3a}
The electric displacement field in the time domain is
\begin{dmath}\label{eqn:emtProblemSet5Problem3:80}
\BD(\Bx, t)
= \inv{2\pi} \int_{-\infty}^\infty \epsilon(\omega) \BE(\Bx, \Bomega) e^{-j \omega t} d\omega.
\end{dmath}
Assuming all integrals are over \( [-\infty, \infty] \) we have
\begin{dmath}\label{eqn:emtProblemSet5Problem3:100}
\BD(\Bx, t)
= \inv{2\pi} \int \epsilon(\omega) \BE(\Bx, \Bomega) e^{-j \omega t} d\omega
= \inv{2\pi} \iiint \lr{ \epsilon(\tau) e^{j \omega \tau} d\tau } \lr{ \BE(\Bx, t') e^{j \omega t'} dt'} e^{-j \omega t} d\omega
= \iint \epsilon(\tau) \BE(\Bx, t') d\tau dt'
\inv{2 \pi} \int e^{j \omega \tau} e^{j \omega t'} e^{-j \omega t} d\omega
= \iint \epsilon(\tau) \BE(\Bx, t') d\tau dt'
\inv{2 \pi} \int e^{j \omega (t' - (t - \tau))} d\omega
= \iint \epsilon(\tau) \BE(\Bx, t') d\tau dt' \delta( t' - (t - \tau) )
= \int \epsilon(\tau) \BE(\Bx, t - \tau) d\tau.
\end{dmath}
To this convolution \( \epsilon_0 \BE(\Bx, t) \) can be simultaneously added and subtracted
\begin{dmath}\label{eqn:emtProblemSet5Problem3:120}
\BD(\Bx, t)
=
\epsilon_0 \BE(\Bx, t)
+
\epsilon_0
\int \lr{ \frac{\epsilon(\tau)}{\epsilon_0} - \delta(\tau) } \BE(\Bx, t - \tau) d\tau.
\end{dmath}
This shows that
\begin{dmath}\label{eqn:emtProblemSet5Problem3:140}
G(\tau)
=
\frac{\epsilon(\tau)}{\epsilon_0} - \delta(\tau)
=
\inv{2 \pi} \int \frac{\epsilon(\omega)}{\epsilon_0} e^{-j \omega \tau} d\tau
- \inv{2 \pi} \int e^{-j \omega \tau} d\tau
=
\inv{2 \pi} \int \lr{ \frac{\epsilon(\omega)}{\epsilon_0} -1 } e^{-j \omega \tau} d\tau. \qedmarker
\end{dmath}
\makeSubAnswer{}{emt:problemSet5:3b}
The permittivity in the frequency domain is
\begin{dmath}\label{eqn:emtProblemSet5Problem3:160}
\epsilon(\omega) = \int \epsilon(t) e^{j \omega t} dt,
\end{dmath}
or
\begin{dmath}\label{eqn:emtProblemSet5Problem3:180}
\epsilon(-\omega) = \int \epsilon(t) e^{-j \omega t} dt.
\end{dmath}
Assuming that \( \epsilon(t) \) is real, this is
\begin{dmath}\label{eqn:emtProblemSet5Problem3:200}
\epsilon(-\omega) = \lr{ \int \epsilon(t) e^{j \omega t} dt }^\conj,
\end{dmath}
or
\begin{dmath}\label{eqn:emtProblemSet5Problem3:220}
\epsilon(-\omega) = \epsilon^\conj(\omega). \qedmarker
\end{dmath}
\makeSubAnswer{}{emt:problemSet5:3c}
Written out explicitly, the frequency domain permittivity is
\begin{dmath}\label{eqn:emtProblemSet5Problem3:240}
\epsilon(\omega)
=
\int \epsilon(t) \cos\lr{\omega t} dt
+ j \int \epsilon(t) \sin\lr{\omega t} dt
=
\int \epsilon(t) \cos\lr{-\omega t} dt
- j \int \epsilon(t) \sin\lr{-\omega t} dt,
\end{dmath}
or
\begin{equation}\label{eqn:emtProblemSet5Problem3:260}
\begin{aligned}
\epsilon'(\omega) &= \epsilon'(-\omega)  \\
\epsilon''(\omega) &= -\epsilon''(-\omega). \qedmarker
\end{aligned}
\end{equation}
}}

   \mychapter{Druid model.}
      %\paragraph{Disclaimer.}
%%
%These notes are mostly a direct transcription of \mo's handwritten notes on this topic, mostly skipped over in class.  These become relevant because this is used in the non-vacuum model of Maxwell's equations.
%%
\section{Druid model.}
\paragraph{Additional references:} A nice vector based derivation of these Druid model results can be found in \citep{ashcroft1976solid}.  The Meissner effect is also discussed in that context.
%
%
In this section we will investigate the optical properties of free electrons, or what is commonly called free electron gas.
%
By free electron gas we mean electrons that do not experience the restoring force which we considered for bound charges in the case of Lorentz model.  In particular, the resonance frequency \( \omega_0 \) for free electrons is zero.
%
There are two typical cases of free electron systems
%
\begin{enumerate}[a]
\item Metals.
\item Doped (n or p type) semiconductors.
\end{enumerate}
%
For the moment we consider the case of metals.
%
Free electrons are responsible for high reflectivity and good thermal conductivity of metals up to optical frequencies.  A model that can be used to describe the high reflectivity of metals is the Drude model.
%
\paragraph{Plasma:} A neutral gas of free electrons and heavy ions is called plasma.  Examples of plasma are metals and doped semiconductors, since these materials are a combination of free electrons and heavy ions which are, in sum, electrically neutral.
%
\paragraph{Drude-Lorentz model,} (or Drude model for short): similar to the case of bound charges we already studied for free electron plasma, we can start with a harmonic oscillator model.  However, in this case, since electrons are free, there is no restoring force (i.e. \(\omega_0 = 0 \).  Recall that in the spring mass model \( \omega_0^2 = S/m \) where \( S \) was the spring tension coefficient.
%
With such a model the Lorentz model equation
%
\begin{dmath}\label{eqn:druid:20}
\frac{d^2 x}{dt^2} + \gamma \ddt{x} + \omega_0^2 x = \frac{Q E_0}{m} e^{j \omega t},
\end{dmath}
%
is reduced to
%
\begin{dmath}\label{eqn:druid:40}
\frac{d^2 x}{dt^2} + \gamma \ddt{x} = \frac{Q E_0}{m} e^{j \omega t},
\end{dmath}
%
Again, assuming a solution of the form \( x_p = x_0 e^{j \omega t} \) for the particular solution and substituting in \cref{eqn:druid:40}, we have
%
\begin{dmath}\label{eqn:druid:80}
x_0 \lr{ (j\omega)^2 + \gamma (j \omega)} = \frac{Q E_0}{m},
\end{dmath}
%
or
\begin{dmath}\label{eqn:druid:60}
x
=
\frac{Q E/m}{-\omega^2 + j \gamma \omega },
\end{dmath}
%
Once more assuming identical particles that are not coupled and a linear isotropic medium and using the fact that \( \BP = N \Bp = N Q \Bx \), and
%
\begin{dmath}\label{eqn:druid:100}
\chi_e = \frac{\Abs{\BP}}{\epsilon_0 \Abs{\BE} },
\end{dmath}
%
we have
%
\begin{dmath}\label{eqn:druid:120}
\chi_e
=
\frac{Q^2 N/m \epsilon_0}{-\omega^2 + j \gamma \omega },
\end{dmath}
%
or with \( \omega_p^2 = Q^2 N/m\epsilon_0\),
%
\begin{dmath}\label{eqn:druid:140}
\epsilon_r
= 1 + \chi_e
=
1+
\frac{\omega_p^2}{-\omega^2 + j \gamma \omega }.
\end{dmath}
%
Plasma frequency, \( \omega_p \), can be understood as the natural resonance frequency by which the free electron gas (plasma) collectively (not individual electrons ) oscillates.
%
Note that if we neglect the last term, i.e., let \( \gamma = 0 \) then
%
\begin{dmath}\label{eqn:druid:160}
\epsilon_r = 1 - \frac{\omega_p^2}{\omega^2}.
\end{dmath}
%
From this it is clear that when \( \omega < \omega_p \), we have \( \epsilon_r < 1 \) and \( n = \sqrt{\epsilon_r} \) is purely imaginary, and the wave attenuates inside the electron plasma.
%
This means that for \( \omega < \omega_p \) electromagnetic waves do not propagate a large distance inside of metal.  However, for \( \omega > \omega_p \) the electron plasma (e.g. metal) is transparent.  The latter is called ultraviolet transparency of metal, because for most metals \( \omega_p \) is in the ultraviolet part of the spectrum.  For example,
%
\begin{itemize}
\item For \ce{Al}:
\begin{equation}\label{eqn:druid:180}
\frac{\omega_p}{2 \pi} = 3.82 \times 10^{15} \si{Hz} \implies \lambda_p = 79 [nm].
\end{equation}
\item For \ce{Au}:
\begin{equation}\label{eqn:druid:200}
\frac{\omega_p}{2 \pi} = 5.9 \times 10^{15} \si{Hz} \implies \lambda_p = 138 [nm].
\end{equation}
\end{itemize}
%
Using \cref{eqn:druid:160} one can calculate
%
\begin{dmath}\label{eqn:druid:220}
\tilde{n} = \sqrt{\epsilon_r},
\end{dmath}
%
and plot the reflectivity \( R \) at normal incidence
%
\begin{dmath}\label{eqn:druid:240}
R = \Abs{ \frac{\tilde{n} - 1 }{\tilde{n} + 1} },
\end{dmath}
%
which will have a shape similar to that of \cref{fig:druidReflectivity:druidReflectivityFig1}.
%
\imageFigure{../figures/ece1228-electromagnetic-theory/druidReflectivityFig1}{Metal reflectivity.}{fig:druidReflectivity:druidReflectivityFig1}{0.3}
%
This figure shows that for \( \omega/\omega_p \ll 1 \) metal reflects most of the incident light, whereas it becomes transparent (it transmits light) for \( \omega/\omega_p \gg 1 \).  This explains the shiny appearance of the metal at optical wavelengths.
%
The fact that plasma reflects EM waves below a \( \omega_p \) frequency can be used to transmit AM radio waves.  The ionosphere can be viewed as a plasma gas due to free electrons generated by cosmic radiation and ultraviolet light from the sun.  The \( \omega_p \) for ionosphere plasma is \( \omega_p = O(1 \si{MHz}) \).  Therefore AM signals modulated at frequencies below or in the range of a \si{MHz} will be reflected from the ionosphere.  But FM signals where the modulation frequency is greater than \si{MHz} will not be reflected, but will travel through the ionosphere and into space.
%
\section{Conductivity}
%
%\ddt{} \lr{ m \ddt{x} } + \gamma \ddt{x} = Q E_0 e^{i \omega t}
%
\begin{dmath}\label{eqn:druid:260}
\spacegrad \cross \BH(\Br, \omega)
= \sigma \BE(\Br, \omega) + j \omega \epsilon_0 \BE(\Br, \omega)
= j \omega \epsilon_0 \lr{ 1 + \frac{\sigma}{j \omega \epsilon_0} } \BE(\Br, \omega)
= j \omega \epsilon_0 \lr{ 1 - \frac{j \sigma}{\omega \epsilon_0} } \BE(\Br, \omega)
\end{dmath}
%
This complex factor is the relative permittivity
%
\begin{dmath}\label{eqn:druid:280}
\epsilon_r
= 1 - \frac{j \sigma}{\omega \epsilon_0},
\end{dmath}
%
and is why we write
%
\begin{dmath}\label{eqn:druid:300}
\epsilon(\omega) = \epsilon'(\omega) - j \epsilon''(\omega).
\end{dmath}
%

%      FIXME: transcribe handwritten notes that were mostly skipped over in class?
      \section{Problems.}
%\mychapter{conductivity} % transcribe?  This was part of L7
   %
% Copyright � 2016 Peeter Joot.  All Rights Reserved.
% Licenced as described in the file LICENSE under the root directory of this GIT repository.
%
%\input{../blogpost.tex}
%\renewcommand{\basename}{emt7}
%\renewcommand{\dirname}{notes/ece1228/}
%\newcommand{\keywords}{ECE1228H}
%\input{../latex/peeter_prologue_print2.tex}
%
%%\usepackage{ece1228}
%\usepackage{peeters_braket}
%%\usepackage{peeters_layout_exercise}
%\usepackage{peeters_figures}
%\usepackage{mathtools}
%\usepackage{siunitx}
%\usepackage{macros_bm}
%
%\beginArtNoToc
%\generatetitle{ECE1228H Electromagnetic Theory.  Lecture 8: Wave equation.  Taught by Prof.\ M. Mojahedi}
%\mychapter{Wave equation.}
%\label{chap:emt7}
%
%\paragraph{Disclaimer}
%
%Peeter's lecture notes from class.  These may be incoherent and rough.
%
%These are notes for the UofT course ECE1228H, Electromagnetic Theory, taught by Prof. M. Mojahedi, covering \textchapref{{1}} \citep{balanis1989advanced} content.
%
\paragraph{Wave equation.}
Using an expansion of the triple cross product in terms of the Laplacian
\begin{dmath}\label{eqn:emtLecture7:40}
\spacegrad \cross \lr{ \spacegrad \cross \Bf }
=
-\spacegrad \cdot \lr{ \spacegrad \wedge \Bf }
=
-\spacegrad^2 \Bf
+ \spacegrad \lr{ \spacegrad \cdot \Bf },
\end{dmath}
we can evaluate the cross products
\begin{equation}\label{eqn:emtLecture7:60}
\begin{aligned}
\spacegrad \cross \lr{ \spacegrad \cross \bcE } &= \spacegrad \cross \lr{ -\PD{t}{\bcB} - \bcM },\\
\spacegrad \cross \lr{ \spacegrad \cross \bcH } &= \spacegrad \cross \lr{ \PD{t}{\bcD} + \bcJ },
\end{aligned}
\end{equation}
or
\begin{equation}\label{eqn:emtLecture7:80}
\begin{aligned}
-\spacegrad^2 \bcE + \spacegrad \lr{ \spacegrad \cdot \bcE } &= -\mu \PD{t}{} \spacegrad \cross \bcH - \spacegrad \cross \bcM, \\
-\spacegrad^2 \bcH + \spacegrad \lr{ \spacegrad \cdot \bcH } &= \epsilon \PD{t}{} \lr{ \spacegrad \cross \bcE } + \spacegrad \cross \bcJ,
\end{aligned}
\end{equation}
%
or
%
\begin{equation}\label{eqn:emtLecture7:100}
\begin{aligned}
-\spacegrad^2 \bcE + \inv{\epsilon} \spacegrad \rho_{ev} &= -\mu \PD{t}{} \lr{ \PD{t}{\bcD} + \bcJ } - \spacegrad \cross \bcM,\\
-\spacegrad^2 \bcH + \inv{\mu} \spacegrad \rho_{mv} &= \epsilon \PD{t}{} \lr{ -\PD{t}{\bcB} - \bcM } + \spacegrad \cross \bcJ.
\end{aligned}
\end{equation}
%
This decouples the equations for the electric and the magnetic fields
%
\begin{equation}\label{eqn:emtLecture7:120}
\begin{aligned}
\spacegrad^2 \bcE &=
   \mu \epsilon \PDSq{t}{\bcE} +
   \inv{\epsilon} \spacegrad \rho_{ev} +
   \mu \PD{t}{\bcJ } +
   \spacegrad \cross \bcM, \\
\spacegrad^2 \bcH &=
   \epsilon \mu \PDSq{t}{\bcH} +
   \inv{\mu} \spacegrad \rho_{mv} +
   \epsilon \PD{t}{\bcM } -
   \spacegrad \cross \bcJ.
\end{aligned}
\end{equation}
%
Splitting the current between induced and bound (?) currents
%
\begin{equation}\label{eqn:emtLecture7:260}
\bcJ = \bcJ_i + \bcJ_c = \bcJ_i + \sigma \bcE,
\end{equation}
%
these become
%
\begin{equation}\label{eqn:emtLecture7:160}
\begin{aligned}
\spacegrad^2 \bcE &=
   \mu \epsilon \PDSq{t}{\bcE} +
   \inv{\epsilon} \spacegrad \rho_{ev} +
   \mu \sigma \PD{t}{\bcE} +
   \spacegrad \cross \bcM +
   \mu \PD{t}{\bcJ_i}, \\
\spacegrad^2 \bcH &=
   \epsilon \mu \PDSq{t}{\bcH} +
   \inv{\mu} \spacegrad \rho_{mv} +
   \epsilon \PD{t}{\bcM } +
   \sigma \mu \PD{t}{\bcH} +
   \sigma \bcM
-
   \spacegrad \cross \bcJ_i
.
\end{aligned}
\end{equation}
%
\paragraph{Time harmonic form.}
%
Assuming time harmonic dependence \( \bcX = \BX e^{j\omega t} \), we find
%
\begin{equation}\label{eqn:emtLecture7:140}
\begin{aligned}
\spacegrad^2 \BE &=
   \lr{ - \omega^2 \mu \epsilon +
   j \omega \mu \sigma } \BE +
   \inv{\epsilon} \spacegrad \rho_{ev} +
   \spacegrad \cross \BM +
   j \omega \mu \BJ_i, \\
\spacegrad^2 \BH &=
   \lr{ -\omega^2 \epsilon \mu +
   j \omega \sigma \mu } \BH +
   \inv{\mu} \spacegrad \rho_{mv} +
   (j \omega \epsilon + \sigma) \BM
-
   \spacegrad \cross \BJ_i.
\end{aligned}
\end{equation}
%
For a lossy medium where \( \epsilon = \epsilon' -j \omega \epsilon'' \), the leading term factor is
%
\begin{dmath}\label{eqn:emtLecture7:180}
- \omega^2 \mu \epsilon + j \omega \mu \sigma
=
- \omega^2 \mu \epsilon' + j \omega \mu \lr{ \sigma + \omega \epsilon'' }.
\end{dmath}
%
With the definition
\begin{equation}\label{eqn:emtLecture7:200}
\gamma^2 = \lr{ \alpha + j \beta }^2 = - \omega^2 \mu \epsilon' + j \omega \mu \lr{ \sigma + \omega \epsilon'' },
\end{equation}
%
the wave equations have the form
%
\begin{equation}\label{eqn:emtLecture7:220}
\begin{aligned}
\spacegrad^2 \BE &=
\gamma^2 \BE +
   \inv{\epsilon} \spacegrad \rho_{ev} +
   \spacegrad \cross \BM +
   j \omega \mu \BJ_i, \\
\spacegrad^2 \BH &=
\gamma^2 \BH +
   \inv{\mu} \spacegrad \rho_{mv} +
   (j \omega \epsilon + \sigma) \BM
-
   \spacegrad \cross \BJ_i.
\end{aligned}
\end{equation}
%
Here
%
\begin{itemize}
\item \( \alpha \) is the attenuation constant [\si{Np/m}],
\item \( \beta \) is the phase velocity [\si{rad/m}],
\item \( \gamma \) is the propagation constant [\si{1/m}].
\end{itemize}

We are usually interested in solutions in regions free of magnetic currents, induced electric currents, and free of any charge densities, in which case the wave equations are just
%
\begin{equation}\label{eqn:emtLecture7:240}
\begin{aligned}
\spacegrad^2 \BE &= \gamma^2 \BE, \\
\spacegrad^2 \BH &= \gamma^2 \BH.
\end{aligned}
\end{equation}
%
%\EndArticle
%\EndNoBibArticle

      \section{Problems.}
         %
% Copyright � 2016 Peeter Joot.  All Rights Reserved.
% Licenced as described in the file LICENSE under the root directory of this GIT repository.
%
\makeproblem{Meissner effect.}{emt:problemSet6:1}{
The constitutive relation for superconductors in weak magnetic fields can be macroscopically
characterized by the first London equation
%
\begin{dmath}\label{eqn:emtproblemSet6Problem1:20}
\PD{t}{\BJ_{\mathrm{sup}}} = \alpha \BE,
\end{dmath}
%
and the second London equation
\begin{dmath}\label{eqn:emtproblemSet6Problem1:40}
\spacegrad \cross \BJ_{\mathrm{sup}} = -\alpha_1 \BB,
\end{dmath}
where \( \BJ_{\mathrm{sup}} \)
stands for the superconducting current,
\( \alpha = n_s q^2 /m \) and \( \alpha_1 \approx \alpha \), with
\( n_s \), \(m\), and \( q\)
denoting, respectively, the number density, the effective mass, and the charge of the Cooper pairs
responsible for the superconductivity in a charged Boson fluid model.
%
\makesubproblem{}{emt:problemSet6:1a}
From the first London equation, derive and equation for \( \dot{\BB} = \PDi{t}{\BB} \)
by using the static
Maxwell equation \( \spacegrad \cross \BH = \BJ_{\mathrm{sup}} \)
without the displacement current. Show that
\begin{equation}\label{eqn:emtproblemSet6Problem1:60}
\spacegrad^2 \dot{\BB} = \mu_0 \alpha \dot{\BB}.
\end{equation}
\makesubproblem{}{emt:problemSet6:1b}
From the second London equation and the Ampere's law stated above derive an equation
for \( \BB \).
\makesubproblem{}{emt:problemSet6:1c}
What are the penetration depths in the
\partref{emt:problemSet6:1a}
and
\partref{emt:problemSet6:1b}
cases? Justify your answer.
%
\paragraph{Remark:} from above analysis we see that both the current and magnetic field are confined to a
thin layer of the order of the penetration depth which is very small. The exclusion of static
magnetic field in a superconductor is known as the Meissner effect experimentally discovered in
1933.
} % makeproblem
%
\skipIfRedacted{
\makeanswer{emt:problemSet6:1}{
\makeSubAnswer{}{emt:problemSet6:1a}
%
Taking the curl of the first London equation \cref{eqn:emtproblemSet6Problem1:20} gives
%
\begin{dmath}\label{eqn:emtproblemSet6Problem1:80}
\spacegrad \cross \dot{\BJ}_{\mathrm{sup}}
= \alpha \spacegrad \cross \BE
= \alpha \lr{ -\dot{\BB} },
\end{dmath}
%
or
\begin{dmath}\label{eqn:emtproblemSet6Problem1:100}
\spacegrad \cross \dot{\BJ}_{\mathrm{sup}}
= -\alpha \dot{\BB},
\end{dmath}
%
which has the same structure as the time derivative of the second London equation, but with \( \alpha \) instead of \( \alpha_1 \).  Taking the curl once more gives
%
\begin{dmath}\label{eqn:emtproblemSet6Problem1:120}
0
=
\PD{t}{} \lr{
\spacegrad \cross \lr{ \spacegrad \cross \BJ_{\mathrm{sup}} } + \alpha \spacegrad \cross \BB
}
=
\PD{t}{} \lr{
-\spacegrad^2 \BJ_{\mathrm{sup}} + \spacegrad \lr{ \spacegrad \cdot \BJ_{\mathrm{sup}} }
+ \alpha \mu_0 \spacegrad \cross \BH
}
=
\PD{t}{} \lr{
-\spacegrad^2 \BJ_{\mathrm{sup}} + \spacegrad \lr{ \spacegrad \cdot \BJ_{\mathrm{sup}} }
+ \alpha \mu_0 \lr{ \BJ_{\mathrm{sup}} + \cancel{ \PD{t}{\BD} } }
},
\end{dmath}
%
or
\begin{dmath}\label{eqn:emtproblemSet6Problem1:140}
\alpha \mu_0 \dot{\BJ}_{\mathrm{sup}} = \spacegrad^2 \dot{\BJ}_{\mathrm{sup}} + \spacegrad \lr{ \spacegrad \cdot \dot{\BJ}_{\mathrm{sup}} }.
\end{dmath}
%
One final application of the curl operator gives
\begin{dmath}\label{eqn:emtproblemSet6Problem1:160}
0
=
-\alpha \mu_0 \spacegrad \cross \dot{\BJ}_{\mathrm{sup}} + \spacegrad \cross \lr{ \spacegrad^2 \dot{\BJ}_{\mathrm{sup}}} + \cancel{\spacegrad \cross \lr{ \spacegrad \lr{ \spacegrad \cdot \dot{\BJ}_{\mathrm{sup}} }}}.
=
-\alpha \mu_0 \lr{ -\alpha \dot{\BB} } + \spacegrad^2 \lr{ \spacegrad \cross \dot{\BJ}_{\mathrm{sup}} }
=
-\alpha \mu_0 \lr{ -\alpha \dot{\BB} } + \spacegrad^2 \lr{ -\alpha \dot{\BB} }
=
-\alpha \lr{ -\mu_0 \alpha \dot{\BB} + \spacegrad^2 \dot{\BB} }.
\end{dmath}
%
Note that this used
the fact that the curl of a gradient is zero, the fact that the curl and the Laplacian commute (\( \spacegrad^2 \epsilon_{rst} \Be_r \partial_s A_t = \epsilon_{rst} \Be_r \partial_s \spacegrad^2 A_t = \spacegrad \cross ( \spacegrad^2 \BA ) \)), and made two substitutions of
\cref{eqn:emtproblemSet6Problem1:100}
.  This gives the desired result
%
\begin{dmath}\label{eqn:emtproblemSet6Problem1:180}
\mu_0 \alpha \dot{\BB} = \spacegrad^2 \dot{\BB}.      \qedmarker
\end{dmath}
%
\makeSubAnswer{}{emt:problemSet6:1b}
%
Taking the double curl of the second London equation, we have
\begin{dmath}\label{eqn:emtproblemSet6Problem1:200}
\spacegrad \cross \lr{ \spacegrad \cross \lr{ -\alpha_1 \BB } }
=
-\alpha_1 \spacegrad \cross \lr{ \spacegrad \cross \BB }
=
-\alpha_1 \mu_0 \spacegrad \cross \lr{ \spacegrad \cross \BH }
=
-\alpha_1 \mu_0 \spacegrad \cross \lr{ \BJ + \cancel{\PD{t}{\BD}} }
=
-\alpha_1 \mu_0 \lr{ -\alpha_1 \BB }
=
-\alpha_1 \lr{ -\spacegrad^2 \BB + \spacegrad \cancel{ \spacegrad \cdot \BB } },
\end{dmath}
%
or
\begin{dmath}\label{eqn:emtproblemSet6Problem1:220}
\spacegrad^2 \BB = \alpha_1 \mu_0 \BB.
\end{dmath}
%
This has the structure of a homogeneous Helmholtz equation \( (\spacegrad^2 + k^2) \BB = 0 \) with an imaginary \( k \).
%
\makeSubAnswer{}{emt:problemSet6:1c}
%
The solution of \cref{eqn:emtproblemSet6Problem1:60} is
%
\begin{dmath}\label{eqn:emtproblemSet6Problem1:240}
\dot{\BB} = \dot{\BB}_0 \exp \lr{ \pm \sqrt{\alpha \mu_0} \kcap \cdot \Br },
\end{dmath}
%
and the solution of \cref{eqn:emtproblemSet6Problem1:220} is
\begin{dmath}\label{eqn:emtproblemSet6Problem1:260}
\BB = \BB_0 \exp \lr{ \pm \sqrt{\alpha_1 \mu_0} \kcap \cdot \Br },
\end{dmath}
%
when \( \alpha_1 = \alpha \), a magnetic field solution that satisfies both is
%
\begin{dmath}\label{eqn:emtproblemSet6Problem1:280}
\BB(\Br, t) = \BB_0 \cos( \omega t ) \exp \lr{ \pm \sqrt{\alpha_1 \mu_0} \kcap \cdot \Br }.
\end{dmath}
%
Anchoring the coordinate system at the boundary of the material, and picking an unbounded solution, requires that at depth \( \delta \) from that surface the exponential goes as
%
\begin{dmath}\label{eqn:emtproblemSet6Problem1:300}
e^{-\sqrt{\alpha \mu_0} \delta}.
\end{dmath}
%
The \( e^{-1} \) point is when
%
\begin{dmath}\label{eqn:emtproblemSet6Problem1:320}
\sqrt{\alpha \mu_0} \delta = 1,
\end{dmath}
%
or
\begin{dmath}\label{eqn:emtproblemSet6Problem1:340}
\delta
= \inv{\sqrt{\alpha \mu_0} }
= \inv{\sqrt{n_s q^2 \mu_0/m} }.
\end{dmath}
%
That is
\boxedEquation{eqn:emtproblemSet6Problem1:360}{
\delta
= \sqrt{ \frac{m}{n_s q^2 \mu_0} }.
}
%
Assuming an electrostatic configuration (where \( \spacegrad \cdot \BJ = 0 \)), the time derivative of the current
in \cref{eqn:emtproblemSet6Problem1:140}
is also seen to be governed by an equation of the form
\cref{eqn:emtproblemSet6Problem1:220}.  This means that both the magnetic field and current are restricted to the same thin layer, with continuing exponential tailoff past the skin depth.
}}

   \mychapter{Wave equation solutions.}
In class, we walked through splitting up the wave equation into components, and separation of variables.  I didn't take notes on that.

Winding down that discussion, however, was a mention of phase and group velocity, and a phenomena called superluminal velocity.  This latter is analogous to quantum electron tunnelling where a wave can make it through an aperture with a damped solution \( e^{-\alpha x} \) in the aperture interval, and sinusoidal solutions in the incident and transmitted regions as sketched in \cref{fig:L7:L7Fig1}.  The time \( \tau \) to get through the aperture is called the tunnelling time.
\imageFigure{../figures/ece1228-electromagnetic-theory/L7Fig1}{Superluminal tunnelling.}{fig:L7:L7Fig1}{0.3}
      \section{Problems.}
          %
% Copyright � 2016 Peeter Joot.  All Rights Reserved.
% Licenced as described in the file LICENSE under the root directory of this GIT repository.
%
\makeproblem{Lossy waves.}{emt:problemSet6:2}{
In the case of lossy medium the wave equation was given by
\begin{equation}\label{eqn:emtproblemSet6Problem2:20}
\spacegrad^2 \BE = \gamma^2 \BE,
\end{equation}
where
\begin{equation}\label{eqn:emtproblemSet6Problem2:40}
\gamma^2 = \lr{ \alpha + j \beta }^2.
\end{equation}
Now consider a medium for which \( \epsilon(\omega) = \epsilon'(\omega) \) (i.e. \( \epsilon''(\omega) = 0 \)), \( \sigma = \sigma_0 \) (i.e. \(\omega \tau \sim 0 \) in the Drude model), and \( \mu \) is a constant and real.
For this case obtain the expression for \( \alpha \) and \(\beta\) in terms of \( \omega, \mu, \epsilon', \sigma_0 \).
}
\skipIfRedacted{
\makeanswer{emt:problemSet6:2}{
The desired values of \( \alpha \) and \( \beta \) are defined by
\begin{equation}\label{eqn:emtproblemSet6Problem2:80}
\begin{aligned}
\gamma^2
&= \lr{\alpha + j \beta}^2
\\ &= -\mu \epsilon \omega^2 + j \omega \mu \sigma
\\ &= -\mu \lr{ \epsilon' - j \epsilon''} \omega^2 + j \omega \mu \sigma
\\ &= -\mu \epsilon' \omega^2 + j \omega \mu \lr{ \sigma + \epsilon'' \omega }.
\end{aligned}
\end{equation}
When \( \epsilon'' = 0 \) and \( \sigma = \sigma_0 \), this is
\begin{equation}\label{eqn:emtproblemSet6Problem2:100}
\begin{aligned}
\lr{\alpha + j \beta}^2
&= -\mu \epsilon' \omega^2 + j \omega \mu \sigma_0
\\ &= \mu \epsilon' \omega^2 \lr{ -1 + j \frac{\sigma_0}{\epsilon' \omega} }.
%\\ &= \sqrt{ \lr{ \mu \epsilon' \omega^2}^2 + \lr{ \omega \mu \sigma_0 }^2 }
%\exp\lr{
%-j \Atan\lr{\frac{ \sigma_0 }{\epsilon' \omega}}
%}
%\\ &=
%\mu \epsilon' \omega^2
%\sqrt{ 1 + \lr{ \frac{\sigma_0}{\epsilon' \omega} }^2 }
%\exp\lr{
%-j \Atan\lr{\frac{ \sigma_0 }{\epsilon' \omega}}
%},
\end{aligned}
\end{equation}
To attempt to take the square root of \( z = -1 + j t \), I wrote
\begin{equation}\label{eqn:emtproblemSet6Problem2:120}
z = \sqrt{ 1 + t^2 } e^{-j\Atan t },
\end{equation}
so
\begin{equation}\label{eqn:emtproblemSet6Problem2:140}
\begin{aligned}
\sqrt{z}
&= \lr{ 1 + t^2 }^{1/4} e^{-\frac{j}{2}\Atan t }
\\ &=
\lr{ 1 + t^2 }^{1/4}
\lr{
\cos\lr{ \inv{2} \Atan t }
- j \sin\lr{ \inv{2} \Atan t } }.
\end{aligned}
\end{equation}
Expanding the trig relations using
\begin{equation}\label{eqn:emtproblemSet6Problem2:160}
\begin{aligned}
\cos\lr{ \inv{2} \Atan t } &= \inv{\sqrt{2}} \sqrt{ 1 + \inv{\sqrt{1 + t^2}}} \\
\sin\lr{ \inv{2} \Atan t } &= \frac{t}{\sqrt{2}} \inv{ \sqrt{ 1 + t^2} \sqrt{ 1 + \inv{\sqrt{1 + t^2}}}},
\end{aligned}
\end{equation}
and doing some simplification, I find
\begin{equation}\label{eqn:emtproblemSet6Problem2:180}
\sqrt{z} = \inv{\sqrt{2}} \sqrt{ 1 + \sqrt{ 1 + t^2} } - j \frac{t}{\sqrt{2} \sqrt{ 1 + \sqrt{ 1 + t^2} }}.
\end{equation}
Curiously, a squaring check shows that this yields \( z = 1 - j t \) and not \( z = -1 + j t \).  This indicates that a \( j^2 \)  phase factor was lost somewhere in the algebra (probably related to the quadrant ambiguity of the arctan).  Inserting that phase factor back in gives
\boxedEquation{eqn:emtproblemSet6Problem2:200}{
\begin{aligned}
\alpha &= \inv{\sqrt{2}} \frac{\sigma_0 \sqrt{\frac{\mu}{\epsilon'}}} { \sqrt{ 1 + \sqrt{ 1 + \lr{\frac{\sigma_0}{\epsilon' \omega} }^2 }} }, \\
\beta &= \inv{\sqrt{2}} \omega \sqrt{\mu\epsilon'} \sqrt{ 1 + \sqrt{ 1 + \lr{\frac{\sigma_0}{\epsilon' \omega} }^2 }}.
\end{aligned}
}
A computation of \( (\alpha + j \beta)^2 \) shows that this gives \( -\mu \epsilon' \omega^2 + j \omega \mu \sigma_0 \) as expected.
}}

          %
% Copyright © 2016 Peeter Joot.  All Rights Reserved.
% Licenced as described in the file LICENSE under the root directory of this GIT repository.
%
\makeproblem{Uniform plane wave.}{emt:problemSet6:3}{
\paragraph{Note:} This seemed like a separate problem, and has been split out from the problem 2 as specified in the original problem set handout.
The uniform plane wave
\begin{equation}\label{eqn:emtproblemSet6Problem3:60}
\bcE(\Br, t) = E_0
\lr{ \xcap \cos\theta - \zcap \sin\theta } \cos\lr{ \omega t -k \sin\theta x - k \cos\theta z },
\end{equation}
is propagating in the \(x-z\) plane as sketched in \cref{fig:emtProblemSet6:emtProblemSet6Fig1}
in a simple medium with \( \sigma = 0\).
\imageFigure{../figures/ece1228-electromagnetic-theory/emtProblemSet6Fig1}{Linear wave front.}{fig:emtProblemSet6:emtProblemSet6Fig1}{0.3}
Here, \( E_0 \) is a real constant and \( k \) is the propagation
constant. Answer the following questions and show all your
work.
\makesubproblem{}{emt:problemSet6:3a}
Determine the associated magnetic field \( \BH(\Br, t) \).
\makesubproblem{}{emt:problemSet6:3b}
Determine the time averaged Poynting vector, \( \expectation{\BS(\Br, t)} \).
\makesubproblem{}{emt:problemSet6:3c}
Determine the stored magnetic energy density, \( W_m(\Br, t) \).
\makesubproblem{}{emt:problemSet6:3d}
Determine the components of phase velocity vector \( \Bv_p \) along x and z.
} % makeproblem
\skipIfRedacted{
\makeanswer{emt:problemSet6:3}{
The wave equation for \( \bcE \) is
\begin{equation}\label{eqn:emtproblemSet6Problem3:80}
\spacegrad^2 \bcE  = \mu \epsilon \PDSq{t}{\bcE} + \mu \sigma \PD{t}{\bcE}.
\end{equation}
With \( \sigma = 0 \) and \( E_0 \) real, the permittivity \( \epsilon \) must also be real.  In the frequency domain, this means that the waves are governed by the equations
\begin{equation}\label{eqn:emtproblemSet6Problem3:100}
\begin{aligned}
\spacegrad^2 \BE &= - \omega^2 \mu \epsilon \BE, \\
\spacegrad^2 \BH &= - \omega^2 \mu \epsilon \BH, \\
\spacegrad \cross \BE &= -j \omega \mu \BH, \\
\spacegrad \cross \BH &= j \omega \epsilon \BE, \\
\spacegrad \cdot \BH &= 0, \\
\spacegrad \cdot \BE &= 0,
\end{aligned}
\end{equation}
where
\begin{equation}\label{eqn:emtproblemSet6Problem3:120}
\begin{aligned}
\BE &= E_0 \lr{ \xcap \cos\theta - \zcap \sin\theta } e^{-j \Bk \cdot \Br}, \\
\Bk &= k\lr{ \xcap \sin\theta + \zcap \cos\theta }.
\end{aligned}
\end{equation}
We also require that
\begin{equation}\label{eqn:emtproblemSet6Problem3:140}
(-j \Bk)^2 = -\omega^2 \mu \epsilon,
\end{equation}
or
\begin{equation}\label{eqn:emtproblemSet6Problem3:160}
\frac{\omega}{k} = \inv{\sqrt{\mu \epsilon}}.
\end{equation}
\makeSubAnswer{}{emt:problemSet6:3a}
In the frequency domain, the magnetic field can be obtained directly from the electric field
\begin{equation}\label{eqn:emtproblemSet6Problem3:180}
\begin{aligned}
\BH
&= \inv{-j \omega \mu} \spacegrad \cross \BE
\\ &=
\frac{E_0}{-j \omega \mu}
\begin{vmatrix}
\xcap & \ycap & \zcap \\
\partial_x & \partial_y & \partial_z \\
\cos\theta
e^{-j \Bk \cdot \Br} & 0 &
- \sin\theta
e^{-j \Bk \cdot \Br}
\end{vmatrix}
\\ &=
-\frac{E_0 \ycap}{-j \omega \mu}
\lr{
-\sin\theta \partial_x
-\cos\theta \partial_z
}
e^{-j \Bk \cdot \Br}
\\ &=
-\frac{-j k E_0 \ycap}{j \omega \mu}
\lr{
\sin\theta \sin\theta +
\cos\theta \cos\theta
}
e^{-j \Bk \cdot \Br}
\\ &=
\frac{k E_0 \ycap}{\omega \mu}
e^{-j \Bk \cdot \Br}
\\ &=
\frac{\sqrt{\mu \epsilon}E_0 \ycap}{\mu}
e^{-j \Bk \cdot \Br},
\end{aligned}
\end{equation}
or
\boxedEquation{eqn:emtproblemSet6Problem3:200}{
\bcH(\Br, t) = \sqrt{\frac{\epsilon}{\mu}} E_0 \ycap \cos\lr{ \omega t - \Bk \cdot \Br }.
}
\makeSubAnswer{}{emt:problemSet6:3b}
The instantaneous Poynting vector is
\begin{equation}\label{eqn:emtproblemSet6Problem3:220}
\begin{aligned}
\bcS
&= \bcE \cross \bcH
\\ &= \inv{4}
\lr{ \BE e^{j \omega t} + \BE^\conj e^{-j\omega t} }
\cross
\lr{ \BH e^{j \omega t} + \BH^\conj e^{-j\omega t} }
\\ &=
\inv{4}
\lr{
\BE \cross \BH^\conj + \BH \cross \BE^\conj + \BE \cross \BH e^{2 j \omega t} + \BE^\conj \cross \BH^\conj e^{-2 j \omega t}
}
\\ &=
\inv{2}
\Real \lr{
\BE \cross \BH^\conj + \BE \cross \BH e^{2 j \omega t}
},
\end{aligned}
\end{equation}
The average is
\begin{equation}\label{eqn:emtproblemSet6Problem3:240}
\begin{aligned}
\expectation{\bcS}
&=
\inv{2}
\inv{T} \int_0^T
\Real \lr{
\BE \cross \BH^\conj + \BE \cross \BH e^{2 j \omega t}
}
\\ &=
\inv{2} \Real \BE \cross \BH^\conj
\\ &=
\inv{2} E_0^2 \sqrt{\frac{\epsilon}{\mu}}
\begin{vmatrix}
\xcap & \ycap & \zcap \\
\cos\theta & 0 & -\sin\theta \\
0 & 1 & 0
\end{vmatrix}
\\ &=
\inv{2} E_0^2 \sqrt{\frac{\epsilon}{\mu}}
\lr{ \xcap \sin\theta + \zcap \cos\theta },
\end{aligned}
\end{equation}
or
\boxedEquation{eqn:emtproblemSet6Problem3:260}{
\expectation{\bcS}
=
\inv{2} E_0^2 \sqrt{\frac{\epsilon}{\mu}} \kcap.
}
\makeSubAnswer{}{emt:problemSet6:3c}
The stored magnetic energy density is
\begin{equation}\label{eqn:emtproblemSet6Problem3:280}
\begin{aligned}
W_m
&= \inv{2} \mu \Abs{\BH}^2
\\ &= \inv{2} \mu \frac{\epsilon}{\mu} E_0^2
\\ &= \inv{2} \epsilon E_0^2.
\end{aligned}
\end{equation}
This equals the stored electric energy density, as expected.
\makeSubAnswer{}{emt:problemSet6:3d}
The phase velocity are the velocities that satisfy
\begin{equation}\label{eqn:emtproblemSet6Problem3:300}
\begin{aligned}
0
&=
\ddt{} \lr{ \omega t - \Bk \cdot \Br }
\\ &=
\omega - \Bk \cdot \frac{d\Br}{dt},
\end{aligned}
\end{equation}
or
\begin{equation}\label{eqn:emtproblemSet6Problem3:380}
\begin{aligned}
\omega &=
\Bk \cdot \Bv_p
\\ &=
k_x v_x
+
k_z v_z
\\ &=
k \sin\theta v_x
+
k \cos\theta v_z.
\end{aligned}
\end{equation}
This has many solutions, including superluminal phase velocities such as:
\begin{equation}\label{eqn:emtproblemSet6Problem3:420}
\begin{aligned}
\Bv_p &= (\omega/(k\sin\theta), 0, 0), \\
\Bv_p &= (0, 0, \omega/(k\cos\theta)), \\
\Bv_p &= \frac{\omega}{2 k}(1/\sin\theta, 0, 1/\cos\theta).
\end{aligned}
\end{equation}
I'm unsure if those are physically relevant.  However,
it is reasonable to assume that a solution where \( \Bv_p \) is colinear with the energy propagation direction \( \Bk \) is the solution of interest.  In that case, the components of the phase velocity are
\begin{equation}\label{eqn:emtproblemSet6Problem3:320}
\begin{aligned}
v_x &= \frac{\omega}{k}\sin\theta, \\
v_z &= \frac{\omega}{k}\cos\theta,
\end{aligned}
\end{equation}
or
\boxedEquation{eqn:emtproblemSet6Problem3:400}{
\begin{aligned}
v_x &= c \sin\theta, \\
v_z &= c \cos\theta,
\end{aligned}
}
where
\begin{equation}\label{eqn:emtproblemSet6Problem3:360}
c = \inv{\sqrt{\mu \epsilon}}.
\end{equation}
%Both of these phase velocity components are greater than the speed of the wave.
}}

   \mychapter{Wave equation solutions.}
      %
% Copyright � 2016 Peeter Joot.  All Rights Reserved.
% Licenced as described in the file LICENSE under the root directory of this GIT repository.
%
%\input{../blogpost.tex}
%\renewcommand{\basename}{emt8}
%\renewcommand{\dirname}{notes/ece1228/}
%\newcommand{\keywords}{ECE1228H}
%\input{../latex/peeter_prologue_print2.tex}
%
%%\usepackage{ece1228}
%\usepackage{peeters_braket}
%%\usepackage{peeters_layout_exercise}
%\usepackage{peeters_figures}
%\usepackage{mathtools}
%\usepackage{siunitx}
%
%\beginArtNoToc
%\generatetitle{ECE1228H Electromagnetic Theory.  Lecture 8: Waves.  Taught by Prof.\ M. Mojahedi}
%%\chapter{Waves}
%\label{chap:emt8}
%
%\paragraph{Disclaimer}
%
%Peeter's lecture notes from class.  These may be incoherent and rough.
%
%These are notes for the UofT course ECE1228H, Electromagnetic Theory, taught by Prof. M. Mojahedi, covering \textchapref{{1}} \citep{balanis1989advanced} content.
%
\paragraph{Cylindrical coordinates.}
%
Seek a function
%
\begin{dmath}\label{eqn:emtLecture8:20}
\BE = E_\rho \rhocap + E_\phi \phicap + E_z \zcap,
\end{dmath}
%
solving
%
\begin{dmath}\label{eqn:emtLecture8:40}
\spacegrad^2 \BE = -\beta^2 \BE.
\end{dmath}
%
One way to find the Laplacian in cylindrical coordinates is to use
%
\begin{dmath}\label{eqn:emtLecture8:60}
\spacegrad^2 \BE =
\spacegrad \lr{ \spacegrad \cdot \BE }
-\spacegrad \cross \lr{ \spacegrad \cross \BE },
\end{dmath}
%
where
%
\begin{dmath}\label{eqn:emtLecture8:80}
\spacegrad = \rhocap \PD{\rho}{} + \frac{\phicap}{\rho} \PD{\phi}{} + \zcap \PD{z}{}.
\end{dmath}
%
It can be shown that:
\begin{dmath}\label{eqn:emtLecture8:100}
\spacegrad \cdot \BE = \inv{\rho} \PD{\rho}{} \lr{ \rho E_\rho } + \inv{\rho}\PD{\phi}{E_\phi} + \PD{z}{E_z},
\end{dmath}
%
and
\begin{dmath}\label{eqn:emtLecture8:120}
\spacegrad \cross \BE
%=
%\begin{vmatrix}
%\rhocap & \phicap & \zcap \\
%\partial_\rho & \inv{\rho}\partial_\phi & \partial_z \\
%E_\rho & \rho E_\phi & E_z
%\end{vmatrix}
=
\rhocap  \lr{ \inv{\rho} \partial_\phi E_z - \partial_z E_\phi }
+\phicap \lr{ \partial_z E_\rho - \partial_\rho E_z }
+\zcap   \lr{ \inv{\rho} \partial_\rho (\rho E_\phi) - \inv{\rho} \partial_\phi E_\rho }.
\end{dmath}
%
This gives
\begin{dmath}\label{eqn:emtLecture8:200}
\spacegrad^2 \psi =
\PDSq{\rho}{\psi}
+\inv{\rho} \PD{\rho}{\psi}
+\inv{\rho^2} \PDSq{\phi}{\psi}
+\PDSq{z}{\psi},
\end{dmath}
and
\begin{equation}\label{eqn:emtLecture8:220}
\begin{aligned}
\spacegrad^2 E_\rho &= \lr{ -\frac{E_\rho}{\rho^2} - \frac{2}{\rho^2} \PD{\phi}{E_\phi} }, \\
\spacegrad^2 E_\phi &= \lr{ -\frac{E_\phi}{\rho^2} + \frac{2}{\rho^2} \PD{\phi}{E_\rho} }, \\
\spacegrad^2 E_z    &= -\beta^2 E_\phi.
\end{aligned}
\end{equation}
%
This is explored in \cref{chap:laplacianCylindrical}.
%
%Note that with \( i = \Be_1 \Be_2 \),
%
%\begin{dmath}\label{eqn:emtLecture8:140}
%\rhocap = \Be_1 e^{i \phi}
%\end{dmath}
%
%so
%\begin{equation}\label{eqn:emtLecture8:160}
%\PD{\phi}{\rhocap} = \Be_2 e^{i \phi} = \thetacap
%\end{equation}
%
%... the end result is
%
%\begin{dmath}\label{eqn:emtLecture8:180}
%\end{dmath}
%
\paragraph{TEM:} If we want to have a TEM mode it can be shown that we need an axial distribution mechanism, such as the core of a co-axial cable.
%
These are messy to solve in general, but we can solve the z-component without too much pain
%
\begin{dmath}\label{eqn:emtLecture8:240}
\PDSq{\rho}{E_z}
+\inv{\rho} \PD{\rho}{E_z}
+\inv{\rho^2} \PDSq{\phi}{E_z}
+\PDSq{z}{E_z}
=
-\beta^2 E_z.
\end{dmath}
%
Solving this using separation of variables with
%
\begin{dmath}\label{eqn:emtLecture8:260}
E_z = R(\rho) P(\phi) Z(z),
\end{dmath}
%
\begin{dmath}\label{eqn:emtLecture8:280}
\inv{R}\lr{R'' + \inv{\rho} R'} + \inv{\rho^2 P} P'' + \frac{Z''}{Z} = -\beta^2.
\end{dmath}
%
Assuming for some constant \( \beta_z \) that we have
\begin{dmath}\label{eqn:emtLecture8:300}
\frac{Z''}{Z} = -\beta_z^2,
\end{dmath}
%
then
%
\begin{dmath}\label{eqn:emtLecture8:320}
\inv{R}\lr{\rho^2 R'' + \rho R'} + \inv{P} P'' + \rho^2 \lr{\beta^2 - \beta_z^2} = 0.
\end{dmath}
%
Now assume that
\begin{dmath}\label{eqn:emtLecture8:340}
\inv{P} P'' = -m^2,
\end{dmath}
%
and let \( \beta^2 - \beta_z^2 = \beta_\rho^2 \), which leaves
%
\begin{dmath}\label{eqn:emtLecture8:360}
\rho^2 R'' + \rho R' + \lr{ \rho^2 \beta_\rho^2 -m^2 } R = 0.
\end{dmath}
%
This is the Bessel differential equation, with travelling wave solution
%
\begin{dmath}\label{eqn:emtLecture8:380}
R(\rho) =
A H_m^{(1)}(\beta_\rho \rho)
+B H_m^{(2)}(\beta_\rho \rho),
\end{dmath}
%
and standing wave solutions
\begin{dmath}\label{eqn:emtLecture8:400}
R(\rho) =
A J_m(\beta_\rho \rho)
+B Y_m(\beta_\rho \rho).
\end{dmath}
%
Here \( H_m^{(1)}, H_m^{(2)} \) are Hankel functions of the first and second kinds, and
\( J_m, Y_m \) are the Bessel functions of the first and second kinds.
%
For \( P(\phi) \)
\begin{dmath}\label{eqn:emtLecture8:460}
P'' = -m^2 P.
\end{dmath}
%
%
\paragraph{Waves.}
%
\begin{itemize}
\item The field is a modification of space-time
\item Mode is a particular field configuration for a given boundary value problem.  Many field configurations can satisfy Maxwell equations (wave equation).  These usually are referred to as modes.  A mode is a self-consistent field distribution.
\item In a TEM mode, \( \BE \) and \( \BH \) are every point in space are constrained in a local plane, independent of time.  This plane is called the equiphase plane.  In general equiphase planes are not parallel at two different points along the trajectory of the wave.
%\item If equiphase planes are parallel (i.e. the space orientation of the planes for TEM mode...
%... next time.
\end{itemize}
%
%}
%\EndNoBibArticle

      \section{Problems.}
         %
% Copyright � 2016 Peeter Joot.  All Rights Reserved.
% Licenced as described in the file LICENSE under the root directory of this GIT repository.
%
\makeoproblem{Spherical wave solutions.}{emt:problemSet7:2}{2016 ps7.}{
Suppose under some circumstances (e.g. \(\mathrm{TE}^r\)
or \(\mathrm{TM}^r\) modes), the partial differential equations
for the wavefunction \( \psi \)
can further be simplified to
%
\begin{dmath}\label{eqn:emtproblemSet7Problem2:20}
\spacegrad^2 \psi(r, \theta, \phi) = - \beta^2 \psi(r, \theta, \phi).
\end{dmath}
%
Using
separation of variables
%
\begin{dmath}\label{eqn:emtproblemSet7Problem2:40}
\psi(r, \theta, \phi) = R(r) T(\theta) P(\phi),
\end{dmath}
%
find the differential equations governing the
behavior of \( R, T, P \).  Comment on the differential equations found and their possible solutions.
\paragraph{Remarks:} To have a more uniform answer, making it easier to mark the questions, use the
following conventions (notations) in your answer.
%
%
\begin{itemize}
\item Use \( -m^2 \) as the constant of separation for the differential equation governing \( P(\phi) \).
\item Use \( -n(n+1) \) as the constant of separation for the differential equation governing \( T(\theta) \).
\item Show that \( R(r) \) follows the differential equation associated with spherical Bessel or Hankel functions.
\end{itemize}
%
} % makeproblem
%
\skipIfRedacted{
\makeanswer{emt:problemSet7:2}{
%
The Laplacian in spherical coordinates is
%
\begin{dmath}\label{eqn:emtproblemSet7Problem2:60}
\spacegrad^2 \psi
=
  \inv{r^2} \partial_r \lr{ r^2 \partial_r \psi }
+ \inv{r^2 S_\theta} \partial_\theta \lr{ S_\theta \partial_\theta \psi }
+ \inv{r^2 S_\theta^2} \partial_{\phi\phi} \psi,
\end{dmath}
%
where I've written \( S_\theta = \sin\theta \) for brevity.  Substituting \cref{eqn:emtproblemSet7Problem2:40} into \cref{eqn:emtproblemSet7Problem2:20} gives
%
\begin{dmath}\label{eqn:emtproblemSet7Problem2:80}
0
= \spacegrad^2 \psi + \beta^2 \psi
=
  \inv{r^2} \partial_r \lr{ r^2 \partial_r (R T P) }
+ \inv{r^2 S_\theta} \partial_\theta \lr{ S_\theta \partial_\theta (R T P) }
+ \inv{r^2 S_\theta^2} \partial_{\phi\phi} (R T P)
+ \beta^2 (R T P)
=
  T P \inv{r^2} \lr{ r^2 R' }'
+ R P \inv{r^2 S_\theta} \lr{ S_\theta T' }'
+ R T \inv{r^2 S_\theta^2} P''
+ \beta^2 (R T P).
\end{dmath}
%
Here primes denote differentiation by the respective coordinates.  Multiplying by \( r^2 S_\theta^2/(R T P) \) this is
%
\begin{dmath}\label{eqn:emtproblemSet7Problem2:100}
0 =
  S_\theta^2 \lr{ \inv{R} \lr{ r^2 R' }' + r^2 \beta^2 }
+ \frac{S_\theta}{ T} \lr{ S_\theta T' }'
+ \inv{P} P''.
\end{dmath}
%
\paragraph{Solution for the \( \phi \) dependent function.}
%
The last term is separable, so we can equate
%
\begin{dmath}\label{eqn:emtproblemSet7Problem2:120}
\inv{P} P'' = -m^2,
\end{dmath}
%
which has exponential (or trigonometric) solutions
%
%\begin{dmath}\label{eqn:emtproblemSet7Problem2:440}
\boxedEquation{eqn:emtproblemSet7Problem2:460}{
P(\phi) = a e^{j m \phi} + b e^{-j m \phi}.
}
%\end{dmath}
%
\paragraph{Equation for the \( \theta \) dependent function.}
%
Utilizing this separation constant and dividing \cref{eqn:emtproblemSet7Problem2:100} through by \( S_\theta^2 \) gives
%
\begin{dmath}\label{eqn:emtproblemSet7Problem2:140}
0 =
  \inv{R} \lr{ \lr{ r^2 R' }' + r^2 \beta^2 }
+ \frac{1}{ T S_\theta^2} \lr{ S_\theta T' }' - \frac{m^2}{S_\theta^2}.
\end{dmath}
%
This is once again separable.  Let
%
\begin{dmath}\label{eqn:emtproblemSet7Problem2:160}
\frac{1}{ T S_\theta^2} \lr{ S_\theta T' }' - \frac{m^2}{S_\theta^2} = - n(n+1),
\end{dmath}
%
or
\begin{dmath}\label{eqn:emtproblemSet7Problem2:420}
0 = \lr{ S_\theta T' }' + \lr{ n(n+1)S_\theta^2 - m^2 } T.
\end{dmath}
%
Let's defer considering the nature of the solutions to this equation temporarily.
%
\paragraph{Solution for the \( r \) dependent function.}
%
The separation constant also
fixes the radial equation.  That radial equation, after multiplication by \( R \) is
%
\begin{dmath}\label{eqn:emtproblemSet7Problem2:180}
0 = \lr{ r^2 R' }' + \lr{ r^2 \beta^2 - n(n+1)} R,
\end{dmath}
%
or
\begin{dmath}\label{eqn:emtproblemSet7Problem2:200}
0 = r^2 R'' + 2 r R' + \lr{ r^2 \beta^2 - n(n+1) } R.
\end{dmath}
%
After a variable substitution, \( z = r \beta \), this is
%
%\begin{dmath}\label{eqn:emtproblemSet7Problem2:220}
\boxedEquation{eqn:emtproblemSet7Problem2:260}{
0 = z^2 \frac{d^2 R}{dz^2} + 2 z \frac{d R}{dz} + \lr{ z^2 - n(n+1) } R,
}
%\end{dmath}
%
which is the differential equation solved by spherical Bessel functions \( j_n(z), y_n(z) \).  This equation can be put into the standard Bessel equation form with a multiplicative transformation.  The standard Bessel equation is
%
\begin{dmath}\label{eqn:emtproblemSet7Problem2:240}
0 = z^2 \frac{d^2 y}{dz^2} + z \frac{d y}{dz} + \lr{ z^2 - \alpha^2 } y,
\end{dmath}
%
Let \( R = q y \), so that
\( R' = q' y + q y' \) and \( R'' = q'' y + 2 q' y' + q y'' \), let \( z^2 - n(n+1) = \mu \), and substitute these into \cref{eqn:emtproblemSet7Problem2:200}.  This gives
%
\begin{dmath}\label{eqn:emtproblemSet7Problem2:280}
0
=
z^2 \lr{ q'' y + 2 q' y' + q y'' } + 2 z \lr{ q' y + q y' } + \mu q y
=
q \lr{
z^2 y'' + z \frac{ 2 z q' + 2 q }{q} y' + \frac{ z^2 q'' + 2 z q' + \mu q }{q} y
}
\end{dmath}
%
We want
\begin{dmath}\label{eqn:emtproblemSet7Problem2:300}
2 z \frac{q'}{q} + 2 = 1,
\end{dmath}
%
which has solution
%
\begin{dmath}\label{eqn:emtproblemSet7Problem2:320}
q = z^{-1/2}.
\end{dmath}
%
We also have
%
\begin{equation}\label{eqn:emtproblemSet7Problem2:340}
\begin{aligned}
z^2 q'' /q + 2 z q'/q
&=
z^2 (-1/2)(-3/2) z^{-5/2} / z^{-1/2} + 2 z (-1/2) z^{-3/2}/z^{-1/2} \\
&=
\frac{3}{4}
- 1 \\
&=
-\inv{4},
\end{aligned}
\end{equation}
%
which gives
%
\begin{dmath}\label{eqn:emtproblemSet7Problem2:360}
0
= z^2 y'' + z y' + \lr{ z^2 - n(n+1) - \frac{1}{4} } y.
\end{dmath}
%
Note that
\begin{dmath}\label{eqn:emtproblemSet7Problem2:380}
n(n+1) + \inv{4}
=
\inv{4} \lr{ 4 n^2 + 4 n + 1 }
=
\inv{4} \lr{ 2 n + 1 }^2
=
\lr{ n + \inv{2} }^2,
\end{dmath}
%
so the radial equation has the general solution
%
%\begin{dmath}\label{eqn:emtproblemSet7Problem2:400}
\boxedEquation{eqn:emtproblemSet7Problem2:400}{
R( r)
=
A \frac{Y_{n+ 1/2}(\beta r)}{\sqrt{\beta r}}
+
B \frac{J_{n+ 1/2}(\beta r)}{\sqrt{\beta r}}.
}
%\end{dmath}
%
\paragraph{Nature of solution for the \(\theta\) dependent equation.}
%
To determine the nature of the solution of the \( \theta \) dependent equation \cref{eqn:emtproblemSet7Problem2:420}, it seems natural to make a change of variables \( x = \sin\theta \).  The derivative operator becomes
%
\begin{dmath}\label{eqn:emtproblemSet7Problem2:480}
\frac{d}{d\theta}
=
\frac{dx}{d\theta} \frac{d}{dx}
=
\cos\theta
\frac{d}{dx}
=
\sqrt{ 1 - x^2 }
\frac{d}{dx},
\end{dmath}
%
so the differential equation becomes
%
\begin{dmath}\label{eqn:emtproblemSet7Problem2:500}
0
=
\frac{d}{d\theta} \lr{ x \frac{d}{d\theta} T }
+ \lr{ n(n+1) x^2 - m^2 } T
=
\sqrt{1 - x^2} \frac{d}{dx} \lr{ x \sqrt{ 1 - x^2} \frac{d}{dx} T }
+ \lr{ n(n+1) x^2 - m^2 } T.
\end{dmath}
%
Note that
%
\begin{dmath}\label{eqn:emtproblemSet7Problem2:520}
\frac{d}{dx} \lr{ x \sqrt{ 1 - x^2} }
=
\sqrt{ 1 - x^2}
+ \frac{-x^2}{\sqrt{1 - x^2}}
=
\frac{1 - 2 x^2}{\sqrt{1 - x^2}},
\end{dmath}
%
so
%
\begin{dmath}\label{eqn:emtproblemSet7Problem2:540}
0
=
x (1-x^2) T''
+
\lr{ 1 - 2 x^2 } T'
+
\lr{ n(n+1) x^2 - m^2 } T.
\end{dmath}
%
Digging out my old differential equations text \citep{boyce1969elementary}, I see that this equation has three regular singular points (\(x_0 \in \setlr{ 0, 1, -1 }\)), and admits a power series solution around each such point of the form
%
\begin{dmath}\label{eqn:emtproblemSet7Problem2:560}
T(x) = (x - x_0)^r \sum_{n=0}^\infty a_n (x - x_0)^n.
\end{dmath}
%
The power \( r \), called the exponent of the singularity, is determined after substitution by the lowest power term, and recurrence relations for the \( a_n \) coefficients can be determined after that.
}}

         %
% Copyright � 2016 Peeter Joot.  All Rights Reserved.
% Licenced as described in the file LICENSE under the root directory of this GIT repository.
%
\makeproblem{Orthogonality conditions for the fields.}{emt:problemSet7:3}{
Consider plane waves
\begin{equation}\label{eqn:emtproblemSet7Problem3:20}
\begin{aligned}
\BE &= \BE_0 e^{-j \Bk \cdot \Br + j \omega t } \\
\BH &= \BH_0 e^{-j \Bk \cdot \Br + j \omega t }
\end{aligned}
\end{equation}
propagating in a homogeneous, lossless, source free region for which \( \epsilon > 0 \), \( \mu > 0 \), and where \( \BE_0, \BH_0 \) are constant.
\makesubproblem{}{emt:problemSet7:3a}
Show that
\( \Bk \perp \BE \) and \( \Bk \perp \BH \).
\makesubproblem{}{emt:problemSet7:3b}
Show that  \( \Bk, \BE, \BH \) form a right hand triplet as indicated in \cref{fig:kEHrightHandTriplet:kEHrightHandTripletFig1}.
\imageFigure{../figures/ece1228-electromagnetic-theory/kEHrightHandTripletFig1}{Right handed triplet.}{fig:kEHrightHandTriplet:kEHrightHandTripletFig1}{0.2}
\paragraph{Hint:} show that
\( \Bk \cross \BE = \omega \mu \BH \) and
\( \Bk \cross \BH = -\omega \epsilon \BE \).
\makesubproblem{}{emt:problemSet7:3c}
Now suppose \( \epsilon, \mu < 0 \), how does the figure change?  Redraw the figure.
} % makeproblem
\makeanswer{emt:problemSet7:3}{\withproblemsetsParagraph{
\makeSubAnswer{}{emt:problemSet7:3a}
Since \( \spacegrad \cdot \BD = 0 \) in a source free region, application of the divergence to the electric field of
\cref{eqn:emtproblemSet7Problem3:20} we have
\begin{dmath}\label{eqn:emtproblemSet7Problem3:40}
0
= \spacegrad \cdot \BE
= \BE_0 \cdot \spacegrad e^{ -j \Bk \cdot \Br + j \omega t }
= \sum_m \BE_0 \cdot \Be_m \partial_m e^{ -j \Bk \cdot \Br + j \omega t }
= \sum_m \BE_0 \cdot \Be_m (-j k_m) e^{ -j \Bk \cdot \Br + j \omega t }
= (\BE_0 \cdot (-j \Bk)) e^{ -j \Bk \cdot \Br + j \omega t },
\end{dmath}
so
\begin{dmath}\label{eqn:emtproblemSet7Problem3:60}
\BE_0 \cdot \Bk = 0.
\end{dmath}
For the magnetic field we also have zero divergence, and must also have this same perpendicular relationship, which is easily verified
\begin{dmath}\label{eqn:emtproblemSet7Problem3:120}
0
= \spacegrad \cdot \BB
= \mu \BH_0 \cdot \spacegrad e^{ -j \Bk \cdot \Br + j \omega t }
= \mu (\BH_0 \cdot (-j \Bk)) e^{ -j \Bk \cdot \Br + j \omega t },
\end{dmath}
so
\begin{dmath}\label{eqn:emtproblemSet7Problem3:140}
\BH_0 \cdot \Bk = 0.
\end{dmath}
This shows that \( \Bk \perp \BE \), and \( \Bk \perp \BH \) for plane waves in source free simple media.
\makeSubAnswer{}{emt:problemSet7:3b}
We can relate the electric and magnetic fields to each other with the time harmonic cross product equations
\begin{equation}\label{eqn:emtproblemSet7Problem3:80}
\begin{aligned}
\spacegrad \cross \BE &= - j \omega \BB \\
\spacegrad \cross \BH &= j \omega \BD.
\end{aligned}
\end{equation}
For the magnetic field, we find
\begin{dmath}\label{eqn:emtproblemSet7Problem3:100}
\BH
= \frac{1}{-j\mu \omega} \spacegrad \cross \BE
= \sum_m \frac{1}{-j\mu \omega} \Be_m \cross \BE_0 \partial_m e^{ -j \Bk \cdot \Br + j \omega t }
= \sum_m \frac{1}{-j\mu \omega} \Be_m \cross \BE_0 (-j k_m) e^{ -j \Bk \cdot \Br + j \omega t }
= \frac{1}{-j\mu \omega} \Bk \cross \BE_0 e^{ -j \Bk \cdot \Br + j \omega t }
= \frac{1}{\mu \omega} \Bk \cross \BE,
\end{dmath}
and for the electric field
\begin{dmath}\label{eqn:emtproblemSet7Problem3:180}
\BE
= \frac{1}{j \epsilon \omega} \spacegrad \cross \BH
= \sum_m \frac{1}{j \epsilon \omega} \Be_m \cross \BH_0 \partial_m e^{ -j \Bk \cdot \Br + j \omega t }
= \sum_m \frac{1}{j \epsilon \omega} \Be_m \cross \BH_0 (-j k_m) e^{ -j \Bk \cdot \Br + j \omega t }
= \frac{1}{j \epsilon \omega} \Bk \cross \BH_0 e^{ -j \Bk \cdot \Br + j \omega t }
= -\frac{1}{\epsilon \omega} \Bk \cross \BH.
\end{dmath}
When \( \mu, \epsilon \) are both positive, this demonstrates a right handed relationship between \( \BH, \Bk \) and \( \BE \)
\boxedEquation{eqn:emtproblemSet7Problem3:200}{
\begin{aligned}
\BH &= \frac{1}{\mu \omega} \Bk \cross \BE \\
\BE &= \frac{1}{\epsilon \omega} \BH \cross \Bk.
\end{aligned}
}
\makeSubAnswer{}{emt:problemSet7:3c}
When \( \mu, \epsilon \) are both less than zero, we have
\boxedEquation{eqn:emtproblemSet7Problem3:220}{
\begin{aligned}
\BH &= \frac{1}{\Abs{\mu} \omega} \BE \cross \Bk \\
\BE &= \frac{1}{\Abs{\epsilon} \omega} \Bk \cross \BH.
\end{aligned}
}
Now \( \Bk, \BH, \BE \) form a right handed triplet, which means that \( \Bk, \BE, \BH \) is a left handed triplet.  Both the fields are still perpendicular to \( \Bk \).  This is sketched in \cref{fig:ps7p3:ps7p3Fig1}.
\imageFigure{../figures/ece1228-electromagnetic-theory/ps7p3Fig1}{Negative permittivities field orientation.}{fig:ps7p3:ps7p3Fig1}{0.2}
}}

   \mychapter{Quadrupole expansion.}
      %
% Copyright © 2016 Peeter Joot.  All Rights Reserved.
% Licenced as described in the file LICENSE under the root directory of this GIT repository.
%
%
\paragraph{Quadrupole potential}
%
In Jackson
\citep{jackson1975cew}
,
is the following
%
\begin{dmath}\label{eqn:emtLecture8:420}
\inv{\Abs{\Bx - \Bx'}}
=
4 \pi \sum_{l= 0}^\infty \sum_{m = -l}^l \inv{2 l + 1} \frac{(r')^l}{r^{l+1}}
Y^\conj_{l,m}(\theta', \phi')
Y_{l,m}(\theta, \phi),
\end{dmath}
%
where \( Y_{l,m} \) are the spherical harmonics.  It appears that this is actually just an orthogonal function expansion of the inverse distance (for a region outside of the charge density).  The proof of this in is scattered through chapter 3, dependent on a similar expansion in Legendre polynomials, for an the azimuthally symmetric configuration.
%
It looks like quite a project to get comfortable enough with these special functions to fully reproduce the proof of this identity.  We are forced to play engineer, and assume the mathematics works out.  If we do that and plug this inverse distance formula into
the potential we have
%
\begin{dmath}\label{eqn:emtLecture8:440}
\phi(\Bx)
= \inv{4 \pi \epsilon_0} \int \frac{\rho(\Bx') d^3 x'}{\Abs{\Bx - \Bx'}}
=
\inv{4 \pi \epsilon_0} \int \rho(\Bx') d^3 x' \lr{
4 \pi \sum_{l= 0}^\infty \sum_{m = -l}^l \inv{2 l + 1} \frac{(r')^l}{r^{l+1}}
Y^\conj_{l,m}(\theta', \phi')
Y_{l,m}(\theta, \phi)
}
=
\inv{\epsilon_0}
\sum_{l= 0}^\infty \sum_{m = -l}^l \inv{2 l + 1}
\int \rho(\Bx') d^3 x' \lr{
\frac{(r')^l}{r^{l+1}}
Y^\conj_{l,m}(\theta', \phi')
Y_{l,m}(\theta, \phi)
}
=
\inv{\epsilon_0}
\sum_{l= 0}^\infty \sum_{m = -l}^l \inv{2 l + 1}
\lr{
\int (r')^l \rho(\Bx')
Y^\conj_{l,m}(\theta', \phi')
d^3 x'
}
\frac{
Y_{l,m}(\theta, \phi)
}
{
r^{l+1}
}
\end{dmath}
%
The integral terms are called the coefficients of the multipole moments, denoted
\begin{dmath}\label{eqn:emtLecture8:480}
q_{l,m} =
\int (r')^l \rho(\Bx')
Y^\conj_{l,m}(\theta', \phi')
d^3 x',
\end{dmath}
%
The \( l = 0,1,2\) terms are, respectively, called the monopole, dipole, and quadrupole terms of the potential
\begin{dmath}\label{eqn:emtLecture8:500}
\rho(\Bx) =
\inv{4 \pi \epsilon_0}
\sum_{l= 0}^\infty \sum_{m = -l}^l \frac{4\pi} {2 l + 1}
q_{l,m}
\frac{
Y_{l,m}(\theta, \phi)
}
{
r^{l+1}
}.
\end{dmath}
%
Note the power of this expansion.  Should we wish to compute the electric field, we have only to compute the gradient of  the last (\(Y_{l,m} r^{-l-1} \)) portion (since \( q_{l,m} \) is a constant).
%
\begin{dmath}\label{eqn:emtLecture8:520}
q_{1,1}
=
-\int \sqrt{\frac{3}{8 \pi}} \sin\theta' e^{-i\phi'} r' \rho(\Bx') dV'
=
-\sqrt{\frac{3}{8 \pi}} \int \sin\theta' \lr{ \cos\phi' - i\sin\phi'} r' \rho(\Bx') dV'
=
-\sqrt{\frac{3}{8 \pi}} \lr{
\int x' \rho(\Bx') dV'
-i \int y' \rho(\Bx') dV'
}
=
-\sqrt{\frac{3}{8 \pi}} \lr{
p_x - i p_y
}.
\end{dmath}
%
Here we've used
\begin{equation}\label{eqn:emtLecture8:540}
\begin{aligned}
x' &= r' \sin\theta' \cos\phi' \\
y' &= r' \sin\theta' \sin\phi' \\
z' &= r' \cos\theta'
\end{aligned}
\end{equation}
%
and the \( Y_{11} \) representation
%
\begin{equation}\label{eqn:emtLecture8:560}
\begin{aligned}
Y_{00} &= -\sqrt{\frac{1}{4 \pi}} \\
Y_{11} &= -\sqrt{\frac{3}{8 \pi}} \sin\theta e^{i\phi} \\
Y_{10} &=  \sqrt{\frac{3}{4 \pi}} \cos\theta  \\
Y_{22} &= -\inv{4} \sqrt{\frac{15}{2 \pi}} \sin^2\theta e^{2 i\phi} \\
Y_{21} &=  \inv{2} \sqrt{\frac{15}{2 \pi}} \sin\theta \cos\theta e^{i\phi} \\
Y_{20} &=  \inv{4} \sqrt{\frac{5}{\pi}} \lr{ 3 \cos^2\theta - 1 } \\
\end{aligned}
\end{equation}
%
%NOTE: compute a few of the more tedious moment coeffients.  These have been exam questions in the past.
%
With the usual dipole moment expression
%
\begin{dmath}\label{eqn:emtLecture8:580}
\Bp = \int \Bx' \rho(\Bx') d^3 x',
\end{dmath}
%
and a quadrupole moment defined as
\begin{dmath}\label{eqn:emtLecture8:600}
Q_{i,j} = \int \lr{ 3 x_i' x_j' - \delta_{ij} (r')^2 } \rho(\Bx') d^3 x',
\end{dmath}
%
the first order terms of the potential are now fully specified
\begin{dmath}\label{eqn:emtLecture8:620}
\phi(\Bx)
=
\inv{4 \pi \epsilon_0}
\lr{
q + \frac{\Bp \cdot \Bx}{r^3} +
\inv{2} \sum_{ij} Q_{ij} \frac{x_i x_j}{r^5}
}.
\end{dmath}

      %
% Copyright � 2016 Peeter Joot.  All Rights Reserved.
% Licenced as described in the file LICENSE under the root directory of this GIT repository.
%
%{
%\input{../blogpost.tex}
%\renewcommand{\basename}{momentCoeffiecients}
%%\renewcommand{\dirname}{notes/phy1520/}
%\renewcommand{\dirname}{notes/ece1228-electromagnetic-theory/}
%%\newcommand{\dateintitle}{}
%%\newcommand{\keywords}{}
%
%\input{../latex/peeter_prologue_print2.tex}
%
%\usepackage{peeters_layout_exercise}
%\usepackage{peeters_braket}
%\usepackage{peeters_figures}
%\usepackage{siunitx}
%%\usepackage{txfonts} % \ointclockwise
%
%\beginArtNoToc
%
%\generatetitle{Dipole and Quadrupole electrostatic potential moments and coefficients}
%\chapter{Dipole and Quadropole electrostatic potential moments and coefficents}
%\label{chap:momentCoeffiecients}
%
\paragraph{Explicit moment and quadrupole expansion.}
%
We calculated the \( q_{1,1} \) coefficient of the electrostatic moment, as covered in \citep{jackson1975cew} chapter 4.  Let's verify the rest, as well as the tensor sum formula for the quadrupole moment, and the spherical harmonic sum that yields the dipole moment potential.
%
%%%XX
%%%\begin{dmath}\label{eqn:momentCoeffiecients:20}
%%%q_{l,m} =
%%%\int (r')^l \rho(\Bx')
%%%Y^\conj_{l,m}(\theta', \phi')
%%%d^3 x',
%%%\end{dmath}
%%%
%%%The class notes also give the results for \( q_{0,0}, q_{1,0}, q_{2,2}, q_{2,1}, q_{2,0} \).  Let's verify those
%%%
%%%\paragraph{\(q_{0,0}\)}
%%%
%%%\begin{dmath}\label{eqn:momentCoeffiecients:40}
%%%q_{0,0}
%%%=
%%%\int (r')^0 \rho(\Bx')
%%%Y^\conj_{0,0}(\theta', \phi')
%%%d^3 x'
%%%=
%%%\inv{4\pi}
%%%\int \rho(\Bx') d^3 x'
%%%=
%%%\frac{q}{4\pi}
%%%\end{dmath}
%%%
%%%\paragraph{\(q_{1,0}\)}
%%%
%%%\begin{dmath}\label{eqn:momentCoeffiecients:60}
%%%q_{1,0}
%%%=
%%%\int r' \rho(\Bx')
%%%Y^\conj_{1,0}(\theta', \phi')
%%%d^3 x'
%%%=
%%%\sqrt{\frac{3}{4\pi}}
%%%\int r' \rho(\Bx')
%%%\cos\theta'
%%%d^3 x'
%%%=
%%%\sqrt{\frac{3}{4\pi}}
%%%\int r' \rho(\Bx') \cos\theta' d^3 x'
%%%=
%%%\sqrt{\frac{3}{4\pi}}
%%%\int z' \rho(\Bx') d^3 x'
%%%=
%%%\sqrt{\frac{3}{4\pi}} p_z
%%%\end{dmath}
%%%
%%%\paragraph{\(q_{2,2}\)}
%%%
%%%\begin{dmath}\label{eqn:momentCoeffiecients:80}
%%%q_{2,2}
%%%=
%%%\int (r')^2 \rho(\Bx')
%%%Y^\conj_{2,2}(\theta', \phi')
%%%d^3 x'
%%%=
%%%\sqrt{\frac{15}{32 \pi}}
%%%\int (r')^2 \rho(\Bx')
%%%\sin^2 \theta e^{-2 i\phi}
%%%d^3 x'
%%%=
%%%\sqrt{\frac{15}{32 \pi}}
%%%\int (r')^2 \rho(\Bx')
%%%\sin^2 \theta \lr{ \cos \phi - i \sin\phi }^2
%%%d^3 x'
%%%=
%%%\sqrt{\frac{15}{32 \pi}}
%%%\int (r')^2 \rho(\Bx')
%%%\sin^2 \theta \lr{ \cos^2\phi - \sin^2\phi - 2 i \cos\phi \sin\phi }
%%%d^3 x'
%%%=
%%%\sqrt{\frac{15}{32 \pi}}
%%%\int \rho(\Bx') \lr{
%%%(x')^2
%%%- (y')^2
%%%- 2 i x' y' } d^3 x'
%%%=
%%%\sqrt{\frac{15}{32 \pi}}
%%%\int \rho(\Bx') \lr{
%%%x' - i y'
%%%}^2 d^3 x'
%%%%=
%%%%\sqrt{\frac{15}{32 \pi}}
%%%%\lr{ p_x - i p_y }^2
%%%\end{dmath}
%%%
%%%\paragraph{\(q_{2,1}\)}
%%%
%%%\begin{dmath}\label{eqn:momentCoeffiecients:100}
%%%q_{2,1}
%%%=
%%%\int (r')^2 \rho(\Bx')
%%%Y^\conj_{2,1}(\theta', \phi')
%%%d^3 x'
%%%=
%%%-\sqrt{\frac{15}{8 \pi}}
%%%\int (r')^2 \rho(\Bx')
%%%\sin\theta' \cos\theta' e^{-i \phi}
%%%d^3 x'
%%%=
%%%-\sqrt{\frac{15}{8 \pi}}
%%%\int (r')^2 \rho(\Bx')
%%%\sin\theta' \cos\theta' \lr{ \cos\phi - i \sin\phi }
%%%d^3 x'
%%%=
%%%-\sqrt{\frac{15}{8 \pi}}
%%%\int \rho(\Bx')
%%%\lr{ x' z' - i y' z' }
%%%d^3 x'
%%%%=
%%%%-\sqrt{\frac{15}{8 \pi}} p_z \lr{ p_x - i p_y }.
%%%\end{dmath}
%%%
%%%\paragraph{\(q_{2,0}\)}
%%%
%%%\begin{dmath}\label{eqn:momentCoeffiecients:260}
%%%q_{2,0}
%%%=
%%%\int (r')^2 \rho(\Bx')
%%%Y^\conj_{2,0}(\theta', \phi')
%%%d^3 x'
%%%=
%%%\int (r')^2 \rho(\Bx') \sqrt{\frac{5}{4\pi}} \lr{ \frac{3}{2} \cos^2\theta - \inv{2} }
%%%d^3 x'
%%%=
%%%\inv{2} \sqrt{\frac{5}{4\pi}}
%%%\int \rho(\Bx')
%%%\lr{ 3 (z')^2 - (r')^2 }
%%%d^3 x'.
%%%\end{dmath}
%%%
%%%\paragraph{\(Q_{ij}\)}
%%%XX
The quadrupole term of the potential was stated to be
%
\begin{equation}\label{eqn:momentCoeffiecients:120}
\begin{aligned}
\inv{4 \pi \epsilon_0} &\frac{4 \pi}{5 r^3} \sum_{m=-2}^2 \int (r')^2 \rho(\Bx') Y_{lm}^\conj(\theta', \phi') Y_{lm}(\theta, \phi) \\
&=
\inv{2} \sum_{ij} Q_{ij} \frac{x_i x_j}{r^5},
\end{aligned}
\end{equation}
%
where
%
\begin{dmath}\label{eqn:momentCoeffiecients:140}
Q_{i,j} = \int \lr{ 3 x_i' x_j' - \delta_{ij} (r')^2 } \rho(\Bx') d^3 x'.
\end{dmath}
%
Let's verify this.  First note that
%
\begin{dmath}\label{eqn:momentCoeffiecients:160}
Y_{l,m} = \sqrt{\frac{2 l + 1}{4 \pi} \frac{(l-m)!}{(l+m)!}} P_l^m(\cos\theta) e^{i m \phi},
\end{dmath}
%
and
\begin{dmath}\label{eqn:momentCoeffiecients:180}
P_l^{-m}(x) =
(-1)^m \frac{(l-m)!}{(l+m)!} P_l^m(x),
\end{dmath}
%
so
\begin{dmath}\label{eqn:momentCoeffiecients:200}
Y_{l,-m}
= \sqrt{\frac{2 l + 1}{4 \pi} \frac{(l+m)!}{(l-m)!} }
P_l^{-m}(\cos\theta)
e^{-i m \phi}
=
(-1)^m
\sqrt{\frac{2 l + 1}{4 \pi} \frac{(l-m)!}{(l+m)!} }
P_l^m(x)
e^{-i m \phi}
=
(-1)^m Y_{l,m}^\conj.
\end{dmath}
%
That means
%
\begin{dmath}\label{eqn:momentCoeffiecients:220}
q_{l,-m}
=
\int (r')^l \rho(\Bx')
Y^\conj_{l,-m}(\theta', \phi')
d^3 x'
=
(-1)^m
\int (r')^l \rho(\Bx')
Y_{l,m}(\theta', \phi')
d^3 x'
=
(-1)^m q_{lm}^\conj.
\end{dmath}
%
In particular, for \( m \ne 0 \)
%
\begin{dmath}\label{eqn:momentCoeffiecients:320}
(r')^l Y_{l, m}^\conj (\theta', \phi') r^l Y_{l, m}(\theta, \phi)
+ (r')^l Y_{l, -m}^\conj (\theta', \phi') r^l Y_{l, -m}(\theta, \phi)
=
(r')^l Y_{l, m}^\conj (\theta', \phi') r^l Y_{l, m}(\theta, \phi)
+ (r')^l Y_{l, m} (\theta', \phi') r^l Y_{l, m}^\conj(\theta, \phi) ,
\end{dmath}
%
or
\begin{dmath}\label{eqn:momentCoeffiecients:340}
(r')^l Y_{l, m}^\conj (\theta', \phi') r^l Y_{l, m}(\theta, \phi)
+ (r')^l Y_{l, -m}^\conj (\theta', \phi') r^l Y_{l, -m}(\theta, \phi)
=
2 \Real \lr{ (r')^l Y_{l, m}^\conj (\theta', \phi') r^l Y_{l, m}(\theta, \phi) }.
\end{dmath}
%
To verify the quadrupole expansion formula in a compact way it is helpful to compute some intermediate results.
%
\begin{dmath}\label{eqn:momentCoeffiecients:360}
r Y_{1, 1}
= -r \sqrt{\frac{3}{8 \pi}} \sin\theta e^{i\phi}
= -\sqrt{\frac{3}{8 \pi}} (x + i y),
\end{dmath}
%
\begin{dmath}\label{eqn:momentCoeffiecients:380}
r Y_{1, 0}
= r \sqrt{\frac{3}{4 \pi}} \cos\theta
= \sqrt{\frac{3}{4 \pi}} z,
\end{dmath}
%
\begin{dmath}\label{eqn:momentCoeffiecients:400}
r^2 Y_{2, 2}
= -r^2 \sqrt{\frac{15}{32 \pi}} \sin^2\theta e^{2 i\phi}
= - \sqrt{\frac{15}{32 \pi}} (x + i y)^2,
\end{dmath}
%
\begin{dmath}\label{eqn:momentCoeffiecients:420}
r^2 Y_{2, 1}
= r^2 \sqrt{\frac{15}{8 \pi}} \sin\theta \cos\theta e^{i\phi}
= \sqrt{\frac{15}{8 \pi}} z ( x + i y ),
\end{dmath}
%
\begin{dmath}\label{eqn:momentCoeffiecients:440}
r^2 Y_{2, 0}
= r^2 \sqrt{\frac{5}{16 \pi}} \lr{ 3 \cos^2\theta - 1 }
= \sqrt{\frac{5}{16 \pi}} \lr{ 3 z^2 - r^2 }.
\end{dmath}
%
Given primed coordinates and integrating the conjugate of each of these with \( \rho(\Bx') dV' \), we obtain the \( q_{lm} \) moment coefficients.  Those are
%
\begin{dmath}\label{eqn:momentCoeffiecients:460}
q_{11}
= -\sqrt{\frac{3}{8 \pi}} \int d^3 x' \rho(\Bx') (x - i y),
\end{dmath}
%
\begin{dmath}\label{eqn:momentCoeffiecients:480}
q_{1, 0}
= \sqrt{\frac{3}{4 \pi}} \int d^3 x' \rho(\Bx') z',
\end{dmath}
%
\begin{dmath}\label{eqn:momentCoeffiecients:500}
q_{2, 2}
= - \sqrt{\frac{15}{32 \pi}} \int d^3 x' \rho(\Bx') (x' - i y')^2,
\end{dmath}
%
\begin{dmath}\label{eqn:momentCoeffiecients:520}
q_{2, 1}
= \sqrt{\frac{15}{8 \pi}} \int d^3 x' \rho(\Bx') z' ( x' - i y' ),
\end{dmath}
%
\begin{dmath}\label{eqn:momentCoeffiecients:540}
q_{2, 0}
= \sqrt{\frac{5}{16 \pi}} \int d^3 x' \rho(\Bx') \lr{ 3 (z')^2 - (r')^2 }.
\end{dmath}
%
For the potential we are interested in
%
\begin{dmath}\label{eqn:momentCoeffiecients:560}
2 \Real q_{11} r^2 Y_{11}(\theta, \phi)
= 2 \frac{3}{8 \pi} \int d^3 x' \rho(\Bx') \Real \lr{ (x' - i y')( x + i y) }
= \frac{3}{4 \pi} \int d^3 x' \rho(\Bx') \lr{ x x' + y y' },
\end{dmath}
%
\begin{dmath}\label{eqn:momentCoeffiecients:580}
q_{1, 0} r Y_{1,0}(\theta, \phi)
= \frac{3}{4 \pi} \int d^3 x' \rho(\Bx') z' z,
\end{dmath}
%
\begin{dmath}\label{eqn:momentCoeffiecients:600}
2 \Real q_{22} r^2 Y_{22}(\theta, \phi)
= 2 \frac{15}{32 \pi} \int d^3 x' \rho(\Bx') \Real \lr{
(x' - i y')^2
(x + i y)^2
}
= \frac{15}{16 \pi} \int d^3 x' \rho(\Bx') \Real \lr{
((x')^2 - 2 i x' y' -(y')^2)
(x^2 + 2 i x y -y^2)
}
= \frac{15}{16 \pi} \int d^3 x' \rho(\Bx') \lr{
((x')^2 -(y')^2) (x^2 -y^2)
+ 4 x x' y y'
},
\end{dmath}
%
\begin{dmath}\label{eqn:momentCoeffiecients:620}
2 \Real q_{21} r^2 Y_{21}(\theta, \phi)
= 2 \frac{15}{8 \pi} \int d^3 x' \rho(\Bx') z \Real \lr{ ( x' - i y' ) (x + i y) }
= \frac{15}{4 \pi} \int d^3 x' \rho(\Bx') z \lr{ x x' + y y' },
\end{dmath}
%
and
\begin{dmath}\label{eqn:momentCoeffiecients:640}
q_{2, 0} r^2 Y_{20}(\theta, \phi)
= \frac{5}{16 \pi} \int d^3 x' \rho(\Bx') \lr{ 3 (z')^2 - (r')^2 } \lr{ 3 z^2 - r^2 }.
\end{dmath}
%
The dipole term of the potential is
%
\begin{dmath}\label{eqn:momentCoeffiecients:660}
\inv{ 4 \pi \epsilon_0 } \frac{4 \pi}{3 r^3}
\lr{
\frac{3}{4 \pi} \int d^3 x' \rho(\Bx') \lr{ x x' + y y' }
+
\frac{3}{4 \pi} \int d^3 x' \rho(\Bx') z' z
}
=
\inv{ 4 \pi \epsilon_0 r^3}
\Bx \cdot \int d^3 x' \rho(\Bx') \Bx'
=
\frac{\Bx \cdot \Bp}{ 4 \pi \epsilon_0 r^3},
\end{dmath}
%
as obtained directly when a strict dipole approximation was used.
%
Summing all the terms for the quadrupole gives
%
\begin{equation}\label{eqn:momentCoeffiecients:680}
\begin{aligned}
\inv{ 4 \pi \epsilon r^5 } \frac{ 4 \pi }{5}
\biglr{
&\frac{15}{16 \pi} \int d^3 x' \rho(\Bx') \lr{
((x')^2 -(y')^2) (x^2 -y^2)
+ 4 x x' y y'
} \\
&+
\frac{15}{4 \pi} \int d^3 x' \rho(\Bx') z z' \lr{ x x' + y y' } \\
&+
\frac{5}{16 \pi} \int d^3 x' \rho(\Bx') \lr{ 3 (z')^2 - (r')^2 } \lr{ 3 z^2 - r^2 }
} \\
=
\inv{ 4 \pi \epsilon r^5 }
\int d^3 x' \rho(\Bx')
\inv{4}
\biglr{
   &3
   \lr{
   ((x')^2 -(y')^2) (x^2 -y^2)
   + 4 x x' y y'
   } \\
   &+
   12
   z z' \lr{ x x' + y y' } \\
   &+
   \lr{ 3 (z')^2 - (r')^2 } \lr{ 3 z^2 - r^2 }
}.
\end{aligned}
\end{equation}
%
The portion in brackets is
%
\begin{equation}\label{eqn:momentCoeffiecients:700}
\begin{aligned}
   3
   &\lr{
      ((x')^2 -(y')^2) (x^2 -y^2)
      + 4 x x' y y'
   } \\
   +
   12
   & z z' \lr{ x x' + y y' }  \\
   +
   &\lr{ 2 (z')^2 - (x')^2 - (y')^2} \lr{ 2 z^2 - x^2 -y^2 } \\
=
x^2 &\lr{
     3 (x')^2 - 3(y')^2
-
   \lr{ 2 (z')^2 - (x')^2 - (y')^2}
} \\
+
y^2 &\lr{
      -3 (x')^2 + 3 (y')^2
-
   \lr{ 2 (z')^2 - (x')^2 - (y')^2}
} \\
+
2 z^2 &\lr{
   2 (z')^2 - (x')^2 - (y')^2
} \\
+
&12{ x x' y y' + x x' z z' + y y' z z' } \\
=
2 x^2 &\lr{
     2 (x')^2 - (y')^2 - (z')^2
} \\
+
2 y^2 &\lr{
     2 (y')^2 - (x')^2 - (z')^2
} \\
+
2 z^2 &\lr{
   2 (z')^2 - (x')^2 - (y')^2
} \\
+
&12{ x x' y y' + x x' z z' + y y' z z' }.
\end{aligned}
\end{equation}
%
The quadrupole sum can now be written as
\begin{dmath}\label{eqn:momentCoeffiecients:720}
\inv{2}
\inv{ 4 \pi \epsilon r^5 }
\int d^3 x' \rho(\Bx')
\biglr{
x^2 \lr{ 3 (x')^2 - (r')^2 }
+y^2 \lr{ 3 (y')^2 - (r')^2 }
+z^2 \lr{ 3 (z')^2 - (r')^2 }
+
3 \lr{
x y x' y'
+y x y' x'
+x z x' z'
+z x z' x'
+y z y' z'
+z y z' y'
}
},
\end{dmath}
%
which is precisely \cref{eqn:momentCoeffiecients:120}, the quadrupole potential stated in the text and class notes.
%
%}
%\EndArticle

      \section{Problems.}
         %
% Copyright � 2016 Peeter Joot.  All Rights Reserved.
% Licenced as described in the file LICENSE under the root directory of this GIT repository.
%
%{
%\input{../blogpost.tex}
%\renewcommand{\basename}{dipoleFromSphericalMoments}
%%\renewcommand{\dirname}{notes/phy1520/}
%\renewcommand{\dirname}{notes/ece1228-electromagnetic-theory/}
%%\newcommand{\dateintitle}{}
%%\newcommand{\keywords}{}
%
%\input{../latex/peeter_prologue_print2.tex}
%
%\usepackage{peeters_layout_exercise}
%\usepackage{peeters_braket}
%\usepackage{peeters_figures}
%\usepackage{siunitx}
%%\usepackage{mhchem} % \ce{}
%%\usepackage{macros_bm} % \bcM
%%\usepackage{txfonts} % \ointclockwise
%
%\beginArtNoToc
%
%\generatetitle{Dipole field from multipole moment sum}
%\chapter{Dipole field from multipole moment sum}
%\label{chap:dipoleFromSphericalMoments}
%
\makeproblem{Dipole multipole moment.}{problem:dipoleFromSphericalMoments:1}{
Following Jackson \citep{jackson1975cew}, derive the electric field contribution from the dipole terms of the multipole sum, but don't skip the details.
} % problem
%
\makeanswer{problem:dipoleFromSphericalMoments:1}{
%\withproblemsetsParagraph{
The components of the electric field can be obtained directly from the multipole moments
%
\begin{dmath}\label{eqn:dipoleFromSphericalMoments:20}
\Phi(\Bx)
= \inv{4 \pi \epsilon_0} \sum \frac{4 \pi}{ (2 l + 1) r^{l + 1} } q_{l m} Y_{l m},
\end{dmath}
%
so for the \( l,m \) contribution to this sum the components of the electric field are
%
\begin{dmath}\label{eqn:dipoleFromSphericalMoments:40}
E_r
=
\inv{\epsilon_0} \sum \frac{l+1}{ (2 l + 1) r^{l + 2} } q_{l m} Y_{l m},
\end{dmath}
%
\begin{dmath}\label{eqn:dipoleFromSphericalMoments:60}
E_\theta
= -\inv{\epsilon_0} \sum \frac{1}{ (2 l + 1) r^{l + 2} } q_{l m} \partial_\theta Y_{l m}
\end{dmath}
%
\begin{dmath}\label{eqn:dipoleFromSphericalMoments:80}
E_\phi
= -\inv{\epsilon_0} \sum \frac{1}{ (2 l + 1) r^{l + 2} \sin\theta } q_{l m} \partial_\phi Y_{l m}
= -\inv{\epsilon_0} \sum \frac{j m}{ (2 l + 1) r^{l + 2} \sin\theta } q_{l m} Y_{l m}.
\end{dmath}
%
Here I've translated from CGS to SI.  Let's calculate the \( l = 1 \) electric field components directly from these expressions and check against the previously calculated results.
%
\begin{dmath}\label{eqn:dipoleFromSphericalMoments:100}
E_r
=
\inv{\epsilon_0} \frac{2}{ 3 r^{3} }
\lr{
   2 \lr{ -\sqrt{\frac{3}{8\pi}} }^2 \Real \lr{
      (p_x - j p_y) \sin\theta e^{j\phi}
   }
   +
   \lr{ \sqrt{\frac{3}{4\pi}} }^2 p_z \cos\theta
}
=
\frac{2}{4 \pi \epsilon_0 r^3}
\lr{
   p_x \sin\theta \cos\phi + p_y \sin\theta \sin\phi + p_z \cos\theta
}
=
\frac{1}{4 \pi \epsilon_0 r^3} 2 \Bp \cdot \rcap.
\end{dmath}
%
Note that
%
\begin{dmath}\label{eqn:dipoleFromSphericalMoments:120}
\partial_\theta Y_{11} = -\sqrt{\frac{3}{8\pi}} \cos\theta e^{j \phi},
\end{dmath}
%
and
%
\begin{dmath}\label{eqn:dipoleFromSphericalMoments:140}
\partial_\theta Y_{1,-1} = \sqrt{\frac{3}{8\pi}} \cos\theta e^{-j \phi},
\end{dmath}
%
so
%
\begin{dmath}\label{eqn:dipoleFromSphericalMoments:160}
E_\theta
=
-\inv{\epsilon_0} \frac{1}{ 3 r^{3} }
\lr{
   2 \lr{ -\sqrt{\frac{3}{8\pi}} }^2 \Real \lr{
      (p_x - j p_y) \cos\theta e^{j\phi}
   }
   -
   \lr{ \sqrt{\frac{3}{4\pi}} }^2 p_z \sin\theta
}
=
-\frac{1}{4 \pi \epsilon_0 r^3}
\lr{
   p_x \cos\theta \cos\phi + p_y \cos\theta \sin\phi - p_z \sin\theta
}
=
-\frac{1}{4 \pi \epsilon_0 r^3} \Bp \cdot \thetacap.
\end{dmath}
%
For the \(\phicap\) component, the \( m = 0 \) term is killed.  This leaves
%
\begin{dmath}\label{eqn:dipoleFromSphericalMoments:180}
E_\phi
=
-\frac{1}{\epsilon_0} \frac{1}{ 3 r^{3} \sin\theta }
\lr{
j q_{11} Y_{11} - j q_{1,-1} Y_{1,-1}
}
=
-\frac{1}{3 \epsilon_0 r^{3} \sin\theta }
\lr{
j q_{11} Y_{11} - j (-1)^{2m} q_{11}^\conj Y_{11}^\conj
}
=
\frac{2}{\epsilon_0} \frac{1}{ 3 r^{3} \sin\theta }
\Imag q_{11} Y_{11}
=
\frac{2}{3 \epsilon_0 r^{3} \sin\theta }
\Imag \lr{
   \lr{ -\sqrt{\frac{3}{8\pi}} }^2 (p_x - j p_y) \sin\theta e^{j \phi}
}
=
\frac{1}{ 4 \pi \epsilon_0 r^{3} }
\Imag \lr{
   (p_x - j p_y) e^{j \phi}
}
=
\frac{1}{ 4 \pi \epsilon_0 r^{3} }
\lr{
   p_x \sin\phi - p_y \cos\phi
}
=
-\frac{\Bp \cdot \phicap}{ 4 \pi \epsilon_0 r^3}.
\end{dmath}
%
That is
%\begin{dmath}\label{eqn:dipoleFromSphericalMoments:200}
\boxedEquation{eqn:dipoleFromSphericalMoments:200}{
\begin{aligned}
E_r &=
\frac{2}{4 \pi \epsilon_0 r^3}
\Bp \cdot \rcap \\
E_\theta &= -
\frac{1}{4 \pi \epsilon_0 r^3}
\Bp \cdot \thetacap \\
E_\phi &= -
\frac{1}{4 \pi \epsilon_0 r^3}
\Bp \cdot \phicap.
\end{aligned}
}
%\end{dmath}
%
These are consistent with equations (4.12) from the text for when \( \Bp \) is aligned with the z-axis.
%
Observe that we can sum each of the projections of \( \BE \) to construct the total electric field due to this \( l = 1 \) term of the multipole moment sum
%
\begin{dmath}\label{eqn:dipoleFromSphericalMoments:n}
\BE
=
\frac{1}{4 \pi \epsilon_0 r^3}
\lr{
2 \rcap (\Bp \cdot \rcap)
-
\phicap ( \Bp \cdot \phicap)
-
\thetacap ( \Bp \cdot \thetacap)
}
=
\frac{1}{4 \pi \epsilon_0 r^3}
\lr{
3 \rcap (\Bp \cdot \rcap)
-
\Bp
},
\end{dmath}
%
which recovers the expected dipole moment approximation.
%}
} % answer
%
%}
%\EndArticle

   \mychapter{Fresnel relations.}
      %
% Copyright � 2016 Peeter Joot.  All Rights Reserved.
% Licenced as described in the file LICENSE under the root directory of this GIT repository.
%
%\input{../blogpost.tex}
%\renewcommand{\basename}{emt10}
%\renewcommand{\dirname}{notes/ece1228/}
%\newcommand{\keywords}{ECE1228H}
%\input{../latex/peeter_prologue_print2.tex}
%
%%\usepackage{ece1228}
%\usepackage{peeters_braket}
%%\usepackage{peeters_layout_exercise}
%\usepackage{peeters_figures}
%\usepackage{mathtools}
%\usepackage{siunitx}
%\usepackage{macros_bm}
%
%\beginArtNoToc
%\generatetitle{ECE1228H Electromagnetic Theory.  Lecture 10: Fresnel relations.  Taught by Prof.\ M. Mojahedi}
%%\chapter{Fresnel relations}
%\label{chap:emt10}
%
%\paragraph{Motivation}
%
%In class, an overview of the Fresnel relations for a TE mode electric field were presented.  Here's a fleshing out of the details is presented, as well as the equivalent for the TM mode.
%
%Peeter's lecture notes from class.  These may be incoherent and rough.
%
%These are notes for the UofT course ECE1228H, Electromagnetic Theory, taught by Prof. M. Mojahedi, covering \textchapref{{1}} \citep{balanis1989advanced} content.
%
\section{Single interface TE mode.}
%
The Fresnel reflection geometry for an electric field \( \BE \) parallel to the interface (TE mode) is sketched in \cref{fig:fresnelTE:fresnelTEFig1}.
%
\imageFigure{../figures/ece1228-electromagnetic-theory/fresnelTEFig1}{Electric field TE mode Fresnel geometry.}{fig:fresnelTE:fresnelTEFig1}{0.3}
%
\begin{dmath}\label{eqn:emtLecture10:20}
   \bcE_i = \Be_2 E_i e^{j \omega t - j \Bk_{i} \cdot \Bx },
\end{dmath}
%
with an assumption that this field maintains it's polarization in both its reflected and transmitted components, so that
%
\begin{dmath}\label{eqn:emtLecture10:40}
   \bcE_r = \Be_2 r E_i e^{j \omega t - j \Bk_{r} \cdot \Bx },
\end{dmath}
%
and
\begin{dmath}\label{eqn:emtLecture10:60}
   \bcE_t = \Be_2 t E_i e^{j \omega t - j \Bk_{t} \cdot \Bx },
\end{dmath}
%
Measuring the angles \( \theta_i, \theta_r, \theta_t \) from the normal, with \( i = \Be_3 \Be_1 \) the wave vectors are
%
\begin{equation}\label{eqn:emtLecture10:620}
\begin{aligned}
\Bk_{i} &= \Be_3 k_1 e^{i\theta_i} = k_1\lr{ \Be_3 \cos\theta_i + \Be_1\sin\theta_i }, \\
\Bk_{r} &= -\Be_3 k_1 e^{-i\theta_r} = k_1 \lr{ -\Be_3 \cos\theta_r + \Be_1 \sin\theta_r }, \\
\Bk_{t} &= \Be_3 k_2 e^{i\theta_t} = k_2 \lr{ \Be_3 \cos\theta_t + \Be_1 \sin\theta_t }.
\end{aligned}
\end{equation}
%
So the time harmonic electric fields are
%
\begin{equation}\label{eqn:emtLecture10:640}
\begin{aligned}
   \BE_i &= \Be_2 E_i \exp\lr{ - j k_1 \lr{ z\cos\theta_i + x \sin\theta_i} }, \\
   \BE_r &= \Be_2 r E_i \exp\lr{ - j k_1 \lr{ -z \cos\theta_r + x \sin\theta_r}}, \\
   \BE_t &= \Be_2 t E_i \exp\lr{ - j k_2 \lr{ z \cos\theta_t + x \sin\theta_t}}.
\end{aligned}
\end{equation}
%
The magnetic fields follow from Faraday's law
%
\begin{dmath}\label{eqn:emtLecture10:900}
\BH
= \inv{-j \omega \mu } \spacegrad \cross \BE
= \inv{-j \omega \mu } \spacegrad \cross \Be_2 e^{-j \Bk \cdot \Bx}
= \inv{j \omega \mu } \Be_2 \cross \spacegrad e^{-j \Bk \cdot \Bx}
= -\inv{\omega \mu } \Be_2 \cross \Bk e^{-j \Bk \cdot \Bx}
= \inv{\omega \mu } \Bk \cross \BE.
\end{dmath}
%
We have
%
\begin{equation}\label{eqn:emtLecture10:920}
\begin{aligned}
\kcap_{i} \cross \Be_2 &= -\Be_1 \cos\theta_i + \Be_3\sin\theta_i  \\
\kcap_{r} \cross \Be_2 &= \Be_1 \cos\theta_r + \Be_3 \sin\theta_r  \\
\kcap_{t} \cross \Be_2 &= -\Be_1 \cos\theta_t + \Be_3 \sin\theta_t,
\end{aligned}
\end{equation}
%
Note that
\begin{dmath}\label{eqn:emtLecture10:1500}
\frac{k}{\omega \mu}
=
\frac{k}{k v \mu}
=
\frac{\sqrt{\mu\epsilon}}{\mu}
=\sqrt
{
\frac{\epsilon}{\mu}
}
=
\inv{\eta}.
\end{dmath}
%
so
\begin{equation}\label{eqn:emtLecture10:940}
\begin{aligned}
\BH_{i} &= \frac{ E_i}{\eta_1} \lr{ -\Be_1 \cos\theta_i + \Be_3\sin\theta_i } \exp\lr{ - j k_1 \lr{ z\cos\theta_i + x \sin\theta_i} } \\
\BH_{r} &= \frac{ r E_i}{\eta_1} \lr{ \Be_1 \cos\theta_r + \Be_3 \sin\theta_r } \exp\lr{ - j k_1 \lr{ -z \cos\theta_r + x \sin\theta_r}} \\
\BH_{t} &= \frac{ t E_i}{\eta_2} \lr{ -\Be_1 \cos\theta_t + \Be_3 \sin\theta_t } \exp\lr{ - j k_2 \lr{ z \cos\theta_t + x \sin\theta_t}}.
\end{aligned}
\end{equation}
%
The boundary conditions at \( z = 0 \) with \( \ncap = \Be_3 \) are
%
\begin{equation}\label{eqn:emtLecture10:960}
\begin{aligned}
\ncap \cross \BH_1 &= \ncap \cross \BH_2, \\
\ncap \cdot \BB_1 &= \ncap \cdot \BB_2, \\
\ncap \cross \BE_1 &= \ncap \cross \BE_2, \\
\ncap \cdot \BD_1 &= \ncap \cdot \BD_2.
\end{aligned}
\end{equation}
%
%which gives
%
%\begin{subequations}
%\label{eqn:emtLecture10:980}
%\begin{dmath}\label{eqn:emtLecture10:1000}
%-\frac{k_1 }{\mu_1} \cos\theta_i \exp\lr{ - j k_1 x \sin\theta_i }
%+
%\frac{k_1 r }{\mu_1} \cos\theta_r \exp\lr{ - j k_1 x \sin\theta_r }
%=
%-\frac{k_2 t }{\mu_2} \cos\theta_t \exp\lr{ - j k_2 x \sin\theta_t },
%\end{dmath}
%\begin{dmath}\label{eqn:emtLecture10:1020}
%k_1 \sin\theta_i \exp\lr{ - j k_1 x \sin\theta_i }
%+
%k_1 r \sin\theta_r \exp\lr{ + j k_1 x \sin\theta_r }
%=
%k_2 t \sin\theta_t \exp\lr{ - j k_2 x \sin\theta_t }
%\end{dmath}
%\begin{dmath}\label{eqn:emtLecture10:1040}
%\exp\lr{ - j k_1 \lr{ x \sin\theta_i} }
%+
%r \exp\lr{ - j k_1 \lr{ x \sin\theta_r}}
%=
%t \exp\lr{ - j k_2 \lr{ x \sin\theta_t}}.
%\end{dmath}
%\end{subequations}
%
At \( x = 0 \), this is
%
\begin{equation}\label{eqn:emtLecture10:1060}
\begin{aligned}
-\frac{1}{\eta_1} \cos\theta_i + \frac{r }{\eta_1} \cos\theta_r &= -\frac{t }{\eta_2} \cos\theta_t  \\
k_1 \sin\theta_i + k_1 r \sin\theta_r &= k_2 t \sin\theta_t  \\
1 + r &= t
\end{aligned}
\end{equation}
%
When \( t = 0 \) the latter two equations give Shell's first law
%
%\begin{dmath}\label{eqn:emtLecture10:1080}
\boxedEquation{eqn:emtLecture10:1080}{
\sin\theta_i = \sin\theta_r.
}
%\end{dmath}
%
Assuming this holds for all \( r, t \) we have
%
\begin{dmath}\label{eqn:emtLecture10:1120}
k_1 \sin\theta_i (1 + r ) = k_2 t \sin\theta_t,
\end{dmath}
%
which is Snell's second law in disguise
\begin{dmath}\label{eqn:emtLecture10:1140}
k_1 \sin\theta_i = k_2 \sin\theta_t.
\end{dmath}
%
With
\begin{dmath}\label{eqn:emtLecture10:1540}
k
= \frac{\omega}{v}
= \frac{\omega}{c} \frac{c}{v}
= \frac{\omega}{c} n,
\end{dmath}
%
so \cref{eqn:emtLecture10:1140} takes the form
%
%\begin{dmath}\label{eqn:emtLecture10:1560}
\boxedEquation{eqn:emtLecture10:1560}{
n_1 \sin\theta_i = n_2 \sin\theta_t.
}
%\end{dmath}
%
With
\begin{equation}\label{eqn:emtLecture10:1200}
\begin{aligned}
k_{1z} &= k_1 \cos\theta_i \\
k_{2z} &= k_2 \cos\theta_t,
\end{aligned}
\end{equation}
%
we can solve for \( r, t \) by inverting
%
\begin{dmath}\label{eqn:emtLecture10:1180}
\begin{bmatrix}
\mu_2 k_{1z} & \mu_1 k_{2z} \\
-1 & 1 \\
\end{bmatrix}
\begin{bmatrix}
r \\
t
\end{bmatrix}
=
\begin{bmatrix}
\mu_2 k_{1z} \\
1
\end{bmatrix},
\end{dmath}
%
which gives
%
\begin{dmath}\label{eqn:emtLecture10:1220}
\begin{bmatrix}
r \\
t
\end{bmatrix}
=
\begin{bmatrix}
1 & -\mu_1 k_{2z} \\
1 &  \mu_2 k_{1z}
\end{bmatrix}
\begin{bmatrix}
\mu_2 k_{1z} \\
1
\end{bmatrix},
\end{dmath}
%
or
%\begin{dmath}\label{eqn:emtLecture10:1240}
\boxedEquation{eqn:emtLecture10:1260}{
\begin{aligned}
r &= \frac{\mu_2 k_{1z} - \mu_1 k_{2z}}{\mu_2 k_{1z} + \mu_1 k_{2z}}, \\
t &= \frac{2 \mu_2 k_{1z}}{\mu_2 k_{1z} + \mu_1 k_{2z}}.
\end{aligned}
}
%\end{dmath}
%
There are many ways that this can be written.  Dividing both the numerator and denominator by \( \mu_1 \mu_2 \omega/c \), and noting that \( k = \omega n/c \), we have
%
\begin{equation}\label{eqn:emtLecture10:1680}
\begin{aligned}
r &= \frac
{ \frac{n_1}{\mu_1} \cos\theta_i - \frac{n_2}{\mu_2} \cos\theta_t }
{ \frac{n_1}{\mu_1} \cos\theta_i + \frac{n_2}{\mu_2} \cos\theta_t } \\
t &=
\frac{ 2 \frac{n_1}{\mu_1} \cos\theta_i }
{ \frac{n_1}{\mu_1} \cos\theta_i + \frac{n_2}{\mu_2} \cos\theta_t },
\end{aligned}
\end{equation}
%
which checks against (4.32,4.33) in \citep{hecht1998hecht}.
%
\section{Single interface TM mode.}
%
For completeness, now consider the TM mode.
%
Faraday's law also can provide the electric field from the magnetic
%
\begin{dmath}\label{eqn:emtLecture10:1280}
\kcap \cross \BH
= \eta \kcap \cross \lr{ \kcap \cross \BE }
= -\eta \kcap \cdot \lr{ \kcap \wedge \BE }
= -\eta \lr{ \BE - \kcap \lr{ \kcap \cdot \BE } }
= -\eta \BE.
\end{dmath}
%
so
%
\begin{dmath}\label{eqn:emtLecture10:1300}
\BE = \eta \BH \cross \kcap.
\end{dmath}
%
So the magnetic and electric fields are
%
\begin{subequations}
\label{eqn:emtLecture10:1520}
\begin{equation}\label{eqn:emtLecture10:1320}
\begin{aligned}
   \BH_i &= \Be_2 \frac{E_i}{\eta_1} \exp\lr{ - j k_1 \lr{ z\cos\theta_i + x \sin\theta_i} } \\
   \BH_r &= \Be_2 r \frac{E_i}{\eta_1} \exp\lr{ - j k_1 \lr{ -z \cos\theta_r + x \sin\theta_r}} \\
   \BH_t &= \Be_2 t \frac{E_i}{\eta_2} \exp\lr{ - j k_2 \lr{ z \cos\theta_t + x \sin\theta_t}}
\end{aligned}
\end{equation}
\begin{equation}\label{eqn:emtLecture10:1340}
\begin{aligned}
   \BE_{i} &= -E_i \lr{ -\Be_1 \cos\theta_i + \Be_3\sin\theta_i } \exp\lr{ - j k_1 \lr{ z\cos\theta_i + x \sin\theta_i} } \\
   \BE_{r} &= -r E_i \lr{ \Be_1 \cos\theta_r + \Be_3 \sin\theta_r } \exp\lr{ - j k_1 \lr{ -z \cos\theta_r + x \sin\theta_r}} \\
   \BE_{t} &= -t E_i \lr{ -\Be_1 \cos\theta_t + \Be_3 \sin\theta_t } \exp\lr{ - j k_2 \lr{ z \cos\theta_t + x \sin\theta_t}}.
\end{aligned}
\end{equation}
\end{subequations}
%
Imposing the constraints \cref{eqn:emtLecture10:960}, at \( x = z = 0 \) we have
%
\begin{equation}\label{eqn:emtLecture10:1440}
\begin{aligned}
\inv{\eta_1}\lr{1 + r} &= \frac{t}{\eta_2} \\
\cos\theta_i - r \cos\theta_r &= t \cos\theta_t  \\
\epsilon_1 \lr{ \sin\theta_i  + r \sin\theta_r}  &= t \epsilon_2 \sin\theta_t.
\end{aligned}
\end{equation}
%
At \( t = 0 \), the first and third of these give \( \theta_i = \theta_r \).  Assuming this incident and reflection angle equality holds for all values of \( t \), we have
%
\begin{equation}\label{eqn:emtLecture10:1580}
\begin{aligned}
\sin\theta_i(1  + r)  &= t \frac{\epsilon_2}{\epsilon_1} \sin\theta_t \\
\sin\theta_i \frac{\eta_1}{\eta_2} t &=
\end{aligned}
\end{equation}
%
or
\begin{dmath}\label{eqn:emtLecture10:1600}
\epsilon_1 \eta_1 \sin\theta_i = \epsilon_2 \eta_2 \sin\theta_t.
\end{dmath}
%
This is also Snell's second law \cref{eqn:emtLecture10:1560} in disguise, which can be seen by
%
\begin{dmath}\label{eqn:emtLecture10:1620}
\epsilon_1 \eta_1
=
\epsilon_1 \sqrt{\frac{\mu_1}{\epsilon_1}}
=
\sqrt{\epsilon_1 \mu_1}
=
\inv{v}
=
\frac{n}{c}.
\end{dmath}
%
The remaining equations in matrix form are
%
\begin{dmath}\label{eqn:emtLecture10:1460}
\begin{bmatrix}
\cos\theta_i & \cos\theta_t \\
-1 & \frac{\eta_1}{\eta_2}
\end{bmatrix}
\begin{bmatrix}
r \\
t
\end{bmatrix}
=
\begin{bmatrix}
\cos\theta_i \\
1
\end{bmatrix},
\end{dmath}
%
the inverse of which is
\begin{dmath}\label{eqn:emtLecture10:1480}
\begin{bmatrix}
r \\
t
\end{bmatrix}
=
\inv{ \frac{\eta_1}{\eta_2} \cos\theta_i + \cos\theta_t }
\begin{bmatrix}
\frac{\eta_1}{\eta_2} & - \cos\theta_t \\
1 & \cos\theta_i
\end{bmatrix}
\begin{bmatrix}
\cos\theta_i \\
1
\end{bmatrix}
=
\inv{ \frac{\eta_1}{\eta_2} \cos\theta_i + \cos\theta_t }
\begin{bmatrix}
\frac{\eta_1}{\eta_2} \cos\theta_i - \cos\theta_t \\
2 \cos\theta_i
\end{bmatrix},
\end{dmath}
%
or
%\begin{dmath}\label{eqn:emtLecture10:1640}
\boxedEquation{eqn:emtLecture10:1660}{
\begin{aligned}
r
&=
\frac{\eta_1 \cos\theta_i - \eta_2 \cos\theta_t }{ \eta_1 \cos\theta_i + \eta_2 \cos\theta_t } \\
t &=
\frac{2 \eta_2 \cos\theta_i}{ \eta_1 \cos\theta_i + \eta_2 \cos\theta_t }.
\end{aligned}
}
%\end{dmath}
%
Multiplication of the numerator and denominator by \( c/\eta_1 \eta_2 \), noting that \( c/\eta = n/\mu \) gives
%
\begin{equation}\label{eqn:emtLecture10:1700}
\begin{aligned}
r
&=
\frac{\frac{n_2}{\mu_2} \cos\theta_i - \frac{n_1}{\mu_1} \cos\theta_t }{ \frac{n_2}{\mu_2} \cos\theta_i + \frac{n_1}{\mu_1} \cos\theta_t }, \\
t &=
\frac{2 \frac{n_1}{\mu_1} \cos\theta_i }{ \frac{n_2}{\mu_2} \cos\theta_i + \frac{n_1}{\mu_1} \cos\theta_t },
\end{aligned}
\end{equation}
%
which checks against (4.38,4.39) in \citep{hecht1998hecht}.
%
%\EndArticle

      %
% Copyright � 2016 Peeter Joot.  All Rights Reserved.
% Licenced as described in the file LICENSE under the root directory of this GIT repository.
%
%{
%\input{../blogpost.tex}
%\renewcommand{\basename}{twoInterfaceNormal}
%%\renewcommand{\dirname}{notes/phy1520/}
%\renewcommand{\dirname}{notes/ece1228-electromagnetic-theory/}
%%\newcommand{\dateintitle}{}
%%\newcommand{\keywords}{}
%
%\input{../latex/peeter_prologue_print2.tex}
%
%\usepackage{peeters_layout_exercise}
%\usepackage{peeters_braket}
%\usepackage{peeters_figures}
%\usepackage{siunitx}
%\usepackage{enumerate}
%%\usepackage{mhchem} % \ce{}
%%\usepackage{macros_bm} % \bcM
%%\usepackage{txfonts} % \ointclockwise
%
%\beginArtNoToc
%
\section{Normal transmission and reflection through two interfaces.}
%\chapter{Normal transmission and reflection through two interfaces}
%\label{chap:twoInterfaceNormal}
%
%\paragraph{Motivation}
%
%In class an outline of normal transmission through a slab was presented.  Let's go through the details.
%
%\section{Two interfaces, normal incidence.}
%
The geometry of a two interface configuration is sketched in \cref{fig:l10TwoInterfaces:l10TwoInterfacesFig1}.
%
\imageFigure{../figures/ece1228-electromagnetic-theory/l10TwoInterfacesFig1}{Two interface transmission.}{fig:l10TwoInterfaces:l10TwoInterfacesFig1}{0.2}
%
Given a normal incident ray with magnitude \( A \), the respective forward and backwards rays in each the mediums can be written as
%
\begin{enumerate}[I]
\item
\begin{equation}\label{eqn:twoInterfaceNormal:20}
\begin{aligned}
\rightarrow &\qquad A e^{-j k_{1z} z} \\
\leftarrow &\qquad A r e^{j k_{1z} z} \\
\end{aligned}
\end{equation}
\item
\begin{equation}\label{eqn:twoInterfaceNormal:40}
\begin{aligned}
\rightarrow &\qquad C e^{-j k_{2z} z} \\
\leftarrow &\qquad D e^{j k_{2z} z} \\
\end{aligned}
\end{equation}
\item
\begin{equation}\label{eqn:twoInterfaceNormal:60}
\begin{aligned}
\rightarrow &\qquad A t e^{-j k_{3z} (z-d)}
\end{aligned}
\end{equation}
\end{enumerate}
%
Matching at \( z = 0 \) gives
\begin{equation}\label{eqn:twoInterfaceNormal:80}
\begin{aligned}
A t_{12} + r_{21} D &= C \\
A r      &= A r_{12} + D t_{21},
\end{aligned}
\end{equation}
%
whereas matching at \( z = d \) gives
%
\begin{equation}\label{eqn:twoInterfaceNormal:100}
\begin{aligned}
A t &= C e^{-j k_{2z} d} t_{23} \\
D e^{j k_{2z} d} &= C e^{-j k_{2z} d} r_{23}.
\end{aligned}
\end{equation}
%
We have four linear equations in four unknowns \( r, t, C, D \), but only care about solving for \( r, t \).  Let's write \(
\gamma = e^{ j k_{2z} d }, C' = C/A, D' = D/A \), for
%
\begin{equation}\label{eqn:twoInterfaceNormal:120}
\begin{aligned}
t_{12} + r_{21} D' &= C' \\
r      &= r_{12} + D' t_{21} \\
t \gamma &= C' t_{23} \\
D' \gamma^2 &= C' r_{23}.
\end{aligned}
\end{equation}
%
Solving for \( C', D' \) we get
%
\begin{equation}\label{eqn:twoInterfaceNormal:140}
\begin{aligned}
D' \lr{ \gamma^2 - r_{21} r_{23} } &= t_{12} r_{23} \\
C' \lr{ \gamma^2 - r_{21} r_{23} } &= t_{12} \gamma^2,
\end{aligned}
\end{equation}
%
so
%
\begin{equation}\label{eqn:twoInterfaceNormal:160}
\begin{aligned}
r &= r_{12} + \frac{t_{12} t_{21} r_{23} }{\gamma^2 - r_{21} r_{23} } \\
t &= t_{23} \frac{ t_{12} \gamma }{\gamma^2 - r_{21} r_{23} }.
\end{aligned}
\end{equation}
%
With \( \phi = -j k_{2z} d \), or \( \gamma = e^{-j\phi} \), we have
%
%\begin{dmath}\label{eqn:twoInterfaceNormal:180}
\boxedEquation{eqn:twoInterfaceNormal:180}{
\begin{aligned}
r &= r_{12} + \frac{t_{12} t_{21} r_{23} e^{2 j \phi} }{1 - r_{21} r_{23} e^{2 j \phi}} \\
t &= \frac{ t_{12} t_{23} e^{j\phi}}{1 - r_{21} r_{23} e^{2 j \phi}}.
\end{aligned}
}
%\end{dmath}
%
\paragraph{A slab}
%
When the materials in region I, and III are equal, then \( r_{12} = r_{32} \).  For a TE mode, we have
%
\begin{equation}\label{eqn:twoInterfaceNormal:200}
r_{12} =
\frac{\mu_2 k_{1z} - \mu_1 k_{2z}}{\mu_2 k_{1z} + \mu_1 k_{2z}}
= -r_{21}.
\end{equation}
%
so the reflection and transmission coefficients are
%
\begin{equation}\label{eqn:twoInterfaceNormal:220}
\begin{aligned}
r^{\textrm{TE}} &= r_{12} \lr{ 1 - \frac{t_{12} t_{21} e^{2 j \phi} }{1 - r_{21}^2 e^{2 j \phi}} } \\
t^{\textrm{TE}} &= \frac{ t_{12} t_{21} e^{j\phi}}{1 - r_{21}^2 e^{2 j \phi}}.
\end{aligned}
\end{equation}
%
It's possible to produce a matched condition for which \( r_{12} = r_{21} = 0 \), by selecting
%
\begin{dmath}\label{eqn:twoInterfaceNormal:240}
0
= \mu_2 k_{1z} - \mu_1 k_{2z}
= \mu_1 \mu_2 \lr{ \inv{\mu_1} k_{1z} - \inv{\mu_2} k_{2z} }
= \mu_1 \mu_2 \omega \lr{ \frac{1}{v_1 \mu_1} \theta_1 - \frac{1}{v_2 \mu_2} \theta_2 },
\end{dmath}
%
or
%
\begin{dmath}\label{eqn:twoInterfaceNormal:260}
\inv{\eta_1} \cos\theta_1 = \inv{\eta_2} \cos\theta_2,
\end{dmath}
%
so the matching condition for normal incidence is just
%
\begin{dmath}\label{eqn:twoInterfaceNormal:280}
\eta_1 = \eta_2.
\end{dmath}
%
Given this matched condition, the transmission coefficient for the 1,2 interface is
%
\begin{dmath}\label{eqn:twoInterfaceNormal:300}
t_{12}
= \frac{2 \mu_2 k_{1z}}{\mu_2 k_{1z} + \mu_1 k_{2z}}
= \frac{2 \mu_2 k_{1z}}{2 \mu_2 k_{1z} }
= 1,
\end{dmath}
%
so the matching condition yields
\begin{dmath}\label{eqn:twoInterfaceNormal:320}
t
=
t_{12} t_{21} e^{j\phi}
=
e^{j\phi}
=
e^{-j k_{2z} d}.
\end{dmath}

Normal transmission through a matched slab only introduces a phase delay.
%
%}
%\EndNoBibArticle

      %
% Copyright � 2016 Peeter Joot.  All Rights Reserved.
% Licenced as described in the file LICENSE under the root directory of this GIT repository.
%
%{
%\input{../blogpost.tex}
%\renewcommand{\basename}{brewsters}
%%\renewcommand{\dirname}{notes/phy1520/}
%\renewcommand{\dirname}{notes/ece1228-electromagnetic-theory/}
%%\newcommand{\dateintitle}{}
%%\newcommand{\keywords}{}
%
%\input{../latex/peeter_prologue_print2.tex}
%
%\usepackage{peeters_layout_exercise}
%\usepackage{peeters_braket}
%\usepackage{peeters_figures}
%\usepackage{siunitx}
%%\usepackage{mhchem} % \ce{}
%%\usepackage{macros_bm} % \bcM
%%\usepackage{macros_qed} % \qedmarker
%%\usepackage{txfonts} % \ointclockwise
%
%\beginArtNoToc
%
%\generatetitle{Total internal reflection and Brewster's angles}
%\chapter{Total internal reflection and Brewster's angles}
%%\label{chap:brewsters}
%% \citep{griffiths1999introduction}
%
\section{Total internal reflection.}
From Snell's second law we have
%
\begin{dmath}\label{eqn:brewsters:20}
\theta_t = \arcsin\lr{ \frac{n_i}{n_t} \sin\theta_i }.
\end{dmath}
%
This is plotted in \cref{fig:reflectionForBoth:reflectionForBothFig3}.
%
\imageFigure{../figures/ece1228-electromagnetic-theory/reflectionForBothFig3}{Transmission angle vs incident angle.}{fig:reflectionForBoth:reflectionForBothFig3}{0.3}
%
For the \( n_i > n_t \) case, for example, like shining from glass into air, there is a critical incident angle beyond which there is no real value of \( \theta_t \).  That critical incident angle occurs when \( \theta_t = \pi/2 \), which is
%
\begin{dmath}\label{eqn:brewsters:40}
\sin\theta_{ic} = \frac{n_t}{n_i} \sin(\pi/2).
\end{dmath}
%
With
\begin{dmath}\label{eqn:brewsters:340}
n = n_t/n_i,
\end{dmath}
%
the critical angle is
\begin{dmath}\label{eqn:brewsters:60}
\theta_{ic} = \arcsin(n).
\end{dmath}
%
Note that Snell's law can also be expressed in terms of this critical angle, allowing for the solution of the transmission angle in a convenient way
\begin{dmath}\label{eqn:brewsters:360}
\sin\theta_i
= \frac{n_t}{n_i} \sin\theta_t
= n \sin\theta_t
= \sin\theta_{ic} \sin\theta_t,
\end{dmath}
%
or
%
\begin{dmath}\label{eqn:brewsters:380}
\sin\theta_t = \frac{\sin\theta_i}{\sin\theta_{ic}}.
\end{dmath}
%
Still for \( n_i > n_t \), at angles past \( \theta_{ic} \), the transmitted wave angle becomes complex as outlined in
\citep{jackson1975cew}
, namely
%
\begin{dmath}\label{eqn:brewsters:400}
\cos^2\theta_t
=
1 - \sin^2 \theta_t
=
1 -
\frac{\sin^2\theta_i}{\sin^2\theta_{ic}}
=
-\lr{
\frac{\sin^2\theta_i}{\sin^2\theta_{ic}}
-1
},
\end{dmath}
%
or
\begin{dmath}\label{eqn:brewsters:420}
\cos\theta_t =
j \sqrt{
\frac{\sin^2\theta_i}{\sin^2\theta_{ic}}
-1
}.
\end{dmath}
%
Following the convention that puts the normal propagation direction along z, and the interface along x, the wave vector direction is
\begin{dmath}\label{eqn:brewsters:440}
\kcap_t
= \Be_3 e^{ \Be_{31} \theta_t }
= \Be_3 \cos\theta_t + \Be_1 \sin\theta_t.
\end{dmath}
%
The phase factor for the transmitted field is
%
\begin{dmath}\label{eqn:brewsters:460}
\exp\lr{ j \omega t \pm j \Bk_t \cdot \Bx }
=
\exp\lr{ j \omega t \pm j k \kcap_t \cdot \Bx }
=
\exp\lr{ j \omega t \pm j k \lr{ z \cos\theta_t + x \sin\theta_t } }
=
\exp\lr{
   j \omega t
   \pm j k \lr{ z j \sqrt{ \frac{\sin^2\theta_i}{\sin^2\theta_{ic}} -1 } + x \frac{\sin\theta_i}{\sin\theta_{ic}} }
}
=
\exp\lr{
   j \omega t \pm k
\lr{
    j x \frac{\sin\theta_i}{\sin\theta_{ic}}
   - z \sqrt{ \frac{\sin^2\theta_i}{\sin^2\theta_{ic}} -1 }
}
}.
\end{dmath}
%
The propagation is channelled along the x axis, but the propagation into the second medium decays exponentially (or unphysically grows exponentially), only getting into the surface a small amount.
%
What is the average power transmission into the medium?  We are interested in the time average of the normal component of the Poynting vector \( \BS \cdot \ncap \).
%
\begin{dmath}\label{eqn:brewsters:480}
\BS
= \inv{2} \BE \cross \BH^\conj
= \inv{2} \BE \cross \lr{ \inv{\eta} \kcap_t \cross \BE^\conj }
= -\inv{2 \eta} \BE \cdot \lr{ \kcap_t \wedge \BE^\conj }
= -\inv{2 \eta} \lr{
(\BE \cdot \kcap_t) \BE^\conj
-
\kcap_t \BE \cdot \BE^\conj
}
=
\inv{2 \eta}
\kcap_t \Abs{\BE}^2.
\end{dmath}
%
\begin{dmath}\label{eqn:brewsters:500}
\kcap_t \cdot \ncap
= \lr{ \Be_3 \cos\theta_t + \Be_1 \sin\theta_t } \cdot \Be_3
= \cos\theta_t
=
j \sqrt{
\frac{\sin^2\theta_i}{\sin^2\theta_{ic}}
-1
}.
\end{dmath}
%
Note that this is purely imaginary.  The time average real power transmission is
%
\begin{dmath}\label{eqn:brewsters:520}
\expectation{\BS \cdot \ncap}
=
\Real \lr{
j \sqrt{
\frac{\sin^2\theta_i}{\sin^2\theta_{ic}}
-1
}
\frac{1}{2 \eta} \Abs{\BE}^2
}
= 0.
\end{dmath}
%
There is no power transmission into the second medium at or past the critical angle for total internal reflection.
%
\section{Brewster's angle.}
%
Brewster's angle is the angle for which there the amplitude of the reflected component of the field is zero.  Recall that when the electric field is parallel(perpendicular) to the plane of incidence, the reflection amplitude (\citep{hecht1998hecht} eq. 4.38)
%
\begin{dmath}\label{eqn:brewsters:80}
r_\parallel
=
\frac
{
\frac{ n_t }{\mu_t} \cos \theta_i
-\frac{ n_i }{\mu_i} \cos \theta_t
}
{
\frac{ n_t }{\mu_t} \cos \theta_i
+\frac{ n_i }{\mu_i} \cos \theta_t
},
\end{dmath}
\begin{dmath}\label{eqn:brewsters:100}
r_\perp
=
\frac
{
\frac{ n_i }{\mu_i} \cos \theta_i
-\frac{ n_t }{\mu_t} \cos \theta_t
}
{
\frac{ n_i }{\mu_i} \cos \theta_i
+\frac{ n_t }{\mu_t} \cos \theta_t
}.
\end{dmath}
%
There are limited conditions for which \( r_\perp \) is zero, at least for \( \mu_i = \mu_t \).  Using Snell's second law \( n_i \sin\theta_i = n_t \sin\theta_t \), that zero is found at
%
\begin{dmath}\label{eqn:brewsters:120}
n_i \cos \theta_i
= n_t \cos \theta_t
= n_t \sqrt{ 1 - \sin^2 \theta_t }
= n_t \sqrt{ 1 - \frac{n_i^2}{n_t^2} \sin^2 \theta_i },
\end{dmath}
%
or
%
\begin{dmath}\label{eqn:brewsters:140}
\frac{n_i^2}{n_t^2} \cos^2 \theta_i = 1 - \frac{n_i^2}{n_t^2} \sin^2 \theta_i,
\end{dmath}
%
or
\begin{dmath}\label{eqn:brewsters:160}
\frac{n_i^2}{n_t^2} \lr{ \cos^2 \theta_i + \sin^2 \theta_i } = 1.
\end{dmath}
%
This has solutions only when \( n_i = \pm n_t \).  The \( n_i = n_t \) case is of no interest, since that is just propagation, so naturally there is no reflection.  The \( n_i = -n_t \) case is possible with the transmission into a negative index of refraction material that is matched in absolute magnitude with the index of refraction in the incident medium.
%
There are richer solutions for the \( r_\parallel \) zero.  Again considering \( \mu_1 = \mu_2 \) those occur when
%
\begin{dmath}\label{eqn:brewsters:180}
n_t \cos \theta_i
= n_i \cos \theta_t
= n_i \sqrt{ 1 - \frac{n_i^2}{n_t^2} \sin^2 \theta_i }
= n_i \sqrt{ 1 - \frac{n_i^2}{n_t^2} \sin^2 \theta_i }.
\end{dmath}
%
Let \( n = n_t/n_i \), and square both sides.  This gives
%
\begin{dmath}\label{eqn:brewsters:200}
n^2 \cos^2 \theta_i
= 1 - \inv{n^2} \sin^2 \theta_i
= 1 - \inv{n^2} (1 - \cos^2 \theta_i),
\end{dmath}
%
or
%
\begin{dmath}\label{eqn:brewsters:220}
\cos^2 \theta_i \lr{ n^2 + \inv{n^2}} = 1 - \inv{n^2},
\end{dmath}
%
or
\begin{dmath}\label{eqn:brewsters:240}
\cos^2 \theta_i
= \frac{1 - \inv{n^2}}{ n^2 - \inv{n^2} }
= \frac{n^2 - 1}{ n^4 - 1 }
= \frac{n^2 - 1}{ (n^2 - 1)(n^2 + 1) }
= \frac{1}{ n^2 + 1 }.
\end{dmath}
%
We also have
%
\begin{dmath}\label{eqn:brewsters:260}
\sin^2 \theta_i
=
1 - \frac{1}{ n^2 + 1 }
=
\frac{n^2}{ n^2 + 1 },
\end{dmath}
%
so
\begin{dmath}\label{eqn:brewsters:280}
\tan^2 \theta_i = n^2,
\end{dmath}
%
and
\begin{dmath}\label{eqn:brewsters:300}
\tan \theta_{iB} = \pm n,
\end{dmath}
%
For normal media where \( n_i > 0, n_t > 0 \), only the positive solution is physically relevant, which is
%
\boxedEquation{eqn:brewsters:320}{
\theta_{iB} = \arctan\lr{ \frac{n_t}{n_i} }.
}
%
%}
%\EndArticle

      \section{Problems.}
         %
% Copyright � 2016 Peeter Joot.  All Rights Reserved.
% Licenced as described in the file LICENSE under the root directory of this GIT repository.
%
%{
%\input{../blogpost.tex}
%\renewcommand{\basename}{fresnelSumAndDifferenceAngleFormulas}
%%\renewcommand{\dirname}{notes/phy1520/}
%\renewcommand{\dirname}{notes/ece1228-electromagnetic-theory/}
%%\newcommand{\dateintitle}{}
%%\newcommand{\keywords}{}
%
%\input{../latex/peeter_prologue_print2.tex}
%
%\usepackage{peeters_layout_exercise}
%\usepackage{peeters_braket}
%\usepackage{peeters_figures}
%\usepackage{siunitx}
%%\usepackage{mhchem} % \ce{}
%%\usepackage{macros_bm} % \bcM
%%\usepackage{txfonts} % \ointclockwise
%
%\beginArtNoToc
%
%\generatetitle{Fresnel angular sum and difference formulas}
%\chapter{Fresnel angular sum and difference formulas}
%\label{chap:fresnelSumAndDifferenceAngleFormulas}
%
\makeoproblem{Fresnel sum and difference formulas.}{problem:fresnelSumAndDifferenceAngleFormulas:1}{\citep{hecht1998hecht} pr. 4.39}{
%
Given a \( \mu_1 = \mu_2 \) constraint, show that the Fresnel equations have the form
%
\begin{subequations}
\label{eqn:fresnelSumAndDifferenceAngleFormulas:260}
\begin{dmath}\label{eqn:fresnelSumAndDifferenceAngleFormulas:280}
r^{\textrm{TE}}
=
\frac {
\sin( \theta_t - \theta_i )
} {
\sin( \theta_t + \theta_i )
}
\end{dmath}
\begin{dmath}\label{eqn:fresnelSumAndDifferenceAngleFormulas:300}
r^{\textrm{TM}}
=
\frac
{\tan(\theta_i -\theta_t)}
{\tan(\theta_i +\theta_t)}
\end{dmath}
\begin{dmath}\label{eqn:fresnelSumAndDifferenceAngleFormulas:320}
t^{\textrm{TE}}
= \frac{ 2  \sin\theta_t \cos\theta_i }
{ \sin(\theta_i + \theta_t) }
\end{dmath}
\begin{dmath}\label{eqn:fresnelSumAndDifferenceAngleFormulas:340}
t^{\textrm{TM}}
=
{ \sin(\theta_i + \theta_t) \cos(\theta_i - \theta_t) }.
\end{dmath}
\end{subequations}
} % problem
%
\makeanswer{problem:fresnelSumAndDifferenceAngleFormulas:1}{\withproblemsetsParagraph{
%
We need a couple trig identities to start with.
%
\begin{dmath}\label{eqn:fresnelSumAndDifferenceAngleFormulas:20}
\sin(a + b)
=
\Imag\lr{ e^{j(a + b)} }
=
\Imag\lr{
e^{ja} e^{+ jb}
}
=
\Imag\lr{
(\cos a + j \sin a) (\cos b + j \sin b)
}
=
\sin a \cos b + \cos a \sin b.
\end{dmath}
%
Allowing for both signs we have
%
\begin{equation}\label{eqn:fresnelSumAndDifferenceAngleFormulas:240}
\begin{aligned}
\sin(a + b) &= \sin a \cos b + \cos a \sin b \\
\sin(a - b) &= \sin a \cos b - \cos a \sin b.
\end{aligned}
\end{equation}
%
The mixed sine and cosine product can be expressed as a sum of sines
%
\begin{dmath}\label{eqn:fresnelSumAndDifferenceAngleFormulas:40}
2 \sin a \cos b = \sin(a + b) + \sin(a - b).
\end{dmath}
%
With \( 2 x = a + b, 2 y = a - b \), or \( a = x + y, b = x - y \), we find
%
\begin{equation}\label{eqn:fresnelSumAndDifferenceAngleFormulas:60}
\begin{aligned}
2 \sin(x + y) \cos (x - y) &= \sin( 2 x ) + \sin( 2 y ) \\
2 \sin(x - y) \cos (x + y) &= \sin( 2 x ) - \sin( 2 y ).
\end{aligned}
\end{equation}
%
Returning to the problem.  When \( \mu_1 = \mu_2 \) the Fresnel equations were found to be
%
\begin{equation}\label{eqn:fresnelSumAndDifferenceAngleFormulas:100}
\begin{aligned}
r^{\textrm{TE}} &= \frac { n_1 \cos\theta_i - n_2 \cos\theta_t } { n_1 \cos\theta_i + n_2 \cos\theta_t } \\
r^{\textrm{TM}} &= \frac{n_2 \cos\theta_i - n_1 \cos\theta_t }{ n_2 \cos\theta_i + n_1 \cos\theta_t } \\
t^{\textrm{TE}} &= \frac{ 2 n_1 \cos\theta_i } { n_1 \cos\theta_i + n_2 \cos\theta_t } \\
t^{\textrm{TM}} &= \frac{2 n_1 \cos\theta_i }{ n_2 \cos\theta_i + n_1 \cos\theta_t }.
\end{aligned}
\end{equation}
%
Using Snell's law, one of \( n_1, n_2 \) can be eliminated, for example
%
\begin{dmath}\label{eqn:fresnelSumAndDifferenceAngleFormulas:120}
n_1 = n_2 \frac{\sin \theta_t}{\sin\theta_i}.
\end{dmath}
%
Inserting this and proceeding with the application of the trig identities above, we have
%
\begin{subequations}
\label{eqn:fresnelSumAndDifferenceAngleFormulas:140}
\begin{dmath}\label{eqn:fresnelSumAndDifferenceAngleFormulas:160}
r^{\textrm{TE}}
= \frac { n_2 \frac{\sin\theta_t}{\sin\theta_i} \cos\theta_i - n_2 \cos\theta_t } { n_2 \frac{\sin\theta_t}{\sin\theta_i} \cos\theta_i + n_2 \cos\theta_t }
=
\frac {
\sin\theta_t \cos\theta_i - \cos\theta_t \sin\theta_i
} {
\sin\theta_t \cos\theta_i + \cos\theta_t \sin\theta_i
}
=
\frac {
\sin( \theta_t - \theta_i )
} {
\sin( \theta_t + \theta_i )
}
\end{dmath}
\begin{dmath}\label{eqn:fresnelSumAndDifferenceAngleFormulas:180}
r^{\textrm{TM}}
= \frac{n_2 \cos\theta_i - n_2 \frac{\sin\theta_t}{\sin\theta_i} \cos\theta_t }{ n_2 \cos\theta_i + n_2 \frac{\sin\theta_t}{\sin\theta_i} \cos\theta_t }
= \frac{
\sin\theta_i \cos\theta_i - \sin\theta_t \cos\theta_t
}{
\sin\theta_i \cos\theta_i + \sin\theta_t \cos\theta_t
}
= \frac{\inv{2} \sin(2 \theta_i) -  \inv{2} \sin(2 \theta_t) }{ \inv{2} \sin(2 \theta_i) +  \inv{2} \sin(2 \theta_t) }
= \frac
{\sin(\theta_i - \theta_t)\cos(\theta_i + \theta_t) }
{\sin(\theta_i + \theta_t)\cos(\theta_i - \theta_t) }
=
\frac
{\tan(\theta_i -\theta_t)}
{\tan(\theta_i +\theta_t)}
\end{dmath}
\begin{dmath}\label{eqn:fresnelSumAndDifferenceAngleFormulas:200}
t^{\textrm{TE}}
= \frac{ 2 n_2 \frac{\sin\theta_t}{\sin\theta_i} \cos\theta_i } { n_2 \frac{\sin\theta_t}{\sin\theta_i} \cos\theta_i + n_2 \cos\theta_t }
= \frac{ 2  \sin\theta_t \cos\theta_i } { \sin\theta_t \cos\theta_i + \cos\theta_t \sin\theta_i }
= \frac{ 2  \sin\theta_t \cos\theta_i }
{ \sin(\theta_i + \theta_t) }
\end{dmath}
\begin{dmath}\label{eqn:fresnelSumAndDifferenceAngleFormulas:220}
t^{\textrm{TM}}
= \frac{2 n_2 \frac{\sin\theta_t}{\sin\theta_i} \cos\theta_i }{ n_2 \cos\theta_i + n_2 \frac{\sin\theta_t}{\sin\theta_i} \cos\theta_t }
= \frac{2  \sin\theta_t \cos\theta_i }{ \sin\theta_i \cos\theta_i +  \sin\theta_t \cos\theta_t }
= \frac{2  \sin\theta_t \cos\theta_i }
{ \inv{2} \sin(2 \theta_i) +  \inv{2} \sin(2 \theta_t) }
= \frac{2 \sin\theta_t \cos\theta_i }
{ \sin(\theta_i + \theta_t) \cos(\theta_i - \theta_t) }
\end{dmath}
\end{subequations}
}} % answer
%
%}
%\EndArticle

         %
% Copyright � 2016 Peeter Joot.  All Rights Reserved.
% Licenced as described in the file LICENSE under the root directory of this GIT repository.
%
\makeproblem{Fresnel TM equations.}{emt:problemSet8:1}{
For the geometry shown in \cref{fig:ps8:ps8Fig1}, obtain the TM (E)
Fresnel reflection and transmission coefficients. Express your
results in terms of the propagation constant \( k_{1z} \) and \( k_{2z} \),
(i.e., the projection of
\( \Bk_1 \) and
\( \Bk_2 \)
along z-direction.) Note that the
interface is at \( z=0 \) plane.
\imageFigure{../figures/ece1228-electromagnetic-theory/ps8Fig1}{TM mode geometry.}{fig:ps8:ps8Fig1}{0.2}
} % makeproblem
\makeanswer{emt:problemSet8:1}{\withproblemsetsParagraph{
From the figure, with \( i = \Be_3 \Be_1 \) the propagation unit vectors are
\begin{subequations}
\label{eqn:emtproblemSet8Problem1:20}
\begin{dmath}\label{eqn:emtproblemSet8Problem1:40}
\kcap_1
= \Be_3 e^{i \theta_1}
= \Be_3 \cos\theta_1 + \Be_1 \sin\theta_1
\end{dmath}
\begin{dmath}\label{eqn:emtproblemSet8Problem1:60}
\kcap_1'
= -\Be_3 e^{-i \theta_1'}
= -\Be_3 \cos\theta_1' + \Be_1 \sin\theta_1'
\end{dmath}
\begin{dmath}\label{eqn:emtproblemSet8Problem1:80}
\kcap_2
= \Be_3 e^{i \theta_2}
= \Be_3 \cos\theta_2 + \Be_1 \sin\theta_2.
\end{dmath}
\end{subequations}
Recall that Faraday's law shows that \( \Bk, \BE, \BH \) is a right handed triple.  In particular
\begin{dmath}\label{eqn:emtproblemSet8Problem1:100}
-j \omega \mu \BH
=
\spacegrad \cross \BE
=
-\BE_0 \cross \spacegrad e^{j \omega t - j \Bk \cdot \Bx}
=
-\BE_0 \cross (-j \Bk) e^{j \omega t - j \Bk \cdot \Bx}
=
j \BE \cross \Bk,
\end{dmath}
or
\begin{dmath}\label{eqn:emtproblemSet8Problem1:120}
\BH
=
\inv{-j \omega \mu} j \BE \cross \Bk
=
\inv{\omega \mu} \Bk \cross \BE.
=
\inv{\eta} \kcap \cross \BE.
\end{dmath}
This means that \( \BH_i, \BH_t \) must be directed along the \( +\Be_2 \) direction, whereas \( \BH_r \) is directed in the \( -\Be_2 \) direction.  Note that the phase of this reflected magnetic field is opposite to what might be considered a natural choice, so we should that \( r \) is negative compared to a reference that picks the opposite phase convention.

The electric field directions from the figure are
\begin{subequations}
\label{eqn:emtproblemSet8Problem1:140}
\begin{dmath}\label{eqn:emtproblemSet8Problem1:160}
\Ecap_i
=
\kcap_1 i
= \Be_3 e^{i \theta_1} i
= \Be_1 e^{i \theta_1}
= \Be_1 \cos\theta_1 - \Be_3 \sin\theta_1
\end{dmath}
\begin{dmath}\label{eqn:emtproblemSet8Problem1:180}
\Ecap_r
=
\kcap_1' (-i)
= -\Be_3 e^{-i \theta_1'} (-i)
= \Be_1 e^{-i \theta_1'}
= \Be_1 \cos\theta_1' + \Be_3 \sin\theta_1'
\end{dmath}
\begin{dmath}\label{eqn:emtproblemSet8Problem1:200}
\Ecap_t
=
\kcap_2 i
= \Be_3 e^{i \theta_2} i
= \Be_1 e^{i \theta_2}
= \Be_1 \cos\theta_2 - \Be_3 \sin\theta_2.
\end{dmath}
\end{subequations}
The boundary value conditions, with \( \ncap = \Be_3 \), are
\begin{equation}\label{eqn:emtproblemSet8Problem1:220}
\begin{aligned}
\ncap \cross \lr{ \BH_1 - \BH_2 } &= 0 \\
\ncap \cdot \lr{ \BB_1 - \BB_2 } &= 0 \\
\ncap \cross \lr{ \BE_1 - \BE_2 } &= 0 \\
\ncap \cdot \lr{ \BD_1 - \BD_2 } &= 0,
\end{aligned}
\end{equation}
where
\begin{subequations}
\label{eqn:emtproblemSet8Problem1:240}
\begin{dmath}\label{eqn:emtproblemSet8Problem1:260}
\BE_1
=
E_0 \lr{ \Be_1 \cos\theta_1 - \Be_3 \sin\theta_1 } e^{-j \Bk_1 \cdot \Bx }
+
E_0 r \lr{ \Be_1 \cos\theta_1' + \Be_3 \sin\theta_1' } e^{-j \Bk_1' \cdot \Bx }
\end{dmath}
\begin{dmath}\label{eqn:emtproblemSet8Problem1:280}
\BE_2
=
E_0 t \lr{ \Be_1 \cos\theta_2 - \Be_3 \sin\theta_2 } e^{-j \Bk_2 \cdot \Bx }
\end{dmath}
\begin{dmath}\label{eqn:emtproblemSet8Problem1:300}
\BH_1
=
\Be_2 \frac{E_0}{\eta_1} e^{-j \Bk_1 \cdot \Bx }
-
\Be_2 \frac{E_0 r}{\eta_1} e^{-j \Bk_1' \cdot \Bx }
\end{dmath}
\begin{dmath}\label{eqn:emtproblemSet8Problem1:320}
\BH_2
=
\Be_2 \frac{E_0 t}{\eta_2} e^{-j \Bk_2 \cdot \Bx }.
\end{dmath}
\end{subequations}

The boundary value constraints can be seen to resolve to the following set of scalar equations
\begin{subequations}
\label{eqn:emtproblemSet8Problem1:340}
\begin{dmath}\label{eqn:emtproblemSet8Problem1:360}
\frac{E_0}{\eta_1} e^{-j \Bk_1 \cdot \Bx}
-\frac{E_0 r}{\eta_1} e^{-j \Bk_1' \cdot \Bx}
=
\frac{t E_0}{\eta_2} e^{-j \Bk_2 \cdot \Bx}
\end{dmath}
\begin{dmath}\label{eqn:emtproblemSet8Problem1:380}
E_0 \cos\theta_1 e^{-j \Bk_1 \cdot \Bx } + E_0 r \cos\theta_1' e^{-j \Bk_1' \cdot \Bx }
=
E_0 t \cos\theta_2 e^{-j \Bk_2 \cdot \Bx }
\end{dmath}
\begin{equation}\label{eqn:emtproblemSet8Problem1:400}
-\epsilon_1 &E_0 \sin\theta_1 e^{-j \Bk_1 \cdot \Bx }
+
\epsilon_1 E_0 r \sin\theta_1' e^{-j \Bk_1' \cdot \Bx } \\
&=
-\epsilon_2 E_0 t \sin\theta_2 e^{-j \Bk_2 \cdot \Bx },
\end{equation}
\end{subequations}
where equality is required at all points \( \Bx = x \Be_1 \) along the \( z = 0 \) axis.
In order for the phase factors to cancel out, as they do at the origin, we require
\begin{equation}\label{eqn:emtproblemSet8Problem1:420}
\Bk_1 \cdot \Be_1 = \Bk_1' \cdot \Be_1 = \Bk_2 \cdot \Be_2.
\end{equation}
The \( \Bk_1, \Bk_1' \) equality is Snell's first law, a requirement that the incident angle equals the reflection angle
\begin{dmath}\label{eqn:emtproblemSet8Problem1:440}
k_1 \sin\theta_1 = k_1 \sin\theta_1'.
\end{dmath}

The remaining phase equality is Snell's second law in disguise
\begin{dmath}\label{eqn:emtproblemSet8Problem1:460}
0
= \Bk_1 \cdot \Be_1 - \Bk_2 \cdot \Be_2
= k_1 \sin\theta_1 - k_2 \sin\theta_2
= \frac{\omega}{v_1} \sin\theta_1 - \frac{\omega}{v_2} \sin\theta_2
= \frac{\omega}{c}\frac{c}{v_1} \sin\theta_1 - \frac{\omega}{c}\frac{c}{v_2} \sin\theta_2
= \frac{\omega}{c} \lr{ n_1 \sin\theta_1 - n_2 \sin\theta_2 },
\end{dmath}
or
\begin{dmath}\label{eqn:emtproblemSet8Problem1:480}
n_1 \sin\theta_1 = n_2 \sin\theta_2.
\end{dmath}

With equality of all the phase terms, we are left with
\begin{subequations}
\label{eqn:emtproblemSet8Problem1:500}
\begin{dmath}\label{eqn:emtproblemSet8Problem1:520}
\frac{1}{\eta_1}
-\frac{r}{\eta_1}
=
\frac{t}{\eta_2}
\end{dmath}
\begin{dmath}\label{eqn:emtproblemSet8Problem1:540}
\cos\theta_1(1 + r)
=
t \cos\theta_2
\end{dmath}
\begin{dmath}\label{eqn:emtproblemSet8Problem1:560}
-\epsilon_1 \sin\theta_1 (1 - r)
=
-\epsilon_2 t \sin\theta_2.
\end{dmath}
\end{subequations}

Since \( \epsilon \eta = \sqrt{\epsilon\mu} = n/c \), one of these is redundant since the first and last just re-express Snell's law.  That leaves two equations in two unknowns (\(r,t\))
\begin{dmath}\label{eqn:emtproblemSet8Problem1:580}
\begin{bmatrix}
r \\
t
\end{bmatrix}
=
{
\begin{bmatrix}
1 & \eta_1/\eta_2 \\
-\cos\theta_1 & \cos\theta_2
\end{bmatrix}
}^{-1}
\begin{bmatrix}
1 \\
\cos\theta_1
\end{bmatrix}
=
\inv{ \eta_1 \cos\theta_1 + \eta_2 \cos\theta_2 }
\begin{bmatrix}
\eta_2 \cos\theta_2 & -\eta_1 \\
\eta_2 \cos\theta_1 & \eta_2
\end{bmatrix}
\begin{bmatrix}
1 \\
\cos\theta_1
\end{bmatrix},
\end{dmath}
or
\begin{equation}\label{eqn:emtproblemSet8Problem1:600}
\begin{aligned}
r &=
\frac{ \eta_2 \cos\theta_2 -\eta_1 \cos\theta_1 }
{ \eta_1 \cos\theta_1 + \eta_2 \cos\theta_2 } \\
t &=
\frac{ 2 \eta_2 \cos\theta_1 }
{ \eta_1 \cos\theta_1 + \eta_2 \cos\theta_2 }.
\end{aligned}
\end{equation}
As expected, this reflection coefficient has a different sign, than \citep{hecht1998hecht} (4.38), where the magnetic fields were all aligned along \( +\Be_2 \).

To express this in terms of \( k_{1z}, k_{2z} \) we have to rewrite expressions of the form
\begin{dmath}\label{eqn:emtproblemSet8Problem1:620}
\eta_1 \cos\theta_1
=
\frac{\eta_1}{k_1} k_{1z}
=
\frac{\eta_1 v_1}{\omega} k_{1z}
=
\frac{k_{1z}}{\omega} \sqrt{\frac{\mu_1}{\epsilon_1}} \inv{\sqrt{\epsilon_1 \mu_1}}
=
\frac{k_{1z}}{\epsilon_1 \omega}.
\end{dmath}
This gives
\begin{equation}\label{eqn:emtproblemSet8Problem1:640}
\begin{aligned}
r &=
\frac{  \frac{k_{2z}}{\epsilon_2} - \frac{k_{1z}}{\epsilon_1} }
{  \frac{k_{1z}}{\epsilon_1} +  \frac{k_{2z}}{\epsilon_2} } \\
t &=
\frac{ 2 \frac{\eta_2}{\eta_1}  \frac{k_{1z}}{\epsilon_1} }
{  \frac{k_{1z}}{\epsilon_1} +  \frac{k_{2z}}{\epsilon_2} },
\end{aligned}
\end{equation}
or
\boxedEquation{eqn:emtproblemSet8Problem1:660}{
\begin{aligned}
r &=
\frac{  \epsilon_1 k_{2z} - \epsilon_2 k_{1z} }
{  \epsilon_2 k_{1z} +  \epsilon_1 k_{2z} } \\
t &=
\frac{ 2 \frac{\eta_2}{\eta_1}  \epsilon_2 k_{1z} }
{  \epsilon_2 k_{1z} +  \epsilon_1 k_{2z} }.
\end{aligned}
}
}}

         %
% Copyright � 2016 Peeter Joot.  All Rights Reserved.
% Licenced as described in the file LICENSE under the root directory of this GIT repository.
%
\makeproblem{Two interfaces.}{emt:problemSet8:2}{
\makesubproblem{}{emt:problemSet8:2a}
Give the TE transmission function \( T^{\textrm{TE}}(\omega) \) for a slab of
length \( d \) with permittivity and permeability \( \epsilon_2, \mu_2 \), surrounded
by medium characterized by \( \epsilon_1 \)
and \( \mu_1 \)
as shown in \cref{fig:ps8:ps8Fig2}.
Make sure you provide the expressions for the terms appearing in the transmission function \( T^{\textrm{TE}}(\omega) \).
\makesubproblem{}{emt:problemSet8:2b}
Suppose medium (II) is a meta-material with \( \epsilon_2 = -\epsilon_1 \)
and \( \mu_2 = -\mu_1 \), where \( \epsilon_1 > 0 \) and \( \mu_1 > 0 \).
What is the transmission function \( T^{\textrm{TE}}(\omega) \) in this case. Express your results in terms
of the propagation constant in medium (I), i.e. \( k_{1z} \).
\makesubproblem{}{emt:problemSet8:2c}
Now consider a source located at \( z=0 \) generating
a uniform plane wave, and for simplicity suppose a one-dimensional propagation.
What is the field at the second interface \(z=2d\). What is the meaning of your results?
\imageFigure{../figures/ece1228-electromagnetic-theory/ps8Fig2}{Slab geometry.}{fig:ps8:ps8Fig2}{0.2}
} % makeproblem
\makeanswer{emt:problemSet8:2}{\withproblemsetsParagraph{
\makeSubAnswer{}{emt:problemSet8:2a}
As outlined in class,
%Assuming normal incidence,
we can solve the system assuming waves of the form sketched in \cref{fig:twoInterfaces:twoInterfacesFig2}.
\imageFigure{../figures/ece1228-electromagnetic-theory/twoInterfacesFig2}{Wave functions crossing two interfaces.}{fig:twoInterfaces:twoInterfacesFig2}{0.3}
%\cref{fig:threeMediaWave:threeMediaWaveFig1}.
%\imageFigure{../figures/ece1228-electromagnetic-theory/threeMediaWaveFig1}{Waves crossing two interfaces.}{fig:threeMediaWave:threeMediaWaveFig1}{0.25}

I've utilized the freedom to include arbitrary constant phase factors in the input and output wave functions.  The boundary conditions at \( z = d \) are
\begin{equation}\label{eqn:emtproblemSet8Problem2:20}
\begin{aligned}
A t_{12} + D r_{21} &= C \\
A r &= A r_{12} + D t_{21},
\end{aligned}
\end{equation}
and the boundary conditions at \( z = 2d \) are
\begin{equation}\label{eqn:emtproblemSet8Problem2:40}
\begin{aligned}
C t_{23} e^{-j k_{2z} d } &= A t \\
C e^{-j k_{2z} d} r_{23} &= D e^{+j k_{2z} d}.
\end{aligned}
\end{equation}
Setting \( A = 1 \) for convenience, we have two equations relating \( C, D \)
\begin{equation}\label{eqn:emtproblemSet8Problem2:60}
\begin{aligned}
t_{12} + D r_{21} &= C \\
C e^{-j k_{2z} d} r_{23} &= D e^{+j k_{2z} d},
\end{aligned}
\end{equation}
where the reflection and transmitted amplitude scaling factors \( r, t \) both follow immediately once we have solved for \( C, D \)
\begin{equation}\label{eqn:emtproblemSet8Problem2:80}
\begin{aligned}
t &= C t_{23} e^{-j k_{2z} d } \\
r &= r_{12} + D t_{21}.
\end{aligned}
\end{equation}

Writing \( \gamma = e^{ j k_{2z} d } \) we can solve for \( C, D \) immediately
\begin{dmath}\label{eqn:emtproblemSet8Problem2:100}
\begin{bmatrix}
C \\
D
\end{bmatrix}
=
{\begin{bmatrix}
1 & - r_{21}  \\
r_{23} & - \gamma^2
\end{bmatrix}}^{-1}
\begin{bmatrix}
t_{12} \\
0
\end{bmatrix}
=
\inv{ \gamma^2 - r_{21} r_{23} }
\begin{bmatrix}
\gamma^2 & -r_{21}  \\
r_{23} & -1
\end{bmatrix}
\begin{bmatrix}
t_{12} \\
0
\end{bmatrix}
=
\frac{t_{12}}{ \gamma^2 - r_{21} r_{23} }
\begin{bmatrix}
\gamma^2  \\
r_{23}
\end{bmatrix}.
\end{dmath}

For the reflection amplitude we have
\begin{dmath}\label{eqn:emtproblemSet8Problem2:120}
r
= r_{12} + D t_{21}
= r_{12} +
\frac{t_{21} t_{12} r_{23}}{ \gamma^2 - r_{21} r_{23} }
= r_{12} +
\frac{t_{21} t_{12} r_{23} e^{-2 j k_{2z} d}}{ 1 - r_{21} r_{23} e^{-2 j k_{2z} d}},
\end{dmath}
and for the transmission amplitude, we have
\begin{dmath}\label{eqn:emtproblemSet8Problem2:140}
t
= C \frac{t_{23}}{\gamma}
=
\frac{t_{23} t_{12}\gamma^2}{\gamma( \gamma^2 - r_{21} r_{23}) }
=
\frac{t_{23} t_{12}\gamma}{\gamma^2 - r_{21} r_{23} }
=
\frac{t_{23} t_{12}e^{-j k_{2z} d}}{1 - r_{21} r_{23} e^{-2 j k_{2z} d} }.
\end{dmath}
Noting that for normal incident TE mode, the individual amplitudes are
\begin{equation}\label{eqn:emtproblemSet8Problem2:160}
\begin{aligned}
r_{12} &= \frac{\mu_2 k_{1z} - \mu_1 k_{2z}}{\mu_2 k_{1z} + \mu_1 k_{2z}} = - r_{21} \\
t_{12} &= \frac{2 \mu_2 k_{1z}}{\mu_2 k_{1z} + \mu_1 k_{2z}} \\
t_{21} &= \frac{2 \mu_1 k_{2z}}{\mu_1 k_{2z} + \mu_2 k_{1z}},
\end{aligned}
\end{equation}
and setting \( 3 = 1 \), the transmission function for this problem is
\begin{dmath}\label{eqn:emtproblemSet8Problem2:180}
T(\omega)
=
\frac{t_{21} t_{12}e^{-j k_{2z} d}}{1 - r_{21} r_{21} e^{-2 j k_{2z} d} },
\end{dmath}
we have
\begin{equation}\label{eqn:emtproblemSet8Problem2:200}
\begin{aligned}
r_{21}^2 &=
\frac{(\mu_2 k_{1z} - \mu_1 k_{2z})^2}{(\mu_2 k_{1z} + \mu_1 k_{2z})^2} \\
t_{12} t_{21} &=
4 \frac{(\mu_2 k_{1z})(\mu_1 k_{2z})}{(\mu_2 k_{1z} + \mu_1 k_{2z})^2},
\end{aligned}
\end{equation}
so the transmission function is
\begin{dmath}\label{eqn:emtproblemSet8Problem2:220}
T(\omega)
=
\frac{4 \mu_1 \mu_2 k_{1z} k_{2z} e^{-j k_{2z} d}}{(\mu_2 k_{1z} + \mu_1 k_{2z})^2 - (\mu_2 k_{1z} - \mu_1 k_{2z})^2 e^{-2 j k_{2z} d} }.
\end{dmath}

The angular frequency dependence is carried by \( k_{iz} = (\omega/c) n_i(\omega) \cos\theta_i \).  In particular, when \( n_i \) does not depend on frequency, this means that all the frequency dependence is carried by the
%  In the one interface problem, that completely kills any explicit frequency dependence, but we have such dependence here,
phase factor \( e^{-j k_{2z} d } \) terms.
%  That is
%
%%\begin{dmath}\label{eqn:emtproblemSet8Problem2:240}
%\boxedEquation{eqn:emtproblemSet8Problem2:240}{
%T(\omega)
%=
%\frac{4 \mu_1 \mu_2 n_1 \cos\theta_1 n_2 \cos\theta_2 e^{-j \omega n_2 \cos\theta_2 d/c}}{(\mu_2 n_1 \cos\theta_1 + \mu_1 n_2\cos\theta_2 )^2 - (\mu_2 n_1 \cos\theta_1 - \mu_1 n_2 \cos\theta_2)^2 e^{-2 j \omega n_2 \cos\theta_2 d/c} }.
%}
%%\end{dmath}
%
%Here \( \theta_2 \) is related to \( \theta_1 \) are related by Snell's second law
%
%\begin{dmath}\label{eqn:emtproblemSet8Problem2:360}
%\theta_2 = \sin^{-1} \lr{ \frac{n_1}{n_2} \sin\theta_1 }.
%\end{dmath}
%
\makeSubAnswer{}{emt:problemSet8:2b}
The index of refraction, in terms of \( \mu, \epsilon \) is
\begin{dmath}\label{eqn:emtproblemSet8Problem2:260}
n = \frac{c}{v} = c \sqrt{ \mu \epsilon },
\end{dmath}
so
\begin{dmath}\label{eqn:emtproblemSet8Problem2:280}
n_2
= c \sqrt{ \mu_2 \epsilon_2 }
= c \sqrt{ (-\mu_1)(-\epsilon_1) }
= n_1.
\end{dmath}

The slab wave vector \( k_{2z} \) can be expressed in terms of \( k_{1z} \)
\begin{dmath}\label{eqn:emtproblemSet8Problem2:380}
k_{2z}
= \frac{\omega}{c} n_2 \cos\theta_2
= \frac{\omega}{c} n_1 \cos\theta_2
= k_{1z} \frac{\cos\theta_2}{\cos\theta_1}.
\end{dmath}

Let
\begin{dmath}\label{eqn:emtproblemSet8Problem2:400}
\alpha
=
\frac{\cos\theta_2}{\cos\theta_1}
=
\frac{\cos\lr{ \sin^{-1} \lr{ \frac{n_1}{n_2} \sin\theta_1 } }}
{\cos\theta_1},
\end{dmath}
The transmission function is
\begin{dmath}\label{eqn:emtproblemSet8Problem2:300}
T(\omega)
=
\frac{4 \mu_1 (-\mu_1) k_{1z}^2 \alpha e^{-j k_{1z} \alpha d}}{((-\mu_1) k_{1z} + \mu_1 k_{1z} \alpha)^2 - ((-\mu_1) k_{1z} - \mu_1 k_{1z} \alpha)^2 e^{-2 j k_{1z} \alpha d} }
=
\frac{-4 \alpha e^{-j k_{1z} \alpha d}}{(-1 + \alpha)^2 - (1 + \alpha)^2 e^{-2 j k_{1z} \alpha d} }.
%\frac{4 \mu_1 \mu_2 e^{-j \omega n_1 d/c}}{(\mu_2 + \mu_1 )^2 - (\mu_2 - \mu_1 )^2 e^{-2 j \omega n_1 d/c} }
%=
%\frac{-4 \mu_1^2 e^{-j \omega n_1 d/c}}{(-\mu_1 + \mu_1 )^2 - (-\mu_1 - \mu_1 )^2 e^{-2 j \omega n_1 d/c} }
%=
%\frac{e^{-j \omega n_1 d/c}}{e^{-2 j \omega n_1 d/c} },
\end{dmath}
\makeSubAnswer{}{emt:problemSet8:2c}
From Snell's second law, observe that a normal incident source remains normal in the slab
\begin{equation}\label{eqn:emtproblemSet8Problem2:460}
n_1 \sin 0       = n_2 \sin\theta_2,
\end{equation}
so \( \theta_2 = 0 \).  For small angles, say \( \theta_1 = \epsilon \), we have
\begin{dmath}\label{eqn:emtproblemSet8Problem2:480}
\theta_2
= \sin^{-1} \lr{ \frac{ n_1 \epsilon }{n_2} }
\approx \frac{ n_1 }{n_2} \epsilon,
\end{dmath}
so 1D propagation can be realized provided \( (\ifrac{ n_1 }{n_2}) \theta_1 \ll 1 \).
%A 1D propagation is not
%physically realizable, since normal incident light will no longer have normal direction within the slab due to diffraction.
%However, one can suppose that there are situations where \( \theta_1 \approx \theta_2 \) with sufficient closeness that \( \cos\theta_2/\cos\theta_1 \approx 1 \).
Given such a normal source,
the transmission function reduces to just a phase shift (a time shift in the time domain)
\begin{dmath}\label{eqn:emtproblemSet8Problem2:420}
T(\omega)
=
\frac{-4 e^{-j k_{1z} d}}{(-2)^2 e^{-2 j k_{1z} d} }
=
e^{j k_{1z} d}.
\end{dmath}
The field at the second interface, assuming the input wave amplitude \( A \) is real, is
\begin{dmath}\label{eqn:emtproblemSet8Problem2:440}
\psi(2d, t)
= \Real \lr{ A e^{j k_{1z} d} e^{j \omega t} }
= A \cos\lr{ \omega t + k_{1z} d }
= A \cos\lr{ \omega \lr{ t + \frac{k_{1z}}{\omega} d } }.
\end{dmath}

Instead of a phase delay, the specified meta-material results in an output wave that has a phase lead with respect to the source.
}}

         %
% Copyright � 2016 Peeter Joot.  All Rights Reserved.
% Licenced as described in the file LICENSE under the root directory of this GIT repository.
%
\makeproblem{One dimensional photonic crystal.}{emt:problemSet8:3}{
Consider an infinitely periodic one dimensional photonic crystal (1DPC) shown in \cref{fig:ps8:ps8Fig3} below
where \( n_i \)
and
\( n_j \)
are the indices of refractions (in general complex) associated with the regions \( i \)
and \( j \) having thicknesses \( d_i \) and \( d_j \). The one period transfer matrix \( \BM \) relates the fields according to
\begin{subequations}
\label{eqn:emtproblemSet8Problem3:120}
\begin{dmath}\label{eqn:emtproblemSet8Problem3:20}
\begin{bmatrix}
E_{l,i}' \\
E_{r,i}'
\end{bmatrix}
=
\BM
\begin{bmatrix}
E_{l,i+1}' \\
E_{r,i+1}'
\end{bmatrix}
\end{dmath}
\begin{dmath}\label{eqn:emtproblemSet8Problem3:40}
\BM =
g
\begin{bmatrix}
a & b \\
\hat{b} & \hat{a}
\end{bmatrix}
\end{dmath}
\begin{dmath}\label{eqn:emtproblemSet8Problem3:60}
g = \inv{1 - \rho_{i,j}^2},
\end{dmath}
\end{subequations}
and \( \rho_{i,j} \) is the Fresnel coefficient.  Give the expressions for \( a, \hat{a}, b, \hat{b} \) in terms of \( \beta_i \), \(\beta_j\), and \( \rho_{i,j} \) where
the phase constants in regions \( i \) and \( j \) are
%
\begin{subequations}
\label{eqn:emtproblemSet8Problem3:140}
\begin{dmath}\label{eqn:emtproblemSet8Problem3:80}
\beta_i = \frac{\omega}{c} n_i d_i \cos\theta_i
\end{dmath}
\begin{dmath}\label{eqn:emtproblemSet8Problem3:100}
\beta_j = \frac{\omega}{c} n_j d_j \cos\theta_j,
\end{dmath}
\end{subequations}
%
and \( \theta_i \) or \( \theta_j \) are the incident angles.
%
%\makesubproblem{}{emt:problemSet8:3a}
\imageFigure{../figures/ece1228-electromagnetic-theory/ps8Fig3}{1DPC photonic crystal.}{fig:ps8:ps8Fig3}{0.3}
} % makeproblem
%
\makeanswer{emt:problemSet8:3}{\withproblemsetsParagraph{
%
The fields inside and outside the slab are sketched in \cref{fig:1DPC:1DPCFig1}.  The wave amplitudes that we are interested in relating are
%
\begin{equation}\label{eqn:emtproblemSet8Problem3:160}
\begin{aligned}
E'_{l,i} &= A \\
E'_{r,i} &= B \\
E'_{l,i+1} &= E \gamma' \\
E'_{r,i+1} &= F /\gamma',
\end{aligned}
\end{equation}
%
\imageFigure{../figures/ece1228-electromagnetic-theory/1DPCFig1}{Fields inside and outside of a slab.}{fig:1DPC:1DPCFig1}{0.4}
where
\begin{dmath}\label{eqn:emtproblemSet8Problem3:180}
\gamma' = e^{ j \beta_{i+1} d_{i+1} }.
\end{dmath}
%
Relating fields at \( z = 0 \), with \( \rho_{ij} \) and \( \tau_{ij} \) for the Fresnel reflection and transmission coefficients, we have
%
\begin{equation}\label{eqn:emtproblemSet8Problem3:200}
\begin{aligned}
A &= C \tau_{ji} + B \rho_{ij} \\
D &= B \tau_{ij} + C \rho_{ji}.
\end{aligned}
\end{equation}
%
To express the boundary conditions for the fields at \( z = d_j \), let
%
\begin{dmath}\label{eqn:emtproblemSet8Problem3:220}
\gamma = e^{ j \beta_j d_j } .
\end{dmath}
%
Those boundary conditions are
\begin{equation}\label{eqn:emtproblemSet8Problem3:240}
\begin{aligned}
C \gamma &= E \tau_{i+1, j} + \frac{D}{\gamma} \rho_{j,i+1} \\
F &= E \rho_{i+1,j} + \frac{D}{\gamma} \tau_{j,i+1}.
\end{aligned}
\end{equation}
%
The periodicity constraint is
%
\begin{equation}\label{eqn:emtproblemSet8Problem3:260}
\begin{aligned}
d_{i+1} &= d_i \\
d_{j+1} &= d_j \\
\beta_{i+1} &= \beta_i \\
\beta_{j+1} &= \beta_j,
\end{aligned}
\end{equation}
%
so we have
%
\begin{equation}\label{eqn:emtproblemSet8Problem3:280}
\begin{aligned}
\begin{bmatrix}
1 & - \rho_{ij} \\
0 &   \tau_{ij}
\end{bmatrix}
\begin{bmatrix}
A \\
B
\end{bmatrix}
&=
\begin{bmatrix}
  \tau_{ji} & 0 \\
- \rho_{ji} & 1
\end{bmatrix}
\begin{bmatrix}
C \\
D
\end{bmatrix} \\
\begin{bmatrix}
  \gamma^2 &- \rho_{j i} \\
  0        &  \tau_{j i}
\end{bmatrix}
\begin{bmatrix}
C \\
D
\end{bmatrix}
&=
\gamma
\begin{bmatrix}
     \tau_{i j} & 0 \\
 -   \rho_{i j} & 1
\end{bmatrix}
\begin{bmatrix}
E \\
F
\end{bmatrix}.
\end{aligned}
\end{equation}
%
This gives
\begin{dmath}\label{eqn:emtproblemSet8Problem3:300}
\begin{bmatrix}
A \\
B
\end{bmatrix}
=
\gamma
{\begin{bmatrix}
1 & - \rho_{ij} \\
0 &   \tau_{ij}
\end{bmatrix}}^{-1}
\begin{bmatrix}
  \tau_{ji} & 0 \\
- \rho_{ji} & 1
\end{bmatrix}
{\begin{bmatrix}
  \gamma^2 &- \rho_{j i} \\
  0        &  \tau_{j i}
\end{bmatrix}}^{-1}
\begin{bmatrix}
     \tau_{i j} & 0 \\
 -   \rho_{i j} & 1
\end{bmatrix}
\begin{bmatrix}
E \\
F
\end{bmatrix}
=
\frac{1}{\gamma \tau_{ij} \tau_{ji}}
\begin{bmatrix}
\tau_{ij} & \rho_{ij} \\
0 &   1
\end{bmatrix}
\begin{bmatrix}
  \tau_{ji} & 0 \\
- \rho_{ji} & 1
\end{bmatrix}
\begin{bmatrix}
    \tau_{ji} &  \rho_{j i} \\
  0        &  \gamma^2
\end{bmatrix}
\begin{bmatrix}
     \tau_{i j} & 0 \\
 -   \rho_{i j} & 1
\end{bmatrix}
\begin{bmatrix}
E \\
F
\end{bmatrix}
=
\frac{1}{\gamma \tau_{ij} \tau_{ji}}
\begin{bmatrix}
\tau_{ij}\tau_{ji} - \rho_{ij} \rho_{ji} &  \rho_{ij} \\
-\rho_{ji}                      &  1
\end{bmatrix}
\begin{bmatrix}
\tau_{ji} \tau_{i j} - \rho_{ij} \rho_{ji} & \rho_{j i} \\
-\gamma^2 \rho_{ij}               & \gamma^2
\end{bmatrix}
\begin{bmatrix}
E \\
F
\end{bmatrix}
=
\frac{1}{\gamma \tau_{ij} \tau_{ji}}
\begin{bmatrix}
\tau_{ij}\tau_{ji} + \rho_{ij}^2 &  \rho_{ij} \\
\rho_{ij}                  &  1
\end{bmatrix}
\begin{bmatrix}
\tau_{ji} \tau_{i j} + \rho_{ij}^2 & -\rho_{i j} \\
-\gamma^2 \rho_{ij}          & \gamma^2
\end{bmatrix}
\begin{bmatrix}
E \\
F
\end{bmatrix}.
\end{dmath}
%
A digression is now required, since the problem asks for this matrix in terms of the reflection coefficients, and we have both the reflection coefficients and transmission coefficients in the mix.  For TE we have
%
\begin{equation}\label{eqn:emtproblemSet8Problem3:320}
\begin{aligned}
\rho_{12} &=
\frac
{\mu_2 \beta_1 - \mu_1 \beta_2}
{\mu_2 \beta_1 + \mu_1 \beta_2} \\
\rho_{21} &= -\rho_{12} \\
\tau_{12} &= 2 \frac
{\mu_2 \beta_1}
{\mu_2 \beta_1 + \mu_1 \beta_2} \\
\tau_{21} &= 2 \frac
{\mu_1 \beta_2}
{\mu_2 \beta_1 + \mu_1 \beta_2}.
\end{aligned}
\end{equation}
%
Rearranging for \( \tau_{12} \) and \( \tau_{21} \) gives
%
\begin{equation}\label{eqn:emtproblemSet8Problem3:340}
\begin{aligned}
2 \rho_{12} &= \tau_{12} - \tau_{21} \\
2        &= \tau_{12} + \tau_{21},
\end{aligned}
\end{equation}
%
or
\begin{equation}\label{eqn:emtproblemSet8Problem3:360}
\begin{aligned}
\tau_{12} &= 1 + \rho_{12} \\
\tau_{21} &= 1 - \rho_{12}.
\end{aligned}
\end{equation}
%
In particular, we have
\begin{dmath}\label{eqn:emtproblemSet8Problem3:380}
\tau_{12} \tau_{21} = 1 - \rho_{12}^2.
\end{dmath}
%
Does this relation also hold for TM modes?  In problem 1 for TM, we found
%
\begin{equation}\label{eqn:emtproblemSet8Problem3:400}
\begin{aligned}
\rho_{12} &=
\frac
{\epsilon_1 \beta_2 - \epsilon_2 \beta_1}
{\epsilon_2 \beta_1 + \epsilon_1 \beta_2} \\
\rho_{21} &= -\rho_{12} \\
\tau_{12} &=
2 \frac{\eta_2}{\eta_1} \frac
{\epsilon_2 \beta_1}
{\epsilon_2 \beta_1 + \epsilon_1 \beta_2} \\
\tau_{21} &=
2 \frac{\eta_1}{\eta_2} \frac
{\epsilon_1 \beta_2}
{\epsilon_2 \beta_1 + \epsilon_1 \beta_2} \\
\end{aligned}.
\end{equation}
%
Rearranging for \( \tau_{12} \) and \( \tau_{21} \) gives
%
\begin{equation}\label{eqn:emtproblemSet8Problem3:420}
\begin{aligned}
2 \rho_{12} &= -\tau_{12} \frac{\eta_1}{\eta_2} + \tau_{21}\frac{\eta_2}{\eta_1}  \\
2        &= \tau_{12} \frac{\eta_1}{\eta_2} + \tau_{21}\frac{\eta_2}{\eta_1} ,
\end{aligned}
\end{equation}
%
or
%
\begin{equation}\label{eqn:emtproblemSet8Problem3:440}
\begin{aligned}
\tau_{21}\frac{\eta_2}{\eta_1} &= 1 + \rho_{12} \\
\tau_{12} \frac{\eta_1}{\eta_2} &= 1 - \rho_{12},
\end{aligned}
\end{equation}
%
so we also find
\begin{dmath}\label{eqn:emtproblemSet8Problem3:460}
\tau_{12} \tau_{21} = 1 - \rho_{12}^2.
\end{dmath}
%
Inserting this into \cref{eqn:emtproblemSet8Problem3:300}, we have
%
\begin{dmath}\label{eqn:emtproblemSet8Problem3:480}
\begin{bmatrix}
E'_{l,i} \\
E'_{r,i} \\
\end{bmatrix}
=
\begin{bmatrix}
A \\
B
\end{bmatrix}
=
\frac{1}{\gamma (1 - \rho_{12}^2)}
\begin{bmatrix}
1                       &  \rho_{ij} \\
\rho_{ij}                  &  1
\end{bmatrix}
\begin{bmatrix}
1                       & -\rho_{i j} \\
-\gamma^2 \rho_{ij}        & \gamma^2
\end{bmatrix}
\begin{bmatrix}
E \\
F
\end{bmatrix}
=
\frac{1}{\gamma (1 - \rho_{12}^2)}
\begin{bmatrix}
1 - \gamma^2 \rho_{ij}^2 & - \rho_{ij}( 1 - \gamma^2 ) \\
\rho_{ij}( 1 - \gamma^2 ) & \gamma^2 - \rho_{ij}^2
\end{bmatrix}
\begin{bmatrix}
E \\
F
\end{bmatrix}.
\end{dmath}
%
Since
\begin{dmath}\label{eqn:emtproblemSet8Problem3:500}
\begin{bmatrix}
E \\
F
\end{bmatrix}
=
\begin{bmatrix}
E'_{l,i+1}/\gamma' \\
E'_{r,i+1} \gamma'
\end{bmatrix}
=
\begin{bmatrix}
1/\gamma' & 0 \\
0         & \gamma'
\end{bmatrix}
\begin{bmatrix}
E'_{l,i+1} \\
E'_{r,i+1}
\end{bmatrix},
\end{dmath}
%
so
\begin{dmath}\label{eqn:emtproblemSet8Problem3:520}
\begin{bmatrix}
E'_{l,i} \\
E'_{r,i} \\
\end{bmatrix}
=
\frac{1}{\gamma (1 - \rho_{12}^2)}
\begin{bmatrix}
1 - \gamma^2 \rho_{ij}^2 & - \rho_{ij}( 1 - \gamma^2 ) \\
\rho_{ij}( 1 - \gamma^2 ) & \gamma^2 - \rho_{ij}^2
\end{bmatrix}
\begin{bmatrix}
1/\gamma' & 0 \\
0         & \gamma'
\end{bmatrix}
\begin{bmatrix}
E'_{l,i+1} \\
E'_{r,i+1}
\end{bmatrix}
=
\frac{1}{\gamma (1 - \rho_{12}^2)}
\begin{bmatrix}
(1 - \gamma^2 \rho_{ij}^2)/\gamma' & - \rho_{ij}( 1 - \gamma^2 ) \gamma' \\
\rho_{ij}( 1 - \gamma^2 )/\gamma' & (\gamma^2 - \rho_{ij}^2) \gamma'
\end{bmatrix}
\begin{bmatrix}
E'_{l,i+1} \\
E'_{r,i+1}
\end{bmatrix}.
\end{dmath}
%
With the leading coefficient including a \( g = 1/(1 - \rho_{ij}^2) \) factor, the desired matrix elements can be read off
%
\begin{equation}\label{eqn:emtproblemSet8Problem3:540}
\begin{aligned}
a &= \inv{\gamma \gamma'} (1 - \gamma^2 \rho_{ij}^2) \\
\hat{a} &= (\gamma^2 - \rho_{ij}^2) \frac{\gamma'}{\gamma} \\
b &= - \rho_{ij}( 1 - \gamma^2 ) \frac{\gamma'}{\gamma} \\
\hat{b} &= \inv{\gamma \gamma'} \rho_{ij}( 1 - \gamma^2 ).
\end{aligned}
\end{equation}
%
With \( \gamma = e^{-j \phi_j } = e^{j \beta_j d_j}, \gamma' = e^{-j\phi_i} = e^{j \beta_i d_i} \), these are
%
\begin{equation}\label{eqn:emtproblemSet8Problem3:660}
\begin{aligned}
a &= e^{j(\phi_j + \phi_i)} (1 - e^{-2j \phi_j} \rho_{ij}^2) \\
\hat{a} &= (e^{-2j \phi_j} - \rho_{ij}^2) e^{j(\phi_j - \phi_i)} \\
b &= - \rho_{ij}( 1 - e^{-2 j\phi_j} ) e^{j(\phi_j - \phi_i)} \\
\hat{b} &= e^{j(\phi_j + \phi_i)} \rho_{ij}( 1 - e^{-2 j \phi_j} ),
\end{aligned}
\end{equation}
%
or
%
\boxedEquation{eqn:emtproblemSet8Problem3:640}{
\begin{aligned}
a &=
(e^{j \phi_j} - e^{-j \phi_j} \rho_{ij}^2)
e^{ j\phi_i }
\\
\hat{a} &= (e^{-j \phi_j} - e^{j \phi_j} \rho_{ij}^2) e^{-j \phi_i} \\
b &= - \rho_{ij}( e^{j \phi_j} - e^{- j\phi_j} ) e^{-j\phi_i} \\
\hat{b} &=
\rho_{ij}( e^{j\phi_j} - e^{- j \phi_j} )
e^{j\phi_i}
.
\end{aligned}
}
%
The complete transfer matrix is
\begin{equation}\label{eqn:emtproblemSet8Problem3:680}
\begin{aligned}
\BM
&=
\inv{1 - \rho_{i,j}^2} \times \\
&\qquad \begin{bmatrix}
(e^{j \phi_j} - e^{-j \phi_j} \rho_{ij}^2)
e^{ j\phi_i }
& - \rho_{ij}( e^{j \phi_j} - e^{- j\phi_j} ) e^{-j\phi_i} \\
\rho_{ij}( e^{j\phi_j} - e^{- j \phi_j} ) e^{j\phi_i} &
(e^{-j \phi_j} - e^{j \phi_j} \rho_{ij}^2) e^{-j \phi_i} \\
\end{bmatrix}.
\end{aligned}
\end{equation}
%
%
%%\begin{dmath}\label{eqn:emtproblemSet8Problem3:560}
%\boxedEquation{eqn:emtproblemSet8Problem3:560}{
%\begin{aligned}
%a &= e^{-j(\beta_i d_i + \beta_j d_j)} \lr{ 1 - e^{2 j \beta_j d_j} \rho_{ij}^2 } \\
%\hat{a} &= \lr{e^{2 j \beta_j d_j} - \rho_{ij}^2 } e^{j (\beta_i d_i - \beta_j d_j)} \\
%b &= - \rho_{ij}\lr{ 1 - e^{2 j \beta_j d_j} } e^{j (\beta_i d_i - \beta_j d_j)} \\
%\hat{b} &= e^{-j(\beta_i d_i + \beta_j d_j)} \rho_{ij} \lr{ 1 - e^{2 j \beta_j d_j} }
%\end{aligned}
%}
%%\end{dmath}
%
%%%The \( a,b \) and \( \hat{a}, \hat{b} \) differ from each other by phase factors.  Factoring those out we have
%%%
%%%\begin{dmath}\label{eqn:emtproblemSet8Problem3:580}
%%%\hat{a}
%%%= e^{2 j \beta_j d_j} \lr{1 - e^{-2 j \beta_j d_j} \rho_{ij}^2 } e^{j (\beta_i d_i - \beta_j d_j)}
%%%= \lr{1 - e^{-2 j \beta_j d_j} \rho_{ij}^2 } e^{j (\beta_i d_i - 3 \beta_j d_j)}
%%%= e^{-j(\beta_i d_i + \beta_j d_j)} \lr{1 - e^{-2 j \beta_j d_j} \rho_{ij}^2 } e^{2 j (\beta_i d_i - \beta_j d_j)}
%%%= e^{2 j (\beta_i d_i - \beta_j d_j)} a,
%%%\end{dmath}
%%%
%%%and
%%%
%%%\begin{dmath}\label{eqn:emtproblemSet8Problem3:600}
%%%\hat{b}
%%%= e^{-j(\beta_i d_i + \beta_j d_j)} \rho_{ij} \lr{ 1 - e^{2 j \beta_j d_j} }
%%%= e^{-2 j(\beta_i d_i) } e^{j(\beta_i d_i - \beta_j d_j)} \rho_{ij} \lr{ 1 - e^{2 j \beta_j d_j} }
%%%= -e^{-2 j(\beta_i d_i) } b,
%%%\end{dmath}
%%%
%%%\boxedEquation{eqn:emtproblemSet8Problem3:620}{
%%%\begin{aligned}
%%%a &= e^{-j(\beta_i d_i + \beta_j d_j)} \lr{ 1 - e^{2 j \beta_j d_j} \rho_{ij}^2 } \\
%%%b &= - \rho_{ij}\lr{ 1 - e^{2 j \beta_j d_j} } e^{j (\beta_i d_i - \beta_j d_j)} \\-e^{-2 j(\beta_i d_i) } b \\
%%%\hat{a} &= e^{2 j (\beta_i d_i - \beta_j d_j)} a \\
%%%\hat{b} &= -e^{-2 j(\beta_i d_i) } b \\
%%%\end{aligned}
%%%}
}}

         %
% Copyright � 2016 Peeter Joot.  All Rights Reserved.
% Licenced as described in the file LICENSE under the root directory of this GIT repository.
%
\makeproblem{Finite length photonic crystal.}{emt:problemSet9:1}{
%
Consider a truncated (finite length) one dimensional photonic crystal shown in \cref{fig:finitePhotonic:finitePhotonicFig1} below,
in which there are \( N \) dielectric slabs of index \( n_j \)
and length \( d_j \). Find the transmission and reflection functions for this structure as a function of
\( \lambda_1 \),
\( \lambda_2 \),
\( a \),
\( b \),
\( g \)
and
\( \beta_i \), where
\( \lambda_1 \) and
\( \lambda_2 \)
are the
eigenvalues of the one period matrix \( \BM \) given in problem 3 of last week and
\( a \),
\( b \),
\( g \),
and
\( \beta_i \)
are also defined in the same problem.
%
\imageFigure{../figures/ece1228-electromagnetic-theory/finitePhotonicFig1}{Finite photonic crystal.}{fig:finitePhotonic:finitePhotonicFig1}{0.15}
} % makeproblem
%
\makeanswer{emt:problemSet9:1}{\withproblemsetsParagraph{
%
In problem set 8, it was
found that the one period fields were related by
%
\begin{dmath}\label{eqn:emtproblemSet9Problem1:20}
\begin{bmatrix}
E_{l,i}' \\
E_{r,i}'
\end{bmatrix}
=
\BM
\begin{bmatrix}
E_{l,i+1}' \\
E_{r,i+1}'
\end{bmatrix},
\end{dmath}
%
where \( \rho_{i,j} \) was the Fresnel coefficient for the \((i,j)\) interface, and
%
\begin{dmath}\label{eqn:emtproblemSet9Problem1:40}
\BM
=
\inv{1 - \rho_{i,j}^2}
\begin{bmatrix}
(e^{j \phi_j} - e^{-j \phi_j} \rho_{ij}^2)
e^{ j\phi_i }
& - \rho_{ij}( e^{j \phi_j} - e^{- j\phi_j} ) e^{-j\phi_i} \\
\rho_{ij}( e^{j\phi_j} - e^{- j \phi_j} ) e^{j\phi_i} &
(e^{-j \phi_j} - e^{j \phi_j} \rho_{ij}^2) e^{-j \phi_i} \\
\end{bmatrix}.
\end{dmath}
%
Before using this and potentially accumulating errors, let's check this against the slab reflection and transmission amplitudes calculated in class.  If the wave amplitude exiting the slab is \( E \), then \( E_{r,i+1}' = E e^{-j \beta_i d_i} = E e^{j \phi_i} \), so we have
%
\begin{dmath}\label{eqn:emtproblemSet9Problem1:60}
\begin{bmatrix}
E_{l,i}' \\
E_{r,i}'
\end{bmatrix}
=
\inv{1 - \rho_{i,j}^2}
\begin{bmatrix}
(e^{j \phi_j} - e^{-j \phi_j} \rho_{ij}^2)
e^{ j\phi_i }
& - \rho_{ij}( e^{j \phi_j} - e^{- j\phi_j} ) e^{-j\phi_i} \\
\rho_{ij}( e^{j\phi_j} - e^{- j \phi_j} ) e^{j\phi_i} &
(e^{-j \phi_j} - e^{j \phi_j} \rho_{ij}^2) e^{-j \phi_i} \\
\end{bmatrix}
\begin{bmatrix}
0 \\
E e^{j \phi_i}
\end{bmatrix}
=
\frac{E}{1 - \rho_{i,j}^2}
\begin{bmatrix}
- \rho_{ij}( e^{j \phi_j} - e^{- j\phi_j} ) \\
e^{-j \phi_j} - e^{j \phi_j} \rho_{ij}^2 \\
\end{bmatrix}
\end{dmath}
%
The transmission coefficient is
%
\begin{dmath}\label{eqn:emtproblemSet9Problem1:80}
t
= \frac{E}{E_{r,i}'}
= \frac{ 1 - \rho_{i,j}^2 }{ e^{-j \phi_j} - e^{j \phi_j} \rho_{ij}^2 }
= \frac{ (1 - \rho_{i,j}^2) e^{j \phi_j} }{ 1 - e^{2 j \phi_j} \rho_{ij}^2 }
= \frac{ \tau_{ij} \tau_{ji} e^{j \phi_j} }{ 1 - e^{2 j \phi_j} \rho_{ij}^2 }.
\end{dmath}
%
This matches what we found in class.  For the reflection amplitude we have
%
\begin{dmath}\label{eqn:emtproblemSet9Problem1:100}
r
= \frac{E_{l,i}'}{E_{r,i}'}
=
\frac{
- \rho_{ij}( e^{j \phi_j} - e^{- j\phi_j} )
}
{
e^{-j \phi_j} - e^{j \phi_j} \rho_{ij}^2
}
=
\frac{
\rho_{ij}( 1 - e^{2 j \phi_j}  )
}
{
1 - e^{2 j \phi_j} \rho_{ij}^2
}.
\end{dmath}
%
Compare this to the value found in class
\begin{dmath}\label{eqn:emtproblemSet9Problem1:120}
r
=
\rho_{ij} + \frac{\tau_{ij} \tau_{ji} \rho_{ji} e^{ 2 j \phi_j} }{1 - \rho_{ij}^2 e^{2 j \phi_j}}
=
\rho_{ij} - \frac{ (1 - \rho_{ij}^2) \rho_{ij} e^{ 2 j \phi_j} }{1 - \rho_{ij}^2 e^{2 j \phi_j}}
=
\rho_{ij}
\lr{
   1 -
      \frac{ (1 - \rho_{ij}^2) e^{ 2 j \phi_j} }{1 - \rho_{ij}^2 e^{2 j \phi_j} }
}
=
\rho_{ij}
   \frac{ 1 - \rho_{ij}^2 e^{2 j \phi_j} - (1 - \rho_{ij}^2) e^{ 2 j \phi_j} }{1 - \rho_{ij}^2 e^{2 j \phi_j}}
=
\rho_{ij}
   \frac{ 1 - e^{ 2 j \phi_j} }{1 - \rho_{ij}^2 e^{2 j \phi_j}},
\end{dmath}
%
which matches \cref{eqn:emtproblemSet9Problem1:100} as desired, and provides at least partial confidence before continuing.  We can now state the relation between the inputs and outputs for the entire system
%
\begin{dmath}\label{eqn:emtproblemSet9Problem1:140}
\begin{bmatrix}
E_{l,i}' \\
E_{r,i}'
\end{bmatrix}
=
\BM^N
\begin{bmatrix}
0 \\
E e^{j \phi_i}
\end{bmatrix}.
\end{dmath}
%
If \( \BM \) satisfies the eigenvalue relations
%
\begin{equation}\label{eqn:emtproblemSet9Problem1:160}
\begin{aligned}
\BM \Be_1 &= \lambda_1 \Be_1 \\
\BM \Be_2 &= \lambda_2 \Be_2 \\
\end{aligned},
\end{equation}
%
then we can form
\begin{dmath}\label{eqn:emtproblemSet9Problem1:180}
\BM
\begin{bmatrix}
\Be_1 & \Be_2
\end{bmatrix}
=
\begin{bmatrix}
\Be_1 \lambda_1 & \Be_2 \lambda_2
\end{bmatrix}
=
\begin{bmatrix}
\Be_1 & \Be_2
\end{bmatrix}
\begin{bmatrix}
\lambda_1 & 0 \\
0 & \lambda_2
\end{bmatrix}.
\end{dmath}
%
Let
\begin{equation}\label{eqn:emtproblemSet9Problem1:200}
\begin{aligned}
\BE &=
\begin{bmatrix}
\Be_1 & \Be_2
\end{bmatrix} \\
\BD
&=
\begin{bmatrix}
\lambda_1 & 0 \\
0 & \lambda_2
\end{bmatrix}
\end{aligned},
\end{equation}
%
so that
%
\begin{dmath}\label{eqn:emtproblemSet9Problem1:220}
\BM = \BE \BD \BE^{-1}.
\end{dmath}
%
Integer powers of \( \BM \) can now be computed easily
%
\begin{dmath}\label{eqn:emtproblemSet9Problem1:240}
\BM^N
=
(\BE \BD \BE^{-1}) (\BE \BD \BE^{-1}) \cdots (\BE \BD \BE^{-1})
=
\BE \BD^N \BE^{-1}
=
\BE
\begin{bmatrix}
\lambda_1^N & 0 \\
0 & \lambda_2^N
\end{bmatrix}
\BE^{-1}
.
\end{dmath}
%
This gives
%
\begin{dmath}\label{eqn:emtproblemSet9Problem1:260}
\begin{bmatrix}
E_{l,i}' \\
E_{r,i}'
\end{bmatrix}
=
E
\BE \BD^N \BE^{-1}
\begin{bmatrix}
0 \\
e^{j \phi_i}
\end{bmatrix}
.
\end{dmath}
The symbolic expansion of the eigenvalues and eigenvectors for this problem are not particularly illuminating
\begin{equation}\label{eqn:emtproblemSet9Problem1:n}
\begin{aligned}
\lambda_1 &=
 \frac{
e^{-j \left(\phi_i+\phi_j\right)}
}{2 \left(\rho_{ij}^2-1\right)}
\biglr{
   e^{2 j \phi_i} \rho_{ij}^2
   +
   e^{2 j \phi_j} \left(\rho_{ij}^2
   -
   e^{2 j \phi_i}\right) \\
&-2
\sqrt{
   e^{2 j \left(\phi_i+\phi_j\right)}
   \left(\left(\cos  \left(\phi_i+\phi_j\right)-\left(\cos  \left(\phi_i-\phi_j\right)\right) \rho_{ij}^2\right){}^2-\left(\rho_{ij}^2-1\right){}^2\right)}-1
}
\\
\lambda_2 &= \frac{
e^{-j \left(\phi_i+\phi_j\right)}
}{2 \left(\rho_{ij}^2-1\right)}
\biglr{
   e^{2 j \phi_i} \rho_{ij}^2
   +
   e^{2 j \phi_j} \left(\rho_{ij}^2
   -
   e^{2 j \phi_i}
   \right) \\
&
+2 \sqrt{
   e^{2 j \left(\phi_i+\phi_j\right)}
    \left(\left(\cos  \left(\phi_i+\phi_j\right)-\left(\cos  \left(\phi_i-\phi_j\right)\right) \rho_{ij}^2\right){}^2-\left(\rho_{ij}^2-1\right){}^2\right)}-1
}
\\
\end{aligned}
\end{equation}
The corresponding eigenvectors are even messier.  They can be found in
\nbcite{ps9:ps9p1Eigenvalues.nb}
{ps9/ps9p1Eigenvalues.nb} if interested.
%
The transmission and reflection amplitude coefficients are
\boxedEquation{eqn:emtproblemSet9Problem1:280}{
\begin{aligned}
t &= \frac{E} { E_{r,i}' } =
   \inv{
   \begin{bmatrix}
   0 & 1
   \end{bmatrix}
   \BE \BD^N \BE^{-1}
   \begin{bmatrix}
   0 \\
   e^{j \phi_i}
   \end{bmatrix}
   } \\
r &= \frac{ E_{l,i}' } { E_{r,i}' } =
   \frac
   {
   \begin{bmatrix}
   1 & 0
   \end{bmatrix}
      \BE \BD^N \BE^{-1}
      \begin{bmatrix}
      0 \\
      e^{j \phi_i}
      \end{bmatrix}
   }
   {
   \begin{bmatrix}
   0 & 1
   \end{bmatrix}
      \BE \BD^N \BE^{-1}
      \begin{bmatrix}
      0 \\
      e^{j \phi_i}
      \end{bmatrix}
   }
\end{aligned}
}
}}

         %
% Copyright � 2016 Peeter Joot.  All Rights Reserved.
% Licenced as described in the file LICENSE under the root directory of this GIT repository.
%
\makeproblem{Eccostock example.}{emt:problemSet9:2}{
Use the expression for transmission function obtained above and the values and instructions below to plot the following at normal incidence:
%
\makesubproblem{}{emt:problemSet9:2a}
Transmission magnitude and phase as a function of frequency for the case \(N=3\).
\makesubproblem{}{emt:problemSet9:2b}
The group delay as a function of frequency for the cases \(N=1, 2, 3, 4\).
\makesubproblem{}{emt:problemSet9:2c}
The group velocity as a function of frequency for the cases \(N=1, 2, 3\).
%
\begin{itemize}
\item \( n_i = 1 \) (this is air), \( n_j = 3.4 - j 0.002 \) (this is Eccostock).
\item \( d_i = 1.76 \,[\si{cm}] \)
\item \( d_j = 1.33 \,[\si{cm}] \)
\item \( L_{\textrm{PC}} = (N-1)(d_i + d_j) + d_j \)
\item Frequency range for all plots: \( 20 \,[\si{GHz}] \) to \( 23 \,[\si{GHz}] \).
\item Use linear scale for transmission magnitude (not \si{dB}) and express the transmission phase in Degrees.
\item Plot the group delay in nanosecond.
\item Plot the group velocity in units of \( V_g/c \), where \(c\) is the speed of light in vacuum.
\end{itemize}
%
} % makeproblem
%
\makeanswer{emt:problemSet9:2}{\withproblemsetsParagraph{
%
Note: These are apparently wrong.  Haven't seen the graded results.
%
\makeSubAnswer{}{emt:problemSet9:2a}
%
Plotted in \cref{fig:ps9PartaTransmissionFunction:ps9PartaTransmissionFunctionFig1}.
%
\imageFigure{../figures/ece1228-electromagnetic-theory/ps9PartaTransmissionFunctionFig1}{Transmission function.}{fig:ps9PartaTransmissionFunction:ps9PartaTransmissionFunctionFig1}{0.3}
%
\makeSubAnswer{}{emt:problemSet9:2b}
%
Plotted in \cref{fig:ps9PartbGroupDelay:ps9PartbGroupDelayFig2}.
%
\imageFigure{../figures/ece1228-electromagnetic-theory/ps9PartbGroupDelayFig2}{Group delay.}{fig:ps9PartbGroupDelay:ps9PartbGroupDelayFig2}{0.3}
%
\makeSubAnswer{}{emt:problemSet9:2c}
%
Plotted in \cref{fig:ps9PartcGroupVelocity:ps9PartcGroupVelocityFig3}.
%
\imageFigure{../figures/ece1228-electromagnetic-theory/ps9PartcGroupVelocityFig3}{Group velocity.}{fig:ps9PartcGroupVelocity:ps9PartcGroupVelocityFig3}{0.3}
}}

   \mychapter{Gauge freedom.}
      \section{Problems.}
         %
% Copyright � 2016 Peeter Joot.  All Rights Reserved.
% Licenced as described in the file LICENSE under the root directory of this GIT repository.
%
\makeproblem{Potentials under different gauges.}{emt:problemSet9:3}{
%
Using the non-existence of magnetic monopole and Faraday's law
%
\makesubproblem{}{emt:problemSet9:3a}
Define the vector and scalar vector potentials
\( \BA(\Br, t) \) and
\( V(\Br, t) \).
%
\makesubproblem{}{emt:problemSet9:3b}
%
Let \( \BJ = \BJ_i + \BJ_c \) be
the current \,[\si{A/m}] and
\( \rho \)
be the charge \,[\si{C/m}] densities.
Assuming a simple medium and Lorentz gauge, derive the decoupled non-homogeneous wave equations for
\( \BA(\Br, t) \) and
\( V(\Br, t) \).
%
\makesubproblem{}{emt:problemSet9:3c}
Replace the Lorentz gauge of
\partref{emt:problemSet9:3b}
%part (b)
with the Coulomb gauge, and obtain the non-homogeneous differential equations for
\( \BA(\Br, t) \) and
\( V(\Br, t) \).
%
\makesubproblem{}{emt:problemSet9:3d}
What fundamental theorem allows us to use different gauges in
\partref{emt:problemSet9:3b}
and
\partref{emt:problemSet9:3b}
%parts (b) and (c)
? (Justify your answer.)
%
\paragraph{Note:} From the problem's statement, it should be clear that I want the results for the instantaneous fields and not in the form of time harmonic fields.
} % makeproblem
%
\skipIfRedacted{
\makeanswer{emt:problemSet9:3}{
\makeSubAnswer{}{emt:problemSet9:3a}
%
The non-existence of magnetic monopoles means that we are working with the standard form of Maxwell's equations, with no volume magnetic charge density, nor any magnetic currents
%
\begin{subequations}
\label{eqn:emtproblemSet9Problem3:20}
\begin{equation}\label{eqn:emtproblemSet9Problem3:40}
\spacegrad \cross \BE = -\partial_t \BB
\end{equation}
\begin{equation}\label{eqn:emtproblemSet9Problem3:60}
\spacegrad \cross \BH = \BJ + \partial_t \BD
\end{equation}
\begin{equation}\label{eqn:emtproblemSet9Problem3:80}
\spacegrad \cross \BD = \rho
\end{equation}
\begin{equation}\label{eqn:emtproblemSet9Problem3:100}
\spacegrad \cross \BB = 0.
\end{equation}
\end{subequations}
%
Helmholtz's theorem applied to \cref{eqn:emtproblemSet9Problem3:100}, indicates that the magnetic field \( \BB \) must be a curl
%
\begin{equation}\label{eqn:emtproblemSet9Problem3:120}
\BB = \spacegrad \cross \BA,
\end{equation}
%
for some \( \BA \).  Plugging this into \cref{eqn:emtproblemSet9Problem3:40} we have
\begin{equation}\label{eqn:emtproblemSet9Problem3:140}
\spacegrad \cross \BE = -\partial_t \lr{ \spacegrad \cross \BA },
\end{equation}
%
or
\begin{equation}\label{eqn:emtproblemSet9Problem3:160}
\spacegrad \cross \lr{ \BE + \partial_t \BA } = 0.
\end{equation}
%
This is satisfied by setting
%
\begin{equation}\label{eqn:emtproblemSet9Problem3:180}
\BE + \partial_t \BA = \grad \psi,
\end{equation}
%
for some \( \psi \), say \( \psi = -V \), or
%
\begin{equation}\label{eqn:emtproblemSet9Problem3:200}
\BE = -\partial_t \BA - \spacegrad V.
\end{equation}
%
Ignoring gauge freedom temporarily, in simple media where \( \BB = \mu \BH \) and \( \BD = \epsilon \BE \), the Ampere-Maxwell equation is
%
\begin{equation}\label{eqn:emtproblemSet9Problem3:220}
\begin{aligned}
0
&= \spacegrad \cross \BH - \BJ - \partial_t \BD
\\ &= \spacegrad \cross \frac{ \spacegrad \cross \BA }{\mu} - \BJ - \epsilon \partial_t \lr{ -\partial_t \BA - \spacegrad V },
\end{aligned}
\end{equation}
%
or
%
%\begin{equation}\label{eqn:emtproblemSet9Problem3:240}
\boxedEquation{eqn:emtproblemSet9Problem3:240}{
-\spacegrad^2 \BA + \mu\epsilon \partial_{tt} \BA + \spacegrad \lr{ \spacegrad \cdot \BA + \epsilon\mu \partial_t V } = \mu \BJ.
}
%\end{equation}
%
Gauss's law takes the form
%
\begin{equation}\label{eqn:emtproblemSet9Problem3:300}
\begin{aligned}
\frac{\rho}{\epsilon}
&=
\spacegrad \cdot \BE
\\ &=
\spacegrad \cdot \lr{
-\partial_t \BA - \spacegrad V
},
\end{aligned}
\end{equation}
%
or
%\begin{equation}\label{eqn:emtproblemSet9Problem3:320}
\boxedEquation{eqn:emtproblemSet9Problem3:320}{
-\spacegrad^2 V
-\partial_t \spacegrad \cdot \BA
=
\frac{\rho}{\epsilon}.
}
%\end{equation}
%
\makeSubAnswer{}{emt:problemSet9:3b}
%
The Lorentz gauge is usually defined in vacuum where
%
\begin{equation}\label{eqn:emtproblemSet9Problem3:260}
\begin{aligned}
0
&= \partial_\mu A^\mu
\\ &= \partial_0 (V/c) + \partial_k A^k
\\ &= \inv{c^2} \partial_t V + \spacegrad \cdot \BA.
\end{aligned}
\end{equation}
%
Generalizing this to simple media, we can define the Lorentz gauge condition as
%
%\begin{equation}\label{eqn:emtproblemSet9Problem3:280}
\boxedEquation{eqn:emtproblemSet9Problem3:280}{
0 = \spacegrad \cdot \BA + \epsilon\mu \partial_t V,
}
%\end{equation}
%
which reduces to \cref{eqn:emtproblemSet9Problem3:260} when \( \epsilon\mu = \epsilon_0 \mu_0 = 1/c^2 \).
%
Inserting this into the Ampere-Maxwell equation, we are left with a non-homogeneous wave equation for \( BA \)
%
\boxedEquation{eqn:emtproblemSet9Problem3:360}{
-\spacegrad^2 \BA + \mu\epsilon \partial_{tt} \BA = \mu \BJ.
}
%
Inserting into Gauss's law, we have
%
\begin{equation}\label{eqn:emtproblemSet9Problem3:380}
\begin{aligned}
\frac{\rho}{\epsilon}
&=
-\spacegrad^2 V
-\partial_t \lr{ \spacegrad \cdot \BA }
\\ &=
-\spacegrad^2 V
-\partial_t \lr{ -\epsilon\mu \partial_t V },
\end{aligned}
\end{equation}
%
a second non-homogeneous wave equation for the potential \( V \)
%
%\begin{equation}\label{eqn:emtproblemSet9Problem3:400}
\boxedEquation{eqn:emtproblemSet9Problem3:420}{
-\spacegrad^2 V + \epsilon\mu \partial_{tt} V
=
\frac{\rho}{\epsilon}.
}
%\end{equation}
%
Note that if \( \BJ = \BJ_i + \BJ_c = \BJ_i + \sigma \BE \), then \cref{eqn:emtproblemSet9Problem3:360} is not fully decoupled.  Instead we have
%
\begin{equation}\label{eqn:emtproblemSet9Problem3:440}
-\spacegrad^2 \BA + \mu\epsilon \partial_{tt} \BA
= \mu \lr{
   \BJ_i + \sigma \lr{
      -\partial_t \BA - \spacegrad V
   }
}.
\end{equation}
%
To decouple this, we can take the divergence of both sides
\begin{equation}\label{eqn:emtproblemSet9Problem3:460}
\begin{aligned}
-\spacegrad^2 \spacegrad \cdot \BA + \mu\epsilon \partial_{tt} \spacegrad \cdot \BA
&= \mu \lr{
   \spacegrad \cdot \BJ_i + \sigma \lr{
      -\partial_t \spacegrad \cdot \BA - \spacegrad^2 V
   }
}
\\ &= \mu \lr{
   \spacegrad \cdot \BJ_i + \sigma \lr{
      -\partial_t \spacegrad \cdot \BA + \frac{\rho}{\epsilon} + \partial_t \spacegrad \cdot \BA
   }
}
%\\ &= \mu \lr{
%   \spacegrad \cdot \BJ_i + \sigma \lr{
%\frac{\rho}{\epsilon}
%   }
%}
\\ &=
\mu \spacegrad \cdot \BJ_i
 + \frac{\mu \sigma}{\epsilon} \rho,
\end{aligned}
\end{equation}
%
which yields a third order equation for \( \BA \)
%
\begin{equation}\label{eqn:emtproblemSet9Problem3:480}
\spacegrad \cdot \lr{
-\spacegrad^2 \BA + \mu\epsilon \partial_{tt} \BA
}
=
\mu \spacegrad \cdot \BJ_i
 + \frac{\mu \sigma}{\epsilon} \rho.
\end{equation}
%
\makeSubAnswer{}{emt:problemSet9:3c}
%
In the Coulomb gauge we set \( \spacegrad \cdot \BA = 0 \), leaving
%
\boxedEquation{eqn:emtproblemSet9Problem3:500}{
-\spacegrad^2 V = \frac{\rho}{\epsilon},
}
%
and
\begin{equation}\label{eqn:emtproblemSet9Problem3:520}
-\spacegrad^2 \BA + \mu\epsilon \partial_{tt} \BA + \epsilon\mu \partial_t \spacegrad V = \mu \BJ.
\end{equation}
%
This can clearly be separated by taking the divergence of both sides
%
\begin{equation}\label{eqn:emtproblemSet9Problem3:540}
\spacegrad \cdot \lr{ -\spacegrad^2 \BA + \mu\epsilon \partial_{tt} \BA } + \epsilon\mu \partial_t \lr{ -\frac{\rho}{\epsilon}} = \mu \spacegrad \cdot \BJ.
\end{equation}
%
Using the continuity equation, this is a wave equation for the divergence of the vector potential \( \BA \)
%
%\begin{equation}\label{eqn:emtproblemSet9Problem3:740}
\boxedEquation{eqn:emtproblemSet9Problem3:760}{
-\spacegrad^2 \lr{ \spacegrad \cdot \BA } + \mu\epsilon \partial_{tt} \lr{ \spacegrad \cdot \BA } = 0.
}
%\end{equation}
%
Another way to look at this, as outlined in \citep{jackson1975cew} is to note that
the potential equation \cref{eqn:emtproblemSet9Problem3:540}
depends only on the divergence of \( \BJ \), and not on the curl of \( \BJ \).  This suggests that a Helmholtz decomposition of the current into its divergence and divergence free components in the second order equation \cref{eqn:emtproblemSet9Problem3:520} may help eliminate the coupling, without resorting to a third order PDE.
%
That decomposition follows from expanding \( \spacegrad^2 \BJ/R \) in two ways using the delta function \( -4 \pi \delta(\Bx - \Bx') = \spacegrad^2 1/R \) representation, as well as directly
%
\begin{equation}\label{eqn:emtproblemSet9Problem3:560}
\begin{aligned}
- 4 \pi \BJ(\Bx)
&=
\int \spacegrad^2 \frac{\BJ(\Bx')}{\Abs{\Bx - \Bx'}} d^3 x'
\\ &=
\spacegrad
\int \spacegrad \cdot \frac{\BJ(\Bx')}{\Abs{\Bx - \Bx'}} d^3 x'
+
\spacegrad \cdot
\int \spacegrad \wedge \frac{\BJ(\Bx')}{\Abs{\Bx - \Bx'}} d^3 x'
\\ &=
-\spacegrad
\int \BJ(\Bx') \cdot \spacegrad' \inv{\Abs{\Bx - \Bx'}} d^3 x'
+
\spacegrad \cdot \lr{ \spacegrad \wedge
\int \frac{\BJ(\Bx')}{\Abs{\Bx - \Bx'}} d^3 x'
}
\\ &=
-\spacegrad
\int \spacegrad' \cdot \frac{\BJ(\Bx')}{\Abs{\Bx - \Bx'}} d^3 x'
+\spacegrad
\int \frac{\spacegrad' \cdot \BJ(\Bx')}{\Abs{\Bx - \Bx'}} d^3 x'
-
\spacegrad \cross \lr{
\spacegrad \cross
\int \frac{\BJ(\Bx')}{\Abs{\Bx - \Bx'}} d^3 x'
}.
\end{aligned}
\end{equation}
%
The first term can be converted to a surface integral
%
\begin{equation}\label{eqn:emtproblemSet9Problem3:580}
-\spacegrad
\int \spacegrad' \cdot \frac{\BJ(\Bx')}{\Abs{\Bx - \Bx'}} d^3 x'
=
-\spacegrad
\int d\BA' \cdot \frac{\BJ(\Bx')}{\Abs{\Bx - \Bx'}},
\end{equation}
%
so provided the currents are either localized or \( \Abs{\BJ}/R \rightarrow 0 \) on an infinite sphere, we can make the identification
%
\begin{equation}\label{eqn:emtproblemSet9Problem3:600}
\BJ(\Bx)
=
\spacegrad
\inv{4\pi} \int \frac{\spacegrad' \cdot \BJ(\Bx')}{\Abs{\Bx - \Bx'}} d^3 x'
-
\spacegrad \cross \spacegrad \cross \inv{4 \pi} \int \frac{\BJ(\Bx')}{\Abs{\Bx - \Bx'}} d^3 x'
\equiv
\BJ_l +
\BJ_t,
\end{equation}
%
where \( \spacegrad \cross \BJ_l = 0 \) (irrotational, or longitudinal), whereas \( \spacegrad \cdot \BJ_t = 0 \) (solenoidal or transverse).  The irrotational property is clear from inspection, and the transverse property can be verified readily
%
\begin{equation}\label{eqn:emtproblemSet9Problem3:620}
\begin{aligned}
\spacegrad \cdot \lr{ \spacegrad \cross \lr{ \spacegrad \cross \BX } }
&=
-\spacegrad \cdot \lr{ \spacegrad \cdot \lr{ \spacegrad \wedge \BX } }
\\ &=
-\spacegrad \cdot \lr{ \spacegrad^2 \BX - \spacegrad \lr{ \spacegrad \cdot \BX } }
\\ &=
-\spacegrad \cdot \lr{\spacegrad^2 \BX} + \spacegrad^2 \lr{ \spacegrad \cdot \BX }
\\ &= 0.
\end{aligned}
\end{equation}
%
Since
%
\begin{equation}\label{eqn:emtproblemSet9Problem3:640}
V(\Bx, t)
=
\inv{4 \pi \epsilon} \int \frac{\rho(\Bx', t)}{\Abs{\Bx - \Bx'}} d^3 x',
\end{equation}
%
we have
%
\begin{equation}\label{eqn:emtproblemSet9Problem3:660}
\begin{aligned}
\spacegrad \PD{t}{V}
&=
\inv{4 \pi \epsilon} \spacegrad \int \frac{\partial_t \rho(\Bx', t)}{\Abs{\Bx - \Bx'}} d^3 x'
\\ &=
\inv{4 \pi \epsilon} \spacegrad \int \frac{-\spacegrad' \cdot \BJ}{\Abs{\Bx - \Bx'}} d^3 x'
\\ &=
\frac{\BJ_l}{\epsilon}.
\end{aligned}
\end{equation}
%
The Ampere-Maxwell equation \cref{eqn:emtproblemSet9Problem3:520} is reduced to
%
\begin{equation}\label{eqn:emtproblemSet9Problem3:680}
-\spacegrad^2 \BA + \mu\epsilon \partial_{tt} \BA + \epsilon\mu \lr{ \frac{\BJ_l}{\epsilon} } = \mu \BJ,
\end{equation}
%
or
\boxedEquation{eqn:emtproblemSet9Problem3:700}{
-\spacegrad^2 \BA + \mu\epsilon \partial_{tt} \BA = \mu \BJ_t,
}
%
where
\begin{equation}\label{eqn:emtproblemSet9Problem3:720}
\BJ_t =
-
\spacegrad \cross \spacegrad \cross \inv{4 \pi} \int \frac{\BJ(\Bx')}{\Abs{\Bx - \Bx'}} d^3 x'.
\end{equation}
%
\makeSubAnswer{}{emt:problemSet9:3d}
%
Because the magnetic field is defined in terms of the curl of a potential, we can alter that potential by any gradient, and not change the magnetic field
%
\begin{equation}\label{eqn:emtproblemSet9Problem3:780}
\BB = \spacegrad \lr{ \BA + \spacegrad \chi }.
\end{equation}
%
What corresponding change to the scalar potential must be made after a mapping
%
\begin{equation}\label{eqn:emtproblemSet9Problem3:800}
\BA \rightarrow \BA + \spacegrad \chi = \BA'.
\end{equation}
%
The electric field is
%
\begin{equation}\label{eqn:emtproblemSet9Problem3:820}
\begin{aligned}
\BE
&= -\spacegrad V - \partial_t \BA
\\ &= -\spacegrad V - \partial_t \lr{ \BA' - \spacegrad \chi }
\\ &= -\spacegrad \lr{ V - \partial_t \chi } - \partial_t \BA'.
\end{aligned}
\end{equation}
%
We see that the electric and magnetic fields will be unchanged provided the vector and scalar potential are changed in pairs as
%\begin{equation}\label{eqn:emtproblemSet9Problem3:840}
\boxedEquation{eqn:emtproblemSet9Problem3:860}{
\begin{aligned}
\BA' &= \BA + \spacegrad \chi \\
V' &= V - \partial_t \chi.
\end{aligned}
}
%
We should also expect that the form of Maxwell's equations will be left unchanged.  Plugging these transformed potentials into Gauss's law \cref{eqn:emtproblemSet9Problem3:320}, we recover the original field equation in terms of the unaltered potentials
%
\begin{equation}\label{eqn:emtproblemSet9Problem3:880}
\begin{aligned}
\frac{\rho}{\epsilon}
&=
-\spacegrad^2 V'
-\partial_t \spacegrad \cdot \BA'
\\ &=
-\spacegrad^2 \lr{ V - \partial_t \chi }
-\partial_t \spacegrad \cdot \lr{ \BA + \spacegrad \chi }
\\ &=
-\spacegrad^2 V
-\partial_t \spacegrad \cdot \BA
- \spacegrad \partial_t \chi
-\partial_t \spacegrad^2 \chi
\\ &=
-\spacegrad^2 V
-\partial_t \spacegrad \cdot \BA.
\end{aligned}
\end{equation}
%
Similarly, plugging in the transformed potentials into \cref{eqn:emtproblemSet9Problem3:240} gives
%
\begin{equation}\label{eqn:emtproblemSet9Problem3:900}
\begin{aligned}
\mu \BJ
&=
-\spacegrad^2 \BA' + \mu\epsilon \partial_{tt} \BA' + \spacegrad \lr{ \spacegrad \cdot \BA' + \epsilon\mu \partial_t V' }
\\ &=
-\spacegrad^2 \lr{ \BA + \spacegrad \chi }
+ \mu\epsilon \partial_{tt} \lr{ \BA + \spacegrad \chi }
+ \spacegrad \lr{ \spacegrad \cdot \lr{ \BA + \spacegrad \chi } + \epsilon\mu \partial_t \lr{ V - \partial_t \chi } }
\\ &=
-\spacegrad^2 \BA
+ \mu\epsilon \partial_{tt} \BA
+ \spacegrad \lr{ \spacegrad \cdot \BA } + \epsilon\mu \partial_t V
-\spacegrad^2 \spacegrad \chi
+ \mu\epsilon \partial_{tt} \spacegrad \chi
+ \spacegrad \spacegrad^2 \chi - \epsilon\mu \spacegrad \partial_{tt} \chi
\\ &=
-\spacegrad^2 \BA
+ \mu\epsilon \partial_{tt} \BA
+ \spacegrad \lr{ \spacegrad \cdot \BA } + \epsilon\mu \partial_t V,
\end{aligned}
\end{equation}
%
showing that both pairs of potential equations are unaltered by a transformation of the form \cref{eqn:emtproblemSet9Problem3:860}.
}}

