%
% Copyright � 2016 Peeter Joot.  All Rights Reserved.
% Licenced as described in the file LICENSE under the root directory of this GIT repository.
%
%----------------------------------------------------------------------------------------
%
% Prof. decoder ring:
%
% antigreat      == integrate
% ambeeguis      == ambiguous
% eikenvector    == eigenvector
% esmal          == small
% sine "of a" x  == sin(x)
% beta suba x    == beta_x
% constrain      == constraint
% havea expr     == have expr (example: havea e^x), plug this expression for beta -> plug this expression for abeta.
% isa            == is : f_{c1} isa bigger than f_{c2}
% den            == then
% ispectra       == spectrum
% smood          == smooth
% togeder        == together
%
%----------------------------------------------------------------------------------------
%\part{Lecture notes}
   %\mychapter{Electromagnetic fields}
\mychapter{Introduction.}
   %
% Copyright � 2016 Peeter Joot.  All Rights Reserved.
% Licenced as described in the file LICENSE under the root directory of this GIT repository.
%
%\input{../blogpost.tex}
%\renewcommand{\basename}{emt1}
%\renewcommand{\dirname}{notes/ece1228/}
%\newcommand{\keywords}{ECE1228H}
%\input{../latex/peeter_prologue_print2.tex}
%
%%\usepackage{ece1228}
%\usepackage{peeters_braket}
%%\usepackage{peeters_layout_exercise}
%\usepackage{peeters_figures}
%\usepackage{mathtools}
%\usepackage{siunitx}
%
%\beginArtNoToc
%\generatetitle{ECE1228H Electromagnetic Theory.  Lecture 1: Introduction.  Taught by Prof.\ M. Mojahedi}
%\mychapter{Introduction.}
%\label{chap:emt1}
%
%\paragraph{Disclaimer}
%
%Peeter's lecture notes from class.  These may be incoherent and rough.
%
%These are notes for the UofT course ECE1228H, Electromagnetic Theory, taught by Prof. M. Mojahedi, covering \textchapref{{1}} \%citep{balanis1989advanced} content.
%
\section{Conventions for Maxwell's equations.}
\index{Maxwell's equations!time domain}
%
In these course notes, Maxwell's equations will be written in one of two forms.  The first is the standard bold face vectors, where the fields are assumed to be real.
%
\begin{itemize}
\item Faraday's Law
\begin{dmath}\label{eqn:emtLecture1:20}
\spacegrad \cross \BE( \Br, t ) = - \PD{t}{\BB}(\Br, t) - \BM_i,
\end{dmath}
\item Ampere-Maxwell equation
\begin{dmath}\label{eqn:emtLecture1:40}
\spacegrad \cross \BH( \Br, t ) = \BJ_\txtc(\Br, t) + \PD{t}{\BD}(\Br, t),
\end{dmath}
\item Gauss's law
\begin{dmath}\label{eqn:emtLecture1:80}
\spacegrad \cdot \BD(\Br, t) = \rho_{\txte\txtv}(\Br, t),
\end{dmath}
\item Gauss's law for magnetism
\begin{dmath}\label{eqn:emtLecture1:100}
\spacegrad \cdot \BB(\Br, t) = \rho_{\txtm\txtv}(\Br, t).
\end{dmath}
\end{itemize}
%
In chapters where frequency domain analysis is used, Maxwell's equations will be written in script
%
\begin{equation}\label{eqn:emtLecture1:160}
\begin{aligned}
\spacegrad \cross \bcE &= -\PD{t}{\bcB} - \bcM \\
\spacegrad \cross \bcH &= \PD{t}{\bcD} + \bcJ \\
\spacegrad \cross \bcB &= q_{mv} \\
\spacegrad \cross \bcD &= q_{ev} \\
\end{aligned}
\end{equation}
%
with bold face reserved for complex valued field variables.  In the frequency domain (called time harmonic form in this class), the frequency dependence is of the form
%
\begin{dmath}\label{eqn:emtLecture1:140}
   \bcX = \Real\lr{ \BX e^{j\omega t} }.
\end{dmath}
%
In this form, Maxwell's equations are
%
\begin{equation}\label{eqn:emtLecture1:180}
\begin{aligned}
\spacegrad \cross \BE &= -j \omega \BB - \BM \\
\spacegrad \cross \BH &= j \omega \BD + \BJ \\
\spacegrad \cross \BB &= \rho_{mv} \\
\spacegrad \cross \BD &= \rho_{ev}.
\end{aligned}
\end{equation}
%
Where there is no ambiguity, bold face vectors will be used, even in the time domain.
%
\section{Units.}
%
Regardless of the conventions, after unpacking, we have a total of eight equations, with four vectoral field variables, and 8 sources, all interrelated by partial derivatives in space and time coordinates.
%
It will be left to homework to show that without the displacement current \( \PDi{t}{\BD} \), these equations will not satisfy conservation relations.
%
The fields are and sources are
\index{units}
\begin{itemize}
\item \( \BE \) Electric field intensity \si{V/m},
\item \( \BB \) Magnetic flux density \si{V s/m^2} (or Tesla),
\item \( \BH \) Magnetic field intensity \si{A/m},
\item \( \BD \) Electric flux density \si{C/m^2},
\item \( \rho_{\txte\txtv} \) Electric charge volume density,
\item \( \rho_{\txtm\txtv} \) Magnetic charge volume density,
\item \( \BJ_{\txtc} \) Impressed (source) electric current ,ensity \si{A/m^2}.  This is the charge passing through a plane in a unit time.  Here \( \txtc \) is for ``conduction''.
\item \( \BM_{\txti} \) Impressed (source) magnetic current density \si{V/m^2}.
\end{itemize}
%
In an undergrad context we'll have seen the electric and magnetic fields in the Lorentz force law
%
\begin{dmath}\label{eqn:emtLecture1:120}
\BF = q \Bv \cross \BB + q\BE.
\end{dmath}
%
In SI there are 7 basic units.  These include
%
\begin{itemize}
\item length \si{m},
\item mass \si{kg},
\item time \si{s},
\item ampere \si{A},
\index{unit!ampere}
\item kelvin \si{K} (temperature),
\index{unit!kelvin}
\item candela (luminous intensity),
\index{unit!candela}
\item mole (amount of substance),
\index{unit!mole}
\end{itemize}
%
\index{unit!coulomb}
Note that the coulomb is not a fundamental unit, but the ampere is.  This is because it is easier to measure.

For homework: show that magnetic field lines must close on themselves when there are no magnetic sources (zero divergence).  This is opposed to electric fields that spread out from the charge.
%
%\EndNoBibArticle

\mychapter{Boundaries.}
   \input{emtLecture2.tex}
   \section{Problems.}
      \input{continuityDisplacement.tex}
      \input{griffithsEM2_7.tex}
      \input{Set1Problem1.tex}
      \input{Set1Problem2.tex}
      \input{Set1Problem3.tex}
      \input{Set1Problem4.tex}
      \input{Set1Problem5.tex}
      \input{Set1Problem6.tex}
      %\input{Set1Appendix.tex}
      \input{Set2Problem2.tex}
      \input{Set2Problem3.tex}
      \input{Set2Problem5.tex}
\mychapter{Electrostatics and dipoles.}
   \input{emtLecture3.tex}
   \section{Problems.}
      \input{Set2Problem1.tex}
      \input{Set2Problem4.tex}
      \input{dipoleMoment.tex}
      %
% Copyright � 2016 Peeter Joot.  All Rights Reserved.
% Licenced as described in the file LICENSE under the root directory of this GIT repository.
%
%{
%\input{../blogpost.tex}
%\renewcommand{\basename}{dipolePotential}
%%\renewcommand{\dirname}{notes/phy1520/}
%\renewcommand{\dirname}{notes/ece1228-electromagnetic-theory/}
%%\newcommand{\dateintitle}{}
%%\newcommand{\keywords}{}
%
%\input{../latex/peeter_prologue_print2.tex}
%
%\usepackage{peeters_layout_exercise}
%\usepackage{peeters_braket}
%\usepackage{peeters_figures}
%\usepackage{siunitx}
%
%\beginArtNoToc
%
%\generatetitle{Electric dipole potential}
%\chapter{Electric dipole potential}
%\label{chap:dipolePotential}
%
\makeproblem{Electric dipole potential.}{problem:dipolePotential:1}{
\index{potential!electric dipole}
\index{dipole!potential}
%
Having shown that
%
\begin{equation}\label{eqn:dipolePotential:20}
\BE =
\frac{1}{4 \pi \epsilon_0 r^3} \lr{
3 \rcap \lr{ \rcap \cdot \Bp }
-\Bp
},
\end{equation}
%
find the expression for the electric potential for this field.
} % problem
%
\makeanswer{problem:dipolePotential:1}{
%\withproblemsetsParagraph{
%
With the electric potential defined indirectly by
\begin{equation}\label{eqn:dipolePotential:40}
\BE = -\spacegrad V,
\end{equation}
%
we can integrate to find the difference in potential between two points
\begin{equation}\label{eqn:dipolePotential:60}
\begin{aligned}
\int_\Ba^\Bb \BE \cdot d\Bl &=
- \int
\int_\Ba^\Bb \spacegrad V \cdot d\Bl
\\ &=
- \lr{ V(\Bb) - V(\Ba) },
\end{aligned}
\end{equation}
%
or
\begin{equation}\label{eqn:dipolePotential:80}
V(\Bb) - V(\Ba) = -
\int_\Ba^\Bb \BE \cdot d\Bl.
\end{equation}
%
Since the dipole potential is zero at \( \Br = \infty \), we have
%
\begin{equation}\label{eqn:dipolePotential:100}
V(\Br)
= -\int_\infty^\Br \BE \cdot d\Bl.
\end{equation}
%
Let's integrate this on the radial path \( \Br(r') = r'\rcap \), for \( r' \in [\infty, r] \)
%
\begin{equation}\label{eqn:dipolePotential:120}
\begin{aligned}
V(\Br)
&= -\int_\infty^\Br \BE \cdot d\Bl
\\ &= -\int_\infty^\Br \BE \cdot \rcap dr'
\\ &=
-
\frac{1}{4 \pi \epsilon_0 }
\int_\infty^r \frac{dr'}{{r'}^3}
\rcap
\cdot
\lr{
3 \rcap \lr{ \rcap \cdot \Bp }
-\Bp
}
\\ &=
-\frac{2}{4 \pi \epsilon_0 }
\int_\infty^r dr' \frac{\rcap\cdot \Bp}{{r'}^3}
\\ &=
\frac{\rcap \cdot \Bp}{4 \pi \epsilon_0 } \evalrange{ \inv{{r'}^2} }{\infty}{r},
\end{aligned}
\end{equation}
%
so
%\begin{equation}\label{eqn:dipolePotential:160}
\boxedEquation{eqn:dipolePotential:140}{
V(\Br) =
\frac{ \rcap \cdot \Bp}{4 \pi \epsilon_0 }.
}
%\end{equation}
%}
} % answer
%
%}
%\EndNoBibArticle

\mychapter{Magnetic moment, and Boundary value conditions.}
   %
% Copyright � 2016 Peeter Joot.  All Rights Reserved.
% Licenced as described in the file LICENSE under the root directory of this GIT repository.
%
%\input{../blogpost.tex}
%\renewcommand{\basename}{emt4}
%\renewcommand{\dirname}{notes/ece1228/}
%\newcommand{\keywords}{ECE1228H}
%\input{../latex/peeter_prologue_print2.tex}
%
%%\usepackage{ece1228}
%\usepackage{peeters_braket}
%%\usepackage{peeters_layout_exercise}
%\usepackage{peeters_figures}
%\usepackage{mathtools}
%\usepackage{siunitx}
%
%\beginArtNoToc
%\generatetitle{ECE1228H Electromagnetic Theory.  Lecture 4: Magnetic moment, and Boundary value conditions.  Taught by Prof.\ M. Mojahedi}
%\mychapter{Magnetic moment, and Boundary value conditions.}
%\label{chap:emt4}
%
%\paragraph{Disclaimer}
%
%Peeter's lecture notes from class.  These may be incoherent and rough.
%
%These are notes for the UofT course ECE1228H, Electromagnetic Theory, taught by Prof. M. Mojahedi, covering \textchapref{{1}} \citep{balanis1989advanced} content.
%
\section{Magnetic moment.}
\index{magnetic moment}
\index{moment!magnetic}
%
Using a semi-classical model of an electron, assuming that the electron circles the nuclei.  This is a completely wrong model, but useful.  In reality, electrons are random and probabilistic and do not follow defined paths.  We do however have a magnetic moment associated with the electron, and one associated with the spin of the electron, and a moment associated with the spin of the nuclei.  All of these concepts can be used to describe a more accurate model and such a model is discussed in \citep{jackson1975cew} chapters 11,12,13.
%
Ignoring the details of how the moments really occur physically, we can take it as a given that they exist, and model them as elemental magnetic dipole moments of the form
%
\begin{dmath}\label{eqn:emtLecture4:20}
d\Bm_i = \ncap_i I_i ds_i \qquad [\si{A m^2}].
\end{dmath}
%
Note that \( ds_i \) is an element of surface area, not arc length!
%
Here the normal is defined in terms of the right hand rule with respect to the direction of the current as sketched in \cref{fig:emtLecture4:emtLecture4Fig1}.
\imageFigure{../figures/ece1228-electromagnetic-theory/emtLecture4Fig1}{Orientation of current loop.}{fig:emtLecture4:emtLecture4Fig1}{0.3}
%
Such dipole moments are actually what an MRI measures.  The noises that people describe from MRI machines are actually when the very powerful magnets are being rotated, allowing for the magnetic moments in the atoms of the body to be measured in different directions.
%
\index{magnetic polarization}
\index{magnetization}
The magnetic polarization, or magnetization \( \BM \), in [\si{A/m}]] is given by
%
\begin{dmath}\label{eqn:emtLecture4:40}
\BM
= \lim_{\Delta v \rightarrow 0} \lr{ \inv{\Delta v} \Bm_i }
= \lim_{\Delta v \rightarrow 0} \lr{ \inv{\Delta v} \sum_{i = 1}^{N \delta v} d\Bm_i }
= \lim_{\Delta v \rightarrow 0} \lr{ \inv{\Delta v} \sum_{i = 1}^{N \delta v} \ncap_i I_i ds_i } .
\end{dmath}
%
In materials the magnetization within the atoms are usually random, however, application of a magnetic field can force these to line up, as sketched in \cref{fig:emtLecture4:emtLecture4Fig2}.
%
\imageFigure{../figures/ece1228-electromagnetic-theory/emtLecture4Fig2}{External magnetic field alignment of magnetic moments.}{fig:emtLecture4:emtLecture4Fig2}{0.3}
%
\index{torque}
This is accomplished because an applied magnetic field acting on the magnetic moment introduces a torque, as also occurred with dipole moments under applied electric fields
%
\begin{equation}\label{eqn:emtLecture4:60}
\begin{aligned}
\Btau_B &= d\Bm \cross \BB_a, \\
\Btau_E &= d\Bp \cross \BE_a.
\end{aligned}
\end{equation}
%
\index{energy!torque}
There is an energy associated with this torque
%
\begin{equation}\label{eqn:emtLecture4:80}
\begin{aligned}
\Delta U_B &= -d\Bm \cdot \BB_a \\
\Delta U_E &= -d\Bp \cdot \BE_a.
\end{aligned}
\end{equation}
%
In analogy with the electric dipole moment analysis, it can be assumed that there is a linear relationship between the magnetic polarization and the applied magnetic field
%
\begin{dmath}\label{eqn:emtLecture4:100}
\BB = \mu_0 \BH_a + \mu_0 \BM = \mu_0\lr{ \BH_a + \BM },
\end{dmath}
%
where
\begin{dmath}\label{eqn:emtLecture4:120}
\BM = \chi_m \BH_a,
\end{dmath}
%
so
\begin{equation}\label{eqn:emtLecture4:140}
\BB
= \mu_0\lr{ 1 + \chi_m } \BH_a
\equiv \mu \BH_a.
\end{equation}
%
Like electric dipoles, in a volume, we can have bound currents on the surface [\si{A/m}], as well as bound volume currents [\si{A/m^2}].
% as sketched in
%
%F3
%
It can be shown, as with the electric dipoles related bound charge densities of \cref{eqn:emtLecture3:620}, that magnetic currents can be defined
%
\begin{equation}\label{eqn:emtLecture4:160}
\begin{aligned}
\BJ_{sm} &= \BM \cross \ncap, \\
\BJ_{vm} &= \spacegrad \cross \BM.
\end{aligned}
\end{equation}
%
\section{Conductivity.}
\index{conductivity}
%
\index{constitutive relationships}
We have two constitutive relationships so far
\begin{equation}\label{eqn:emtLecture4:180}
\begin{aligned}
\BD &= \epsilon \BE,\\
\BB &= \mu \BH,
\end{aligned}
\end{equation}
%
but these need to be augmented by Ohm's law
%
\index{Ohm's law}
\begin{dmath}\label{eqn:emtLecture4:200}
\BJ_c = \epsilon \BE.
\end{dmath}
%
There are a couple ways to discuss this.  One is to model \( \epsilon \) as a complex number.  Such a model is not entirely unconstrained.  Like with the Cauchy-Riemann conditions that relate derivatives of the real and imaginary parts of a complex number, there is a relationship (Kramers-Kronig \citep{wiki:kramersKronig}), an integral relationship that relates the real and imaginary parts of the permittivity \( \epsilon \).
%

      \section{Problems.}
      \input{magneticMomentJackson.tex}
      \input{vectorAreaGriffiths.tex}
      \input{Set3Problem1.tex}
      \input{Set3Problem2.tex}
      %
% Copyright � 2016 Peeter Joot.  All Rights Reserved.
% Licenced as described in the file LICENSE under the root directory of this GIT repository.
%
\makeproblem{Electric field across dielectric boundary.}{emt:problemSet3:3}{
\index{boundary conditions!electric field}
\index{dielectric!boundary conditions}
The plane \( 3x + 2y + z = 12 \) [\si{m}] describes the interface between a dielectric and free space.
The origin side of the interface has \( \epsilon_{r 1} = 3 \) and \( \BE_1 = 2 \xcap + 5 \zcap \) [\si{V/m}]. What is \(\BE_2\)
(the field on the other side of the interface)?
} % makeproblem
%
\skipIfRedacted{
\makeanswer{emt:problemSet3:3}{
The geometry of the problem is sketched roughly in \cref{fig:ps3Problem3Plane:ps3Problem3PlaneFig1}.
\imageFigure{../figures/ece1228-electromagnetic-theory/ps3Problem3PlaneFig1}{Interfaces on sides of a plane.}{fig:ps3Problem3Plane:ps3Problem3PlaneFig1}{0.3}
Assuming that there are no sources, the relationships between the fields on each side of the interface are
\begin{equation}\label{eqn:emtProblemSet3Problem3:20}
\begin{aligned}
%\ncap \cdot \lr{ \epsilon_0 \epsilon_{r2} \BE_2 - \epsilon_0 \epsilon_{r1} \BE_1 } &= 0
\ncap \cdot \lr{ \BD_2 - \BD_1 } &= 0,\\
\ncap \cross \lr{ \BE_2 - \BE_1 } &= 0.
\end{aligned}
\end{equation}
%
%Since \( \epsilon_{r2} = 1 \),
After cancelling common factors of \( \epsilon_0 \) the first relationship can be written
\begin{equation}\label{eqn:emtProblemSet3Problem3:40}
\ncap \cdot \BE_2 = \frac{\epsilon_{r1}}{\epsilon_{r2}} \ncap \cdot \BE_1.
\end{equation}
Adding the normal and the tangential components of \( \BE_2 \), we have
\begin{equation}\label{eqn:emtProblemSet3Problem3:60}
\begin{aligned}
\BE_2
&=
\ncap \lr{ \ncap \cdot \BE_2 } -
\ncap \cross \lr{ \ncap \cross \BE_2 }
\\ &=
\frac{\epsilon_{r1}}{\epsilon_{r2}} \ncap \lr{ \ncap \cdot \BE_1 }
- \ncap \cross \lr{ \ncap \cross \BE_1 }.
\end{aligned}
\end{equation}
By expanding the tangential projection (normal rejection) of a vector as
\begin{equation}\label{eqn:emtProblemSet3Problem3:160}
\begin{aligned}
\BA_t
&=
- \ncap \cross \lr{ \ncap \cross \BA }
\\ &=
\BA - \ncap (\ncap \cdot \BA),
\end{aligned}
\end{equation}
we find
\begin{equation}\label{eqn:emtProblemSet3Problem3:180}
\begin{aligned}
\BE_2
&=
\frac{\epsilon_{r1}}{\epsilon_{r2}} \ncap \lr{ \ncap \cdot \BE_1 }
+ \lr{ \BE_1 - \ncap (\ncap \cdot \BE_1) }
\\ &=
\BE_1 + \lr{\frac{\epsilon_{r1}}{\epsilon_{r2}} -1} \ncap \lr{ \ncap \cdot \BE_1 },
\end{aligned}
\end{equation}
or
\boxedEquation{eqn:emtProblemSet3Problem3:320}{
\BE_2
=
\BE_1 + \frac{\frac{\epsilon_{r1}}{\epsilon_{r2}} -1}{\Abs{\Bn}^2} \Bn \lr{ \Bn \cdot \BE_1 }.
}
The rest of the problem is routine algebra.
\begin{subequations}
\label{eqn:emtProblemSet3Problem3:220}
\begin{equation}\label{eqn:emtProblemSet3Problem3:240}
\begin{aligned}
\Bn^2 &= (3,2,1) \cdot (3,2,1) \\ &= 9 + 4 + 1 \\ &= 14,
\end{aligned}
\end{equation}
\begin{equation}\label{eqn:emtProblemSet3Problem3:260}
\begin{aligned}
\Bn \cdot \BE_1
&=
(3,2,1) \cdot (2,0,5)
\\ &=
6+5
\\ &=11,
\end{aligned}
\end{equation}
\end{subequations}
so
\begin{equation}\label{eqn:emtProblemSet3Problem3:280}
\begin{aligned}
\BE_2
&=
(2,0,5) + \frac{2 \times 11}{14} (3,2,1)
\\ &=
\inv{7}( (14,0,35) + (33,22,11) ),
\end{aligned}
\end{equation}
which is
\boxedEquation{eqn:emtProblemSet3Problem3:300}{
\BE_2
=
\frac{47}{7} \xcap + \frac{22}{7} \ycap + \frac{46}{7} \zcap
\qquad[\si{V/m}].
}
}}

      \input{Set3Problem4.tex}
      %\input{emtProblemSet3Appendix.tex}
      \input{Set4Problem4.tex}
      \input{magneticFieldFromMoment.tex}
\mychapter{Poynting vector, and time harmonic (phasor) fields.}
   %
% Copyright � 2016 Peeter Joot.  All Rights Reserved.
% Licenced as described in the file LICENSE under the root directory of this GIT repository.
%
%\input{../blogpost.tex}
%\renewcommand{\basename}{emt5}
%\renewcommand{\dirname}{notes/ece1228/}
%\newcommand{\keywords}{ECE1228H}
%\input{../latex/peeter_prologue_print2.tex}
%
%%\usepackage{ece1228}
%\usepackage{peeters_braket}
%%\usepackage{peeters_layout_exercise}
%\usepackage{peeters_figures}
%\usepackage{macros_cal}
%\usepackage{macros_bm}
%\usepackage{mathtools}
%\usepackage{siunitx}
%
%\beginArtNoToc
%\generatetitle{ECE1228H Electromagnetic Theory.  Lecture 5: Poynting vector.  Taught by Prof.\ M. Mojahedi}
%\mychapter{Poynting vector, and time harmonic (phasor) fields.}
\label{chap:emt5}
%
%\section{Poynting.}
\index{Poynting vector}
\index{Poynting theorem}
%
The cross product terms of Maxwell's equation are
\begin{equation}\label{eqn:emtLecture5:120}
\spacegrad \cross \BE
= -\BM_i - \PD{t}{\BB}
= -\BM_i - \BM_d,
\end{equation}
%
where \(\BM_d\) is called the magnetic displacement current here.  For the magnetic curl we have
%
\begin{equation}\label{eqn:emtLecture5:140}
\spacegrad \cross \BH
= \BJ_i + \BJ_c + \PD{t}{\BD}
= \BJ_i + \BJ_c + \BJ_d.
\end{equation}
%
It is left as an exercise to show that
%
\begin{equation}\label{eqn:emtLecture5:160}
\spacegrad \cdot \lr{ \BE \cross \BH } + \BH \cdot \lr{ \BM_i + \BM_d }  + \BE \cdot \lr{ \BJ_i + \BJ_c + \BJ_d } = 0,
\end{equation}
%
or
\begin{equation}\label{eqn:emtLecture5:180}
\begin{aligned}
\oint &d\Ba \cdot \lr{ \BE \cross \BH } + \int dV \lr{ \BH \cdot \lr{ \BM_i + \BM_d }  + \BE \cdot \lr{ \BJ_i + \BJ_c + \BJ_d }} \\
&= 0,
\end{aligned}
\end{equation}
%
or
\begin{equation}\label{eqn:emtLecture5:200}
\begin{aligned}
   0 &=
   \oint d\Ba \cdot \lr{ \BE \cross \BH } \\
   &+ \int dV \BH \cdot \BM_i
+ \int dV \BE \cdot \BJ_i
+ \int dV \BE \cdot \BJ_c \\
&+ \int dV \lr{ \BH \cdot \PD{t}{\BB} + \BE \cdot \PD{t}{\BD} }.
\end{aligned}
\end{equation}
%
\index{supplied power density}
Define a supplied power density \( \rho_{\textrm{supp}} \)
%
\begin{equation}\label{eqn:emtLecture5:220}
-\rho_{\textrm{supp}}
=
 \int dV \BH \cdot \BM_i
+ \int dV \BE \cdot \BJ_i.
\end{equation}
%
When the medium is not dispersive or lossy, we have
%
\begin{equation}\label{eqn:emtLecture5:240}
\begin{aligned}
\int dV \BH \cdot \PD{t}{\BB}
&=
\mu \int dV \BH \cdot \PD{t}{\BH}
\\ &=
\PD{t}{} \int dV \mu \Abs{\BH}^2.
\end{aligned}
\end{equation}
%
\index{magnetic energy density}
% FIXME: Is this supposed to be Wb (webers)?  chatgpt says \mu_0 H^2 has dimensions of Pa (scals).
The units of \( [\mu \Abs{\BH}^2] \) are \si{W}, so one can defined a magnetic energy density \( \mu \Abs{\BH}^2 \), and
%
\begin{equation}\label{eqn:emtLecture5:260}
W_m =
\int dV \mu \Abs{\BH}^2,
\end{equation}
%
for
%
\begin{equation}\label{eqn:emtLecture5:280}
\int dV \BH \cdot \PD{t}{\BB}
=
\PD{t}{W_m}.
\end{equation}
%
\index{stored magnetic energy}
This is the rate of change of stored magnetic energy [\si{J/s} = \si{W}].
%
Similarly
\begin{equation}\label{eqn:emtLecture5:300}
\begin{aligned}
\int dV \BE \cdot \PD{t}{\BD}
&=
\epsilon
\int dV \BE \cdot \PD{t}{\BE}
\\ &=
\PD{t}{} \int dV \epsilon \Abs{\BE}^2.
\end{aligned}
\end{equation}
%
\index{electric energy density}
The electric energy density is \( \epsilon \Abs{\BE}^2 \).  Let
%
\begin{equation}\label{eqn:emtLecture5:320}
W_e =
\int dV \epsilon \Abs{\BE}^2,
\end{equation}
%
and
\begin{equation}\label{eqn:emtLecture5:340}
\int dV \BE \cdot \PD{t}{\BD}
=
\PD{t}{W_e}.
\end{equation}
%
We also have a term
%
\begin{equation}\label{eqn:emtLecture5:360}
\begin{aligned}
\int dV \BE \cdot \BJ_c
&=
\int dV \BE \cdot (\sigma \BE)
\\ &=
\int dV \sigma \Abs{\BE}^2.
%\equiv ...
\end{aligned}
\end{equation}
%
\index{stored electric energy}
This is the rate of change of stored electric energy.
%
The remaining term is
\begin{equation}\label{eqn:emtLecture5:380}
\oint d\Ba \cdot \lr{ \BE \cross \BH }.
\end{equation}
%
This is a density of the power that is leaving the volume.  The vector \( \BE \cross \BH \) is special, called the Poynting vector, and coincidentally points in the direction that the energy leaves the bounding surface per unit time.  We write
%
\begin{equation}\label{eqn:emtLecture5:400}
\BS = \BE \cross \BH.
\end{equation}
%
In vacuum the phase velocity \( \Bv_p \), group velocity \( \Bv_g \) and packet(?) velocity \( \Bv_p \) all line up.  This isn't the case in the media.
%
It turns out that without dissipation
%
\begin{equation}\label{eqn:emtLecture5:420}
\int \BH \cdot \PD{t}{\BB} = \int \BE \cdot \PD{t}{\BD}.
\end{equation}
%
\index{LC circuit}
For example in an LC circuit \cref{fig:lecture4LCCircuit:lecture4LCCircuitFig1}
half the cycle the energy is stored in the inductor, and in the other half of the cycle the energy is stored in the capacitor.
%
\imageFigure{../figures/ece1228-electromagnetic-theory/lecture4LCCircuitFig1}{LC circuit.}{fig:lecture4LCCircuit:lecture4LCCircuitFig1}{0.2}
%
Summarizing
%
\begin{equation}\label{eqn:emtLecture5:440}
\oint \lr{ \BE \cross \BH } \cdot d\Ba = P_{\textrm{exit}}.
\end{equation}
%

      \section{Problems.}
      \input{Set4Problem1.tex}
      \input{Set4Problem2.tex}
      \input{Set4Problem3.tex}
      %\input{poynting.tex}
      \input{Set7Problem1.tex}
      \input{poyntingTimeHarmonic.tex}
\mychapter{Lorentz-Lorenz dispersion.}
   \input{emtLecture6.tex}
      \section{Problems.}
      \input{Set5Problem1.tex}
      \input{Set5Problem2.tex}
      \input{Set5Problem3.tex}
\mychapter{Druid model.}
      \input{druid.tex}
%      FIXME: transcribe handwritten notes that were mostly skipped over in class?
%      \section{Problems.}
%\mychapter{conductivity} % transcribe?  This was part of L7
\mychapter{Wave equation.}
   \input{emtLecture7.tex}
      \section{Problems.}
         \input{Set6Problem1.tex}
   \mychapter{Wave equation solutions.}
In class, we walked through splitting up the wave equation into components, and separation of variables.  I didn't take notes on that.

Winding down that discussion, however, was a mention of phase and group velocity, and a phenomena called superluminal velocity.  This latter is analogous to quantum electron tunnelling where a wave can make it through an aperture with a damped solution \( e^{-\alpha x} \) in the aperture interval, and sinusoidal solutions in the incident and transmitted regions as sketched in \cref{fig:L7:L7Fig1}.  The time \( \tau \) to get through the aperture is called the tunnelling time.
\imageFigure{../figures/ece1228-electromagnetic-theory/L7Fig1}{Superluminal tunnelling.}{fig:L7:L7Fig1}{0.3}
      \section{Problems.}
          \input{Set6Problem2.tex}
          \input{Set6Problem3.tex}
   \mychapter{Wave equation solutions.}
      %
% Copyright � 2016 Peeter Joot.  All Rights Reserved.
% Licenced as described in the file LICENSE under the root directory of this GIT repository.
%
%\input{../blogpost.tex}
%\renewcommand{\basename}{emt8}
%\renewcommand{\dirname}{notes/ece1228/}
%\newcommand{\keywords}{ECE1228H}
%\input{../latex/peeter_prologue_print2.tex}
%
%%\usepackage{ece1228}
%\usepackage{peeters_braket}
%%\usepackage{peeters_layout_exercise}
%\usepackage{peeters_figures}
%\usepackage{mathtools}
%\usepackage{siunitx}
%
%\beginArtNoToc
%\generatetitle{ECE1228H Electromagnetic Theory.  Lecture 8: Waves.  Taught by Prof.\ M. Mojahedi}
%%\chapter{Waves}
%\label{chap:emt8}
%
%\paragraph{Disclaimer}
%
%Peeter's lecture notes from class.  These may be incoherent and rough.
%
%These are notes for the UofT course ECE1228H, Electromagnetic Theory, taught by Prof. M. Mojahedi, covering \textchapref{{1}} \citep{balanis1989advanced} content.
%
\section{Cylindrical coordinates.}
%
Seek a function
%
\begin{equation}\label{eqn:emtLecture8:20}
\BE = E_\rho \rhocap + E_\phi \phicap + E_z \zcap,
\end{equation}
%
solving
%
\begin{equation}\label{eqn:emtLecture8:40}
\spacegrad^2 \BE = -\beta^2 \BE.
\end{equation}
%
One way to find the Laplacian in cylindrical coordinates is to use
%
\begin{equation}\label{eqn:emtLecture8:60}
\spacegrad^2 \BE =
\spacegrad \lr{ \spacegrad \cdot \BE }
-\spacegrad \cross \lr{ \spacegrad \cross \BE },
\end{equation}
%
where
%
\begin{equation}\label{eqn:emtLecture8:80}
\spacegrad = \rhocap \PD{\rho}{} + \frac{\phicap}{\rho} \PD{\phi}{} + \zcap \PD{z}{}.
\end{equation}
%
It can be shown that:
\begin{equation}\label{eqn:emtLecture8:100}
\spacegrad \cdot \BE = \inv{\rho} \PD{\rho}{} \lr{ \rho E_\rho } + \inv{\rho}\PD{\phi}{E_\phi} + \PD{z}{E_z},
\end{equation}
%
and
\begin{equation}\label{eqn:emtLecture8:120}
\begin{aligned}
\spacegrad \cross \BE
%&=
%\begin{vmatrix}
%\rhocap & \phicap & \zcap \\
%\partial_\rho & \inv{\rho}\partial_\phi & \partial_z \\
%E_\rho & \rho E_\phi & E_z
%\end{vmatrix}
\\ &=
\rhocap  \lr{ \inv{\rho} \partial_\phi E_z - \partial_z E_\phi }
+\phicap \lr{ \partial_z E_\rho - \partial_\rho E_z }
+\zcap   \lr{ \inv{\rho} \partial_\rho (\rho E_\phi) - \inv{\rho} \partial_\phi E_\rho }.
\end{aligned}
\end{equation}
%
This gives
\begin{equation}\label{eqn:emtLecture8:200}
\spacegrad^2 \psi =
\PDSq{\rho}{\psi}
+\inv{\rho} \PD{\rho}{\psi}
+\inv{\rho^2} \PDSq{\phi}{\psi}
+\PDSq{z}{\psi},
\end{equation}
and
\begin{equation}\label{eqn:emtLecture8:220}
\begin{aligned}
\spacegrad^2 E_\rho &= \lr{ -\frac{E_\rho}{\rho^2} - \frac{2}{\rho^2} \PD{\phi}{E_\phi} }, \\
\spacegrad^2 E_\phi &= \lr{ -\frac{E_\phi}{\rho^2} + \frac{2}{\rho^2} \PD{\phi}{E_\rho} }, \\
\spacegrad^2 E_z    &= -\beta^2 E_\phi.
\end{aligned}
\end{equation}
%
This is explored in \cref{chap:laplacianCylindrical}.
%
%Note that with \( i = \Be_1 \Be_2 \),
%
%\begin{equation}\label{eqn:emtLecture8:140}
%\rhocap = \Be_1 e^{i \phi}
%\end{equation}
%
%so
%\begin{equation}\label{eqn:emtLecture8:160}
%\PD{\phi}{\rhocap} = \Be_2 e^{i \phi} = \thetacap
%\end{equation}
%
%... the end result is
%
%\begin{equation}\label{eqn:emtLecture8:180}
%\end{equation}
%
\paragraph{TEM:} If we want to have a TEM mode it can be shown that we need an axial distribution mechanism, such as the core of a co-axial cable.
%
These are messy to solve in general, but we can solve the z-component without too much pain
%
\begin{equation}\label{eqn:emtLecture8:240}
\PDSq{\rho}{E_z}
+\inv{\rho} \PD{\rho}{E_z}
+\inv{\rho^2} \PDSq{\phi}{E_z}
+\PDSq{z}{E_z}
=
-\beta^2 E_z.
\end{equation}
%
Solving this using separation of variables with
%
\begin{equation}\label{eqn:emtLecture8:260}
E_z = R(\rho) P(\phi) Z(z),
\end{equation}
%
\begin{equation}\label{eqn:emtLecture8:280}
\inv{R}\lr{R'' + \inv{\rho} R'} + \inv{\rho^2 P} P'' + \frac{Z''}{Z} = -\beta^2.
\end{equation}
%
Assuming for some constant \( \beta_z \) that we have
\begin{equation}\label{eqn:emtLecture8:300}
\frac{Z''}{Z} = -\beta_z^2,
\end{equation}
%
then
%
\begin{equation}\label{eqn:emtLecture8:320}
\inv{R}\lr{\rho^2 R'' + \rho R'} + \inv{P} P'' + \rho^2 \lr{\beta^2 - \beta_z^2} = 0.
\end{equation}
%
Now assume that
\begin{equation}\label{eqn:emtLecture8:340}
\inv{P} P'' = -m^2,
\end{equation}
%
and let \( \beta^2 - \beta_z^2 = \beta_\rho^2 \), which leaves
%
\begin{equation}\label{eqn:emtLecture8:360}
\rho^2 R'' + \rho R' + \lr{ \rho^2 \beta_\rho^2 -m^2 } R = 0.
\end{equation}
%
This is the Bessel differential equation, with travelling wave solution
%
\begin{equation}\label{eqn:emtLecture8:380}
R(\rho) =
A H_m^{(1)}(\beta_\rho \rho)
+B H_m^{(2)}(\beta_\rho \rho),
\end{equation}
%
and standing wave solutions
\begin{equation}\label{eqn:emtLecture8:400}
R(\rho) =
A J_m(\beta_\rho \rho)
+B Y_m(\beta_\rho \rho).
\end{equation}
%
Here \( H_m^{(1)}, H_m^{(2)} \) are Hankel functions of the first and second kinds, and
\( J_m, Y_m \) are the Bessel functions of the first and second kinds.
%
For \( P(\phi) \)
\begin{equation}\label{eqn:emtLecture8:460}
P'' = -m^2 P.
\end{equation}
%
%
\section{Waves.}
%
\begin{itemize}
\item The field is a modification of space-time
\item Mode is a particular field configuration for a given boundary value problem.  Many field configurations can satisfy Maxwell equations (wave equation).  These usually are referred to as modes.  A mode is a self-consistent field distribution.
\item In a TEM mode, \( \BE \) and \( \BH \) are every point in space are constrained in a local plane, independent of time.  This plane is called the equiphase plane.  In general equiphase planes are not parallel at two different points along the trajectory of the wave.
%\item If equiphase planes are parallel (i.e. the space orientation of the planes for TEM mode...
%... next time.
\end{itemize}
%
%}
%\EndNoBibArticle

      \section{Problems.}
         \input{Set7Problem2.tex}
         \input{Set7Problem3.tex}
   \mychapter{Quadrupole expansion.}
      \input{emtLecture8Quad.tex}
      \input{momentCoeffiecients.tex}
      \section{Problems.}
         \input{dipoleFromSphericalMoments.tex}
   \mychapter{Fresnel relations.}
      \input{emtLecture10.tex}
      \input{twoInterfaceNormal.tex}
      \input{brewsters.tex}
      \section{Problems.}
         \input{fresnelSumAndDifferenceAngleFormulas.tex}
         \input{Set8Problem1.tex}
         \input{Set8Problem2.tex}
         \input{Set8Problem3.tex}
         \input{Set9Problem1.tex}
         %
% Copyright � 2016 Peeter Joot.  All Rights Reserved.
% Licenced as described in the file LICENSE under the root directory of this GIT repository.
%
\makeproblem{Eccostock example.}{emt:problemSet9:2}{
Use the expression for transmission function obtained above and the values and instructions below to plot the following at normal incidence:

\makesubproblem{}{emt:problemSet9:2a}
Transmission magnitude and phase as a function of frequency for the case \(N=3\).
\makesubproblem{}{emt:problemSet9:2b}
The group delay as a function of frequency for the cases \(N=1, 2, 3, 4\).
\makesubproblem{}{emt:problemSet9:2c}
The group velocity as a function of frequency for the cases \(N=1, 2, 3\).

\begin{itemize}
\item \( n_i = 1 \) (this is air), \( n_j = 3.4 - j 0.002 \) (this is Eccostock).
\item \( d_i = 1.76 \,[\si{cm}] \)
\item \( d_j = 1.33 \,[\si{cm}] \)
\item \( L_{\textrm{PC}} = (N-1)(d_i + d_j) + d_j \)
\item Frequency range for all plots: \( 20 \,[\si{GHz}] \) to \( 23 \,[\si{GHz}] \).
\item Use linear scale for transmission magnitude (not \si{dB}) and express the transmission phase in Degrees.
\item Plot the group delay in nanosecond.
\item Plot the group velocity in units of \( V_g/c \), where \(c\) is the speed of light in vacuum.
\end{itemize}

} % makeproblem

\makeanswer{emt:problemSet9:2}{

Note: These are apparently wrong.  Haven't seen the graded results.

\makeSubAnswer{}{emt:problemSet9:2a}

Plotted in \cref{fig:ps9PartaTransmissionFunction:ps9PartaTransmissionFunctionFig1}.

\imageFigure{../figures/ece1228-electromagnetic-theory/ps9PartaTransmissionFunctionFig1}{Transmission function.}{fig:ps9PartaTransmissionFunction:ps9PartaTransmissionFunctionFig1}{0.3}

\makeSubAnswer{}{emt:problemSet9:2b}

Plotted in \cref{fig:ps9PartbGroupDelay:ps9PartbGroupDelayFig2}.

\imageFigure{../figures/ece1228-electromagnetic-theory/ps9PartbGroupDelayFig2}{Group delay.}{fig:ps9PartbGroupDelay:ps9PartbGroupDelayFig2}{0.3}

\makeSubAnswer{}{emt:problemSet9:2c}

Plotted in \cref{fig:ps9PartcGroupVelocity:ps9PartcGroupVelocityFig3}.

\imageFigure{../figures/ece1228-electromagnetic-theory/ps9PartcGroupVelocityFig3}{Group velocity.}{fig:ps9PartcGroupVelocity:ps9PartcGroupVelocityFig3}{0.3}
}

   \mychapter{Gauge freedom.}
      \section{Problems.}
         \input{Set9Problem3.tex}
