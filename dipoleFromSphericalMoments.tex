%
% Copyright � 2016 Peeter Joot.  All Rights Reserved.
% Licenced as described in the file LICENSE under the root directory of this GIT repository.
%
%{
%\input{../blogpost.tex}
%\renewcommand{\basename}{dipoleFromSphericalMoments}
%%\renewcommand{\dirname}{notes/phy1520/}
%\renewcommand{\dirname}{notes/ece1228-electromagnetic-theory/}
%%\newcommand{\dateintitle}{}
%%\newcommand{\keywords}{}
%
%\input{../latex/peeter_prologue_print2.tex}
%
%\usepackage{peeters_layout_exercise}
%\usepackage{peeters_braket}
%\usepackage{peeters_figures}
%\usepackage{siunitx}
%%\usepackage{mhchem} % \ce{}
%%\usepackage{macros_bm} % \bcM
%%\usepackage{txfonts} % \ointclockwise
%
%\beginArtNoToc
%
%\generatetitle{Dipole field from multipole moment sum}
%\chapter{Dipole field from multipole moment sum}
%\label{chap:dipoleFromSphericalMoments}
%
\makeproblem{Dipole multipole moment.}{problem:dipoleFromSphericalMoments:1}{
Following Jackson \citep{jackson1975cew}, derive the electric field contribution from the dipole terms of the multipole sum, but don't skip the details.
} % problem
%
\makeanswer{problem:dipoleFromSphericalMoments:1}{
%\withproblemsetsParagraph{
The components of the electric field can be obtained directly from the multipole moments
%
\begin{equation}\label{eqn:dipoleFromSphericalMoments:20}
\Phi(\Bx)
= \inv{4 \pi \epsilon_0} \sum \frac{4 \pi}{ (2 l + 1) r^{l + 1} } q_{l m} Y_{l m},
\end{equation}
%
so for the \( l,m \) contribution to this sum the components of the electric field are
%
\begin{equation}\label{eqn:dipoleFromSphericalMoments:40}
E_r
=
\inv{\epsilon_0} \sum \frac{l+1}{ (2 l + 1) r^{l + 2} } q_{l m} Y_{l m},
\end{equation}
%
\begin{equation}\label{eqn:dipoleFromSphericalMoments:60}
E_\theta
= -\inv{\epsilon_0} \sum \frac{1}{ (2 l + 1) r^{l + 2} } q_{l m} \partial_\theta Y_{l m}
\end{equation}
%
\begin{equation}\label{eqn:dipoleFromSphericalMoments:80}
\begin{aligned}
E_\phi
&= -\inv{\epsilon_0} \sum \frac{1}{ (2 l + 1) r^{l + 2} \sin\theta } q_{l m} \partial_\phi Y_{l m}
\\ &= -\inv{\epsilon_0} \sum \frac{j m}{ (2 l + 1) r^{l + 2} \sin\theta } q_{l m} Y_{l m}.
\end{aligned}
\end{equation}
%
Here I've translated from CGS to SI.  Let's calculate the \( l = 1 \) electric field components directly from these expressions and check against the previously calculated results.
%
\begin{equation}\label{eqn:dipoleFromSphericalMoments:100}
\begin{aligned}
E_r
&=
\inv{\epsilon_0} \frac{2}{ 3 r^{3} }
\lr{
   2 \lr{ -\sqrt{\frac{3}{8\pi}} }^2 \Real \lr{
      (p_x - j p_y) \sin\theta e^{j\phi}
   }
   +
   \lr{ \sqrt{\frac{3}{4\pi}} }^2 p_z \cos\theta
}
\\ &=
\frac{2}{4 \pi \epsilon_0 r^3}
\lr{
   p_x \sin\theta \cos\phi + p_y \sin\theta \sin\phi + p_z \cos\theta
}
\\ &=
\frac{1}{4 \pi \epsilon_0 r^3} 2 \Bp \cdot \rcap.
\end{aligned}
\end{equation}
%
Note that
%
\begin{equation}\label{eqn:dipoleFromSphericalMoments:120}
\partial_\theta Y_{11} = -\sqrt{\frac{3}{8\pi}} \cos\theta e^{j \phi},
\end{equation}
%
and
%
\begin{equation}\label{eqn:dipoleFromSphericalMoments:140}
\partial_\theta Y_{1,-1} = \sqrt{\frac{3}{8\pi}} \cos\theta e^{-j \phi},
\end{equation}
%
so
%
\begin{equation}\label{eqn:dipoleFromSphericalMoments:160}
\begin{aligned}
E_\theta
&=
-\inv{\epsilon_0} \frac{1}{ 3 r^{3} }
\lr{
   2 \lr{ -\sqrt{\frac{3}{8\pi}} }^2 \Real \lr{
      (p_x - j p_y) \cos\theta e^{j\phi}
   }
   -
   \lr{ \sqrt{\frac{3}{4\pi}} }^2 p_z \sin\theta
}
\\ &=
-\frac{1}{4 \pi \epsilon_0 r^3}
\lr{
   p_x \cos\theta \cos\phi + p_y \cos\theta \sin\phi - p_z \sin\theta
}
\\ &=
-\frac{1}{4 \pi \epsilon_0 r^3} \Bp \cdot \thetacap.
\end{aligned}
\end{equation}
%
For the \(\phicap\) component, the \( m = 0 \) term is killed.  This leaves
%
\begin{equation}\label{eqn:dipoleFromSphericalMoments:180}
\begin{aligned}
E_\phi
&=
-\frac{1}{\epsilon_0} \frac{1}{ 3 r^{3} \sin\theta }
\lr{
j q_{11} Y_{11} - j q_{1,-1} Y_{1,-1}
}
\\ &=
-\frac{1}{3 \epsilon_0 r^{3} \sin\theta }
\lr{
j q_{11} Y_{11} - j (-1)^{2m} q_{11}^\conj Y_{11}^\conj
}
\\ &=
\frac{2}{\epsilon_0} \frac{1}{ 3 r^{3} \sin\theta }
\Imag q_{11} Y_{11}
\\ &=
\frac{2}{3 \epsilon_0 r^{3} \sin\theta }
\Imag \lr{
   \lr{ -\sqrt{\frac{3}{8\pi}} }^2 (p_x - j p_y) \sin\theta e^{j \phi}
}
\\ &=
\frac{1}{ 4 \pi \epsilon_0 r^{3} }
\Imag \lr{
   (p_x - j p_y) e^{j \phi}
}
\\ &=
\frac{1}{ 4 \pi \epsilon_0 r^{3} }
\lr{
   p_x \sin\phi - p_y \cos\phi
}
\\ &=
-\frac{\Bp \cdot \phicap}{ 4 \pi \epsilon_0 r^3}.
\end{aligned}
\end{equation}
%
That is
%\begin{equation}\label{eqn:dipoleFromSphericalMoments:200}
\boxedEquation{eqn:dipoleFromSphericalMoments:200}{
\begin{aligned}
E_r &=
\frac{2}{4 \pi \epsilon_0 r^3}
\Bp \cdot \rcap \\
E_\theta &= -
\frac{1}{4 \pi \epsilon_0 r^3}
\Bp \cdot \thetacap \\
E_\phi &= -
\frac{1}{4 \pi \epsilon_0 r^3}
\Bp \cdot \phicap.
\end{aligned}
}
%\end{equation}
%
These are consistent with equations (4.12) from the text for when \( \Bp \) is aligned with the z-axis.
%
Observe that we can sum each of the projections of \( \BE \) to construct the total electric field due to this \( l = 1 \) term of the multipole moment sum
%
\begin{equation}\label{eqn:dipoleFromSphericalMoments:n}
\begin{aligned}
\BE
&=
\frac{1}{4 \pi \epsilon_0 r^3}
\lr{
2 \rcap (\Bp \cdot \rcap)
-
\phicap ( \Bp \cdot \phicap)
-
\thetacap ( \Bp \cdot \thetacap)
}
\\ &=
\frac{1}{4 \pi \epsilon_0 r^3}
\lr{
3 \rcap (\Bp \cdot \rcap)
-
\Bp
},
\end{aligned}
\end{equation}
%
which recovers the expected dipole moment approximation.
%}
} % answer
%
%}
%\EndArticle
