%
% Copyright � 2016 Peeter Joot.  All Rights Reserved.
% Licenced as described in the file LICENSE under the root directory of this GIT repository.
%
\makeproblem{Passive medium.}{emt:problemSet5:1}{
Parameters for \ce{AlGaN} (a passive medium) are given as
\begin{equation}\label{eqn:emtProblemSet5Problem1:20}
\begin{aligned}
\omega_0 &= 1.921 \times 10^{14} \si{rad/s} \\
\omega_p &= 3.328 \times 10^{14} \si{rad/s} \\
\gamma   &= 9.756 \times 10^{12} \si{rad/s} \\
\end{aligned}
\end{equation}
Assuming Lorentz model:
\makesubproblem{}{emt:problemSet5:1a}
Plot the real and imaginary parts of the index of refraction for the range of \( \omega = 0 \) to \( \omega = 6 \times 10^{14} \).
On the figure identify the region of anomalous dispersion.
\makesubproblem{}{emt:problemSet5:1b}
Plot the real and imaginary parts of the relative permittivity for the same range as in \partref{emt:problemSet5:1a}.

On the figure identify the region of anomalous dispersion.
} % makeproblem
\makeanswer{emt:problemSet5:1}{\withproblemsetsParagraph{
\makeSubAnswer{}{emt:problemSet5:1a}
Given the relative permittivity
\begin{equation}\label{eqn:emtProblemSet5Problem1:60}
\epsilon_r = 1 + \chi_e =
1 + \frac{ \omega_{p} }{\omega_{0}^2 - \omega^2 + j \gamma \omega },
\end{equation}
the index of refraction, for \( \mu \approx \mu_0 \), is
\begin{dmath}\label{eqn:emtProblemSet5Problem1:40}
n
= \frac{c}{v}
= \frac{\sqrt{\epsilon_0 \epsilon_r \mu}}{\sqrt{\epsilon_0 \mu_0}}
\approx \sqrt{\epsilon_r}.
\end{dmath}
This is plotted in \cref{fig:p1IndexOfRefraction:p1IndexOfRefractionFig1}.
To be able to flag the portions of the plots that are regions of anomalous dispersion, we have to know what that is.  According to \citep{wiki:opticalDispersion} these are regions where the real part of the index of refraction increases as \( \lambda = 2 \pi/\omega \) increases, or as \( \omega \) decreases.  That is, the portions of the curve where \( d\Real{n}/d\omega < 0 \).  Those portions of the curves are indicated in the plots with a boxed region.
%The plotting code is attached
\mathImageFigure{../figures/ece1228-electromagnetic-theory/p1IndexOfRefractionFig1}{Index of refraction.}{fig:p1IndexOfRefraction:p1IndexOfRefractionFig1}{0.4}{ps5:ps5.nb}
\makeSubAnswer{}{emt:problemSet5:1b}
%\cref{fig:p1EpsilonR:p1EpsilonRFig2}.
\mathImageFigure{../figures/ece1228-electromagnetic-theory/p1EpsilonRFig2}{Relative permittivity.}{fig:p1EpsilonR:p1EpsilonRFig2}{0.4}{ps5:ps5.nb}
}}
