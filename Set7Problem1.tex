%
% Copyright � 2016 Peeter Joot.  All Rights Reserved.
% Licenced as described in the file LICENSE under the root directory of this GIT repository.
%
\makeproblem{Poynting theorem.}{emt:problemSet7:1}{
Using Maxwell's equations given in the class notes, derive the Poynting theorem in both
differential and integral form for instantaneous fields. Assume a linear, homogeneous medium
with no temporal dispersion.
} % makeproblem
%
\skipIfRedacted{
\makeanswer{emt:problemSet7:1}{
%
Given
\begin{equation}\label{eqn:emtproblemSet7Problem1:20}
\spacegrad \cross \BE
= -\BM_i - \PD{t}{\BB},
\end{equation}
%
and
\begin{equation}\label{eqn:emtproblemSet7Problem1:40}
\spacegrad \cross \BH
= \BJ_i + \BJ_c + \PD{t}{\BD},
\end{equation}
%
we want to expand the divergence of \( \BE \cross \BH \) to find the form of the Poynting theorem.
%
%
First we need the chain rule for of this sort of divergence.  Using primes to indicate the scope of the gradient operation
%
\begin{equation}\label{eqn:emtproblemSet7Problem1:60}
\begin{aligned}
\spacegrad \cdot \lr{ \BE \cross \BH }
&=
\spacegrad' \cdot \lr{ \BE' \cross \BH }
-
\spacegrad' \cdot \lr{ \BH' \cross \BE }
\\ &=
\BH \cdot \lr{ \spacegrad' \cross \BE' }
-
\BH \cdot \lr{ \spacegrad' \cross \BH' }
%\\ &= %\gpgradezero{ %\spacegrad \lr{ \BE \cross \BH }
%}
%\\ &=
%\gpgradezero{
%-I \spacegrad \lr{ \BE \wedge \BH }
%}
%\\ &=
%-I \spacegrad \wedge \lr{ \BE \wedge \BH }
%\\ &=
%-I \BH \wedge \lr{ \spacegrad \wedge \BE }
%+I \BE \wedge \lr{ \spacegrad \wedge \BH }
\\ &=
\BH \cdot \lr{ \spacegrad \cross \BE }
-
\BE \cdot \lr{ \spacegrad \cross \BH }.
\end{aligned}
\end{equation}
%
In the second step, cyclic permutation of the triple product was used.
This checks against the inside front cover of Jackson \citep{jackson1975cew}.  Now we can plug in the Maxwell equation cross products.
%
\begin{equation}\label{eqn:emtproblemSet7Problem1:80}
\begin{aligned}
\spacegrad \cdot \lr{ \BE \cross \BH }
&=
\BH \cdot \lr{ -\BM_i - \PD{t}{\BB} }
-
\BE \cdot \lr{ \BJ_i + \BJ_c + \PD{t}{\BD} }
\\ &=
-\BH \cdot \BM_i
-\mu \BH \cdot \PD{t}{\BH}
-
\BE \cdot \BJ_i
-
\BE \cdot \BJ_c
-
\epsilon \BE \cdot \PD{t}{\BE},
\end{aligned}
\end{equation}
%
or
%
%\begin{equation}\label{eqn:emtproblemSet7Problem1:100}
\boxedEquation{eqn:emtproblemSet7Problem1:120}{
0
=
\spacegrad \cdot \lr{ \BE \cross \BH }
+ \frac{\epsilon}{2} \PD{t}{} \Abs{ \BE }^2
+ \frac{\mu}{2} \PD{t}{} \Abs{ \BH }^2
+ \BH \cdot \BM_i
+ \BE \cdot \BJ_i
+ \sigma \Abs{\BE}^2.
}
%\end{equation}
%
In integral form this is
%
\begin{equation}\label{eqn:emtproblemSet7Problem1:140}
\begin{aligned}
0
&=
\int d\BA \cdot \lr{ \BE \cross \BH }
+ \inv{2} \PD{t}{} \int dV \lr{
\epsilon \Abs{ \BE }^2
+ \mu \Abs{ \BH }^2
} 
\\&\quad
+ \int dV \BH \cdot \BM_i
+ \int dV \BE \cdot \BJ_i
+ \sigma \int dV \Abs{\BE}^2.
\end{aligned}
\end{equation}
}}
