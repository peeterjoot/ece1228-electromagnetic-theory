%
% Copyright � 2016 Peeter Joot.  All Rights Reserved.
% Licenced as described in the file LICENSE under the root directory of this GIT repository.
%
%{
%\input{../blogpost.tex}
%\renewcommand{\basename}{vectorAreaGriffiths}
%%\renewcommand{\dirname}{notes/phy1520/}
%\renewcommand{\dirname}{notes/ece1228-electromagnetic-theory/}
%%\newcommand{\dateintitle}{}
%%\newcommand{\keywords}{}
%
%\input{../latex/peeter_prologue_print2.tex}
%
%\usepackage{peeters_layout_exercise}
%\usepackage{peeters_braket}
%\usepackage{peeters_figures}
%\usepackage{siunitx}
%\usepackage{macros_qed}
%\usepackage{txfonts} % \ointclockwise
%
%\beginArtNoToc
%
%\generatetitle{Vector Area}
%%\chapter{Vector Area}
%
%One of the results of this problem is required for a later one on magnetic moments that I'd like to do.
%
\makeoproblem{Vector Area.}{problem:vectorAreaGriffiths:1}{\citep{griffiths1999introduction} pr. 1.61}{
%
The integral
%
\begin{dmath}\label{eqn:vectorAreaGriffiths:20}
\Ba = \int_S d\Ba,
\end{dmath}
%
is sometimes called the vector area of the surface \( S \).
%
\makesubproblem{}{problem:vectorAreaGriffiths:1:a}
%
Find the vector area of a hemispherical bowl of radius \( R \).
\makesubproblem{}{problem:vectorAreaGriffiths:1:b}
%
Show that \( \Ba = 0 \) for any closed surface.
\makesubproblem{}{problem:vectorAreaGriffiths:1:c}
Show that \( \Ba \) is the same for all surfaces sharing the same boundary.
%
\makesubproblem{}{problem:vectorAreaGriffiths:1:d}
%
Show that
\begin{equation}\label{eqn:vectorAreaGriffiths:40}
\Ba = \inv{2} \ointctrclockwise \Br \cross d\Bl,
\end{equation}
%
where the integral is around the boundary line.
%
\makesubproblem{}{problem:vectorAreaGriffiths:1:e}
%
Show that
\begin{equation}\label{eqn:vectorAreaGriffiths:60}
\ointctrclockwise \lr{ \Bc \cdot \Br } d\Bl = \Ba \cross \Bc.
\end{equation}
} % problem
%
\makeanswer{problem:vectorAreaGriffiths:1}{
%\withproblemsetsParagraph{
\makeSubAnswer{}{problem:vectorAreaGriffiths:1:a}
%
\begin{dmath}\label{eqn:vectorAreaGriffiths:80}
\Ba
=
\int_{0}^{\pi/2} R^2 \sin\theta d\theta \int_0^{2\pi} d\phi
\lr{ \sin\theta \cos\phi, \sin\theta \sin\phi, \cos\theta }
=
R^2 \int_{0}^{\pi/2} d\theta \int_0^{2\pi} d\phi
\lr{ \sin^2\theta \cos\phi, \sin^2\theta \sin\phi, \sin\theta\cos\theta }
=
2 \pi R^2 \int_{0}^{\pi/2} d\theta \Be_3
\sin\theta\cos\theta
=
\pi R^2
\Be_3
\int_{0}^{\pi/2} d\theta
\sin(2 \theta)
=
\pi R^2
\Be_3
\evalrange{\lr{\frac{-\cos(2 \theta)}{2}}}{0}{\pi/2}
=
\pi R^2
\Be_3
\lr{ 1 - (-1) }/2
=
\pi R^2
\Be_3.
\end{dmath}
%
\makeSubAnswer{}{problem:vectorAreaGriffiths:1:b}
%
As hinted in the original problem description, this follows from
%
\begin{dmath}\label{eqn:vectorAreaGriffiths:100}
\int dV \spacegrad T = \oint T d\Ba,
\end{dmath}
%
simply by setting \( T = 1 \).
%
\makeSubAnswer{}{problem:vectorAreaGriffiths:1:c}
%
Suppose that two surfaces sharing a boundary are parameterized by vectors \( \Bx(u, v), \Bx(a,b) \) respectively.  The area integral with the first parameterization is
%
\begin{dmath}\label{eqn:vectorAreaGriffiths:120}
\Ba
= \int \PD{u}{\Bx} \cross \PD{v}{\Bx} du dv
= \epsilon_{ijk} \Be_i \int \PD{u}{x_j} \PD{v}{x_k} du dv
=
\epsilon_{ijk} \Be_i \int
\lr{
\PD{a}{x_j}
\PD{u}{a}
+
\PD{b}{x_j}
\PD{u}{b}
}
\lr{
\PD{a}{x_k}
\PD{v}{a}
+
\PD{b}{x_k}
\PD{v}{b}
}
du dv
=
\epsilon_{ijk} \Be_i \int
du dv
\lr{
\PD{a}{x_j}
\PD{u}{a}
\PD{a}{x_k}
\PD{v}{a}
+
\PD{b}{x_j}
\PD{u}{b}
\PD{b}{x_k}
\PD{v}{b}
+
\PD{b}{x_j}
\PD{u}{b}
\PD{a}{x_k}
\PD{v}{a}
+
\PD{a}{x_j}
\PD{u}{a}
\PD{b}{x_k}
\PD{v}{b}
}
=
\epsilon_{ijk} \Be_i \int
du dv
\lr{
\PD{a}{x_j}
\PD{a}{x_k}
\PD{u}{a}
\PD{v}{a}
+
\PD{b}{x_j}
\PD{b}{x_k}
\PD{u}{b}
\PD{v}{b}
}
+
\epsilon_{ijk} \Be_i \int
du dv
\lr{
\PD{b}{x_j}
\PD{a}{x_k}
\PD{u}{b}
\PD{v}{a}
-
\PD{a}{x_k}
\PD{b}{x_j}
\PD{u}{a}
\PD{v}{b}
}.
\end{dmath}
%
In the last step a \( j,k \) index swap was performed for the last term of the second integral.  The first integral is zero, since the integrand is symmetric in \( j,k \).  This leaves
\begin{dmath}\label{eqn:vectorAreaGriffiths:140}
\Ba
=
\epsilon_{ijk} \Be_i \int
du dv
\lr{
\PD{b}{x_j}
\PD{a}{x_k}
\PD{u}{b}
\PD{v}{a}
-
\PD{a}{x_k}
\PD{b}{x_j}
\PD{u}{a}
\PD{v}{b}
}
=
\epsilon_{ijk} \Be_i \int
\PD{b}{x_j}
\PD{a}{x_k}
\lr{
\PD{u}{b}
\PD{v}{a}
-
\PD{u}{a}
\PD{v}{b}
}
du dv
=
\epsilon_{ijk} \Be_i \int
\PD{b}{x_j}
\PD{a}{x_k}
\frac{\partial(b,a)}{\partial(u,v)} du dv
=
-\int
\PD{b}{\Bx} \cross \PD{a}{\Bx} da db
=
\int
\PD{a}{\Bx} \cross \PD{b}{\Bx} da db.
\end{dmath}
%
However, this is the area integral with the second parameterization, proving that the area-integral for any given boundary is independent of the surface.
%
\makeSubAnswer{}{problem:vectorAreaGriffiths:1:d}
%
Having proven that the area-integral for a given boundary is independent of the surface that it is evaluated on, the result follows by illustration as hinted in the full problem description.  Draw a ``cone'', tracing a vector \( \Bx' \) from the origin to the position line element, and divide that cone up into infinitesimal slices as sketched in \cref{fig:coneVectorArea:coneVectorAreaFig1}.
%
\imageFigure{../figures/ece1228-electromagnetic-theory/coneVectorAreaFig1}{Cone configuration.}{fig:coneVectorArea:coneVectorAreaFig1}{0.2}
%
The area of each of these triangular slices is
%
\begin{dmath}\label{eqn:vectorAreaGriffiths:160}
\inv{2} \Bx' \cross d\Bl'.
\end{dmath}
%
Summing those triangles proves the result.
%
\makeSubAnswer{}{problem:vectorAreaGriffiths:1:e}
%
As hinted in the problem, this follows from
%
\begin{dmath}\label{eqn:vectorAreaGriffiths:180}
\int \spacegrad T \cross d\Ba = -\ointctrclockwise T d\Bl.
\end{dmath}
%
Set \( T = \Bc \cdot \Br \), for which
%
\begin{dmath}\label{eqn:vectorAreaGriffiths:240}
\spacegrad T
= \Be_k \partial_k c_m x_m
= \Be_k c_m \delta_{km}
= \Be_k c_k
= \Bc,
\end{dmath}
%
so
\begin{dmath}\label{eqn:vectorAreaGriffiths:200}
(\spacegrad T) \cross d\Ba
=
\int \Bc \cross d\Ba
=
\Bc \cross \int d\Ba
=
\Bc \cross \Ba.
\end{dmath}
%
so
\begin{dmath}\label{eqn:vectorAreaGriffiths:220}
\Bc \cross \Ba = -\ointctrclockwise (\Bc \cdot \Br) d\Bl,
\end{dmath}
%
or
\begin{dmath}\label{eqn:vectorAreaGriffiths:260}
\ointctrclockwise (\Bc \cdot \Br) d\Bl
=
\Ba \cross \Bc. \qedmarker
\end{dmath}
%}
} % answer
%
%}
%\EndArticle
