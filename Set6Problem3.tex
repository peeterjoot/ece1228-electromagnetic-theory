%
% Copyright © 2016 Peeter Joot.  All Rights Reserved.
% Licenced as described in the file LICENSE under the root directory of this GIT repository.
%
\makeproblem{Uniform plane wave.}{emt:problemSet6:3}{
\paragraph{Note:} This seemed like a separate problem, and has been split out from the problem 2 as specified in the original problem set handout.
The uniform plane wave
\begin{dmath}\label{eqn:emtproblemSet6Problem3:60}
\bcE(\Br, t) = E_0
\lr{ \xcap \cos\theta - \zcap \sin\theta } \cos\lr{ \omega t -k \sin\theta x - k \cos\theta z }
\end{dmath}
is propagating in the \(x-z\) plane as sketched in \cref{fig:emtProblemSet6:emtProblemSet6Fig1}
in a simple medium with \( \sigma = 0\).
\imageFigure{../figures/ece1228-electromagnetic-theory/emtProblemSet6Fig1}{Linear wave front.}{fig:emtProblemSet6:emtProblemSet6Fig1}{0.3}
Here, \( E_0 \) is a real constant and \( k \) is the propagation
constant. Answer the following questions and show all your
work.
\makesubproblem{}{emt:problemSet6:3a}
Determine the associated magnetic field \( \BH(\Br, t) \).
\makesubproblem{}{emt:problemSet6:3b}
Determine the time averaged Poynting vector, \( \expectation{\BS(\Br, t)} \).
\makesubproblem{}{emt:problemSet6:3c}
Determine the stored magnetic energy density, \( W_m(\Br, t) \).
\makesubproblem{}{emt:problemSet6:3d}
Determine the components of phase velocity vector \( \Bv_p \) along x and z.
} % makeproblem
\makeanswer{emt:problemSet6:3}{\withproblemsetsParagraph{
The wave equation for \( \bcE \) is
\begin{dmath}\label{eqn:emtproblemSet6Problem3:80}
\spacegrad^2 \bcE  = \mu \epsilon \PDSq{t}{\bcE} + \mu \sigma \PD{t}{\bcE}.
\end{dmath}
With \( \sigma = 0 \) and \( E_0 \) real, the permittivity \( \epsilon \) must also be real.  In the frequency domain, this means that the waves are governed by the equations
\begin{equation}\label{eqn:emtproblemSet6Problem3:100}
\begin{aligned}
\spacegrad^2 \BE &= - \omega^2 \mu \epsilon \BE \\
\spacegrad^2 \BH &= - \omega^2 \mu \epsilon \BH \\
\spacegrad \cross \BE &= -j \omega \mu \BH \\
\spacegrad \cross \BH &= j \omega \epsilon \BE \\
\spacegrad \cdot \BH &= 0 \\
\spacegrad \cdot \BE &= 0,
\end{aligned}
\end{equation}
where
\begin{equation}\label{eqn:emtproblemSet6Problem3:120}
\begin{aligned}
\BE &= E_0 \lr{ \xcap \cos\theta - \zcap \sin\theta } e^{-j \Bk \cdot \Br} \\
\Bk &= k\lr{ \xcap \sin\theta + \zcap \cos\theta }.
\end{aligned}
\end{equation}
We also require that
\begin{dmath}\label{eqn:emtproblemSet6Problem3:140}
(-j \Bk)^2 = -\omega^2 \mu \epsilon,
\end{dmath}
or
\begin{dmath}\label{eqn:emtproblemSet6Problem3:160}
\frac{\omega}{k} = \inv{\sqrt{\mu \epsilon}}.
\end{dmath}
\makeSubAnswer{}{emt:problemSet6:3a}
In the frequency domain, the magnetic field can be obtained directly from the electric field
\begin{dmath}\label{eqn:emtproblemSet6Problem3:180}
\BH
= \inv{-j \omega \mu} \spacegrad \cross \BE,
=
\frac{E_0}{-j \omega \mu}
\begin{vmatrix}
\xcap & \ycap & \zcap \\
\partial_x & \partial_y & \partial_z \\
\cos\theta
e^{-j \Bk \cdot \Br} & 0 &
- \sin\theta
e^{-j \Bk \cdot \Br}
\end{vmatrix}
=
-\frac{E_0 \ycap}{-j \omega \mu}
\lr{
-\sin\theta \partial_x
-\cos\theta \partial_z
}
e^{-j \Bk \cdot \Br}
=
-\frac{-j k E_0 \ycap}{j \omega \mu}
\lr{
\sin\theta \sin\theta +
\cos\theta \cos\theta
}
e^{-j \Bk \cdot \Br}
=
\frac{k E_0 \ycap}{\omega \mu}
e^{-j \Bk \cdot \Br}
=
\frac{\sqrt{\mu \epsilon}E_0 \ycap}{\mu}
e^{-j \Bk \cdot \Br},
\end{dmath}
or
\boxedEquation{eqn:emtproblemSet6Problem3:200}{
\bcH(\Br, t) = \sqrt{\frac{\epsilon}{\mu}} E_0 \ycap \cos\lr{ \omega t - \Bk \cdot \Br }.
}
\makeSubAnswer{}{emt:problemSet6:3b}
The instantaneous Poynting vector is
\begin{dmath}\label{eqn:emtproblemSet6Problem3:220}
\bcS
= \bcE \cross \bcH
= \inv{4}
\lr{ \BE e^{j \omega t} + \BE^\conj e^{-j\omega t} }
\cross
\lr{ \BH e^{j \omega t} + \BH^\conj e^{-j\omega t} }
=
\inv{4}
\lr{
\BE \cross \BH^\conj + \BH \cross \BE^\conj + \BE \cross \BH e^{2 j \omega t} + \BE^\conj \cross \BH^\conj e^{-2 j \omega t}
}
=
\inv{2}
\Real \lr{
\BE \cross \BH^\conj + \BE \cross \BH e^{2 j \omega t}
},
\end{dmath}
The average is
\begin{dmath}\label{eqn:emtproblemSet6Problem3:240}
\expectation{\bcS}
=
\inv{2}
\inv{T} \int_0^T
\Real \lr{
\BE \cross \BH^\conj + \BE \cross \BH e^{2 j \omega t}
}
=
\inv{2} \Real \BE \cross \BH^\conj
=
\inv{2} E_0^2 \sqrt{\frac{\epsilon}{\mu}}
\begin{vmatrix}
\xcap & \ycap & \zcap \\
\cos\theta & 0 & -\sin\theta \\
0 & 1 & 0
\end{vmatrix}
=
\inv{2} E_0^2 \sqrt{\frac{\epsilon}{\mu}}
\lr{ \xcap \sin\theta + \zcap \cos\theta },
\end{dmath}
or
\boxedEquation{eqn:emtproblemSet6Problem3:260}{
\expectation{\bcS}
=
\inv{2} E_0^2 \sqrt{\frac{\epsilon}{\mu}} \kcap.
}
\makeSubAnswer{}{emt:problemSet6:3c}
The stored magnetic energy density is
\begin{dmath}\label{eqn:emtproblemSet6Problem3:280}
W_m
= \inv{2} \mu \Abs{\BH}^2
= \inv{2} \mu \frac{\epsilon}{\mu} E_0^2
= \inv{2} \epsilon E_0^2.
\end{dmath}
This equals the stored electric energy density, as expected.
\makeSubAnswer{}{emt:problemSet6:3d}
The phase velocity are the velocities that satisfy
\begin{dmath}\label{eqn:emtproblemSet6Problem3:300}
0
=
\ddt{} \lr{ \omega t - \Bk \cdot \Br }
=
\omega - \Bk \cdot \frac{d\Br}{dt},
\end{dmath}
or
\begin{dmath}\label{eqn:emtproblemSet6Problem3:380}
\omega =
\Bk \cdot \Bv_p
=
k_x v_x
+
k_z v_z
=
k \sin\theta v_x
+
k \cos\theta v_z.
\end{dmath}
This has many solutions, including superluminal phase velocities such as:
\begin{equation}\label{eqn:emtproblemSet6Problem3:420}
\begin{aligned}
\Bv_p &= (\omega/(k\sin\theta), 0, 0) \\
\Bv_p &= (0, 0, \omega/(k\cos\theta)) \\
\Bv_p &= \frac{\omega}{2 k}(1/\sin\theta, 0, 1/\cos\theta)
\end{aligned}
\end{equation}
I'm unsure if those are physically relevant.  However,
it is reasonable to assume that a solution where \( \Bv_p \) is colinear with the energy propagation direction \( \Bk \) is the solution of interest.  In that case, the components of the phase velocity are
\begin{equation}\label{eqn:emtproblemSet6Problem3:320}
\begin{aligned}
v_x &= \frac{\omega}{k}\sin\theta \\
v_z &= \frac{\omega}{k}\cos\theta,
\end{aligned}
\end{equation}
or
\boxedEquation{eqn:emtproblemSet6Problem3:400}{
\begin{aligned}
v_x &= c \sin\theta \\
v_z &= c \cos\theta
\end{aligned}
}
where
\begin{dmath}\label{eqn:emtproblemSet6Problem3:360}
c = \inv{\sqrt{\mu \epsilon}}.
\end{dmath}
%Both of these phase velocity components are greater than the speed of the wave.
}}
