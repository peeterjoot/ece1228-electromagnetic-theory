%
% Copyright � 2016 Peeter Joot.  All Rights Reserved.
% Licenced as described in the file LICENSE under the root directory of this GIT repository.
%
%{
%\input{../blogpost.tex}
%\renewcommand{\basename}{transverseGauge}
%%\renewcommand{\dirname}{notes/phy1520/}
%\renewcommand{\dirname}{notes/ece1228-electromagnetic-theory/}
%%\newcommand{\dateintitle}{}
%%\newcommand{\keywords}{}
%
%\input{../latex/peeter_prologue_print2.tex}
%
%\usepackage{peeters_layout_exercise}
%\usepackage{peeters_braket}
%\usepackage{peeters_figures}
%\usepackage{siunitx}
%%\usepackage{mhchem} % \ce{}
%%\usepackage{macros_bm} % \bcM
%%\usepackage{txfonts} % \ointclockwise
%
%\beginArtNoToc
%
%\generatetitle{Transverse gauge}
%\label{chap:transverseGauge}
%
Jackson \citep{jackson1975cew} has an interesting presentation of the transverse gauge.  I'd like to walk through the details of this, but first want to translate the preliminaries to SI units (if I had the 3rd edition I'd not have to do this translation step).
%
\paragraph{Gauge freedom.}
%
The starting point is noting that \( \spacegrad \cdot \BB = 0 \) the magnetic field can be expressed as a curl
%
\begin{equation}\label{eqn:transverseGauge:20}
\BB = \spacegrad \cross \BA.
\end{equation}
%
Faraday's law now takes the form
\begin{equation}\label{eqn:transverseGauge:40}
\begin{aligned}
0
&= \spacegrad \cross \BE + \PD{t}{\BB}
\\ &= \spacegrad \cross \BE + \PD{t}{} \lr{ \spacegrad \cross \BA }
\\ &= \spacegrad \cross \lr{ \BE + \PD{t}{\BA} }.
\end{aligned}
\end{equation}
%
Because this curl is zero, the interior sum can be expressed as a gradient
%
\begin{equation}\label{eqn:transverseGauge:60}
\BE + \PD{t}{\BA} \equiv -\spacegrad \Phi.
\end{equation}
%
This can now be substituted into the remaining two Maxwell's equations.
%
\begin{equation}\label{eqn:transverseGauge:80}
\begin{aligned}
\spacegrad \cdot \BD &= \rho_v, \\
\spacegrad \cross \BH &= \BJ + \PD{t}{\BD}.
\end{aligned}
\end{equation}
%
For Gauss's law, in simple media, we have
%
\begin{equation}\label{eqn:transverseGauge:140}
\begin{aligned}
\rho_v
&=
\epsilon \spacegrad \cdot \BE
\\ &=
\epsilon \spacegrad \cdot \lr{ -\spacegrad \Phi - \PD{t}{\BA} }.
\end{aligned}
\end{equation}
%
For simple media again, the Ampere-Maxwell equation is
%
\begin{equation}\label{eqn:transverseGauge:100}
\inv{\mu} \spacegrad \cross \lr{ \spacegrad \cross \BA } = \BJ + \epsilon \PD{t}{} \lr{ -\spacegrad \Phi - \PD{t}{\BA} }.
\end{equation}
%
Expanding \( \spacegrad \cross \lr{ \spacegrad \cross \BA } = -\spacegrad^2 \BA + \spacegrad \lr{ \spacegrad \cdot \BA } \) gives
\begin{equation}\label{eqn:transverseGauge:120}
-\spacegrad^2 \BA + \spacegrad \lr{ \spacegrad \cdot \BA } + \epsilon \mu \PDSq{t}{\BA} = \mu \BJ - \epsilon \mu \spacegrad \PD{t}{\Phi}.
\end{equation}
%
Maxwell's equations are now reduced to
\boxedEquation{eqn:transverseGauge:180}{
\begin{aligned}
\spacegrad^2 \BA - \spacegrad \lr{ \spacegrad \cdot \BA + \epsilon \mu \PD{t}{\Phi}} - \epsilon \mu \PDSq{t}{\BA} &= -\mu \BJ \\
\spacegrad^2 \Phi + \PD{t}{\spacegrad \cdot \BA} &= -\frac{\rho_v }{\epsilon}.
\end{aligned}
}
%
There are two obvious constraints that we can impose
\begin{equation}\label{eqn:transverseGauge:200}
\spacegrad \cdot \BA - \epsilon \mu \PD{t}{\Phi} = 0,
\end{equation}
%
or
\begin{equation}\label{eqn:transverseGauge:220}
\spacegrad \cdot \BA = 0.
\end{equation}
%
The first constraint is the Lorentz gauge, which I've played with previously.  It happens to be really nice in a relativistic context since, in vacuum with a four-vector potential \( A = (\Phi/c, \BA) \), that is a requirement that the four-divergence of the four-potential vanishes (\( \partial_\mu A^\mu = 0 \)).
%
\paragraph{Transverse gauge.}
%
Jackson identifies the latter constraint as the transverse gauge, which I'm less familiar with.  With this gauge selection, we have
%
\begin{subequations}
\label{eqn:transverseGauge:240}
\begin{equation}\label{eqn:transverseGauge:260}
\spacegrad^2 \BA - \epsilon \mu \PDSq{t}{\BA} = -\mu \BJ + \epsilon\mu \spacegrad \PD{t}{\Phi}
\end{equation}
\begin{equation}\label{eqn:transverseGauge:280}
\spacegrad^2 \Phi = -\frac{\rho_v }{\epsilon}.
\end{equation}
\end{subequations}
%
What's not obvious is the fact that the irrotational (zero curl) contribution due to \(\Phi\) in \cref{eqn:transverseGauge:260} cancels the corresponding irrotational term from the current.  Jackson uses a transverse and longitudinal decomposition of the current, related to the Helmholtz theorem to allude to this.
%
That decomposition follows from expanding \( \spacegrad^2 J/R \) in two ways using the delta function \( -4 \pi \delta(\Bx - \Bx') = \spacegrad^2 1/R \) representation, as well as directly
%
\begin{equation}\label{eqn:transverseGauge:300}
\begin{aligned}
- 4 \pi \BJ(\Bx)
&=
\int \spacegrad^2 \frac{\BJ(\Bx')}{\Abs{\Bx - \Bx'}} d^3 x'
\\ &=
\spacegrad
\int \spacegrad \cdot \frac{\BJ(\Bx')}{\Abs{\Bx - \Bx'}} d^3 x'
+
\spacegrad \cdot
\int \spacegrad \wedge \frac{\BJ(\Bx')}{\Abs{\Bx - \Bx'}} d^3 x'
\\ &=
-\spacegrad
\int \BJ(\Bx') \cdot \spacegrad' \inv{\Abs{\Bx - \Bx'}} d^3 x'
+
\spacegrad \cdot \lr{ \spacegrad \wedge
\int \frac{\BJ(\Bx')}{\Abs{\Bx - \Bx'}} d^3 x'
}
\\ &=
-\spacegrad
\int \spacegrad' \cdot \frac{\BJ(\Bx')}{\Abs{\Bx - \Bx'}} d^3 x'
+\spacegrad
\int \frac{\spacegrad' \cdot \BJ(\Bx')}{\Abs{\Bx - \Bx'}} d^3 x' \\
&\quad
-
\spacegrad \cross \lr{
\spacegrad \cross
\int \frac{\BJ(\Bx')}{\Abs{\Bx - \Bx'}} d^3 x'
}.
\end{aligned}
\end{equation}
%
The first term can be converted to a surface integral
%
\begin{equation}\label{eqn:transverseGauge:320}
-\spacegrad
\int \spacegrad' \cdot \frac{\BJ(\Bx')}{\Abs{\Bx - \Bx'}} d^3 x'
=
-\spacegrad
\int d\BA' \cdot \frac{\BJ(\Bx')}{\Abs{\Bx - \Bx'}},
\end{equation}
%
so provided the currents are either localized or \( \Abs{\BJ}/R \rightarrow 0 \) on an infinite sphere, we can make the identification
%
\begin{equation}\label{eqn:transverseGauge:340}
\BJ(\Bx)
=
\spacegrad
\inv{4\pi} \int \frac{\spacegrad' \cdot \BJ(\Bx')}{\Abs{\Bx - \Bx'}} d^3 x'
-
\spacegrad \cross \spacegrad \cross \inv{4 \pi} \int \frac{\BJ(\Bx')}{\Abs{\Bx - \Bx'}} d^3 x'
\equiv
\BJ_l +
\BJ_t,
\end{equation}
%
where \( \spacegrad \cross \BJ_l = 0 \) (irrotational, or longitudinal), whereas \( \spacegrad \cdot \BJ_t = 0 \) (solenoidal or transverse).  The irrotational property is clear from inspection, and the transverse property can be verified readily
%
\begin{equation}\label{eqn:transverseGauge:360}
\begin{aligned}
\spacegrad \cdot \lr{ \spacegrad \cross \lr{ \spacegrad \cross \BX } }
&=
-\spacegrad \cdot \lr{ \spacegrad \cdot \lr{ \spacegrad \wedge \BX } }
\\ &=
-\spacegrad \cdot \lr{ \spacegrad^2 \BX - \spacegrad \lr{ \spacegrad \cdot \BX } }
\\ &=
-\spacegrad \cdot \lr{\spacegrad^2 \BX} + \spacegrad^2 \lr{ \spacegrad \cdot \BX }
\\ &= 0.
\end{aligned}
\end{equation}
%
Since
%
\begin{equation}\label{eqn:transverseGauge:380}
\Phi(\Bx, t)
=
\inv{4 \pi \epsilon} \int \frac{\rho_v(\Bx', t)}{\Abs{\Bx - \Bx'}} d^3 x',
\end{equation}
%
we have
%
\begin{equation}\label{eqn:transverseGauge:400}
\begin{aligned}
\spacegrad \PD{t}{\Phi}
&=
\inv{4 \pi \epsilon} \spacegrad \int \frac{\partial_t \rho_v(\Bx', t)}{\Abs{\Bx - \Bx'}} d^3 x'
\\ &=
\inv{4 \pi \epsilon} \spacegrad \int \frac{-\spacegrad' \cdot \BJ}{\Abs{\Bx - \Bx'}} d^3 x'
\\ &=
\frac{\BJ_l}{\epsilon}.
\end{aligned}
\end{equation}
%
This means that the Ampere-Maxwell equation takes the form
%
\begin{equation}\label{eqn:transverseGauge:420}
\spacegrad^2 \BA - \epsilon \mu \PDSq{t}{\BA}
= -\mu \BJ + \mu \BJ_l
= -\mu \BJ_t.
\end{equation}

This justifies the ``transverse'' in the label transverse gauge.
%
%}
%\EndArticle
