%
% Copyright � 2016 Peeter Joot.  All Rights Reserved.
% Licenced as described in the file LICENSE under the root directory of this GIT repository.
%
\makeproblem{Susceptibility kernel.}{emt:problemSet5:3}{
\makesubproblem{}{emt:problemSet5:3a}
Assuming that a medium is described by the time harmonic relationship \( \BD(\Bx, \omega) = \epsilon(\omega) \BE(\Bx, \Bomega) \), show that the
time domain relation between the electric flux density \( \BD \) and the electric field \( \BE \) is given by,
\begin{equation}\label{eqn:emtProblemSet5Problem3:20}
\BD(\Bx, t) = \epsilon_0
\lr{
\BE(\Bx, t)
+ \int_{-\infty}^\infty G(\tau) \BE(\Bx, t - \tau) d\tau,
}
\end{equation}
where \( G(\tau) \) is the susceptibility kernel given by
\begin{equation}\label{eqn:emtProblemSet5Problem3:40}
G(\tau) =
%\inv{\sqrt{2 \pi}}
\inv{2 \pi}
\int_{-\infty}^\infty
\lr{\frac{\epsilon(\omega)}{\epsilon_0} - 1}
e^{-j \omega t} d\tau.
\end{equation}
\makesubproblem{}{emt:problemSet5:3b}
Show that
\begin{equation}\label{eqn:emtProblemSet5Problem3:60}
\epsilon(-\omega) = \epsilon^\conj(\omega).
\end{equation}
\makesubproblem{}{emt:problemSet5:3c}
Show that for \( \epsilon(\omega) = \epsilon'(\omega) + j \epsilon''(\omega) \),
\( \epsilon'(\omega) \) is even
and \( \epsilon''(\omega) \) is odd.
} % makeproblem
\makeanswer{emt:problemSet5:3}{\withproblemsetsParagraph{
\makeSubAnswer{}{emt:problemSet5:3a}
The electric displacement field in the time domain is
\begin{dmath}\label{eqn:emtProblemSet5Problem3:80}
\BD(\Bx, t)
= \inv{2\pi} \int_{-\infty}^\infty \epsilon(\omega) \BE(\Bx, \Bomega) e^{-j \omega t} d\omega.
\end{dmath}
Assuming all integrals are over \( [-\infty, \infty] \) we have
\begin{dmath}\label{eqn:emtProblemSet5Problem3:100}
\BD(\Bx, t)
= \inv{2\pi} \int \epsilon(\omega) \BE(\Bx, \Bomega) e^{-j \omega t} d\omega
= \inv{2\pi} \iiint \lr{ \epsilon(\tau) e^{j \omega \tau} d\tau } \lr{ \BE(\Bx, t') e^{j \omega t'} dt'} e^{-j \omega t} d\omega
= \iint \epsilon(\tau) \BE(\Bx, t') d\tau dt'
\inv{2 \pi} \int e^{j \omega \tau} e^{j \omega t'} e^{-j \omega t} d\omega
= \iint \epsilon(\tau) \BE(\Bx, t') d\tau dt'
\inv{2 \pi} \int e^{j \omega (t' - (t - \tau))} d\omega
= \iint \epsilon(\tau) \BE(\Bx, t') d\tau dt' \delta( t' - (t - \tau) )
= \int \epsilon(\tau) \BE(\Bx, t - \tau) d\tau.
\end{dmath}
To this convolution \( \epsilon_0 \BE(\Bx, t) \) can be simultaneously added and subtracted
\begin{dmath}\label{eqn:emtProblemSet5Problem3:120}
\BD(\Bx, t)
=
\epsilon_0 \BE(\Bx, t)
+
\epsilon_0
\int \lr{ \frac{\epsilon(\tau)}{\epsilon_0} - \delta(\tau) } \BE(\Bx, t - \tau) d\tau.
\end{dmath}
This shows that
\begin{dmath}\label{eqn:emtProblemSet5Problem3:140}
G(\tau)
=
\frac{\epsilon(\tau)}{\epsilon_0} - \delta(\tau)
=
\inv{2 \pi} \int \frac{\epsilon(\omega)}{\epsilon_0} e^{-j \omega \tau} d\tau
- \inv{2 \pi} \int e^{-j \omega \tau} d\tau
=
\inv{2 \pi} \int \lr{ \frac{\epsilon(\omega)}{\epsilon_0} -1 } e^{-j \omega \tau} d\tau. \qedmarker
\end{dmath}
\makeSubAnswer{}{emt:problemSet5:3b}
The permittivity in the frequency domain is
\begin{dmath}\label{eqn:emtProblemSet5Problem3:160}
\epsilon(\omega) = \int \epsilon(t) e^{j \omega t} dt,
\end{dmath}
or
\begin{dmath}\label{eqn:emtProblemSet5Problem3:180}
\epsilon(-\omega) = \int \epsilon(t) e^{-j \omega t} dt.
\end{dmath}
Assuming that \( \epsilon(t) \) is real, this is
\begin{dmath}\label{eqn:emtProblemSet5Problem3:200}
\epsilon(-\omega) = \lr{ \int \epsilon(t) e^{j \omega t} dt }^\conj,
\end{dmath}
or
\begin{dmath}\label{eqn:emtProblemSet5Problem3:220}
\epsilon(-\omega) = \epsilon^\conj(\omega). \qedmarker
\end{dmath}
\makeSubAnswer{}{emt:problemSet5:3c}
Written out explicitly, the frequency domain permittivity is
\begin{dmath}\label{eqn:emtProblemSet5Problem3:240}
\epsilon(\omega)
=
\int \epsilon(t) \cos\lr{\omega t} dt
+ j \int \epsilon(t) \sin\lr{\omega t} dt
=
\int \epsilon(t) \cos\lr{-\omega t} dt
- j \int \epsilon(t) \sin\lr{-\omega t} dt,
\end{dmath}
or
\begin{equation}\label{eqn:emtProblemSet5Problem3:260}
\begin{aligned}
\epsilon'(\omega) &= \epsilon'(-\omega)  \\
\epsilon''(\omega) &= -\epsilon''(-\omega). \qedmarker
\end{aligned}
\end{equation}
}}
