%
% Copyright � 2016 Peeter Joot.  All Rights Reserved.
% Licenced as described in the file LICENSE under the root directory of this GIT repository.
%
\input{../latex/blogpost.tex}
\renewcommand{\basename}{emt9}
\renewcommand{\dirname}{notes/ece1228/}
\newcommand{\keywords}{ECE1228H}
\input{../latex/peeter_prologue_print2.tex}
%
%\usepackage{ece1228}
\usepackage{peeters_braket}
%\usepackage{peeters_layout_exercise}
\usepackage{peeters_figures}
\usepackage{mathtools}
\usepackage{siunitx}
%
\beginArtNoToc
\generatetitle{ECE1228H Electromagnetic Theory.  Lecture 9: XXX.  Taught by Prof.\ M. Mojahedi}
%\chapter{XXX}
\label{chap:emt9}
%
\paragraph{Disclaimer}
%
Peeter's lecture notes from class.  These may be incoherent and rough.
%
These are notes for the UofT course ECE1228H, Electromagnetic Theory, taught by Prof. M. Mojahedi, covering \textchapref{{1}} \citep{balanis1989advanced} content.
%
\paragraph{YYY}
%
When the magnitude and direction of the fields remain constant, we call this a plane wave.  This is a non-physical entity, but can be made physical with superposition.
%
%
On dimensions, note that
%
\begin{dmath}\label{eqn:emtLecture9:20}
k
= \frac{\omega}{c} n
= \omega \sqrt{ \mu_0 \epsilon_0 } \sqrt{\mu_r} \sqrt{\epsilon_r}
\end{dmath}
%
so the magnetic field for a plane wave is
%
\begin{dmath}\label{eqn:emtLecture9:40}
\BH
= \frac{k}{\omega \mu} \kcap \cross \BE
= \frac{ \omega \sqrt{ \mu_0 \epsilon_0 } \sqrt{\mu_r} \sqrt{\epsilon_r} }{\omega \mu} \kcap \cross \BE
= \frac{ \sqrt{ \mu_0 \epsilon_0 } \sqrt{\mu_r} \sqrt{\epsilon_r} }{\mu} \kcap \cross \BE
= \sqrt{ \frac{ \epsilon_0 \epsilon_r }{\mu_0\mu_r } } \kcap \cross \BE
= \inv{ \sqrt{\mu/\epsilon} } \kcap \cross \BE
= \inv{ \eta } \kcap \cross \BE,
\end{dmath}
%
where
\begin{dmath}\label{eqn:emtLecture9:60}
\eta = \sqrt{\mu/\epsilon}
\end{dmath}
%
is the medium intrinsic impedance and we can define
\begin{dmath}\label{eqn:emtLecture9:80}
\eta_0 = \sqrt{\mu_0/\epsilon_0} = 377 \,\Omega,
\end{dmath}
%
as the free space intrinsic impedance.  This was argued as a definitive superiority of the SI unit system, since this impedance relationship is made clear, something that would not be the case in CGS.
%
\paragraph{Rant on negative index of refraction, diffraction, uncertainty, and related concepts}
%
Note: Poynting vector and \( \Bk \) only colinear in simple isotropic media!
%
Note: When \( \epsilon \) and \( \mu \) are both negative, we must pick the negative sign for \( n = -\sqrt{\epsilon\mu} \).  Recall that we saw negative \( \epsilon \) for low frequencies for metals in the Druid model where \( \epsilon = 1 - \omega_p^2/\omega^2 \).
%
Negative \( \mu \) can be obtained in the microwave domain in the Lorentzian model.  Pandree did this with split ring resonators.
%
Smith did both negative \( epsilon \) and \( \mu \) together and measured the transmission of such a wave.  It was shown that instead of attenuation and reflection there was transmission.
%
One theorized application of negative index of refraction is to construct a ``perfect lens'' to use a flat lens to image a source.  This has been done in very near field applications with such a setup.
%
Of interest is Abbe's diffraction limit for optical systems \citep{wiki:abbe}, defined by
%
\begin{dmath}\label{eqn:emtLecture9:100}
\Delta r \ge \frac{\lambda_0}{2 n}.
\end{dmath}
%
F4
%
which is an application of the constraint equation
%
\begin{dmath}\label{eqn:emtLecture9:120}
\Bk^2 = \lr{\frac{\omega n}{c}}^2
\end{dmath}
%
Also of interest:
%
Negative dielectric paper: Koboyashi
%
\paragraph{Transverse propagation}
%
The plane wave solution to Maxwell's equations, which are transverse, unlike sound (compression waves), begs us to wonder what the wave is travelling through.
%
Maxwell postulated a medium, the Ether, that would carry this wave.  The Michaelson-Morley experiment is considered the definitive proof that no such medium exists.
%
%
%
\EndArticle
%\EndNoBibArticle
