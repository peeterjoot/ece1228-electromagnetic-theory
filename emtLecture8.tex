%
% Copyright � 2016 Peeter Joot.  All Rights Reserved.
% Licenced as described in the file LICENSE under the root directory of this GIT repository.
%
%\input{../blogpost.tex}
%\renewcommand{\basename}{emt8}
%\renewcommand{\dirname}{notes/ece1228/}
%\newcommand{\keywords}{ECE1228H}
%\input{../latex/peeter_prologue_print2.tex}
%
%%\usepackage{ece1228}
%\usepackage{peeters_braket}
%%\usepackage{peeters_layout_exercise}
%\usepackage{peeters_figures}
%\usepackage{mathtools}
%\usepackage{siunitx}
%
%\beginArtNoToc
%\generatetitle{ECE1228H Electromagnetic Theory.  Lecture 8: Waves.  Taught by Prof.\ M. Mojahedi}
%%\chapter{Waves}
%\label{chap:emt8}
%
%\paragraph{Disclaimer}
%
%Peeter's lecture notes from class.  These may be incoherent and rough.
%
%These are notes for the UofT course ECE1228H, Electromagnetic Theory, taught by Prof. M. Mojahedi, covering \textchapref{{1}} \citep{balanis1989advanced} content.
%
\section{Cylindrical coordinates.}
%
Seek a function
%
\begin{dmath}\label{eqn:emtLecture8:20}
\BE = E_\rho \rhocap + E_\phi \phicap + E_z \zcap,
\end{dmath}
%
solving
%
\begin{dmath}\label{eqn:emtLecture8:40}
\spacegrad^2 \BE = -\beta^2 \BE.
\end{dmath}
%
One way to find the Laplacian in cylindrical coordinates is to use
%
\begin{dmath}\label{eqn:emtLecture8:60}
\spacegrad^2 \BE =
\spacegrad \lr{ \spacegrad \cdot \BE }
-\spacegrad \cross \lr{ \spacegrad \cross \BE },
\end{dmath}
%
where
%
\begin{dmath}\label{eqn:emtLecture8:80}
\spacegrad = \rhocap \PD{\rho}{} + \frac{\phicap}{\rho} \PD{\phi}{} + \zcap \PD{z}{}.
\end{dmath}
%
It can be shown that:
\begin{dmath}\label{eqn:emtLecture8:100}
\spacegrad \cdot \BE = \inv{\rho} \PD{\rho}{} \lr{ \rho E_\rho } + \inv{\rho}\PD{\phi}{E_\phi} + \PD{z}{E_z},
\end{dmath}
%
and
\begin{dmath}\label{eqn:emtLecture8:120}
\spacegrad \cross \BE
%=
%\begin{vmatrix}
%\rhocap & \phicap & \zcap \\
%\partial_\rho & \inv{\rho}\partial_\phi & \partial_z \\
%E_\rho & \rho E_\phi & E_z
%\end{vmatrix}
=
\rhocap  \lr{ \inv{\rho} \partial_\phi E_z - \partial_z E_\phi }
+\phicap \lr{ \partial_z E_\rho - \partial_\rho E_z }
+\zcap   \lr{ \inv{\rho} \partial_\rho (\rho E_\phi) - \inv{\rho} \partial_\phi E_\rho }.
\end{dmath}
%
This gives
\begin{dmath}\label{eqn:emtLecture8:200}
\spacegrad^2 \psi =
\PDSq{\rho}{\psi}
+\inv{\rho} \PD{\rho}{\psi}
+\inv{\rho^2} \PDSq{\phi}{\psi}
+\PDSq{z}{\psi},
\end{dmath}
and
\begin{equation}\label{eqn:emtLecture8:220}
\begin{aligned}
\spacegrad^2 E_\rho &= \lr{ -\frac{E_\rho}{\rho^2} - \frac{2}{\rho^2} \PD{\phi}{E_\phi} }, \\
\spacegrad^2 E_\phi &= \lr{ -\frac{E_\phi}{\rho^2} + \frac{2}{\rho^2} \PD{\phi}{E_\rho} }, \\
\spacegrad^2 E_z    &= -\beta^2 E_\phi.
\end{aligned}
\end{equation}
%
This is explored in \cref{chap:laplacianCylindrical}.
%
%Note that with \( i = \Be_1 \Be_2 \),
%
%\begin{dmath}\label{eqn:emtLecture8:140}
%\rhocap = \Be_1 e^{i \phi}
%\end{dmath}
%
%so
%\begin{equation}\label{eqn:emtLecture8:160}
%\PD{\phi}{\rhocap} = \Be_2 e^{i \phi} = \thetacap
%\end{equation}
%
%... the end result is
%
%\begin{dmath}\label{eqn:emtLecture8:180}
%\end{dmath}
%
\paragraph{TEM:} If we want to have a TEM mode it can be shown that we need an axial distribution mechanism, such as the core of a co-axial cable.
%
These are messy to solve in general, but we can solve the z-component without too much pain
%
\begin{dmath}\label{eqn:emtLecture8:240}
\PDSq{\rho}{E_z}
+\inv{\rho} \PD{\rho}{E_z}
+\inv{\rho^2} \PDSq{\phi}{E_z}
+\PDSq{z}{E_z}
=
-\beta^2 E_z.
\end{dmath}
%
Solving this using separation of variables with
%
\begin{dmath}\label{eqn:emtLecture8:260}
E_z = R(\rho) P(\phi) Z(z),
\end{dmath}
%
\begin{dmath}\label{eqn:emtLecture8:280}
\inv{R}\lr{R'' + \inv{\rho} R'} + \inv{\rho^2 P} P'' + \frac{Z''}{Z} = -\beta^2.
\end{dmath}
%
Assuming for some constant \( \beta_z \) that we have
\begin{dmath}\label{eqn:emtLecture8:300}
\frac{Z''}{Z} = -\beta_z^2,
\end{dmath}
%
then
%
\begin{dmath}\label{eqn:emtLecture8:320}
\inv{R}\lr{\rho^2 R'' + \rho R'} + \inv{P} P'' + \rho^2 \lr{\beta^2 - \beta_z^2} = 0.
\end{dmath}
%
Now assume that
\begin{dmath}\label{eqn:emtLecture8:340}
\inv{P} P'' = -m^2,
\end{dmath}
%
and let \( \beta^2 - \beta_z^2 = \beta_\rho^2 \), which leaves
%
\begin{dmath}\label{eqn:emtLecture8:360}
\rho^2 R'' + \rho R' + \lr{ \rho^2 \beta_\rho^2 -m^2 } R = 0.
\end{dmath}
%
This is the Bessel differential equation, with travelling wave solution
%
\begin{dmath}\label{eqn:emtLecture8:380}
R(\rho) =
A H_m^{(1)}(\beta_\rho \rho)
+B H_m^{(2)}(\beta_\rho \rho),
\end{dmath}
%
and standing wave solutions
\begin{dmath}\label{eqn:emtLecture8:400}
R(\rho) =
A J_m(\beta_\rho \rho)
+B Y_m(\beta_\rho \rho).
\end{dmath}
%
Here \( H_m^{(1)}, H_m^{(2)} \) are Hankel functions of the first and second kinds, and
\( J_m, Y_m \) are the Bessel functions of the first and second kinds.
%
For \( P(\phi) \)
\begin{dmath}\label{eqn:emtLecture8:460}
P'' = -m^2 P.
\end{dmath}
%
%
\section{Waves.}
%
\begin{itemize}
\item The field is a modification of space-time
\item Mode is a particular field configuration for a given boundary value problem.  Many field configurations can satisfy Maxwell equations (wave equation).  These usually are referred to as modes.  A mode is a self-consistent field distribution.
\item In a TEM mode, \( \BE \) and \( \BH \) are every point in space are constrained in a local plane, independent of time.  This plane is called the equiphase plane.  In general equiphase planes are not parallel at two different points along the trajectory of the wave.
%\item If equiphase planes are parallel (i.e. the space orientation of the planes for TEM mode...
%... next time.
\end{itemize}
%
%}
%\EndNoBibArticle
