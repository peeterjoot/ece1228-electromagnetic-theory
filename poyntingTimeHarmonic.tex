%
% Copyright � 2016 Peeter Joot.  All Rights Reserved.
% Licenced as described in the file LICENSE under the root directory of this GIT repository.
%
%{
%\input{../blogpost.tex}
%\renewcommand{\basename}{poyntingTimeHarmonic}
%%\renewcommand{\dirname}{notes/phy1520/}
%\renewcommand{\dirname}{notes/ece1228-electromagnetic-theory/}
%%\newcommand{\dateintitle}{}
%%\newcommand{\keywords}{}
%
%\input{../latex/peeter_prologue_print2.tex}
%
%\usepackage{peeters_layout_exercise}
%\usepackage{peeters_braket}
%\usepackage{peeters_figures}
%\usepackage{siunitx}
%\usepackage{macros_bm}
%%\usepackage{txfonts} % \ointclockwise
%
%\beginArtNoToc
%
%\generatetitle{Frequency domain time averaged Poynting theorem}
%\chapter{Frequency domain time averaged Poynting theorem}
%\label{chap:poyntingTimeHarmonic}
\makeproblem{Frequency domain time averaged Poynting theorem.}{problem:poyntingTimeHarmonic:1}{
The time domain Poynting relationship was found to be
\begin{equation}\label{eqn:poyntingTimeHarmonic:20}
\begin{aligned}
0
&=
\spacegrad \cdot \lr{ \BE \cross \BH }
+ \frac{\epsilon}{2} \BE \cdot \PD{t}{\BE}
+ \frac{\mu}{2} \BH \cdot \PD{t}{\BH} \\
&\qquad + \BH \cdot \BM_i
+ \BE \cdot \BJ_i
+ \sigma \BE \cdot \BE.
\end{aligned}
\end{equation}
Derive the equivalent relationship for the time averaged portion of the time-harmonic Poynting vector.
} % problem
\makeanswer{problem:poyntingTimeHarmonic:1}{
%\withproblemsetsParagraph{
The time domain representation of the Poynting vector in terms of the time-harmonic (phasor) vectors is
\begin{equation}\label{eqn:poyntingTimeHarmonic:40}
\begin{aligned}
\bcE \cross \bcH
&= \inv{4}
\lr{
\BE e^{j\omega t}
+ \BE^\conj e^{-j\omega t}
}
\cross
\lr{
\BH e^{j\omega t}
+ \BH^\conj e^{-j\omega t}
}
\\ &=
\inv{2} \Real \lr{ \BE \cross \BH^\conj + \BE \cross \BH e^{2 j \omega t} },
\end{aligned}
\end{equation}
so if we are looking for the relationships that effect only the time averaged Poynting vector, over integral multiples of the period, we are interested in evaluating the divergence of
\begin{equation}\label{eqn:poyntingTimeHarmonic:60}
\inv{2} \BE \cross \BH^\conj.
\end{equation}
The time-harmonic Maxwell's equations are
\begin{equation}\label{eqn:poyntingTimeHarmonic:80}
\begin{aligned}
\spacegrad \cross \BE &= - j \omega \mu \BH - \BM_i, \\
\spacegrad \cross \BH &= j \omega \epsilon \BE + \BJ_i + \sigma \BE.
\end{aligned}
\end{equation}
The latter after conjugation is
\begin{equation}\label{eqn:poyntingTimeHarmonic:100}
\spacegrad \cross \BH^\conj = -j \omega \epsilon^\conj \BE^\conj + \BJ_i^\conj + \sigma^\conj \BE^\conj.
\end{equation}
For the divergence we have
\begin{equation}\label{eqn:poyntingTimeHarmonic:120}
\begin{aligned}
\spacegrad &\cdot \lr{ \BE \cross \BH^\conj } \\
&=
\BH^\conj \cdot \lr{ \spacegrad \cdot \BE }
-\BE \cdot \lr{ \spacegrad \cdot \BH^\conj } \\
&=
\BH^\conj \cdot \lr{ - j \omega \mu \BH - \BM_i }
- \BE \cdot \lr{ -j \omega \epsilon^\conj \BE^\conj + \BJ_i^\conj + \sigma^\conj \BE^\conj },
\end{aligned}
\end{equation}
or
\begin{equation}\label{eqn:poyntingTimeHarmonic:140}
\begin{aligned}
0
&=
\spacegrad \cdot \lr{ \BE \cross \BH^\conj } \\
&\qquad +
\BH^\conj \cdot \lr{ j \omega \mu \BH + \BM_i }
+ \BE \cdot \lr{ -j \omega \epsilon^\conj \BE^\conj + \BJ_i^\conj + \sigma^\conj \BE^\conj },
\end{aligned}
\end{equation}
so
%\begin{equation}\label{eqn:poyntingTimeHarmonic:160}
\boxedEquation{eqn:poyntingTimeHarmonic:160}{
\begin{aligned}
0
&=
\spacegrad \cdot \inv{2} \lr{ \BE \cross \BH^\conj }
+ \inv{2} \lr{ \BH^\conj \cdot \BM_i
+ \BE \cdot \BJ_i^\conj } \\
&\quad + \inv{2} j \omega \lr{ \mu \Abs{\BH}^2 - \epsilon^\conj \Abs{\BE}^2 }
+ \inv{2} \sigma^\conj \Abs{\BE}^2.
\end{aligned}
}
%\end{equation}
%}
} % answer
%}
%\EndNoBibArticle
