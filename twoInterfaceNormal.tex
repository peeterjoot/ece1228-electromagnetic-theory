%
% Copyright � 2016 Peeter Joot.  All Rights Reserved.
% Licenced as described in the file LICENSE under the root directory of this GIT repository.
%
%{
%\input{../blogpost.tex}
%\renewcommand{\basename}{twoInterfaceNormal}
%%\renewcommand{\dirname}{notes/phy1520/}
%\renewcommand{\dirname}{notes/ece1228-electromagnetic-theory/}
%%\newcommand{\dateintitle}{}
%%\newcommand{\keywords}{}
%
%\input{../latex/peeter_prologue_print2.tex}
%
%\usepackage{peeters_layout_exercise}
%\usepackage{peeters_braket}
%\usepackage{peeters_figures}
%\usepackage{siunitx}
%\usepackage{enumerate}
%%\usepackage{mhchem} % \ce{}
%%\usepackage{macros_bm} % \bcM
%%\usepackage{txfonts} % \ointclockwise
%
%\beginArtNoToc
%
\section{Normal transmission and reflection through two interfaces.}
%\chapter{Normal transmission and reflection through two interfaces}
%\label{chap:twoInterfaceNormal}
%
%\paragraph{Motivation}
%
%In class an outline of normal transmission through a slab was presented.  Let's go through the details.
%
%\section{Two interfaces, normal incidence.}
%
The geometry of a two interface configuration is sketched in \cref{fig:l10TwoInterfaces:l10TwoInterfacesFig1}.
%
\imageFigure{../figures/ece1228-electromagnetic-theory/l10TwoInterfacesFig1}{Two interface transmission.}{fig:l10TwoInterfaces:l10TwoInterfacesFig1}{0.2}
%
Given a normal incident ray with magnitude \( A \), the respective forward and backwards rays in each the mediums can be written as
%
\begin{enumerate}[I]
\item
\begin{equation}\label{eqn:twoInterfaceNormal:20}
\begin{aligned}
\rightarrow &\qquad A e^{-j k_{1z} z} \\
\leftarrow &\qquad A r e^{j k_{1z} z} \\
\end{aligned}
\end{equation}
\item
\begin{equation}\label{eqn:twoInterfaceNormal:40}
\begin{aligned}
\rightarrow &\qquad C e^{-j k_{2z} z} \\
\leftarrow &\qquad D e^{j k_{2z} z} \\
\end{aligned}
\end{equation}
\item
\begin{equation}\label{eqn:twoInterfaceNormal:60}
\begin{aligned}
\rightarrow &\qquad A t e^{-j k_{3z} (z-d)}
\end{aligned}
\end{equation}
\end{enumerate}
%
Matching at \( z = 0 \) gives
\begin{equation}\label{eqn:twoInterfaceNormal:80}
\begin{aligned}
A t_{12} + r_{21} D &= C \\
A r      &= A r_{12} + D t_{21},
\end{aligned}
\end{equation}
%
whereas matching at \( z = d \) gives
%
\begin{equation}\label{eqn:twoInterfaceNormal:100}
\begin{aligned}
A t &= C e^{-j k_{2z} d} t_{23} \\
D e^{j k_{2z} d} &= C e^{-j k_{2z} d} r_{23}.
\end{aligned}
\end{equation}
%
We have four linear equations in four unknowns \( r, t, C, D \), but only care about solving for \( r, t \).  Let's write \(
\gamma = e^{ j k_{2z} d }, C' = C/A, D' = D/A \), for
%
\begin{equation}\label{eqn:twoInterfaceNormal:120}
\begin{aligned}
t_{12} + r_{21} D' &= C' \\
r      &= r_{12} + D' t_{21} \\
t \gamma &= C' t_{23} \\
D' \gamma^2 &= C' r_{23}.
\end{aligned}
\end{equation}
%
Solving for \( C', D' \) we get
%
\begin{equation}\label{eqn:twoInterfaceNormal:140}
\begin{aligned}
D' \lr{ \gamma^2 - r_{21} r_{23} } &= t_{12} r_{23} \\
C' \lr{ \gamma^2 - r_{21} r_{23} } &= t_{12} \gamma^2,
\end{aligned}
\end{equation}
%
so
%
\begin{equation}\label{eqn:twoInterfaceNormal:160}
\begin{aligned}
r &= r_{12} + \frac{t_{12} t_{21} r_{23} }{\gamma^2 - r_{21} r_{23} } \\
t &= t_{23} \frac{ t_{12} \gamma }{\gamma^2 - r_{21} r_{23} }.
\end{aligned}
\end{equation}
%
With \( \phi = -j k_{2z} d \), or \( \gamma = e^{-j\phi} \), we have
%
%\begin{dmath}\label{eqn:twoInterfaceNormal:180}
\boxedEquation{eqn:twoInterfaceNormal:180}{
\begin{aligned}
r &= r_{12} + \frac{t_{12} t_{21} r_{23} e^{2 j \phi} }{1 - r_{21} r_{23} e^{2 j \phi}} \\
t &= \frac{ t_{12} t_{23} e^{j\phi}}{1 - r_{21} r_{23} e^{2 j \phi}}.
\end{aligned}
}
%\end{dmath}
%
\paragraph{A slab}
%
When the materials in region I, and III are equal, then \( r_{12} = r_{32} \).  For a TE mode, we have
%
\begin{equation}\label{eqn:twoInterfaceNormal:200}
r_{12} =
\frac{\mu_2 k_{1z} - \mu_1 k_{2z}}{\mu_2 k_{1z} + \mu_1 k_{2z}}
= -r_{21}.
\end{equation}
%
so the reflection and transmission coefficients are
%
\begin{equation}\label{eqn:twoInterfaceNormal:220}
\begin{aligned}
r^{\textrm{TE}} &= r_{12} \lr{ 1 - \frac{t_{12} t_{21} e^{2 j \phi} }{1 - r_{21}^2 e^{2 j \phi}} } \\
t^{\textrm{TE}} &= \frac{ t_{12} t_{21} e^{j\phi}}{1 - r_{21}^2 e^{2 j \phi}}.
\end{aligned}
\end{equation}
%
It's possible to produce a matched condition for which \( r_{12} = r_{21} = 0 \), by selecting
%
\begin{dmath}\label{eqn:twoInterfaceNormal:240}
0
= \mu_2 k_{1z} - \mu_1 k_{2z}
= \mu_1 \mu_2 \lr{ \inv{\mu_1} k_{1z} - \inv{\mu_2} k_{2z} }
= \mu_1 \mu_2 \omega \lr{ \frac{1}{v_1 \mu_1} \theta_1 - \frac{1}{v_2 \mu_2} \theta_2 },
\end{dmath}
%
or
%
\begin{dmath}\label{eqn:twoInterfaceNormal:260}
\inv{\eta_1} \cos\theta_1 = \inv{\eta_2} \cos\theta_2,
\end{dmath}
%
so the matching condition for normal incidence is just
%
\begin{dmath}\label{eqn:twoInterfaceNormal:280}
\eta_1 = \eta_2.
\end{dmath}
%
Given this matched condition, the transmission coefficient for the 1,2 interface is
%
\begin{dmath}\label{eqn:twoInterfaceNormal:300}
t_{12}
= \frac{2 \mu_2 k_{1z}}{\mu_2 k_{1z} + \mu_1 k_{2z}}
= \frac{2 \mu_2 k_{1z}}{2 \mu_2 k_{1z} }
= 1,
\end{dmath}
%
so the matching condition yields
\begin{dmath}\label{eqn:twoInterfaceNormal:320}
t
=
t_{12} t_{21} e^{j\phi}
=
e^{j\phi}
=
e^{-j k_{2z} d}.
\end{dmath}

Normal transmission through a matched slab only introduces a phase delay.
%
%}
%\EndNoBibArticle
