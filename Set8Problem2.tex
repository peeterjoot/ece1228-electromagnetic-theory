%
% Copyright � 2016 Peeter Joot.  All Rights Reserved.
% Licenced as described in the file LICENSE under the root directory of this GIT repository.
%
\makeproblem{Two interfaces.}{emt:problemSet8:2}{
\makesubproblem{}{emt:problemSet8:2a}
Give the TE transmission function \( T^{\textrm{TE}}(\omega) \) for a slab of
length \( d \) with permittivity and permeability \( \epsilon_2, \mu_2 \), surrounded
by medium characterized by \( \epsilon_1 \)
and \( \mu_1 \)
as shown in \cref{fig:ps8:ps8Fig2}.
Make sure you provide the expressions for the terms appearing in the transmission function \( T^{\textrm{TE}}(\omega) \).
\makesubproblem{}{emt:problemSet8:2b}
Suppose medium (II) is a meta-material with \( \epsilon_2 = -\epsilon_1 \)
and \( \mu_2 = -\mu_1 \), where \( \epsilon_1 > 0 \) and \( \mu_1 > 0 \).
What is the transmission function \( T^{\textrm{TE}}(\omega) \) in this case. Express your results in terms
of the propagation constant in medium (I), i.e. \( k_{1z} \).
\makesubproblem{}{emt:problemSet8:2c}
Now consider a source located at \( z=0 \) generating
a uniform plane wave, and for simplicity suppose a one-dimensional propagation.
What is the field at the second interface \(z=2d\). What is the meaning of your results?
\imageFigure{../figures/ece1228-electromagnetic-theory/ps8Fig2}{Slab geometry.}{fig:ps8:ps8Fig2}{0.2}
} % makeproblem
\makeanswer{emt:problemSet8:2}{\withproblemsetsParagraph{
\makeSubAnswer{}{emt:problemSet8:2a}
As outlined in class,
%Assuming normal incidence,
we can solve the system assuming waves of the form sketched in \cref{fig:twoInterfaces:twoInterfacesFig2}.
\imageFigure{../figures/ece1228-electromagnetic-theory/twoInterfacesFig2}{Wave functions crossing two interfaces}{fig:twoInterfaces:twoInterfacesFig2}{0.3}
%\cref{fig:threeMediaWave:threeMediaWaveFig1}.
%\imageFigure{../figures/ece1228-electromagnetic-theory/threeMediaWaveFig1}{Waves crossing two interfaces}{fig:threeMediaWave:threeMediaWaveFig1}{0.25}

I've utilized the freedom to include arbitrary constant phase factors in the input and output wave functions.  The boundary conditions at \( z = d \) are
\begin{equation}\label{eqn:emtproblemSet8Problem2:20}
\begin{aligned}
A t_{12} + D r_{21} &= C \\
A r &= A r_{12} + D t_{21},
\end{aligned}
\end{equation}
and the boundary conditions at \( z = 2d \) are
\begin{equation}\label{eqn:emtproblemSet8Problem2:40}
\begin{aligned}
C t_{23} e^{-j k_{2z} d } &= A t \\
C e^{-j k_{2z} d} r_{23} &= D e^{+j k_{2z} d}
\end{aligned}
\end{equation}
Setting \( A = 1 \) for convenience, we have two equations relating \( C, D \)
\begin{equation}\label{eqn:emtproblemSet8Problem2:60}
\begin{aligned}
t_{12} + D r_{21} &= C \\
C e^{-j k_{2z} d} r_{23} &= D e^{+j k_{2z} d},
\end{aligned}
\end{equation}
where the reflection and transmitted amplitude scaling factors \( r, t \) both follow immediately once we have solved for \( C, D \)
\begin{equation}\label{eqn:emtproblemSet8Problem2:80}
\begin{aligned}
t &= C t_{23} e^{-j k_{2z} d } \\
r &= r_{12} + D t_{21}.
\end{aligned}
\end{equation}

Writing \( \gamma = e^{ j k_{2z} d } \) we can solve for \( C, D \) immediately
\begin{dmath}\label{eqn:emtproblemSet8Problem2:100}
\begin{bmatrix}
C \\
D
\end{bmatrix}
=
{\begin{bmatrix}
1 & - r_{21}  \\
r_{23} & - \gamma^2
\end{bmatrix}}^{-1}
\begin{bmatrix}
t_{12} \\
0
\end{bmatrix}
=
\inv{ \gamma^2 - r_{21} r_{23} }
\begin{bmatrix}
\gamma^2 & -r_{21}  \\
r_{23} & -1
\end{bmatrix}
\begin{bmatrix}
t_{12} \\
0
\end{bmatrix}
=
\frac{t_{12}}{ \gamma^2 - r_{21} r_{23} }
\begin{bmatrix}
\gamma^2  \\
r_{23}
\end{bmatrix}.
\end{dmath}

For the reflection amplitude we have
\begin{dmath}\label{eqn:emtproblemSet8Problem2:120}
r
= r_{12} + D t_{21}
= r_{12} +
\frac{t_{21} t_{12} r_{23}}{ \gamma^2 - r_{21} r_{23} }
= r_{12} +
\frac{t_{21} t_{12} r_{23} e^{-2 j k_{2z} d}}{ 1 - r_{21} r_{23} e^{-2 j k_{2z} d}}
\end{dmath}
and for the transmission amplitude, we have
\begin{dmath}\label{eqn:emtproblemSet8Problem2:140}
t
= C \frac{t_{23}}{\gamma}
=
\frac{t_{23} t_{12}\gamma^2}{\gamma( \gamma^2 - r_{21} r_{23}) }
=
\frac{t_{23} t_{12}\gamma}{\gamma^2 - r_{21} r_{23} }
=
\frac{t_{23} t_{12}e^{-j k_{2z} d}}{1 - r_{21} r_{23} e^{-2 j k_{2z} d} }.
\end{dmath}
Noting that for normal incident TE mode, the individual amplitudes are
\begin{equation}\label{eqn:emtproblemSet8Problem2:160}
\begin{aligned}
r_{12} &= \frac{\mu_2 k_{1z} - \mu_1 k_{2z}}{\mu_2 k_{1z} + \mu_1 k_{2z}} = - r_{21} \\
t_{12} &= \frac{2 \mu_2 k_{1z}}{\mu_2 k_{1z} + \mu_1 k_{2z}} \\
t_{21} &= \frac{2 \mu_1 k_{2z}}{\mu_1 k_{2z} + \mu_2 k_{1z}} \\
\end{aligned}
\end{equation}
and setting \( 3 = 1 \), the transmission function for this problem is
\begin{dmath}\label{eqn:emtproblemSet8Problem2:180}
T(\omega)
=
\frac{t_{21} t_{12}e^{-j k_{2z} d}}{1 - r_{21} r_{21} e^{-2 j k_{2z} d} },
\end{dmath}
we have
\begin{equation}\label{eqn:emtproblemSet8Problem2:200}
\begin{aligned}
r_{21}^2 &=
\frac{(\mu_2 k_{1z} - \mu_1 k_{2z})^2}{(\mu_2 k_{1z} + \mu_1 k_{2z})^2} \\
t_{12} t_{21} &=
4 \frac{(\mu_2 k_{1z})(\mu_1 k_{2z})}{(\mu_2 k_{1z} + \mu_1 k_{2z})^2},
\end{aligned}
\end{equation}
so the transmission function is
\begin{dmath}\label{eqn:emtproblemSet8Problem2:220}
T(\omega)
=
\frac{4 \mu_1 \mu_2 k_{1z} k_{2z} e^{-j k_{2z} d}}{(\mu_2 k_{1z} + \mu_1 k_{2z})^2 - (\mu_2 k_{1z} - \mu_1 k_{2z})^2 e^{-2 j k_{2z} d} }.
\end{dmath}

The angular frequency dependence is carried by \( k_{iz} = (\omega/c) n_i(\omega) \cos\theta_i \).  In particular, when \( n_i \) does not depend on frequency, this means that all the frequency dependence is carried by the
%  In the one interface problem, that completely kills any explicit frequency dependence, but we have such dependence here,
phase factor \( e^{-j k_{2z} d } \) terms.
%  That is
%
%%\begin{dmath}\label{eqn:emtproblemSet8Problem2:240}
%\boxedEquation{eqn:emtproblemSet8Problem2:240}{
%T(\omega)
%=
%\frac{4 \mu_1 \mu_2 n_1 \cos\theta_1 n_2 \cos\theta_2 e^{-j \omega n_2 \cos\theta_2 d/c}}{(\mu_2 n_1 \cos\theta_1 + \mu_1 n_2\cos\theta_2 )^2 - (\mu_2 n_1 \cos\theta_1 - \mu_1 n_2 \cos\theta_2)^2 e^{-2 j \omega n_2 \cos\theta_2 d/c} }.
%}
%%\end{dmath}
%
%Here \( \theta_2 \) is related to \( \theta_1 \) are related by Snell's second law
%
%\begin{dmath}\label{eqn:emtproblemSet8Problem2:360}
%\theta_2 = \sin^{-1} \lr{ \frac{n_1}{n_2} \sin\theta_1 }.
%\end{dmath}
%
\makeSubAnswer{}{emt:problemSet8:2b}
The index of refraction, in terms of \( \mu, \epsilon \) is
\begin{dmath}\label{eqn:emtproblemSet8Problem2:260}
n = \frac{c}{v} = c \sqrt{ \mu \epsilon },
\end{dmath}
so
\begin{dmath}\label{eqn:emtproblemSet8Problem2:280}
n_2
= c \sqrt{ \mu_2 \epsilon_2 }
= c \sqrt{ (-\mu_1)(-\epsilon_1) }
= n_1.
\end{dmath}

The slab wave vector \( k_{2z} \) can be expressed in terms of \( k_{1z} \)
\begin{dmath}\label{eqn:emtproblemSet8Problem2:380}
k_{2z}
= \frac{\omega}{c} n_2 \cos\theta_2
= \frac{\omega}{c} n_1 \cos\theta_2
= k_{1z} \frac{\cos\theta_2}{\cos\theta_1}.
\end{dmath}

Let
\begin{dmath}\label{eqn:emtproblemSet8Problem2:400}
\alpha
=
\frac{\cos\theta_2}{\cos\theta_1}
=
\frac{\cos\lr{ \sin^{-1} \lr{ \frac{n_1}{n_2} \sin\theta_1 } }}
{\cos\theta_1},
\end{dmath}
The transmission function is
\begin{dmath}\label{eqn:emtproblemSet8Problem2:300}
T(\omega)
=
\frac{4 \mu_1 (-\mu_1) k_{1z}^2 \alpha e^{-j k_{1z} \alpha d}}{((-\mu_1) k_{1z} + \mu_1 k_{1z} \alpha)^2 - ((-\mu_1) k_{1z} - \mu_1 k_{1z} \alpha)^2 e^{-2 j k_{1z} \alpha d} }
=
\frac{-4 \alpha e^{-j k_{1z} \alpha d}}{(-1 + \alpha)^2 - (1 + \alpha)^2 e^{-2 j k_{1z} \alpha d} }.
%\frac{4 \mu_1 \mu_2 e^{-j \omega n_1 d/c}}{(\mu_2 + \mu_1 )^2 - (\mu_2 - \mu_1 )^2 e^{-2 j \omega n_1 d/c} }
%=
%\frac{-4 \mu_1^2 e^{-j \omega n_1 d/c}}{(-\mu_1 + \mu_1 )^2 - (-\mu_1 - \mu_1 )^2 e^{-2 j \omega n_1 d/c} }
%=
%\frac{e^{-j \omega n_1 d/c}}{e^{-2 j \omega n_1 d/c} },
\end{dmath}
\makeSubAnswer{}{emt:problemSet8:2c}
From Snell's second law, observe that a normal incident source remains normal in the slab
\begin{equation}\label{eqn:emtproblemSet8Problem2:460}
n_1 \sin 0       = n_2 \sin\theta_2,
\end{equation}
so \( \theta_2 = 0 \).  For small angles, say \( \theta_1 = \epsilon \), we have
\begin{dmath}\label{eqn:emtproblemSet8Problem2:480}
\theta_2
= \sin^{-1} \lr{ \frac{ n_1 \epsilon }{n_2} }
\approx \frac{ n_1 }{n_2} \epsilon,
\end{dmath}
so 1D propagation can be realized provided \( (\ifrac{ n_1 }{n_2}) \theta_1 \ll 1 \).
%A 1D propagation is not
%physically realizable, since normal incident light will no longer have normal direction within the slab due to diffraction.
%However, one can suppose that there are situations where \( \theta_1 \approx \theta_2 \) with sufficient closeness that \( \cos\theta_2/\cos\theta_1 \approx 1 \).
Given such a normal source,
the transmission function reduces to just a phase shift (a time shift in the time domain)
\begin{dmath}\label{eqn:emtproblemSet8Problem2:420}
T(\omega)
=
\frac{-4 e^{-j k_{1z} d}}{(-2)^2 e^{-2 j k_{1z} d} }
=
e^{j k_{1z} d}.
\end{dmath}
The field at the second interface, assuming the input wave amplitude \( A \) is real, is
\begin{dmath}\label{eqn:emtproblemSet8Problem2:440}
\psi(2d, t)
= \Real \lr{ A e^{j k_{1z} d} e^{j \omega t} }
= A \cos\lr{ \omega t + k_{1z} d }
= A \cos\lr{ \omega \lr{ t + \frac{k_{1z}}{\omega} d } }.
\end{dmath}

Instead of a phase delay, the specified meta-material results in an output wave that has a phase lead with respect to the source.
}}
