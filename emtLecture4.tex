%
% Copyright � 2016 Peeter Joot.  All Rights Reserved.
% Licenced as described in the file LICENSE under the root directory of this GIT repository.
%
%\input{../blogpost.tex}
%\renewcommand{\basename}{emt4}
%\renewcommand{\dirname}{notes/ece1228/}
%\newcommand{\keywords}{ECE1228H}
%\input{../latex/peeter_prologue_print2.tex}
%
%%\usepackage{ece1228}
%\usepackage{peeters_braket}
%%\usepackage{peeters_layout_exercise}
%\usepackage{peeters_figures}
%\usepackage{mathtools}
%\usepackage{siunitx}
%
%\beginArtNoToc
%\generatetitle{ECE1228H Electromagnetic Theory.  Lecture 4: Magnetic moment, and Boundary value conditions.  Taught by Prof.\ M. Mojahedi}
%\mychapter{Magnetic moment, and Boundary value conditions.}
%\label{chap:emt4}
%
%\paragraph{Disclaimer}
%
%Peeter's lecture notes from class.  These may be incoherent and rough.
%
%These are notes for the UofT course ECE1228H, Electromagnetic Theory, taught by Prof. M. Mojahedi, covering \textchapref{{1}} \citep{balanis1989advanced} content.
%
\section{Magnetic moment.}
\index{magnetic moment}
\index{moment!magnetic}
%
Using a semi-classical model of an electron, assuming that the electron circles the nuclei.  This is a completely wrong model, but useful.  In reality, electrons are random and probabilistic and do not follow defined paths.  We do however have a magnetic moment associated with the electron, and one associated with the spin of the electron, and a moment associated with the spin of the nuclei.  All of these concepts can be used to describe a more accurate model and such a model is discussed in \citep{jackson1975cew} chapters 11,12,13.
%
Ignoring the details of how the moments really occur physically, we can take it as a given that they exist, and model them as elemental magnetic dipole moments of the form
%
\begin{dmath}\label{eqn:emtLecture4:20}
d\Bm_i = \ncap_i I_i ds_i \qquad [\si{A m^2}].
\end{dmath}
%
Note that \( ds_i \) is an element of surface area, not arc length!
%
Here the normal is defined in terms of the right hand rule with respect to the direction of the current as sketched in \cref{fig:emtLecture4:emtLecture4Fig1}.
\imageFigure{../figures/ece1228-electromagnetic-theory/emtLecture4Fig1}{Orientation of current loop.}{fig:emtLecture4:emtLecture4Fig1}{0.3}
%
Such dipole moments are actually what an MRI measures.  The noises that people describe from MRI machines are actually when the very powerful magnets are being rotated, allowing for the magnetic moments in the atoms of the body to be measured in different directions.
%
\index{magnetic polarization}
\index{magnetization}
The magnetic polarization, or magnetization \( \BM \), in [\si{A/m}]] is given by
%
\begin{dmath}\label{eqn:emtLecture4:40}
\BM
= \lim_{\Delta v \rightarrow 0} \lr{ \inv{\Delta v} \Bm_i }
= \lim_{\Delta v \rightarrow 0} \lr{ \inv{\Delta v} \sum_{i = 1}^{N \delta v} d\Bm_i }
= \lim_{\Delta v \rightarrow 0} \lr{ \inv{\Delta v} \sum_{i = 1}^{N \delta v} \ncap_i I_i ds_i } .
\end{dmath}
%
In materials the magnetization within the atoms are usually random, however, application of a magnetic field can force these to line up, as sketched in \cref{fig:emtLecture4:emtLecture4Fig2}.
%
\imageFigure{../figures/ece1228-electromagnetic-theory/emtLecture4Fig2}{External magnetic field alignment of magnetic moments.}{fig:emtLecture4:emtLecture4Fig2}{0.3}
%
\index{torque}
This is accomplished because an applied magnetic field acting on the magnetic moment introduces a torque, as also occurred with dipole moments under applied electric fields
%
\begin{equation}\label{eqn:emtLecture4:60}
\begin{aligned}
\Btau_B &= d\Bm \cross \BB_a, \\
\Btau_E &= d\Bp \cross \BE_a.
\end{aligned}
\end{equation}
%
\index{energy!torque}
There is an energy associated with this torque
%
\begin{equation}\label{eqn:emtLecture4:80}
\begin{aligned}
\Delta U_B &= -d\Bm \cdot \BB_a \\
\Delta U_E &= -d\Bp \cdot \BE_a.
\end{aligned}
\end{equation}
%
In analogy with the electric dipole moment analysis, it can be assumed that there is a linear relationship between the magnetic polarization and the applied magnetic field
%
\begin{dmath}\label{eqn:emtLecture4:100}
\BB = \mu_0 \BH_a + \mu_0 \BM = \mu_0\lr{ \BH_a + \BM },
\end{dmath}
%
where
\begin{dmath}\label{eqn:emtLecture4:120}
\BM = \chi_m \BH_a,
\end{dmath}
%
so
\begin{equation}\label{eqn:emtLecture4:140}
\BB
= \mu_0\lr{ 1 + \chi_m } \BH_a
\equiv \mu \BH_a.
\end{equation}
%
Like electric dipoles, in a volume, we can have bound currents on the surface [\si{A/m}], as well as bound volume currents [\si{A/m^2}].
% as sketched in
%
%F3
%
It can be shown, as with the electric dipoles related bound charge densities of \cref{eqn:emtLecture3:620}, that magnetic currents can be defined
%
\begin{equation}\label{eqn:emtLecture4:160}
\begin{aligned}
\BJ_{sm} &= \BM \cross \ncap, \\
\BJ_{vm} &= \spacegrad \cross \BM.
\end{aligned}
\end{equation}
%
\section{Conductivity.}
\index{conductivity}
%
\index{constitutive relationships}
We have two constitutive relationships so far
\begin{equation}\label{eqn:emtLecture4:180}
\begin{aligned}
\BD &= \epsilon \BE,\\
\BB &= \mu \BH,
\end{aligned}
\end{equation}
%
but these need to be augmented by Ohm's law
%
\index{Ohm's law}
\begin{dmath}\label{eqn:emtLecture4:200}
\BJ_c = \epsilon \BE.
\end{dmath}
%
There are a couple ways to discuss this.  One is to model \( \epsilon \) as a complex number.  Such a model is not entirely unconstrained.  Like with the Cauchy-Riemann conditions that relate derivatives of the real and imaginary parts of a complex number, there is a relationship (Kramers-Kronig \citep{wiki:kramersKronig}), an integral relationship that relates the real and imaginary parts of the permittivity \( \epsilon \).
%
