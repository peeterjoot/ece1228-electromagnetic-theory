%
% Copyright � 2016 Peeter Joot.  All Rights Reserved.
% Licenced as described in the file LICENSE under the root directory of this GIT repository.
%
%\input{../blogpost.tex}
%\renewcommand{\basename}{emt4}
%\renewcommand{\dirname}{notes/ece1228/}
%\newcommand{\keywords}{ECE1228H}
%\input{../latex/peeter_prologue_print2.tex}
%
%%\usepackage{ece1228}
%\usepackage{peeters_braket}
%%\usepackage{peeters_layout_exercise}
%\usepackage{peeters_figures}
%\usepackage{mathtools}
%\usepackage{siunitx}
%
%\beginArtNoToc
%\generatetitle{ECE1228H Electromagnetic Theory.  Lecture 4: Magnetic moment, and Boundary value conditions.  Taught by Prof.\ M. Mojahedi}
\chapter{Magnetic moment, and Boundary value conditions}
%\label{chap:emt4}

%\paragraph{Disclaimer}
%
%Peeter's lecture notes from class.  These may be incoherent and rough.
%
%These are notes for the UofT course ECE1228H, Electromagnetic Theory, taught by Prof. M. Mojahedi, covering \textchapref{{1}} \citep{balanis1989advanced} content.

\paragraph{Magnetic moment.}
\index{magnetic moment}
\index{moment!magnetic}

Using a semi-classical model of an electron, assuming that the electron circles the nuclei.  This is a completely wrong model, but useful.  In reality, electrons are random and probabilistic and do not follow defined paths.  We do however have a magnetic moment associated with the electron, and one associated with the spin of the electron, and a moment associated with the spin of the nuclei.  All of these concepts can be used to describe a more accurate model and such a model is discussed in \citep{jackson1975cew} chapters 11,12,13.

Ignoring the details of how the moments really occur physically, we can take it as a given that they exist, and model them as elemenetal magnetic dipole moments of the form

\begin{dmath}\label{eqn:emtLecture4:20}
d\Bm_i = \ncap_i I_i ds_i \qquad [\si{A m^2}].
\end{dmath}

Note that \( ds_i \) is an element of surface area, not arc length!

Here the normal is defined in terms of the right hand rule with respect to the direction of the current as sketched in \cref{fig:emtLecture4:emtLecture4Fig1}.
\imageFigure{../figures/ece1228-electromagnetic-theory/emtLecture4Fig1}{Orientation of current loop.}{fig:emtLecture4:emtLecture4Fig1}{0.3}

Such dipole moments are actually what an MRI measures.  The noises that people describe from MRI machines are actually when the very powerful magnets are being rotated, allowing for the magnetic moments in the atoms of the body to be measured in different directions.

\index{magnetic polarization}
\index{magnetization}
The magnetic polarization, or magnetization \( \BM \), in [\si{A/m}]] is given by

\begin{dmath}\label{eqn:emtLecture4:40}
\BM
= \lim_{\Delta v \rightarrow 0} \lr{ \inv{\Delta v} \Bm_i }
= \lim_{\Delta v \rightarrow 0} \lr{ \inv{\Delta v} \sum_{i = 1}^{N \delta v} d\Bm_i }
= \lim_{\Delta v \rightarrow 0} \lr{ \inv{\Delta v} \sum_{i = 1}^{N \delta v} \ncap_i I_i ds_i } .
\end{dmath}

In materials the magnetization within the atoms are usually random, however, application of a magnetic field can force these to line up, as sketched in \cref{fig:emtLecture4:emtLecture4Fig2}.

\imageFigure{../figures/ece1228-electromagnetic-theory/emtLecture4Fig2}{External magnetic field alignment of magnetic moments.}{fig:emtLecture4:emtLecture4Fig2}{0.3}

\index{torque}
This is accomplished because an applied magnetic field acting on the magnetic moment introduces a torque, as also occured with dipole moments under applied electric fields

\begin{dmath}\label{eqn:emtLecture4:60}
\begin{aligned}
\Btau_B &= d\Bm \cross \BB_a \\
\Btau_E &= d\Bp \cross \BE_a.
\end{aligned}
\end{dmath}

\index{energy!torque}
There is an energy associated with this torque

\begin{dmath}\label{eqn:emtLecture4:80}
\begin{aligned}
\Delta U_B &= -d\Bm \cdot \BB_a \\
\Delta U_E &= -d\Bp \cdot \BE_a.
\end{aligned}
\end{dmath}

In analogy with the electric dipole moment analysis, it can be assumed that there is a linear relationship between the magnetic polarization and the applied magnetic field

\begin{dmath}\label{eqn:emtLecture4:100}
\BB = \mu_0 \BH_a + \mu_0 \BM = \mu_0\lr{ \BH_a + \BM },
\end{dmath}

where
\begin{dmath}\label{eqn:emtLecture4:120}
\BM = \chi_m \BH_a,
\end{dmath}

so
\begin{equation}\label{eqn:emtLecture4:140}
\BB
= \mu_0\lr{ 1 + \chi_m } \BH_a
\equiv \mu \BH_a.
\end{equation}

Like electric dipoles, in a volume, we can have bound currents on the surface [\si{A/m}], as well as bound volume currents [\si{A/m^2}].
% as sketched in
%
%F3

It can be shown, as with the electric dipoles related bound charge densities of \cref{eqn:emtLecture3:620}, that magnetic currents can be defined

\begin{dmath}\label{eqn:emtLecture4:160}
\begin{aligned}
\BJ_{sm} &= \BM \cross \ncap \\
\BJ_{vm} &= \spacegrad \cross \BM,
\end{aligned}
\end{dmath}

\paragraph{Conductivity}
\index{conductivity}

\index{constitutive relationships}
We have two constitutive relationships so far
\begin{dmath}\label{eqn:emtLecture4:180}
\begin{aligned}
\BD &= \epsilon \BE \\
\BB &= \mu \BH
\end{aligned}
\end{dmath}

but this needs to be augmented by

\index{Ohm's law}
\begin{dmath}\label{eqn:emtLecture4:200}
\BJ_c = \epsilon \BE.
\end{dmath}

There are a couple ways to discuss this.  One is to model \( \epsilon \) as a complex number.  Such a model is not entirely unconstrained.  Like with the Cauchy-Riemann conditions that relate derivatives of the real and imaginary parts of a complex number, there is a relationship (Kramers-Kronig \citep{wiki:kramersKronig}), an integral relationship that relates the real and imaginary parts of the permittivity \( \epsilon \).

\paragraph{Boundary conditions.}
\index{boundary conditions}

The boundary conditions are

\begin{itemize}
\item \( \ncap \cross \lr{ \BE_2 - \BE_1 } = - \BM_s \)
This means that the tangential components of \( \BE \) is continuous accross the boundary (those components of \(\BE_1,\BE_2\) are equal on the boundary), when \( \BM_s \) is zero.

Here \( \BM_s \) is the (fictitious) magnetic current density in [\si{V/m}].

\item \( \ncap \cross \lr{ \BH_2 - \BH_1 } = \BJ_s \)

\index{tangential field component}
This means that the tangential components of the magnetic fields \( \BH \) are discontinous when the electric surface current density \( \BJ_s \) [\si{A/m}] is non-zero, but continuous otherwise.  The latter is sketched in \cref{fig:emtLecture4:emtLecture4Fig5}.

\imageFigure{../figures/ece1228-electromagnetic-theory/emtLecture4Fig5}{Equal tangential fields.}{fig:emtLecture4:emtLecture4Fig5}{0.2}

Here \( \BJ_s \) is the movement of the free current on the surface.  The bound charges are incorporated into \( \BD \).

\item \( \ncap \cdot \lr{ \BD_2 - \BD_1 } = \rho_{es} \)

\index{normal field component}
Here \( \rho_{es} \) is the electric surface charge density [\si{C/m^2}].

This means that the normal component of the electric displacement field \( \BD \) is discontinuous accross the boundary in the presence of electric surface charge densities, but continuous when that is zero.

\item \( \ncap \cdot \lr{ \BB_2 - \BB_1 } = \rho_{ms} \)

\index{magnetic surface charge density}
Here \( \rho_{ms} \) is the (fictional) magnetic surface charge density [\si{Weber/m^2}].

This means that the magnetic fields \( \BB \) are continous in the abscense of (fictional) magnetic surface charge densities.

\end{itemize}

In the abscence of any free charges or currents, these relationships are considerably simplified

\begin{subequations}
\label{eqn:emtLecture4:220}
\begin{dmath}\label{eqn:emtLecture4:240}
\ncap \cross \lr{ \BE_2 - \BE_1 } = 0
\end{dmath}
\begin{dmath}\label{eqn:emtLecture4:260}
\ncap \cross \lr{ \BH_2 - \BH_1 } = 0
\end{dmath}
\begin{dmath}\label{eqn:emtLecture4:280}
\ncap \cdot \lr{ \BD_2 - \BD_1 } = 0
\end{dmath}
\begin{dmath}\label{eqn:emtLecture4:420}
\ncap \cdot \lr{ \BB_2 - \BB_1 } = 0
\end{dmath}
\end{subequations}

To get an idea where these come from, consider the derivation of \cref{eqn:emtLecture4:260}, relating the tangential components of \( \BH \), as sketched in \cref{fig:emtLecture4:emtLecture4Fig4}.

\imageFigure{../figures/ece1228-electromagnetic-theory/emtLecture4Fig4}{Boundary geometry.}{fig:emtLecture4:emtLecture4Fig4}{0.3}

Integrating over such a loop, the integral version of the Ampere-Maxwell equation \cref{eqn:emtLecture1:40}, with \( \BJ = \sigma \BE \) is

\begin{dmath}\label{eqn:emtLecture4:300}
\oint_C \BH \cdot d\Bl = \int_S \sigma \BE \cdot d\Bs + \PD{t}{} \int_S \BD \cdot d\Bs.
\end{dmath}

In the limit, with the height \( \Delta y \rightarrow 0 \), this is

\begin{dmath}\label{eqn:emtLecture4:320}
\oint_C \BH \cdot d\Bl
\approx
H_1 \cdot (\Delta x \xcap)
-H_2 \cdot (\Delta x \xcap)
\end{dmath}

Similarly
\begin{dmath}\label{eqn:emtLecture4:340}
\int_S \BD \cdot d\Bs
\approx
\BD \cdot \zcap \Delta x \Delta y,
\end{dmath}

and
\begin{equation}\label{eqn:emtLecture4:360}
\int_S \BJ \cdot d\Bs
=
\int_S \sigma \BE \cdot d\Bs
\approx
\sigma \BE \cdot \zcap \Delta x \Delta y,
\end{equation}

However, if \( \Delta y \) approaches zero, both of these terms are killed.

This gives

\begin{dmath}\label{eqn:emtLecture4:380}
\xcap \cdot \lr{ \BH_2 - \BH_1 } = 0.
\end{dmath}

If you were to perform the same calculation using a loop in the y-z plane you'd find

\begin{dmath}\label{eqn:emtLecture4:460}
\zcap \cdot \lr{ \BH_2 - \BH_1 } = 0.
\end{dmath}

Either way, the tangential component of \( \BH \) is continous on the boundary.

This derivation, using explicit components, follows \citep{balanis1989advanced}.  Non coordinate derivations are also possible (reference?).

The idea is that

\begin{dmath}\label{eqn:emtLecture4:440}
\ncap \cross \lr{ (\BH_2 - \BH_{2n}) -(\BH_1 - \BH_{1n}) }
=
\ncap \cross \lr{ \BH_2 - \BH_1 }
= 0.
\end{dmath}

What if there is a surface current?

\begin{dmath}\label{eqn:emtLecture4:400}
\lim_{\Delta y \rightarrow 0} \BJ_{ic} \Delta y = \BJ_s.
\end{dmath}

When this is the case the \( \BJ = \sigma \BE \) needs to be fixed up a bit, and showing how is left to a problem.
%.  In a problem \( \BJ_s \) is the current that is going through the elemental surface considered.

In the notes the other boundary relations are derived.  The normal ones follow by integrating over a pillbox volume.

Variations include the cases when one of the surfaces is made a perfect conductor.  Such a case can be treated by noting that the \( \BE \) field must be zero.

\paragraph{Conducting media.}

It will be left to homework to show, using the continuity equation and Gauss's law that
inside a conductor, that free charges distribute themselves exclusively on the surface on the medium.  Because of this there is no electric field inside the medium (Gauss's law).  What does this imply about the magnetic field in the same medium.  We must have

\begin{dmath}\label{eqn:emtLecture5:20}
\spacegrad \cross \BE = - \PD{t}{\BB}
\end{dmath}

so if \( \BE \) is zero in the medium the magnetic field must be either constant with respect to time, or zero.  In a general electrodynamic configuration, both the magnetic and electric fields vary with time, which seems to imply that \( \BB \) must be zero if \( \BE \) is zero in that space.

However, this is not consistent with what we see with an iron core inductor.  In such an inductor, the iron is used to
concentrate the magnetic field.  Clearly we have magnetic fields in the iron bar, since that is the purpose of it being there.  It turns out that if the frequencies are low enough (and even some smaller GHz frequencies are), then we can consider the system to be quasi-electrostatic, with zero electric fields inside a conductor, yet with finite approximately time independent magnetic fields.  As the frequencies are increased, the magnetic fields are forced out of the conductor into the surrounding space.

The transition point that defines the boundary between electrostatic and quasi-electrostatic will depend on the precision desired.

\paragraph{Boundary conditions with zero magnetic fields in a conductor}

For many calculations, we can proceed with the assumption that there are no appreciable electric nor magnetic fields inside of a conductor.  When that is the case, outside of a conducting medium, we have

\begin{dmath}\label{eqn:emtLecture5:40}
\ncap \cross \BE_2 = 0,
\end{dmath}

so there is no tangential component to an electric field of a conductor.  We also have

\begin{dmath}\label{eqn:emtLecture5:60}
\ncap \cdot \BD_2 = \rho_{es}
\end{dmath}

Assuming there is also no magnetic field either in the conductor, we also have

\begin{dmath}\label{eqn:emtLecture5:80}
\ncap \cross \BH_2 = \BJ_s,
\end{dmath}

and
\begin{dmath}\label{eqn:emtLecture5:100}
\ncap \cdot \BB_2 = 0.
\end{dmath}

There is no normal component to the magnetic field at the surface of a conductor, and the tangential component is determined by the surface current density.

%\EndArticle
