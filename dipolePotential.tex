%
% Copyright � 2016 Peeter Joot.  All Rights Reserved.
% Licenced as described in the file LICENSE under the root directory of this GIT repository.
%
%{
%\input{../blogpost.tex}
%\renewcommand{\basename}{dipolePotential}
%%\renewcommand{\dirname}{notes/phy1520/}
%\renewcommand{\dirname}{notes/ece1228-electromagnetic-theory/}
%%\newcommand{\dateintitle}{}
%%\newcommand{\keywords}{}
%
%\input{../latex/peeter_prologue_print2.tex}
%
%\usepackage{peeters_layout_exercise}
%\usepackage{peeters_braket}
%\usepackage{peeters_figures}
%\usepackage{siunitx}
%
%\beginArtNoToc
%
%\generatetitle{Electric dipole potential}
%\chapter{Electric dipole potential}
%\label{chap:dipolePotential}
%
\makeproblem{Electric dipole potential.}{problem:dipolePotential:1}{
\index{potential!electric dipole}
\index{dipole!potential}
%
Having shown that
%
\begin{dmath}\label{eqn:dipolePotential:20}
\BE =
\frac{1}{4 \pi \epsilon_0 r^3} \lr{
3 \rcap \lr{ \rcap \cdot \Bp }
-\Bp
},
\end{dmath}
%
find the expression for the electric potential for this field.
} % problem
%
\makeanswer{problem:dipolePotential:1}{
%\withproblemsetsParagraph{
%
With the electric potential defined indirectly by
\begin{dmath}\label{eqn:dipolePotential:40}
\BE = -\spacegrad V,
\end{dmath}
%
we can integrate to find the difference in potential between two points
\begin{dmath}\label{eqn:dipolePotential:60}
\int_\Ba^\Bb \BE \cdot d\Bl =
- \int
\int_\Ba^\Bb \spacegrad V \cdot d\Bl
=
- \lr{ V(\Bb) - V(\Ba) },
\end{dmath}
%
or
\begin{dmath}\label{eqn:dipolePotential:80}
V(\Bb) - V(\Ba) = -
\int_\Ba^\Bb \BE \cdot d\Bl.
\end{dmath}
%
Since the dipole potential is zero at \( \Br = \infty \), we have
%
\begin{dmath}\label{eqn:dipolePotential:100}
V(\Br)
= -\int_\infty^\Br \BE \cdot d\Bl.
\end{dmath}
%
Let's integrate this on the radial path \( \Br(r') = r'\rcap \), for \( r' \in [\infty, r] \)
%
\begin{dmath}\label{eqn:dipolePotential:120}
V(\Br)
= -\int_\infty^\Br \BE \cdot d\Bl
= -\int_\infty^\Br \BE \cdot \rcap dr'
=
-
\frac{1}{4 \pi \epsilon_0 }
\int_\infty^r \frac{dr'}{{r'}^3}
\rcap
\cdot
\lr{
3 \rcap \lr{ \rcap \cdot \Bp }
-\Bp
}
=
-\frac{2}{4 \pi \epsilon_0 }
\int_\infty^r dr' \frac{\rcap\cdot \Bp}{{r'}^3}
=
\frac{\rcap \cdot \Bp}{4 \pi \epsilon_0 } \evalrange{ \inv{{r'}^2} }{\infty}{r},
\end{dmath}
%
so
%\begin{dmath}\label{eqn:dipolePotential:160}
\boxedEquation{eqn:dipolePotential:140}{
V(\Br) =
\frac{ \rcap \cdot \Bp}{4 \pi \epsilon_0 }.
}
%\end{dmath}
%}
} % answer
%
%}
%\EndNoBibArticle
