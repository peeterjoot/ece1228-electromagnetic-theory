%
% Copyright � 2016 Peeter Joot.  All Rights Reserved.
% Licenced as described in the file LICENSE under the root directory of this GIT repository.
%
\makeproblem{Conductor charge distribution on surface.}{emt:problemSet4:4}{
\index{conductor!charge dissipation}
\index{continuity equation}
We have stated that the boundary condition for a perfect conductor is such that there is
no electric field or charge distribution inside of the conductor. Here we will study the
dynamics of this process. Start with continuity equation
\( \spacegrad \cdot \BJ = -\PDi{t}{\rho} \), where \( \BJ \)
is the
current density [\si{A/m^2}] and \( \rho \) is the charge density [\si{C/m^3}]. Show that a charge (charge
density) placed inside a conductor will decay in an exponential manner.
} % makeproblem
\makeanswer{emt:problemSet4:4}{\withproblemsetsParagraph{
If we assume that the induced charge density is related by the conductance to the electric field
\begin{dmath}\label{eqn:emtProblemSet4Problem4:20}
\BJ = \sigma \BE,
\end{dmath}
then the continuity equation can be written as
%
\begin{dmath}\label{eqn:emtProblemSet4Problem4:40}
\PD{t}{\rho}
=
-\spacegrad \cdot \BJ
=
- \sigma \spacegrad \cdot \BE
=
- \frac{\sigma}{\epsilon} \rho.
\end{dmath}
%
Now we have an equation for \( \rho \), with solution
%\begin{dmath}\label{eqn:emtProblemSet4Problem4:60}
\boxedEquation{eqn:emtProblemSet4Problem4:80}{
\rho = \rho_0 e^{-\sigma t/\epsilon}.
}
%\end{dmath}
%For copper, we have
%
% http://maxwells-equations.com/materials/conductivity.php
%\sigma = 6.3 * 10^7
Given this dispersion relation for the charge density, we can also find the normal component of the current density using the divergence theorem.  Evaluating the continuity equation in an infinitesimal spherical volume of radius \( R \) surrounding the initial charge \( \rho_0 dV \) that was placed at the centre of that sphere, far enough from the boundary of the conductor that we can ignore it, we find
\begin{dmath}\label{eqn:emtProblemSet4Problem4:100}
\oint \ncap \cdot \BJ dA = - \PD{t}{} \int \rho_0 e^{-\sigma t/\epsilon} dV,
\end{dmath}
or
\begin{dmath}\label{eqn:emtProblemSet4Problem4:120}
J_n \Delta A = \frac{\sigma}{\epsilon} \rho_0 e^{-\sigma t/\epsilon} \Delta V.
\end{dmath}
%
Assuming that such a current is radial by symmetry, this provides the dispersion relation for the current out of this same volume
\begin{equation}\label{eqn:emtProblemSet4Problem4:140}
\begin{aligned}
\BJ &= \BJ_0 e^{-\sigma t/\epsilon} \\
\BJ_0 &= \ncap \frac{\sigma}{\epsilon} \frac{R}{3} \rho_0.
\end{aligned}
\end{equation}
We have a positive current out of the volume that the initial charge \( \rho_0 \) was placed in (i.e. the charge leaving that space).  That charge density also decays with time as the charge dissipates, since there is no charge left to flow out of that space.
}}
