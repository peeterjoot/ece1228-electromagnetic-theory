%
% Copyright � 2016 Peeter Joot.  All Rights Reserved.
% Licenced as described in the file LICENSE under the root directory of this GIT repository.
%
%\input{../blogpost.tex}
%\renewcommand{\basename}{emt10}
%\renewcommand{\dirname}{notes/ece1228/}
%\newcommand{\keywords}{ECE1228H}
%\input{../latex/peeter_prologue_print2.tex}
%
%%\usepackage{ece1228}
%\usepackage{peeters_braket}
%%\usepackage{peeters_layout_exercise}
%\usepackage{peeters_figures}
%\usepackage{mathtools}
%\usepackage{siunitx}
%\usepackage{macros_bm}
%
%\beginArtNoToc
%\generatetitle{ECE1228H Electromagnetic Theory.  Lecture 10: Fresnel relations.  Taught by Prof.\ M. Mojahedi}
%%\chapter{Fresnel relations}
%\label{chap:emt10}
%
%\paragraph{Motivation}
%
%In class, an overview of the Fresnel relations for a TE mode electric field were presented.  Here's a fleshing out of the details is presented, as well as the equivalent for the TM mode.
%
%Peeter's lecture notes from class.  These may be incoherent and rough.
%
%These are notes for the UofT course ECE1228H, Electromagnetic Theory, taught by Prof. M. Mojahedi, covering \textchapref{{1}} \citep{balanis1989advanced} content.
%
\section{Single interface TE mode.}
%
The Fresnel reflection geometry for an electric field \( \BE \) parallel to the interface (TE mode) is sketched in \cref{fig:fresnelTE:fresnelTEFig1}.
%
\imageFigure{../figures/ece1228-electromagnetic-theory/fresnelTEFig1}{Electric field TE mode Fresnel geometry.}{fig:fresnelTE:fresnelTEFig1}{0.3}
%
\begin{dmath}\label{eqn:emtLecture10:20}
   \bcE_i = \Be_2 E_i e^{j \omega t - j \Bk_{i} \cdot \Bx },
\end{dmath}
%
with an assumption that this field maintains it's polarization in both its reflected and transmitted components, so that
%
\begin{dmath}\label{eqn:emtLecture10:40}
   \bcE_r = \Be_2 r E_i e^{j \omega t - j \Bk_{r} \cdot \Bx },
\end{dmath}
%
and
\begin{dmath}\label{eqn:emtLecture10:60}
   \bcE_t = \Be_2 t E_i e^{j \omega t - j \Bk_{t} \cdot \Bx },
\end{dmath}
%
Measuring the angles \( \theta_i, \theta_r, \theta_t \) from the normal, with \( i = \Be_3 \Be_1 \) the wave vectors are
%
\begin{equation}\label{eqn:emtLecture10:620}
\begin{aligned}
\Bk_{i} &= \Be_3 k_1 e^{i\theta_i} = k_1\lr{ \Be_3 \cos\theta_i + \Be_1\sin\theta_i }, \\
\Bk_{r} &= -\Be_3 k_1 e^{-i\theta_r} = k_1 \lr{ -\Be_3 \cos\theta_r + \Be_1 \sin\theta_r }, \\
\Bk_{t} &= \Be_3 k_2 e^{i\theta_t} = k_2 \lr{ \Be_3 \cos\theta_t + \Be_1 \sin\theta_t }.
\end{aligned}
\end{equation}
%
So the time harmonic electric fields are
%
\begin{equation}\label{eqn:emtLecture10:640}
\begin{aligned}
   \BE_i &= \Be_2 E_i \exp\lr{ - j k_1 \lr{ z\cos\theta_i + x \sin\theta_i} }, \\
   \BE_r &= \Be_2 r E_i \exp\lr{ - j k_1 \lr{ -z \cos\theta_r + x \sin\theta_r}}, \\
   \BE_t &= \Be_2 t E_i \exp\lr{ - j k_2 \lr{ z \cos\theta_t + x \sin\theta_t}}.
\end{aligned}
\end{equation}
%
The magnetic fields follow from Faraday's law
%
\begin{dmath}\label{eqn:emtLecture10:900}
\BH
= \inv{-j \omega \mu } \spacegrad \cross \BE
= \inv{-j \omega \mu } \spacegrad \cross \Be_2 e^{-j \Bk \cdot \Bx}
= \inv{j \omega \mu } \Be_2 \cross \spacegrad e^{-j \Bk \cdot \Bx}
= -\inv{\omega \mu } \Be_2 \cross \Bk e^{-j \Bk \cdot \Bx}
= \inv{\omega \mu } \Bk \cross \BE.
\end{dmath}
%
We have
%
\begin{equation}\label{eqn:emtLecture10:920}
\begin{aligned}
\kcap_{i} \cross \Be_2 &= -\Be_1 \cos\theta_i + \Be_3\sin\theta_i  \\
\kcap_{r} \cross \Be_2 &= \Be_1 \cos\theta_r + \Be_3 \sin\theta_r  \\
\kcap_{t} \cross \Be_2 &= -\Be_1 \cos\theta_t + \Be_3 \sin\theta_t,
\end{aligned}
\end{equation}
%
Note that
\begin{dmath}\label{eqn:emtLecture10:1500}
\frac{k}{\omega \mu}
=
\frac{k}{k v \mu}
=
\frac{\sqrt{\mu\epsilon}}{\mu}
=\sqrt
{
\frac{\epsilon}{\mu}
}
=
\inv{\eta}.
\end{dmath}
%
so
\begin{equation}\label{eqn:emtLecture10:940}
\begin{aligned}
\BH_{i} &= \frac{ E_i}{\eta_1} \lr{ -\Be_1 \cos\theta_i + \Be_3\sin\theta_i } \exp\lr{ - j k_1 \lr{ z\cos\theta_i + x \sin\theta_i} } \\
\BH_{r} &= \frac{ r E_i}{\eta_1} \lr{ \Be_1 \cos\theta_r + \Be_3 \sin\theta_r } \exp\lr{ - j k_1 \lr{ -z \cos\theta_r + x \sin\theta_r}} \\
\BH_{t} &= \frac{ t E_i}{\eta_2} \lr{ -\Be_1 \cos\theta_t + \Be_3 \sin\theta_t } \exp\lr{ - j k_2 \lr{ z \cos\theta_t + x \sin\theta_t}}.
\end{aligned}
\end{equation}
%
The boundary conditions at \( z = 0 \) with \( \ncap = \Be_3 \) are
%
\begin{equation}\label{eqn:emtLecture10:960}
\begin{aligned}
\ncap \cross \BH_1 &= \ncap \cross \BH_2, \\
\ncap \cdot \BB_1 &= \ncap \cdot \BB_2, \\
\ncap \cross \BE_1 &= \ncap \cross \BE_2, \\
\ncap \cdot \BD_1 &= \ncap \cdot \BD_2.
\end{aligned}
\end{equation}
%
%which gives
%
%\begin{subequations}
%\label{eqn:emtLecture10:980}
%\begin{dmath}\label{eqn:emtLecture10:1000}
%-\frac{k_1 }{\mu_1} \cos\theta_i \exp\lr{ - j k_1 x \sin\theta_i }
%+
%\frac{k_1 r }{\mu_1} \cos\theta_r \exp\lr{ - j k_1 x \sin\theta_r }
%=
%-\frac{k_2 t }{\mu_2} \cos\theta_t \exp\lr{ - j k_2 x \sin\theta_t },
%\end{dmath}
%\begin{dmath}\label{eqn:emtLecture10:1020}
%k_1 \sin\theta_i \exp\lr{ - j k_1 x \sin\theta_i }
%+
%k_1 r \sin\theta_r \exp\lr{ + j k_1 x \sin\theta_r }
%=
%k_2 t \sin\theta_t \exp\lr{ - j k_2 x \sin\theta_t }
%\end{dmath}
%\begin{dmath}\label{eqn:emtLecture10:1040}
%\exp\lr{ - j k_1 \lr{ x \sin\theta_i} }
%+
%r \exp\lr{ - j k_1 \lr{ x \sin\theta_r}}
%=
%t \exp\lr{ - j k_2 \lr{ x \sin\theta_t}}.
%\end{dmath}
%\end{subequations}
%
At \( x = 0 \), this is
%
\begin{equation}\label{eqn:emtLecture10:1060}
\begin{aligned}
-\frac{1}{\eta_1} \cos\theta_i + \frac{r }{\eta_1} \cos\theta_r &= -\frac{t }{\eta_2} \cos\theta_t  \\
k_1 \sin\theta_i + k_1 r \sin\theta_r &= k_2 t \sin\theta_t  \\
1 + r &= t
\end{aligned}
\end{equation}
%
When \( t = 0 \) the latter two equations give Shell's first law
%
%\begin{dmath}\label{eqn:emtLecture10:1080}
\boxedEquation{eqn:emtLecture10:1080}{
\sin\theta_i = \sin\theta_r.
}
%\end{dmath}
%
Assuming this holds for all \( r, t \) we have
%
\begin{dmath}\label{eqn:emtLecture10:1120}
k_1 \sin\theta_i (1 + r ) = k_2 t \sin\theta_t,
\end{dmath}
%
which is Snell's second law in disguise
\begin{dmath}\label{eqn:emtLecture10:1140}
k_1 \sin\theta_i = k_2 \sin\theta_t.
\end{dmath}
%
With
\begin{dmath}\label{eqn:emtLecture10:1540}
k
= \frac{\omega}{v}
= \frac{\omega}{c} \frac{c}{v}
= \frac{\omega}{c} n,
\end{dmath}
%
so \cref{eqn:emtLecture10:1140} takes the form
%
%\begin{dmath}\label{eqn:emtLecture10:1560}
\boxedEquation{eqn:emtLecture10:1560}{
n_1 \sin\theta_i = n_2 \sin\theta_t.
}
%\end{dmath}
%
With
\begin{equation}\label{eqn:emtLecture10:1200}
\begin{aligned}
k_{1z} &= k_1 \cos\theta_i \\
k_{2z} &= k_2 \cos\theta_t,
\end{aligned}
\end{equation}
%
we can solve for \( r, t \) by inverting
%
\begin{dmath}\label{eqn:emtLecture10:1180}
\begin{bmatrix}
\mu_2 k_{1z} & \mu_1 k_{2z} \\
-1 & 1 \\
\end{bmatrix}
\begin{bmatrix}
r \\
t
\end{bmatrix}
=
\begin{bmatrix}
\mu_2 k_{1z} \\
1
\end{bmatrix},
\end{dmath}
%
which gives
%
\begin{dmath}\label{eqn:emtLecture10:1220}
\begin{bmatrix}
r \\
t
\end{bmatrix}
=
\begin{bmatrix}
1 & -\mu_1 k_{2z} \\
1 &  \mu_2 k_{1z}
\end{bmatrix}
\begin{bmatrix}
\mu_2 k_{1z} \\
1
\end{bmatrix},
\end{dmath}
%
or
%\begin{dmath}\label{eqn:emtLecture10:1240}
\boxedEquation{eqn:emtLecture10:1260}{
\begin{aligned}
r &= \frac{\mu_2 k_{1z} - \mu_1 k_{2z}}{\mu_2 k_{1z} + \mu_1 k_{2z}}, \\
t &= \frac{2 \mu_2 k_{1z}}{\mu_2 k_{1z} + \mu_1 k_{2z}}.
\end{aligned}
}
%\end{dmath}
%
There are many ways that this can be written.  Dividing both the numerator and denominator by \( \mu_1 \mu_2 \omega/c \), and noting that \( k = \omega n/c \), we have
%
\begin{equation}\label{eqn:emtLecture10:1680}
\begin{aligned}
r &= \frac
{ \frac{n_1}{\mu_1} \cos\theta_i - \frac{n_2}{\mu_2} \cos\theta_t }
{ \frac{n_1}{\mu_1} \cos\theta_i + \frac{n_2}{\mu_2} \cos\theta_t } \\
t &=
\frac{ 2 \frac{n_1}{\mu_1} \cos\theta_i }
{ \frac{n_1}{\mu_1} \cos\theta_i + \frac{n_2}{\mu_2} \cos\theta_t },
\end{aligned}
\end{equation}
%
which checks against (4.32,4.33) in \citep{hecht1998hecht}.
%
\section{Single interface TM mode.}
%
For completeness, now consider the TM mode.
%
Faraday's law also can provide the electric field from the magnetic
%
\begin{dmath}\label{eqn:emtLecture10:1280}
\kcap \cross \BH
= \eta \kcap \cross \lr{ \kcap \cross \BE }
= -\eta \kcap \cdot \lr{ \kcap \wedge \BE }
= -\eta \lr{ \BE - \kcap \lr{ \kcap \cdot \BE } }
= -\eta \BE.
\end{dmath}
%
so
%
\begin{dmath}\label{eqn:emtLecture10:1300}
\BE = \eta \BH \cross \kcap.
\end{dmath}
%
So the magnetic and electric fields are
%
\begin{subequations}
\label{eqn:emtLecture10:1520}
\begin{equation}\label{eqn:emtLecture10:1320}
\begin{aligned}
   \BH_i &= \Be_2 \frac{E_i}{\eta_1} \exp\lr{ - j k_1 \lr{ z\cos\theta_i + x \sin\theta_i} } \\
   \BH_r &= \Be_2 r \frac{E_i}{\eta_1} \exp\lr{ - j k_1 \lr{ -z \cos\theta_r + x \sin\theta_r}} \\
   \BH_t &= \Be_2 t \frac{E_i}{\eta_2} \exp\lr{ - j k_2 \lr{ z \cos\theta_t + x \sin\theta_t}}
\end{aligned}
\end{equation}
\begin{equation}\label{eqn:emtLecture10:1340}
\begin{aligned}
   \BE_{i} &= -E_i \lr{ -\Be_1 \cos\theta_i + \Be_3\sin\theta_i } \exp\lr{ - j k_1 \lr{ z\cos\theta_i + x \sin\theta_i} } \\
   \BE_{r} &= -r E_i \lr{ \Be_1 \cos\theta_r + \Be_3 \sin\theta_r } \exp\lr{ - j k_1 \lr{ -z \cos\theta_r + x \sin\theta_r}} \\
   \BE_{t} &= -t E_i \lr{ -\Be_1 \cos\theta_t + \Be_3 \sin\theta_t } \exp\lr{ - j k_2 \lr{ z \cos\theta_t + x \sin\theta_t}}.
\end{aligned}
\end{equation}
\end{subequations}
%
Imposing the constraints \cref{eqn:emtLecture10:960}, at \( x = z = 0 \) we have
%
\begin{equation}\label{eqn:emtLecture10:1440}
\begin{aligned}
\inv{\eta_1}\lr{1 + r} &= \frac{t}{\eta_2} \\
\cos\theta_i - r \cos\theta_r &= t \cos\theta_t  \\
\epsilon_1 \lr{ \sin\theta_i  + r \sin\theta_r}  &= t \epsilon_2 \sin\theta_t.
\end{aligned}
\end{equation}
%
At \( t = 0 \), the first and third of these give \( \theta_i = \theta_r \).  Assuming this incident and reflection angle equality holds for all values of \( t \), we have
%
\begin{equation}\label{eqn:emtLecture10:1580}
\begin{aligned}
\sin\theta_i(1  + r)  &= t \frac{\epsilon_2}{\epsilon_1} \sin\theta_t \\
\sin\theta_i \frac{\eta_1}{\eta_2} t &=
\end{aligned}
\end{equation}
%
or
\begin{dmath}\label{eqn:emtLecture10:1600}
\epsilon_1 \eta_1 \sin\theta_i = \epsilon_2 \eta_2 \sin\theta_t.
\end{dmath}
%
This is also Snell's second law \cref{eqn:emtLecture10:1560} in disguise, which can be seen by
%
\begin{dmath}\label{eqn:emtLecture10:1620}
\epsilon_1 \eta_1
=
\epsilon_1 \sqrt{\frac{\mu_1}{\epsilon_1}}
=
\sqrt{\epsilon_1 \mu_1}
=
\inv{v}
=
\frac{n}{c}.
\end{dmath}
%
The remaining equations in matrix form are
%
\begin{dmath}\label{eqn:emtLecture10:1460}
\begin{bmatrix}
\cos\theta_i & \cos\theta_t \\
-1 & \frac{\eta_1}{\eta_2}
\end{bmatrix}
\begin{bmatrix}
r \\
t
\end{bmatrix}
=
\begin{bmatrix}
\cos\theta_i \\
1
\end{bmatrix},
\end{dmath}
%
the inverse of which is
\begin{dmath}\label{eqn:emtLecture10:1480}
\begin{bmatrix}
r \\
t
\end{bmatrix}
=
\inv{ \frac{\eta_1}{\eta_2} \cos\theta_i + \cos\theta_t }
\begin{bmatrix}
\frac{\eta_1}{\eta_2} & - \cos\theta_t \\
1 & \cos\theta_i
\end{bmatrix}
\begin{bmatrix}
\cos\theta_i \\
1
\end{bmatrix}
=
\inv{ \frac{\eta_1}{\eta_2} \cos\theta_i + \cos\theta_t }
\begin{bmatrix}
\frac{\eta_1}{\eta_2} \cos\theta_i - \cos\theta_t \\
2 \cos\theta_i
\end{bmatrix},
\end{dmath}
%
or
%\begin{dmath}\label{eqn:emtLecture10:1640}
\boxedEquation{eqn:emtLecture10:1660}{
\begin{aligned}
r
&=
\frac{\eta_1 \cos\theta_i - \eta_2 \cos\theta_t }{ \eta_1 \cos\theta_i + \eta_2 \cos\theta_t } \\
t &=
\frac{2 \eta_2 \cos\theta_i}{ \eta_1 \cos\theta_i + \eta_2 \cos\theta_t }.
\end{aligned}
}
%\end{dmath}
%
Multiplication of the numerator and denominator by \( c/\eta_1 \eta_2 \), noting that \( c/\eta = n/\mu \) gives
%
\begin{equation}\label{eqn:emtLecture10:1700}
\begin{aligned}
r
&=
\frac{\frac{n_2}{\mu_2} \cos\theta_i - \frac{n_1}{\mu_1} \cos\theta_t }{ \frac{n_2}{\mu_2} \cos\theta_i + \frac{n_1}{\mu_1} \cos\theta_t }, \\
t &=
\frac{2 \frac{n_1}{\mu_1} \cos\theta_i }{ \frac{n_2}{\mu_2} \cos\theta_i + \frac{n_1}{\mu_1} \cos\theta_t },
\end{aligned}
\end{equation}
%
which checks against (4.38,4.39) in \citep{hecht1998hecht}.
%
%\EndArticle
