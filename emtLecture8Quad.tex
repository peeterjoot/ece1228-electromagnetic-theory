%
% Copyright © 2016 Peeter Joot.  All Rights Reserved.
% Licenced as described in the file LICENSE under the root directory of this GIT repository.
%
%
\paragraph{Quadrupole potential}
%
In Jackson
\citep{jackson1975cew}
,
is the following
%
\begin{dmath}\label{eqn:emtLecture8:420}
\inv{\Abs{\Bx - \Bx'}}
=
4 \pi \sum_{l= 0}^\infty \sum_{m = -l}^l \inv{2 l + 1} \frac{(r')^l}{r^{l+1}}
Y^\conj_{l,m}(\theta', \phi')
Y_{l,m}(\theta, \phi),
\end{dmath}
%
where \( Y_{l,m} \) are the spherical harmonics.  It appears that this is actually just an orthogonal function expansion of the inverse distance (for a region outside of the charge density).  The proof of this in is scattered through chapter 3, dependent on a similar expansion in Legendre polynomials, for an the azimuthally symmetric configuration.
%
It looks like quite a project to get comfortable enough with these special functions to fully reproduce the proof of this identity.  We are forced to play engineer, and assume the mathematics works out.  If we do that and plug this inverse distance formula into
the potential we have
%
\begin{dmath}\label{eqn:emtLecture8:440}
\phi(\Bx)
= \inv{4 \pi \epsilon_0} \int \frac{\rho(\Bx') d^3 x'}{\Abs{\Bx - \Bx'}}
=
\inv{4 \pi \epsilon_0} \int \rho(\Bx') d^3 x' \lr{
4 \pi \sum_{l= 0}^\infty \sum_{m = -l}^l \inv{2 l + 1} \frac{(r')^l}{r^{l+1}}
Y^\conj_{l,m}(\theta', \phi')
Y_{l,m}(\theta, \phi)
}
=
\inv{\epsilon_0}
\sum_{l= 0}^\infty \sum_{m = -l}^l \inv{2 l + 1}
\int \rho(\Bx') d^3 x' \lr{
\frac{(r')^l}{r^{l+1}}
Y^\conj_{l,m}(\theta', \phi')
Y_{l,m}(\theta, \phi)
}
=
\inv{\epsilon_0}
\sum_{l= 0}^\infty \sum_{m = -l}^l \inv{2 l + 1}
\lr{
\int (r')^l \rho(\Bx')
Y^\conj_{l,m}(\theta', \phi')
d^3 x'
}
\frac{
Y_{l,m}(\theta, \phi)
}
{
r^{l+1}
}.
\end{dmath}
%
The integral terms are called the coefficients of the multipole moments, denoted
\begin{dmath}\label{eqn:emtLecture8:480}
q_{l,m} =
\int (r')^l \rho(\Bx')
Y^\conj_{l,m}(\theta', \phi')
d^3 x',
\end{dmath}
%
The \( l = 0,1,2\) terms are, respectively, called the monopole, dipole, and quadrupole terms of the potential
\begin{dmath}\label{eqn:emtLecture8:500}
\rho(\Bx) =
\inv{4 \pi \epsilon_0}
\sum_{l= 0}^\infty \sum_{m = -l}^l \frac{4\pi} {2 l + 1}
q_{l,m}
\frac{
Y_{l,m}(\theta, \phi)
}
{
r^{l+1}
}.
\end{dmath}
%
Note the power of this expansion.  Should we wish to compute the electric field, we have only to compute the gradient of  the last (\(Y_{l,m} r^{-l-1} \)) portion (since \( q_{l,m} \) is a constant).
%
\begin{dmath}\label{eqn:emtLecture8:520}
q_{1,1}
=
-\int \sqrt{\frac{3}{8 \pi}} \sin\theta' e^{-i\phi'} r' \rho(\Bx') dV'
=
-\sqrt{\frac{3}{8 \pi}} \int \sin\theta' \lr{ \cos\phi' - i\sin\phi'} r' \rho(\Bx') dV'
=
-\sqrt{\frac{3}{8 \pi}} \lr{
\int x' \rho(\Bx') dV'
-i \int y' \rho(\Bx') dV'
}
=
-\sqrt{\frac{3}{8 \pi}} \lr{
p_x - i p_y
}.
\end{dmath}
%
Here we've used
\begin{equation}\label{eqn:emtLecture8:540}
\begin{aligned}
x' &= r' \sin\theta' \cos\phi' \\
y' &= r' \sin\theta' \sin\phi' \\
z' &= r' \cos\theta'
\end{aligned}
\end{equation}
%
and the \( Y_{11} \) representation
%
\begin{equation}\label{eqn:emtLecture8:560}
\begin{aligned}
Y_{00} &= -\sqrt{\frac{1}{4 \pi}} \\
Y_{11} &= -\sqrt{\frac{3}{8 \pi}} \sin\theta e^{i\phi} \\
Y_{10} &=  \sqrt{\frac{3}{4 \pi}} \cos\theta  \\
Y_{22} &= -\inv{4} \sqrt{\frac{15}{2 \pi}} \sin^2\theta e^{2 i\phi} \\
Y_{21} &=  \inv{2} \sqrt{\frac{15}{2 \pi}} \sin\theta \cos\theta e^{i\phi} \\
Y_{20} &=  \inv{4} \sqrt{\frac{5}{\pi}} \lr{ 3 \cos^2\theta - 1 } \\
\end{aligned}
\end{equation}
%
%NOTE: compute a few of the more tedious moment coeffients.  These have been exam questions in the past.
%
With the usual dipole moment expression
%
\begin{dmath}\label{eqn:emtLecture8:580}
\Bp = \int \Bx' \rho(\Bx') d^3 x',
\end{dmath}
%
and a quadrupole moment defined as
\begin{dmath}\label{eqn:emtLecture8:600}
Q_{i,j} = \int \lr{ 3 x_i' x_j' - \delta_{ij} (r')^2 } \rho(\Bx') d^3 x',
\end{dmath}
%
the first order terms of the potential are now fully specified
\begin{dmath}\label{eqn:emtLecture8:620}
\phi(\Bx)
=
\inv{4 \pi \epsilon_0}
\lr{
q + \frac{\Bp \cdot \Bx}{r^3} +
\inv{2} \sum_{ij} Q_{ij} \frac{x_i x_j}{r^5}
}.
\end{dmath}
