%\paragraph{Disclaimer.}
%%
%These notes are mostly a direct transcription of \mo's handwritten notes on this topic, mostly skipped over in class.  These become relevant because this is used in the non-vacuum model of Maxwell's equations.
%%
\section{Druid model.}
\paragraph{Additional references:} A nice vector based derivation of these Druid model results can be found in \citep{ashcroft1976solid}.  The Meissner effect is also discussed in that context.
%
%
In this section we will investigate the optical properties of free electrons, or what is commonly called free electron gas.
%
By free electron gas we mean electrons that do not experience the restoring force which we considered for bound charges in the case of Lorentz model.  In particular, the resonance frequency \( \omega_0 \) for free electrons is zero.
%
There are two typical cases of free electron systems
%
\begin{enumerate}[a]
\item Metals.
\item Doped (n or p type) semiconductors.
\end{enumerate}
%
For the moment we consider the case of metals.
%
Free electrons are responsible for high reflectivity and good thermal conductivity of metals up to optical frequencies.  A model that can be used to describe the high reflectivity of metals is the Drude model.
%
\paragraph{Plasma:} A neutral gas of free electrons and heavy ions is called plasma.  Examples of plasma are metals and doped semiconductors, since these materials are a combination of free electrons and heavy ions which are, in sum, electrically neutral.
%
\paragraph{Drude-Lorentz model,} (or Drude model for short): similar to the case of bound charges we already studied for free electron plasma, we can start with a harmonic oscillator model.  However, in this case, since electrons are free, there is no restoring force (i.e. \(\omega_0 = 0 \).  Recall that in the spring mass model \( \omega_0^2 = S/m \) where \( S \) was the spring tension coefficient.
%
With such a model the Lorentz model equation
%
\begin{equation}\label{eqn:druid:20}
\frac{d^2 x}{dt^2} + \gamma \ddt{x} + \omega_0^2 x = \frac{Q E_0}{m} e^{j \omega t},
\end{equation}
%
is reduced to
%
\begin{equation}\label{eqn:druid:40}
\frac{d^2 x}{dt^2} + \gamma \ddt{x} = \frac{Q E_0}{m} e^{j \omega t},
\end{equation}
%
Again, assuming a solution of the form \( x_p = x_0 e^{j \omega t} \) for the particular solution and substituting in \cref{eqn:druid:40}, we have
%
\begin{equation}\label{eqn:druid:80}
x_0 \lr{ (j\omega)^2 + \gamma (j \omega)} = \frac{Q E_0}{m},
\end{equation}
%
or
\begin{equation}\label{eqn:druid:60}
x
=
\frac{Q E/m}{-\omega^2 + j \gamma \omega },
\end{equation}
%
Once more assuming identical particles that are not coupled and a linear isotropic medium and using the fact that \( \BP = N \Bp = N Q \Bx \), and
%
\begin{equation}\label{eqn:druid:100}
\chi_e = \frac{\Abs{\BP}}{\epsilon_0 \Abs{\BE} },
\end{equation}
%
we have
%
\begin{equation}\label{eqn:druid:120}
\chi_e
=
\frac{Q^2 N/m \epsilon_0}{-\omega^2 + j \gamma \omega },
\end{equation}
%
or with \( \omega_p^2 = Q^2 N/m\epsilon_0\),
%
\begin{equation}\label{eqn:druid:140}
\begin{aligned}
\epsilon_r
&= 1 + \chi_e
\\ &=
1+
\frac{\omega_p^2}{-\omega^2 + j \gamma \omega }.
\end{aligned}
\end{equation}
%
Plasma frequency, \( \omega_p \), can be understood as the natural resonance frequency by which the free electron gas (plasma) collectively (not individual electrons ) oscillates.
%
Note that if we neglect the last term, i.e., let \( \gamma = 0 \) then
%
\begin{equation}\label{eqn:druid:160}
\epsilon_r = 1 - \frac{\omega_p^2}{\omega^2}.
\end{equation}
%
From this it is clear that when \( \omega < \omega_p \), we have \( \epsilon_r < 1 \) and \( n = \sqrt{\epsilon_r} \) is purely imaginary, and the wave attenuates inside the electron plasma.
%
This means that for \( \omega < \omega_p \) electromagnetic waves do not propagate a large distance inside of metal.  However, for \( \omega > \omega_p \) the electron plasma (e.g. metal) is transparent.  The latter is called ultraviolet transparency of metal, because for most metals \( \omega_p \) is in the ultraviolet part of the spectrum.  For example,
%
\begin{itemize}
\item For \ce{Al}:
\begin{equation}\label{eqn:druid:180}
\frac{\omega_p}{2 \pi} = 3.82 \times 10^{15} \si{Hz} \implies \lambda_p = 79 [nm].
\end{equation}
\item For \ce{Au}:
\begin{equation}\label{eqn:druid:200}
\frac{\omega_p}{2 \pi} = 5.9 \times 10^{15} \si{Hz} \implies \lambda_p = 138 [nm].
\end{equation}
\end{itemize}
%
Using \cref{eqn:druid:160} one can calculate
%
\begin{equation}\label{eqn:druid:220}
\tilde{n} = \sqrt{\epsilon_r},
\end{equation}
%
and plot the reflectivity \( R \) at normal incidence
%
\begin{equation}\label{eqn:druid:240}
R = \Abs{ \frac{\tilde{n} - 1 }{\tilde{n} + 1} },
\end{equation}
%
which will have a shape similar to that of \cref{fig:druidReflectivity:druidReflectivityFig1}.
%
\imageFigure{../figures/ece1228-electromagnetic-theory/druidReflectivityFig1}{Metal reflectivity.}{fig:druidReflectivity:druidReflectivityFig1}{0.3}
%
This figure shows that for \( \omega/\omega_p \ll 1 \) metal reflects most of the incident light, whereas it becomes transparent (it transmits light) for \( \omega/\omega_p \gg 1 \).  This explains the shiny appearance of the metal at optical wavelengths.
%
The fact that plasma reflects EM waves below a \( \omega_p \) frequency can be used to transmit AM radio waves.  The ionosphere can be viewed as a plasma gas due to free electrons generated by cosmic radiation and ultraviolet light from the sun.  The \( \omega_p \) for ionosphere plasma is \( \omega_p = O(1 \si{MHz}) \).  Therefore AM signals modulated at frequencies below or in the range of a \si{MHz} will be reflected from the ionosphere.  But FM signals where the modulation frequency is greater than \si{MHz} will not be reflected, but will travel through the ionosphere and into space.
%
\section{Conductivity}
%
%\ddt{} \lr{ m \ddt{x} } + \gamma \ddt{x} = Q E_0 e^{i \omega t}
%
\begin{equation}\label{eqn:druid:260}
\begin{aligned}
\spacegrad \cross \BH(\Br, \omega)
&= \sigma \BE(\Br, \omega) + j \omega \epsilon_0 \BE(\Br, \omega)
\\ &= j \omega \epsilon_0 \lr{ 1 + \frac{\sigma}{j \omega \epsilon_0} } \BE(\Br, \omega)
\\ &= j \omega \epsilon_0 \lr{ 1 - \frac{j \sigma}{\omega \epsilon_0} } \BE(\Br, \omega)
\end{aligned}
\end{equation}
%
This complex factor is the relative permittivity
%
\begin{equation}\label{eqn:druid:280}
\epsilon_r
= 1 - \frac{j \sigma}{\omega \epsilon_0},
\end{equation}
%
and is why we write
%
\begin{equation}\label{eqn:druid:300}
\epsilon(\omega) = \epsilon'(\omega) - j \epsilon''(\omega).
\end{equation}
%
