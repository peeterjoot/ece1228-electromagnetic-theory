%
% Copyright � 2016 Peeter Joot.  All Rights Reserved.
% Licenced as described in the file LICENSE under the root directory of this GIT repository.
%
%{
%\input{../blogpost.tex}
%\renewcommand{\basename}{magneticMomentJackson}
%%\renewcommand{\dirname}{notes/phy1520/}
%\renewcommand{\dirname}{notes/ece1228-electromagnetic-theory/}
%%\newcommand{\dateintitle}{}
%%\newcommand{\keywords}{}
%
%\input{../latex/peeter_prologue_print2.tex}
%
%\usepackage{peeters_layout_exercise}
%\usepackage{peeters_braket}
%\usepackage{peeters_figures}
%\usepackage{siunitx}
%%\usepackage{txfonts} % \ointclockwise
%
%\beginArtNoToc
%
%\generatetitle{Magnetic moment for a localized magnetostatic current}
%\chapter{Magnetic moment for a localized magnetostatic current}
%\label{chap:magneticMomentJackson}
% \citep{sakurai2014modern} pr X.Y
% \citep{pozar2009microwave}
% \citep{qftLectureNotes}
% \citep{doran2003gap}
%\paragraph{Motivation.}
%
\makeproblem{Magnetic moment for localized current.}{problem:magneticMomentJackson:1}{
Jackson \citep{jackson1975cew} \S 5.6 derives an expression for the magnetic moment of a localized current distribution, far from the source.  Repeat this derivation, filling in the details.
%.  This time I found that his presentation of magnetic moment didn't really make sense to me.  Here's my own pass through it, filling in a number of details.  As I did last time, I'll also translate into SI units as I go.
} % problem
%
\makeanswer{problem:magneticMomentJackson:1}{
%\withproblemsetsParagraph{
%\paragraph{Vector potential.}
%
\index{Biot-Savart}
The Biot-Savart expression for the magnetic field can be factored into a curl expression using the usual tricks
%
\begin{dmath}\label{eqn:magneticMomentJackson:20}
\BB
= \frac{\mu_0}{4\pi} \int \frac{\BJ(\Bx') \cross (\Bx - \Bx')}{\Abs{\Bx - \Bx'}^3} d^3 x'
= -\frac{\mu_0}{4\pi} \int \BJ(\Bx') \cross \spacegrad \inv{\Abs{\Bx - \Bx'}} d^3 x'
= \frac{\mu_0}{4\pi} \spacegrad \cross \int \frac{\BJ(\Bx')}{\Abs{\Bx - \Bx'}} d^3 x',
\end{dmath}
%
so the vector potential, through its curl, defines the magnetic field \( \BB = \spacegrad \cross \BA \) is given by
%
\begin{dmath}\label{eqn:magneticMomentJackson:40}
\BA(\Bx) = \frac{\mu_0}{4 \pi} \int \frac{J(\Bx')}{\Abs{\Bx - \Bx'}} d^3 x'.
\end{dmath}
%
If the current source is localized (zero outside of some finite region), then there will always be a region for which \( \Abs{\Bx} \gg \Abs{\Bx'} \), so the denominator yields to Taylor expansion
%
\begin{dmath}\label{eqn:magneticMomentJackson:60}
\inv{\Abs{\Bx - \Bx'}}
=
\inv{\Abs{\Bx}} \lr{1 + \frac{\Abs{\Bx'}^2}{\Abs{\Bx}^2} - 2 \frac{\Bx \cdot \Bx'}{\Abs{\Bx}^2} }^{-1/2}
\approx
\inv{\Abs{\Bx}} \lr{ 1 + \frac{\Bx \cdot \Bx'}{\Abs{\Bx}^2} }
=
\inv{\Abs{\Bx}} + \frac{\Bx \cdot \Bx'}{\Abs{\Bx}^3}.
\end{dmath}
%
so the vector potential, far enough away from the current source is
\begin{dmath}\label{eqn:magneticMomentJackson:80}
\BA(\Bx)
=
\frac{\mu_0}{4 \pi} \int \frac{J(\Bx')}{\Abs{\Bx}} d^3 x'
+\frac{\mu_0}{4 \pi} \int \frac{(\Bx \cdot \Bx')J(\Bx')}{\Abs{\Bx}^3} d^3 x'.
\end{dmath}
%
Jackson uses a sneaky trick to show that the first integral is killed for a localized source.  That trick appears to be based on evaluating the following divergence
%
\begin{dmath}\label{eqn:magneticMomentJackson:100}
\spacegrad \cdot (\BJ(\Bx) x_i)
=
(\spacegrad \cdot \BJ) x_i
+
(\spacegrad x_i) \cdot \BJ
=
(\Be_k \partial_k x_i) \cdot\BJ
=
\delta_{ki} J_k
=
J_i.
\end{dmath}
%
Note that this made use of the fact that \( \spacegrad \cdot \BJ = 0 \) for magnetostatics.  This provides a way to rewrite the current density as a divergence
%
\begin{dmath}\label{eqn:magneticMomentJackson:120}
\int \frac{J(\Bx')}{\Abs{\Bx}} d^3 x'
=
\Be_i \int \frac{\spacegrad' \cdot (x_i' \BJ(\Bx'))}{\Abs{\Bx}} d^3 x'
=
\frac{\Be_i}{\Abs{\Bx}} \int \spacegrad' \cdot (x_i' \BJ(\Bx')) d^3 x'
=
\frac{1}{\Abs{\Bx}} \oint \Bx' (d\Ba' \cdot \BJ(\Bx')).
\end{dmath}
%
When \( \BJ \) is localized, this is zero provided we pick the integration surface for the volume outside of that localization region.
%
It is now desired to rewrite \( \int \Bx \cdot \Bx' \BJ \) as a triple cross product since the dot product of such a triple cross product has exactly this term in it
%
\begin{dmath}\label{eqn:magneticMomentJackson:140}
- \Bx \cross \int \Bx' \cross \BJ
=
\int (\Bx \cdot \Bx') \BJ
-
\int (\Bx \cdot \BJ) \Bx'
=
\int (\Bx \cdot \Bx') \BJ
-
\Be_k x_i \int J_i x_k',
\end{dmath}
%
so
\begin{dmath}\label{eqn:magneticMomentJackson:160}
\int (\Bx \cdot \Bx') \BJ
=
- \Bx \cross \int \Bx' \cross \BJ
+
\Be_k x_i \int J_i x_k'.
\end{dmath}
%
To get of this second term, the next sneaky trick is to consider the following divergence
%
\begin{dmath}\label{eqn:magneticMomentJackson:180}
\oint d\Ba' \cdot (\BJ(\Bx') x_i' x_j')
=
\int dV' \spacegrad' \cdot (\BJ(\Bx') x_i' x_j')
=
\int dV' (\spacegrad' \cdot \BJ)
+
\int dV' \BJ \cdot \spacegrad' (x_i' x_j')
=
\int dV' J_k \cdot \lr{ x_i' \partial_k x_j' + x_j' \partial_k x_i' }
=
\int dV' \lr{ J_k x_i' \delta_{kj} + J_k x_j' \delta_{ki} }
=
\int dV' \lr{ J_j x_i' + J_i x_j'}.
\end{dmath}
%
The surface integral is once again zero, which means that we have an antisymmetric relationship in integrals of the form
%
\begin{dmath}\label{eqn:magneticMomentJackson:200}
\int J_j x_i' = -\int J_i x_j'.
\end{dmath}
%
Now we can use the tensor algebra trick of writing \( y = (y + y)/2 \),
%
\begin{dmath}\label{eqn:magneticMomentJackson:220}
\int (\Bx \cdot \Bx') \BJ
=
- \Bx \cross \int \Bx' \cross \BJ
+
\Be_k x_i \int J_i x_k'
=
- \Bx \cross \int \Bx' \cross \BJ
+
\inv{2} \Be_k x_i \int \lr{ J_i x_k' + J_i x_k' }
=
- \Bx \cross \int \Bx' \cross \BJ
+
\inv{2} \Be_k x_i \int \lr{ J_i x_k' - J_k x_i' }
=
- \Bx \cross \int \Bx' \cross \BJ
+
\inv{2} \Be_k x_i \int (\BJ \cross \Bx')_j \epsilon_{ikj}
=
- \Bx \cross \int \Bx' \cross \BJ
-
\inv{2} \epsilon_{kij} \Be_k x_i \int (\BJ \cross \Bx')_j
=
- \Bx \cross \int \Bx' \cross \BJ
-
\inv{2} \Bx \cross \int \BJ \cross \Bx'
=
- \Bx \cross \int \Bx' \cross \BJ
+
\inv{2} \Bx \cross \int \Bx' \cross \BJ
=
-\inv{2} \Bx \cross \int \Bx' \cross \BJ,
\end{dmath}
%
so
%
\begin{dmath}\label{eqn:magneticMomentJackson:240}
\BA(\Bx) \approx \frac{\mu_0}{4 \pi \Abs{\Bx}^3} \lr{ -\frac{\Bx}{2} } \int \Bx' \cross \BJ(\Bx') d^3 x'.
\end{dmath}
%
Letting
%
\index{magnetic moment}
%\begin{dmath}\label{eqn:magneticMomentJackson:260}
\boxedEquation{eqn:magneticMomentJackson:260}{
\Bm = \inv{2} \int \Bx' \cross \BJ(\Bx') d^3 x',
}
%\end{dmath}
%
\index{vector potential}
the far field approximation of the vector potential is
%\begin{dmath}\label{eqn:magneticMomentJackson:280}
\boxedEquation{eqn:magneticMomentJackson:280}{
\BA(\Bx) = \frac{\mu_0}{4 \pi} \frac{\Bm \cross \Bx}{\Abs{\Bx}^3}.
}
%\end{dmath}
%
Note that when the current is restricted to an infinitesimally thin loop, the magnetic moment reduces to
%
\begin{dmath}\label{eqn:magneticMomentJackson:300}
\Bm(\Bx) = \frac{I}{2} \int \Bx \cross d\Bl'.
\end{dmath}
%
Referring to \citep{griffiths1999introduction} (pr. 1.60), this can be seen to be \( I \) times the ``vector-area'' integral.
%
A side effect of having evaluated this approximation is that we have shown that
%
\begin{dmath}\label{eqn:magneticMomentJackson:320}
\int \lr{ \Bx \cdot \Bx' } \BJ(\Bx') d^3 x'
=
\Bm \cross \Bx.
\end{dmath}
%
This will be required again later when evaluating the force due to an applied magnetic field in terms of the magnetic moment.
%}
} % answer
%
%}
%\EndArticle
