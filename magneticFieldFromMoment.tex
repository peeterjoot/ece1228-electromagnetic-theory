%
% Copyright � 2016 Peeter Joot.  All Rights Reserved.
% Licenced as described in the file LICENSE under the root directory of this GIT repository.
%
%{
%\input{../blogpost.tex}
%\renewcommand{\basename}{magneticFieldFromMoment}
%%\renewcommand{\dirname}{notes/phy1520/}
%\renewcommand{\dirname}{notes/ece1228-electromagnetic-theory/}
%%\newcommand{\dateintitle}{}
%%\newcommand{\keywords}{}
%
%\input{../latex/peeter_prologue_print2.tex}
%
%\usepackage{peeters_layout_exercise}
%\usepackage{peeters_braket}
%\usepackage{peeters_figures}
%\usepackage{siunitx}
%%\usepackage{mhchem} % \ce{}
%%\usepackage{macros_bm} % \bcM
%%\usepackage{txfonts} % \ointclockwise
%
%\beginArtNoToc
%
%\generatetitle{Calculating the magnetostatic field from the moment}
%\chapter{Calculating the magnetostatic field from the moment}
%\label{chap:magneticFieldFromMoment}
% \citep{jackson1975cew}
%
\makeproblem{Magnetic field from moment.}{problem:magneticFieldFromMoment:1}{
The vector potential, to first order, for a magnetostatic localized current distribution was found to be
%
\begin{dmath}\label{eqn:magneticFieldFromMoment:20}
\BA(\Bx) = \frac{\mu_0}{4 \pi} \frac{\Bm \cross \Bx}{\Abs{\Bx}^3}.
\end{dmath}
%
Use this to calculate the magnetic field.
} % problem
%
\makeanswer{problem:magneticFieldFromMoment:1}{\withproblemsetsParagraph{
%
\index{magnetic field}
\index{magnetic moment}
\index{triple cross product}
\begin{dmath}\label{eqn:magneticFieldFromMoment:40}
\BB
=
\frac{\mu_0}{4 \pi}
\spacegrad \cross \lr{ \Bm \cross \frac{\Bx}{r^3} }
=
-\frac{\mu_0}{4 \pi}
\spacegrad \cdot \lr{ \Bm \wedge \frac{\Bx}{r^3} }
=
-\frac{\mu_0}{4 \pi}
\lr{
(\Bm \cdot \spacegrad) \frac{\Bx}{r^3}
-\Bm \spacegrad \cdot \frac{\Bx}{r^3}
}
=
\frac{\mu_0}{4 \pi}
\lr{
-\frac{(\Bm \cdot \spacegrad) \Bx}{r^3}
- \lr{ \Bm \cdot \lr{\spacegrad \inv{r^3} }} \Bx
+\Bm (\spacegrad \cdot \Bx) \inv{r^3}
+\Bm \lr{\spacegrad \inv{r^3} } \cdot \Bx
}.
\end{dmath}
%
Here I've used \( \Ba \cross \lr{ \Bb \cross \Bc } = -\Ba \cdot \lr{ \Bb \wedge \Bc } \), and then expanded that with \( \Ba \cdot \lr{ \Bb \wedge \Bc } = (\Ba \cdot \Bb) \Bc - (\Ba \cdot \Bc) \Bb \).  Since one of these vectors is the gradient, care must be taken to have it operate on the appropriate terms in such an expansion.
%
Since we have \( \spacegrad \cdot \Bx = 3 \), \( (\Bm \cdot \spacegrad) \Bx = \Bm \), and \( \spacegrad 1/r^n = -n \Bx/r^{n+2} \), this reduces to
%
\begin{dmath}\label{eqn:magneticFieldFromMoment:60}
\BB
=
\frac{\mu_0}{4 \pi}
\lr{
- \frac{\Bm}{r^3}
+ 3 \frac{(\Bm \cdot \Bx) \Bx}{r^5} %
+ 3 \Bm \inv{r^3}
-3 \Bm \frac{\Bx}{r^5} \cdot \Bx
}
=
\frac{\mu_0}{4 \pi}
\frac{3 (\Bm \cdot \ncap) \ncap -\Bm}{r^3},
\end{dmath}
%
which is the desired result.
}} % answer
%
%}
%\EndNoBibArticle
