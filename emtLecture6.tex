%
% Copyright � 2016 Peeter Joot.  All Rights Reserved.
% Licenced as described in the file LICENSE under the root directory of this GIT repository.
%
%\input{../blogpost.tex}
%\renewcommand{\basename}{emt6}
%\renewcommand{\dirname}{notes/ece1228/}
%\newcommand{\keywords}{ECE1228H}
%\input{../latex/peeter_prologue_print2.tex}
%
%%\usepackage{ece1228}
%\usepackage{peeters_braket}
%%\usepackage{peeters_layout_exercise}
%\usepackage{peeters_figures}
%\usepackage{mathtools}
%\usepackage{siunitx}
%\usepackage{enumerate}
%
%\beginArtNoToc
%\generatetitle{ECE1228H Electromagnetic Theory.  Lecture 6: Lorentz-Lorenz dispersion.  Taught by Prof.\ M. Mojahedi}
%%\chapter{Lorentz-Lorenz dispersion}
%\label{chap:emt6}

%\paragraph{Disclaimer}
%
%Peeter's lecture notes from class.  These may be incoherent and rough.
%
%These are notes for the UofT course ECE1228H, Electromagnetic Theory, taught by Prof. M. Mojahedi, covering \textchapref{{1}} \citep{balanis1989advanced} content.
%
\paragraph{Lorentz-Lorenz Dispersion}
%
We will model the medium using a frequency representation of the permittivity
%
\begin{equation}\label{eqn:emtLecture6:20}
\begin{aligned}
\epsilon(\omega) &= \epsilon'(\omega) - j \epsilon''(\omega) \\
\mu(\omega) &= \mu'(\omega) - j \mu''(\omega)
\end{aligned}
\end{equation}
%
The real part is the phase, whereas the imaginary part is the loss.
%
\begin{dmath}\label{eqn:emtLecture6:40}
n = \frac{c}{v}
= \frac{\sqrt{\epsilon \mu}}{\sqrt{\epsilon_0 \mu_0}}
= \sqrt{\epsilon_r \mu_r}
\end{dmath}
%
We can also write
%
\begin{dmath}\label{eqn:emtLecture6:60}
n(\omega) = n'(\omega) - j n''(\omega)
\end{dmath}
%
If we are considering an electric dipole
%
\begin{dmath}\label{eqn:emtLecture6:80}
\BP_i = Q_i \Bx_i
\end{dmath}
%
With
%
\begin{dmath}\label{eqn:emtLecture6:100}
\BP = \epsilon_0 \chi_e \BE,
\end{dmath}
%
and a time harmonic representation for the electric field
%
\begin{dmath}\label{eqn:emtLecture6:120}
\BE = \BE_0 e^{j \omega t}.
\end{dmath}
%
The dipole moment is assumed to be
%
\begin{dmath}\label{eqn:emtLecture6:140}
\BP = \lim_{\Delta v \rightarrow 0} \frac{ \sum_{i = 1}^{N \Delta v} \BP_i }{\Delta v}
= \frac{ N \Delta v \Bp}{\Delta v}
= N \Bp
= N Q \Bx.
\end{dmath}
%
%F1:
%
We model the oscillating electron and nucleus as a mass and spring.
This electron oscillator model is often called the Lorentz model.  It is not really a model for atoms as such, but the way that an atom responds to pertubation.  At the time when Lorentz formulated the model it was not known that the nuclei havr massive mass as compared to the electrons.
The Lorentz assumption was that in the absence of applied eletric fields the centroids of positive and neagivve charges coincide, but when a field is applied, the electrons will experience a Lorentz force and will be displaced from their equilibrium position.
The wrote ``the displacement immediately gives rise to a new force by which the particle is pulled back towards its original position, and which we may therefore appropriately distinguish by the name of elastic force.''

The forces of interest are
%
\begin{equation}\label{eqn:emtLecture6:160}
\begin{aligned}
F_{\textrm{friction}} &= -D \frac{dx}{dt} = -D v \\
F_{\textrm{elastic}} &= -S x \\
F_{\textrm{external}} &= Q E = Q E_0 e^{j \omega t}
\end{aligned}
\end{equation}
%
Adding all the forces, the electrical system, in one dimension, can be assumed to have the form
%
\begin{equation}\label{eqn:emtLecture6:180}
F = m \frac{d^2 x}{dt^2}
=
-D \frac{dx}{dt}
-D v \\
-S x \\
+ Q E_0 e^{j \omega t},
\end{equation}
%
or
\begin{dmath}\label{eqn:emtLecture6:200}
\frac{d^2 x}{dt^2} + \frac{D}{m} \ddt{x} + \frac{S}{m} x = \frac{Q E_0}{m} e^{j \omega t}
\end{dmath}
%
Let's define
%
\begin{equation}\label{eqn:emtLecture6:220}
\begin{aligned}
\gamma &= \frac{D}{m} \\
\omega_0^2 &= \frac{S}{m},
\end{aligned}
\end{equation}
%
so that
%
\begin{dmath}\label{eqn:emtLecture6:240}
\frac{d^2 x}{dt^2} + \gamma \ddt{x} + \omega_0^2 x = \frac{Q E_0}{m} e^{j \omega t}.
\end{dmath}
%
\paragraph{Calculating the permittivity and susceptibility}
%
With \( x = x_0 e^{j \omega t} \) we have
%
\begin{dmath}\label{eqn:emtLecture6:260}
x_0 \lr{ -\omega^2 + j\gamma \omega + \omega_0^2 } = \frac{Q E_0}{m},
\end{dmath}
%
or (with \( E = E_0 e^{j \omega t} \)), just
%
\begin{equation}\label{eqn:emtLecture6:280}
x = x_0 e^{j\omega t}
= \frac{Q E}{m \lr{ -\omega^2 + j\gamma \omega + \omega_0^2 } }.
\end{equation}
%
\begin{enumerate}[I]
\item Assume that dipoles are identical
\item Assume no coupling between dipoles
\item There are N dipoles per unit volume.  In other words, N is the number of dipoles per unit volume.
\end{enumerate}
%
The polarization \( P(t) \) is given by
%
\begin{dmath}\label{eqn:emtLecture6:300}
P(t) = N Q x,
\end{dmath}
%
where \( Q \) is the charge associate with the unit dipole.  This has dimensions of [\si{\frac{1}{m^3} \times C \times m}], or [\si{C/m^2}].  This polarization is
%
\begin{dmath}\label{eqn:emtLecture6:440}
P(t)
= \frac{Q^2 N E/m}{\omega_0^2 -\omega^2 + j\gamma \omega }.
\end{dmath}
%
In particular, the ratio of the polarization to the electric field magnitude is
%
\begin{dmath}\label{eqn:emtLecture6:320}
\frac{P}{E}
= \frac{Q^2 N/ m}{\omega_0^2 -\omega^2 + j\gamma \omega }.
\end{dmath}
%
With \( P = \epsilon_0 \chi_e E \), we have
%
\begin{dmath}\label{eqn:emtLecture6:340}
\chi_e = \frac{Q^2 N/ m \epsilon_0}{\omega_0^2 -\omega^2 + j\gamma \omega }.
\end{dmath}
%
Define
%
\begin{dmath}\label{eqn:emtLecture6:360}
\omega_p^2 = \frac{ Q^2 N}{m \epsilon_0},
\end{dmath}
%
which has dimensions [\si{1/s^2}].  Then
%
\begin{dmath}\label{eqn:emtLecture6:380}
\chi_e = \frac{\omega_p^2}{\omega_0^2 -\omega^2 + j\gamma \omega }.
\end{dmath}
%
With \( \epsilon_r = 1 + \chi_e \) we have
%
\begin{equation}\label{eqn:emtLecture6:400}
\epsilon_r
= \frac{\epsilon}{\epsilon_0}
= 1 + \frac{\omega_p^2}{\omega_0^2 -\omega^2 + j\gamma \omega }.
\end{equation}
%
%or
%\begin{dmath}\label{eqn:emtLecture6:420}
%\epsilon_r
%= \frac{ \omega_0^2 -\omega^2 + j\gamma \omega + \omega_p^2}{\omega_0^2 -\omega^2 + j\gamma \omega }
%\end{dmath}
%
One can show that \( \epsilon_r = \epsilon_r' -j \epsilon_r'' \) are given bby
%
\begin{dmath}\label{eqn:emtLecture6:460}
\epsilon_r' = \frac{\omega_p^2 \lr{ \omega_0^2 - \omega^2 } }{ (\omega_0^2 - \omega^2)^2 + (\omega \gamma)^2 } + 1,
\end{dmath}
\begin{dmath}\label{eqn:emtLecture6:480}
\epsilon_r'' = \frac{\omega_p^2 \omega \gamma}{ (\omega_0^2 - \omega^2)^2 + (\omega \gamma)^2 }.
\end{dmath}
%
FIXME: calculate this.
%
\paragraph{No damping}
%
With \( D = 0 \), or \( \gamma = 0 \) then \( \epsilon_r'' = 0 \),
%
\begin{dmath}\label{eqn:emtLecture6:500}
x = \frac{Q E_0/m}{\omega^2 - \omega^2} e^{j \omega t},
\end{dmath}
%
and
\begin{equation}\label{eqn:emtLecture6:520}
\epsilon_r
=
\epsilon_r'
= \frac{\epsilon}{\epsilon_0}
=
1 + \frac{\omega_p^2}{\omega_0^2 - \omega^2}.
\end{equation}
%
This has a curve like \cref{fig:emtL6:emtL6Fig5a}.
%
\imageFigure{../figures/ece1228-electromagnetic-theory/emtL6Fig5a}{Undamped resonance.}{fig:emtL6:emtL6Fig5a}{0.3}
%
instead of the normal damped resonance curve like  \cref{fig:emtL6:emtL6Fig5b}.
%
\imageFigure{../figures/ece1228-electromagnetic-theory/emtL6Fig5b}{Damped resonance.}{fig:emtL6:emtL6Fig5b}{0.3}
%
As \( \omega \rightarrow \omega_0 \), then the displacement \( x \rightarrow \infty \).  The frequency \( \omega_0 \) is called the resonance frequency of the system.
%
If the resonance frequency is zero (free charges), then
%
\begin{dmath}\label{eqn:emtLecture6:540}
\epsilon_r = \epsilon_r' = 1 - \frac{\omega_p^2}{\omega^2},
\end{dmath}
%
which is negative for \( \omega_p > \omega \).
%
When damping is present, the resonance frequency is the root of the characteristic equation of the homogeneous part of \cref{eqn:emtLecture6:200}.
%
\paragraph{Multiple resonances}
%
When there are \( N \) molecules per unit volume, and each molecule has
Z electrons per molecule that have a binding frequency \( \omega_i \) and damping constant \( \gamma_i \), then it can be shown that
%
\begin{dmath}\label{eqn:emtLecture6:560}
\epsilon_r = 1 + \frac{Q N^2}{m \epsilon_0} \sum \frac{ f_i }{\omega_0^2 - \omega^2 + j \gamma \omega }
\end{dmath}
%
A quantum mechanical derivation of the transition frequencies is used to derive this multiple resonance result.
%
%\EndArticle
%\EndNoBibArticle
