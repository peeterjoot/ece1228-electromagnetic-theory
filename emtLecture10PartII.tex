%
% Copyright © 2016 Peeter Joot.  All Rights Reserved.
% Licenced as described in the file LICENSE under the root directory of this GIT repository.
%
\input{../latex/blogpost.tex}
\renewcommand{\basename}{emt10}
\renewcommand{\dirname}{notes/ece1228/}
\newcommand{\keywords}{ECE1228H}
\input{../latex/latex/peeter_prologue_print2.tex}
%
%\usepackage{ece1228}
\usepackage{peeters_braket}
%\usepackage{peeters_layout_exercise}
\usepackage{peeters_figures}
\usepackage{mathtools}
\usepackage{siunitx}
\usepackage{macros_bm}
%
\beginArtNoToc
\generatetitle{ECE1228H Electromagnetic Theory.  Lecture 10: Fresnel relations.  Taught by Prof.\ M. Mojahedi}
%\chapter{Fresnel relations}
\label{chap:emt10}
%
\section{Two interface problem.}
%
It turns out that light passing through two interfaces ends up with reflected and transmitted components, but no mode that goes through the slab in a wave guide fashion.  Such a mode is possible when the light is incident in an end-fire configuration (i.e. a waveguide).  It's interesting that such a wave guide like mode isn't possible.
%
%%%XX
%%%Consider a geometry with interfaces at \( z = 0, d \).
%%%
%%%The incident, and reflected wave components in the first medium are
%%%\begin{dmath}\label{eqn:emtLecture10:140}
%%%   A e^{-j k_{1z} z},
%%%\end{dmath}
%%%\begin{dmath}\label{eqn:emtLecture10:160}
%%%   r A e^{j k_{1z} z},
%%%\end{dmath}
%%%
%%%in the second medium
%%%
%%%\begin{dmath}\label{eqn:emtLecture10:180}
%%%   C e^{-j k_{1z} z},
%%%\end{dmath}
%%%\begin{dmath}\label{eqn:emtLecture10:200}
%%%   D e^{j k_{1z} z},
%%%\end{dmath}
%%%
%%%and in the final medium
%%%\begin{dmath}\label{eqn:emtLecture10:220}
%%%   A t e^{-j k_{1z} (z-d)},
%%%\end{dmath}
%%%
%%%\begin{itemize}
%%%   \item \( z = 0 \).
%%%      % FIXME
%%%   %A r_{12} = C + D
%%%   \item \( z = d \).
%%%      % FIXME
%%%\end{itemize}
%%%These relationships can be assembled into matrix form. The end result is
%%%
%%%FIXME.
%%%
%%%Something important to consider is that the index of refraction can be complex valued
%%%
%%%\begin{dmath}\label{eqn:emtLecture10:240}
%%%n(\omega) = n'(\omega) + j n''(\omega),
%%%\end{dmath}
%%%
%%%but all of these derivations are independent of this.
%%%
%%%\paragraph{special case.}
%%%
%%%If medium 1 and 3 are identical then
%%%
%%%\begin{dmath}\label{eqn:emtLecture10:260}
%%%r_{21} = r_{23} = -r_{12},
%%%\end{dmath}
%%%
%%%which gives
%%%
%%%\begin{dmath}\label{eqn:emtLecture10:280}
%%%   t^{\textrm{TE}} = \frac{t_{12} t_{21} e^{j \phi}}{1 - r_{21}^2 e^{2 j \phi}}
%%%\end{dmath}
%%%\begin{dmath}\label{eqn:emtLecture10:300}
%%%   r^{\textrm{TE}} = r_{12} + \frac{t_{12} t_{21} r_{21} e^{2 j \phi}}{1 - r_{21}^2 e^{2 j \phi}}
%%%\end{dmath}
%%%
%%%where
%%%\begin{dmath}\label{eqn:emtLecture10:320}
%%%   \phi = -k_{2z} d = - \frac{\omega}{c} n_2 \cos\theta_2 d
%%%   % \cos(\theta_2 d)?
%%%\end{dmath}
%%%
%%%It's possible to design the material so that there is no reflection from the slab, called the matched condition.
%%%
%%%This requires
%%%
%%%\begin{dmath}\label{eqn:emtLecture10:340}
%%%r_{21} = \frac{-\mu_2 k_{1z} + \mu_1 k_{2z}}{\mu_2 k_{1z} + \mu_1 k_{2z}} = -r_{12}.
%%%\end{dmath}
%%%
%%%To force this to be zero, one can set
%%%\begin{dmath}\label{eqn:emtLecture10:360}
%%%\mu_2 k_{1z} = \mu_1 k_{2z},
%%%\end{dmath}
%%%
%%%or
%%%\begin{dmath}\label{eqn:emtLecture10:380}
%%%   \mu_2 \frac{\omega}{c} n_1 \cos \theta_1 =
%%%   \mu_1 \frac{\omega}{c} n_2 \cos \theta_2
%%%\end{dmath}
%%%
%%%or
%%%\begin{dmath}\label{eqn:emtLecture10:400}
%%%   \mu_2 \sqrt{\epsilon_1 \mu_1} \cos \theta_1 = \mu_1 \sqrt{\epsilon_2 \mu_2} \cos \theta_2,
%%%\end{dmath}
%%%
%%%which is
%%%\begin{dmath}\label{eqn:emtLecture10:420}
%%%   \sqrt{\epsilon_1/\mu_1} \cos \theta_1 = \sqrt{\epsilon_2/\mu_2} \cos \theta_2,
%%%\end{dmath}
%%%
%%%or
%%%\begin{dmath}\label{eqn:emtLecture10:440}
%%%   \eta_1 \cos \theta_1 = \eta_2 \cos \theta_2.
%%%\end{dmath}
%%%
%%%At normal incidence, this reduces to
%%%
%%%\begin{dmath}\label{eqn:emtLecture10:460}
%%%   \eta_1 = \eta_2.
%%%\end{dmath}
%%%
%%%What happens to the transmission coefficient for such a slab.
%%%
%%%One can find
%%%
%%%\begin{equation}\label{eqn:emtLecture10:480}
%%%t_{12} = t_{21} = 1,
%%%\end{equation}
%%%
%%%under a matched condition, so
%%%
%%%\begin{equation}\label{eqn:emtLecture10:500}
%%%   t = t_{12} t_{21} e^{j \phi}.
%%%\end{equation}
%%%
%%%The matched slab only introduces a phase delay (at the specific frequency for which the slab is matched).
%%%XX
%
\section{Group delay.}
%
Under matched conditions where \( t^{\textrm{TE}} = e^{j \phi} = e^{j k_{2z} d} \), we can write
%
\begin{dmath}\label{eqn:emtLecture10:520}
- \PD{\phi}{\omega}
= \PD{\omega}{} \lr{ k_{2z} d }
= d \PD{\omega}{k_{2z}}
= d /\PD{k_{2z}}{\omega}
= d / v_g,
\end{dmath}
%
so
%
\begin{dmath}\label{eqn:emtLecture10:540}
v_g = -\frac{d}{\PD{\phi}{\omega}}.
\end{dmath}
%FIXME: last relation?
%
This is called the \textAndIndex{group delay}, and is essentially the velocity of the peak of the wave form, as sketched in
%
F5.
%
This is different than the phase velocity \( v_p = \omega/k \), as illustrated in the sketch of
%
F2
%
If (FIXME: why?)
%
\begin{dmath}\label{eqn:emtLecture10:660}
v_g
= \PD{k}{\omega}
= 1/\PD{\omega}{k},
\end{dmath}
%
and in an unbounded medium
\begin{dmath}\label{eqn:emtLecture10:560}
   k = \frac{\omega}{c} n(\omega),
\end{dmath}
%
so
\begin{dmath}\label{eqn:emtLecture10:580}
\PD{\omega}{k}
=
\inv{c} \lr{ n(\omega) + \omega \PD{\omega}{n} },
\end{dmath}
%
so
\begin{equation}\label{eqn:emtLecture10:600}
v_g
= \frac{c}{n(\omega) + \omega \PD{\omega}{n} }
= \frac{c}{n_g}
\end{equation}
%
Since the phase is
%
\begin{dmath}\label{eqn:emtLecture10:680}
\phi = -\frac{\omega}{c} n(\omega) d
\end{dmath}
%
we can show that
\begin{dmath}\label{eqn:emtLecture10:700}
v_g
= \frac{d}{-\PD{\omega}{\phi}}
\equiv
\frac{d}{\tau_g}.
\end{dmath}
%
By bounded medium, it is meant that the medium is not matched.
%
\section{Brewster's angle.}
%
Self study (examinable!) Brewster's angle for a single interface: The angle for which there is no reflection.
%
Does it exist for a TE mode, and also for a TM mode, and if not, for which polarization?
%
Should find that there is no Brewster angle for one of the polarizations when \( \mu_1 = \mu_2 \).  It may be that it's TE that always has some reflection.
%
\section{Critical angle.}
%
Self study (examinable!) critical angle, the angle for which \( \Abs{r} = 1 \)
%
Does it exist for a TE mode, and also for a TM mode, and if not, for which polarization?
%
We should find that this only occurs when we go from a medium with a low index of refraction to one where with a higher index of refraction (i.e. \( \theta_i \ge \theta_c \).
%
For such an angle, what is the value of the phase?
%
We should find that there is an electric and magnetic field on the other side of the medium, but that there is no transmitted power on that side of the interface (i.e. Poynting vector is zero (no power transfer) when the angle of incidence is greater than \( \theta_c \)).
%
Can have photon tunnelling through the space between two prisms separated by an air gap
%
F4
%
where, despite the critical angle, the time for the photon to ``tunnel'' through the space between the prisms is less than the time for the photon to go through just the first prism.
%
\section{Vector and electrostatic (scalar) potentials.}
%
In electrostatics where
\begin{dmath}\label{eqn:emtLecture10:720}
\spacegrad \cross \BE  = 0,
\end{dmath}
%
the electric field must be the gradient of some function
\begin{dmath}\label{eqn:emtLecture10:740}
\BE = \pm \spacegrad V,
\end{dmath}
%
We pick negative to have things consistent with the notion of force.
%
In electrodynamics we have
\begin{dmath}\label{eqn:emtLecture10:760}
\spacegrad \cross \BE = - \PD{t}{\BB},
\end{dmath}
%
however, we also have
%
\begin{dmath}\label{eqn:emtLecture10:780}
\spacegrad \cdot \BB = 0,
\end{dmath}
%
so
\begin{dmath}\label{eqn:emtLecture10:800}
   \BB = \spacegrad \cross \BA.
\end{dmath}
%
This gives
\begin{dmath}\label{eqn:emtLecture10:820}
   \spacegrad \cross \BE = - \PD{t}{} \lr{ \spacegrad \cross \BA },
\end{dmath}
%
or
\begin{dmath}\label{eqn:emtLecture10:840}
0 =
\spacegrad \cross \lr{ \BE + \PD{t}{\BA} },
\end{dmath}
%
so this curled quantity can be the gradient of something
%
\begin{dmath}\label{eqn:emtLecture10:860}
\BE + \PD{t}{\BA} = \pm \spacegrad V.
\end{dmath}
%
We pick negative again, so
%
\begin{dmath}\label{eqn:emtLecture10:880}
\BE = -\spacegrad V -\PD{t}{\BA}.
\end{dmath}
%
Observe that when the fields are electrostatic with no time dependence, we would have \( \BE = -\spacegrad V \).
%
%}
%
%\EndArticle
\EndNoBibArticle
