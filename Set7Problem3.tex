%
% Copyright � 2016 Peeter Joot.  All Rights Reserved.
% Licenced as described in the file LICENSE under the root directory of this GIT repository.
%
\makeproblem{Orthogonality conditions for the fields.}{emt:problemSet7:3}{
Consider plane waves
\begin{equation}\label{eqn:emtproblemSet7Problem3:20}
\begin{aligned}
\BE &= \BE_0 e^{-j \Bk \cdot \Br + j \omega t }, \\
\BH &= \BH_0 e^{-j \Bk \cdot \Br + j \omega t },
\end{aligned}
\end{equation}
propagating in a homogeneous, lossless, source free region for which \( \epsilon > 0 \), \( \mu > 0 \), and where \( \BE_0, \BH_0 \) are constant.
\makesubproblem{}{emt:problemSet7:3a}
Show that
\( \Bk \perp \BE \) and \( \Bk \perp \BH \).
\makesubproblem{}{emt:problemSet7:3b}
Show that  \( \Bk, \BE, \BH \) form a right hand triplet as indicated in \cref{fig:kEHrightHandTriplet:kEHrightHandTripletFig1}.
\imageFigure{../figures/ece1228-electromagnetic-theory/kEHrightHandTripletFig1}{Right handed triplet.}{fig:kEHrightHandTriplet:kEHrightHandTripletFig1}{0.2}
\paragraph{Hint:} show that
\( \Bk \cross \BE = \omega \mu \BH \) and
\( \Bk \cross \BH = -\omega \epsilon \BE \).
\makesubproblem{}{emt:problemSet7:3c}
Now suppose \( \epsilon, \mu < 0 \), how does the figure change?  Redraw the figure.
} % makeproblem
\skipIfRedacted{
\makeanswer{emt:problemSet7:3}{
\makeSubAnswer{}{emt:problemSet7:3a}
Since \( \spacegrad \cdot \BD = 0 \) in a source free region, application of the divergence to the electric field of
\cref{eqn:emtproblemSet7Problem3:20} we have
\begin{equation}\label{eqn:emtproblemSet7Problem3:40}
\begin{aligned}
0
&= \spacegrad \cdot \BE
\\ &= \BE_0 \cdot \spacegrad e^{ -j \Bk \cdot \Br + j \omega t }
\\ &= \sum_m \BE_0 \cdot \Be_m \partial_m e^{ -j \Bk \cdot \Br + j \omega t }
\\ &= \sum_m \BE_0 \cdot \Be_m (-j k_m) e^{ -j \Bk \cdot \Br + j \omega t }
\\ &= (\BE_0 \cdot (-j \Bk)) e^{ -j \Bk \cdot \Br + j \omega t },
\end{aligned}
\end{equation}
so
\begin{equation}\label{eqn:emtproblemSet7Problem3:60}
\BE_0 \cdot \Bk = 0.
\end{equation}
For the magnetic field we also have zero divergence, and must also have this same perpendicular relationship, which is easily verified
\begin{equation}\label{eqn:emtproblemSet7Problem3:120}
\begin{aligned}
0
&= \spacegrad \cdot \BB
\\ &= \mu \BH_0 \cdot \spacegrad e^{ -j \Bk \cdot \Br + j \omega t }
\\ &= \mu (\BH_0 \cdot (-j \Bk)) e^{ -j \Bk \cdot \Br + j \omega t },
\end{aligned}
\end{equation}
so
\begin{equation}\label{eqn:emtproblemSet7Problem3:140}
\BH_0 \cdot \Bk = 0.
\end{equation}
This shows that \( \Bk \perp \BE \), and \( \Bk \perp \BH \) for plane waves in source free simple media.
\makeSubAnswer{}{emt:problemSet7:3b}
We can relate the electric and magnetic fields to each other with the time harmonic cross product equations
\begin{equation}\label{eqn:emtproblemSet7Problem3:80}
\begin{aligned}
\spacegrad \cross \BE &= - j \omega \BB, \\
\spacegrad \cross \BH &= j \omega \BD.
\end{aligned}
\end{equation}
For the magnetic field, we find
\begin{equation}\label{eqn:emtproblemSet7Problem3:100}
\begin{aligned}
\BH
&= \frac{1}{-j\mu \omega} \spacegrad \cross \BE
\\ &= \sum_m \frac{1}{-j\mu \omega} \Be_m \cross \BE_0 \partial_m e^{ -j \Bk \cdot \Br + j \omega t }
\\ &= \sum_m \frac{1}{-j\mu \omega} \Be_m \cross \BE_0 (-j k_m) e^{ -j \Bk \cdot \Br + j \omega t }
\\ &= \frac{1}{-j\mu \omega} \Bk \cross \BE_0 e^{ -j \Bk \cdot \Br + j \omega t }
\\ &= \frac{1}{\mu \omega} \Bk \cross \BE,
\end{aligned}
\end{equation}
and for the electric field
\begin{equation}\label{eqn:emtproblemSet7Problem3:180}
\begin{aligned}
\BE
&= \frac{1}{j \epsilon \omega} \spacegrad \cross \BH
\\ &= \sum_m \frac{1}{j \epsilon \omega} \Be_m \cross \BH_0 \partial_m e^{ -j \Bk \cdot \Br + j \omega t }
\\ &= \sum_m \frac{1}{j \epsilon \omega} \Be_m \cross \BH_0 (-j k_m) e^{ -j \Bk \cdot \Br + j \omega t }
\\ &= \frac{1}{j \epsilon \omega} \Bk \cross \BH_0 e^{ -j \Bk \cdot \Br + j \omega t }
\\ &= -\frac{1}{\epsilon \omega} \Bk \cross \BH.
\end{aligned}
\end{equation}
When \( \mu, \epsilon \) are both positive, this demonstrates a right handed relationship between \( \BH, \Bk \) and \( \BE \)
\boxedEquation{eqn:emtproblemSet7Problem3:200}{
\begin{aligned}
\BH &= \frac{1}{\mu \omega} \Bk \cross \BE, \\
\BE &= \frac{1}{\epsilon \omega} \BH \cross \Bk.
\end{aligned}
}
\makeSubAnswer{}{emt:problemSet7:3c}
When \( \mu, \epsilon \) are both less than zero, we have
\boxedEquation{eqn:emtproblemSet7Problem3:220}{
\begin{aligned}
\BH &= \frac{1}{\Abs{\mu} \omega} \BE \cross \Bk, \\
\BE &= \frac{1}{\Abs{\epsilon} \omega} \Bk \cross \BH.
\end{aligned}
}
Now \( \Bk, \BH, \BE \) form a right handed triplet, which means that \( \Bk, \BE, \BH \) is a left handed triplet.  Both the fields are still perpendicular to \( \Bk \).  This is sketched in \cref{fig:ps7p3:ps7p3Fig1}.
\imageFigure{../figures/ece1228-electromagnetic-theory/ps7p3Fig1}{Negative permittivities field orientation.}{fig:ps7p3:ps7p3Fig1}{0.2}
}}
