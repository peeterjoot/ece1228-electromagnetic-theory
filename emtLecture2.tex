%
% Copyright � 2016 Peeter Joot.  All Rights Reserved.
% Licenced as described in the file LICENSE under the root directory of this GIT repository.
%
%\input{../blogpost.tex}
%\renewcommand{\basename}{emt2}
%\renewcommand{\dirname}{notes/ece1228/}
%\newcommand{\keywords}{ECE1228H}
%\input{../latex/peeter_prologue_print2.tex}
%
%%\usepackage{ece1228}
%\usepackage{peeters_braket}
%%\usepackage{peeters_layout_exercise}
%\usepackage{peeters_figures}
%\usepackage{mathtools}
%\usepackage{siunitx}
%
%\beginArtNoToc
%\generatetitle{ECE1228H Electromagnetic Theory.  Lecture 2: Boundaries.  Taught by Prof.\ M. Mojahedi}
%\mychapter{Boundaries.}
\label{chap:emt2}
%
%\section{Disclaimer}
%
%Peeter's lecture notes from class.  These may be incoherent and rough.
%
%These are notes for the UofT course ECE1228H, Electromagnetic Theory, taught by Prof. M. Mojahedi, covering \textchapref{{1}} \citep{balanis1989advanced} content.
%
\section{Integral forms.}
%
Given Maxwell's equations at a point
%
\begin{equation}\label{eqn:emtLecture2:20}
\begin{aligned}
\spacegrad \cross \BE &= -\PD{t}{\BB} \\
\spacegrad \cross \BH &= \BJ + \PD{t}{\BD} \\
\spacegrad \cdot \BD &= \rho_\txtv \\
\spacegrad \cdot \BB &= 0,
\end{aligned}
\end{equation}
%
what happens when we have different fields and currents on two sides of a boundary?  To answer these questions, we want to use the integral forms of Maxwell's equations, over the geometries illustrated in \cref{fig:loopAndPillbox:loopAndPillboxFig1}.
\index{boundary}
%
\imageFigure{../figures/ece1228-electromagnetic-theory/loopAndPillboxFig1}{Loop and pillbox configurations.}{fig:loopAndPillbox:loopAndPillboxFig1}{0.2}
%
To do so, we use Stokes' and the divergence theorems relating the area and volume integrals to the surfaces of those geometries.
%
These are
%
\index{Stokes' theorem}
\index{divergence theorem}
\begin{equation}\label{eqn:emtLecture2:40}
\begin{aligned}
%\iint_S \lr{ \spacegrad \cross \BA } \cdot d\Bs &= \oint_C \BA \cdot d\Bl \\
\iint d\BA \cdot \lr{\spacegrad \cross \Bf} &= \ointctrclockwise \Bf \cdot d\Bl \\
%\iint_V \lr{ \spacegrad \cdot \BA } d\Bs &= \oint_A \BA \cdot d\Bs.
\iiint_{V} \spacegrad \cdot \Bf \, dV &= \iint_{\partial V} \Bf \cdot d\BA.
\end{aligned}
\end{equation}
%
\index{Faraday's law}
Application of Stokes' to Faraday's law we get
%
\begin{dmath}\label{eqn:emtLecture2:60}
\ointctrclockwise \BE \cdot d\Bl = -\PD{t}{} \iint \BB \cdot d\BA,
\end{dmath}
%
which has units \( \si{V} = \si{V/m \times m} \).
%
The quantity
\begin{dmath}\label{eqn:emtLecture2:80}
\iint \BB \cdot d\BA,
\end{dmath}
%
is called the magnetic flux of \( \BB \).  Changing of this flux is responsible for the generation of electromotive force.
\index{magnetic flux}
%
%F2:
Similarly
%
\begin{equation}\label{eqn:emtLecture2:100}
\begin{aligned}
\ointctrclockwise \BH \cdot d\Bl &= \iint \BJ \cdot d\BA + \PD{t}{} \iint \BD \cdot d\BA \\
\iint \BD \cdot d\BA &= \iiint \rho_\txtv dV = Q_\txte \\
\iint \BB \cdot d\BA &= 0.
\end{aligned}
\end{equation}
%
\index{constitutive relationships}
\section{Constitutive relations.}
%
With 12 unknowns in \( \BE, \BB, \BD, \BH \) and 8 equations in Maxwell's equations (or 6 if the divergence equations are considered redundant), things don't look too good for solutions.  In simple media, the fields may be have frequency mode relations of the form
%
\begin{equation}\label{eqn:emtLecture2:120}
\begin{aligned}
\BD( \Br, \omega ) &= \epsilon \BE( \Br, \omega ) \\
\BB( \Br, \omega ) &= \mu \BH( \Br, \omega ).
\end{aligned}
\end{equation}
%
\index{permeability}
\index{macroscopic}
The permeabilities \( \epsilon \) and \( \mu \) are macroscopic beasts, determined either experimentally, or theoretically using an averaging process involving many (millions, or billions, or more) particles.  However, the theoretical determinations that have been attempted do not work well in practise and usually end up considerably different than the measured values.  We are referred to \citep{jackson1975cew} for one attempt to model the statistical microscopic effects non-quantum mechanically to justify the traditional macroscopic form of Maxwell's equations.
%
These can be position dependent, as in the grating sketched in \cref{fig:gratingL2:gratingL2Fig3}.
%
\imageFigure{../figures/ece1228-electromagnetic-theory/gratingL2Fig3}{Grating.}{fig:gratingL2:gratingL2Fig3}{0.2}
%
\index{capacitor}
\index{breakdown voltage}
The permeabilities can also depend on the strength of the fields.  An example, application of an electric field to gallium arsenide or glass can change the behavior in the material.  We can also have non-linear effects, such as the effect on a capacitor when the voltage is increased.  The response near the breakdown point where the capacitor blows up demonstrates this spectacularly.  We can also have materials for which the permeabilities depend on the direction of the field, or the temperature, or the pressure in the environment, the tensile or compression forces on the material, or many other factors.  There are many other possible complicating factors, for example, the electric response \( \epsilon \) can depend on the magnetic field strength \( \Abs{\BB} \).  We could then write
%
\begin{dmath}\label{eqn:emtLecture2:140}
\epsilon = \epsilon( \Br, \Abs{\BE}, \BE/\Abs{\BE}, T, P, \Abs{\Beta}, \omega, k ).
\end{dmath}
%
The complex nature of \( \epsilon \) further complicates things
%
\index{anisotropic}
We can also have anisotropic situations where the electric and displacement fields are not (positive) scalar multiples of each other, as sketched in \cref{fig:constituativeRelationsL2:constituativeRelationsL2Fig4}.
%
\imageFigure{../figures/ece1228-electromagnetic-theory/constituativeRelationsL2Fig4}{Anisotropic field relations.}{fig:constituativeRelationsL2:constituativeRelationsL2Fig4}{0.2}
%
which indicates that the permittivity \( \epsilon \) in the relation
%
\begin{dmath}\label{eqn:emtLecture2:160}
\BD = \epsilon \BE,
\end{dmath}
%
can be modeled as a matrix or as a second rank tensor.  When the off diagonal entries are zero, and the diagonal values are all equal, we have the special case where \( \epsilon \) is reduced to a function.  That function may still be complex-valued, and dependent on many factors, but it least it is scalar valued in this situation.
%
\index{polarization}
\index{magnetization}
\section{Polarization and magnetization.}
%
If we have a material (such as glass), we can generally assume that the induced field can be related to the vacuum field according to
%
\begin{dmath}\label{eqn:emtLecture2:180}
\BE = \BP + \epsilon_0 \BE,
\end{dmath}
and
\begin{dmath}\label{eqn:emtLecture2:200}
\BB = \mu_0 \BM + \mu_0 \BH = \mu_0 \lr{ \BM + \BH }.
\end{dmath}
%
\index{permittivity!vacuum}
Here the vacuum permittivity \( \epsilon_0 \) has the value \( 8.85 \times 10^{-12} \si{F/m} \).  When we are ignoring (fictional) magnetic sources, we have a constant relation between the magnetic fields \( \BB = \mu_0 \BH \).
%
Assuming \( \BP = \epsilon_0 \chi_\txte \BE \), then
%
\begin{dmath}\label{eqn:emtLecture2:220}
\BD
= \epsilon_0 \BE + \epsilon_0 \chi_\txte \BE
= \epsilon_0 ( 1 + \chi_\txte ) \BE ,
\end{dmath}
%
so with \( \epsilon_r = 1 + \chi_\txte \), and \( \epsilon = \epsilon_0 \epsilon_r \) we have
%
\begin{dmath}\label{eqn:emtLecture2:240}
\BD = \epsilon \BE.
\end{dmath}
%
\index{permittivity!relative}
Note that the relative permittivity \( \epsilon_r \) is dimensionless, whereas the vacuum permittivity has units of \si{F/m}.  We call \(\epsilon\) the (unqualified) permittivity.  The relative permittivity \( \epsilon_r\) is sometimes called the relative permittivity.
%
\index{index of refraction}
Another useful quantity is the index of refraction
%
\begin{dmath}\label{eqn:emtLecture2:260}
\eta = \sqrt{ \epsilon_r \mu_r } \approx \sqrt{\epsilon_r}.
\end{dmath}
%
Similar to the above we can write \( \BM = \chi_\txtm \BH \) then
%
\begin{dmath}\label{eqn:emtLecture2:280}
\BM = \mu_0 \BH + \mu_0 \BM = \mu_0 \lr{ 1 + \chi_\txtm } \BH
= \mu_0 \mu_r \BH,
\end{dmath}
%
so with \( \mu_r = 1 + \chi_\txtm \), and \( \mu = \mu_0 \mu_r \) we have
%
\begin{dmath}\label{eqn:emtLecture2:300}
\BB = \mu \BH.
\end{dmath}
%
\section{Linear and angular momentum in light.}
%
\index{photon!momentum}
\index{photon!angular momentum}
It was pointed out that we have two relations in mechanics that relate momentum and forces
%
\begin{equation}\label{eqn:emtLecture2:320}
\begin{aligned}
\BF &= \ddt{\BP} \\
\Btau &= \ddt{\BL},
\end{aligned}
\end{equation}
%
where \( \BP = m \Bv \) is the linear momentum, and \( \BL = \Br \cross \Bp \) is the angular momentum.  In quantum electrodynamics, the photon can be described using a relationship between wave-vector and momentum
%
\begin{dmath}\label{eqn:emtLecture2:340}
\Bp
= \Hbar \Bk
= \Hbar \frac{ 2\pi}{\lambda}
= \frac{h}{2\pi} \frac{ 2\pi}{\lambda}
= \frac{h}{\lambda},
\end{dmath}
%
where \( \hbar = 6.522 \times 10^{-16} \si{ev.s} \).
%
Photons are also governed by
%
\begin{equation}\label{eqn:emtLecture2:360}
E = \Hbar \omega = h \nu.
\end{equation}
%
\index{De-Broglie relation}
(De-Broglie's relations).

\paragraph{ASIDE:} optical fibre at 1550 has the lowest amount of optical attenuation.
%
Since photons have linear momentum, we can move things around using light.  With photons having both linear momentum and energy relationships, and there is a relation between torque and linear momentum, it seems that there must be the possibility of light having angular momentum.
%
Is it possible to utilize the angular momentum to impose patterns on beams (such as laser beams).  For example, what if a beam could have a geometrical pattern along its line of propagation, being off in some regions, on in others.  This is in fact possible, generating beams that are ``self healing''.
%
The question was posed ``Is it possible to solve electromagnetic problems utilizing the force concepts?'', using the Lorentz
force equation
%
\begin{dmath}\label{eqn:emtLecture2:380}
\BF = q \Bv \cross \BB + q\BE.
\end{dmath}
%
This was not thought to be a productive approach due to the complexity.
%
%FIXME: It appeared that this animated talk (probably not captured well) about momentum in light was linked to the idea of the Helmholtz theorem.  Exactly how was not clear to me.
%
\index{Helmholtz's theorem}
\section{Helmholtz's theorem.}
%
Suppose that we have a linear material where
%
\begin{equation}\label{eqn:emtLecture2:400}
\begin{aligned}
\spacegrad \cross \BE &= -\PD{t}{\BB} \\
\spacegrad \cross \BH &= \BJ + \PD{t}{\BD} \\
\spacegrad \cdot \BE &= \frac{\rho_\txtv}{\epsilon_0} \\
\spacegrad \cdot \BH &= 0.
\end{aligned}
\end{equation}
%
We have relations between the divergence and curl of \( \BE \) given the sources.  Is that sufficient to determine \( \BE \) itself?  The answer is yes, which is due to the Helmholtz theorem.

Extra homework question (bonus) : can knowledge of the tangential components of the fields also be used to uniquely determine \( \BE \)?
%
%Also homework: read notes about irrotational fields and solenoidal fields.
%
%\EndArticle
